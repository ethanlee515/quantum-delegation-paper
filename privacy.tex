\section{Private Delegation}

\Ethan{TODO Do some kinda intro to the section}

\subsection{Quantum homomorphic encryption scheme}

\Ethan{This subsection is copied and pasted directly from the other QMPC paper and is kinda a mess}

We present the interface of quantum full homomorphic encryption (QFHE) scheme, $\mathsf{QHE}=(\mathsf{QHE.keygen}, \mathsf{QHE.Enc}, \mathsf{QHE.Dec}, \mathsf{QHE.Eval})$ given in \cite{mahadev_qfhe}. We use this particular QFHE because it allows the use of a classical client. Specifically, it has the following extra properties:
\begin{itemize}
	\item $\mathsf{Keygen}$ can be done classically.
	\item In the case where the plaintext is classical, $\mathsf{Enc}$ can be done classically.
	\item $\mathsf{Dec}$ commutes with standard basis measurement. If the ciphertext is a measurement result, it can also be done classically.
\end{itemize}

\subsection{Generalizing our delegation protocol using QFHE}

We construct a verifiable, private, and constant round $\QPIP_0$ protocol by combining our $\QPIP_0$ protocol earlier with \cite{mahadev_qfhe}.

\floatname{algorithm}{Protocol}
\begin{algorithm}
	\caption{Verifiable, private, and constant round delegation}
	\label{ProtoPrivateDelegation}
	\begin{algorithmic}[1]
		\Procedure{Delegation}{C, x}
			\State $\mathsf{Keygen}\rightarrow(pk, evk, sk)$
			\State Let $\tilde{x}=\mathsf{Enc}_{pk}(x)$
			\State Let $\tilde{C}$ to the circuit that takes $\tilde{x}$ as input to evaluate $\mathsf{Eval}_{evk}(C, \tilde{x})$
			\State The client delegates $\tilde{C}(\tilde{x})$ to the server using our $\QPIP_0$ protocol.
		\EndProcedure
	\end{algorithmic}
\end{algorithm}

\begin{thm}
	\autoref{ProtoPrivateDelegation} is IND-CPA secure.
\end{thm}
\begin{proof}
	TODO
\end{proof}

\begin{thm}
	\autoref{ProtoPrivateDelegation} is a $\QPIP$ scheme for... (TODO) \Ethan{Just copy from earlier and hope I don't need to rewrite too much}
\end{thm}
\begin{proof}
	TODO
\end{proof}
