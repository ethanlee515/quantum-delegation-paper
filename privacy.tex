\section{Private Delegation}

\Ethan{TODO Do some kinda intro to the section}

First we give a definition of CPA security for delegation schemes... TODO

\begin{dfn}
	A $\QPIP_\tau$ protocol is IND-CPA secure if, for any polynomial time prover $\cA$,
	$$\abs{\Pr[\cA]}$$
	where $x$ is the input of the verifier.
\end{dfn}

\subsection{Quantum homomorphic encryption scheme}

\Ethan{This subsection is copied and pasted directly from the other QMPC paper and is kinda a mess}

We present the quantum full homomorphic encryption (QFHE) scheme, $\mathsf{QHE}=(\mathsf{QHE.keygen}, \mathsf{QHE.Enc}, \mathsf{QHE.Dec}, \mathsf{QHE.Eval})$ given in \cite{mahadev_qfhe}. We use this particular QFHE because it allows the use of a classical client. Specifically, it has the following extra properties:
\begin{itemize}
	\item $\mathsf{QHE.Keygen}$ can be done classically.
	\item In the case where the plaintext is classical, $\mathsf{QHE.Enc}$ can be done classically.
	\item $\mathsf{QHE.Dec}$ ``commutes" with measurements. \Ethan{Need to elaborate just a bit more here; i.e. need to mention X/Z only} If the ciphertext is a measurement result, it can also be done classically.
\end{itemize}

\subsection{Generalizing our delegation protocol using QFHE}

\Ethan{TODO intro}

Here's what the client should do.

\begin{algorithm}
	\caption{Verifiable, secure, and constant round delegation}
	\label{ProtoPrivateDelegation}
	\begin{algorithmic}[1]
		\Procedure{Delegation}{C, x}
			\State Compute $\mathsf{QHE.Keygen}\rightarrow(pk, evk, sk)$
			\State Compute $\tilde{x}=\mathsf{Enc}_{pk}(x)$
			\State Let $\tilde{C}$ be the quantum circuit that takes $\tilde{x}$ as input to evaluate $\mathsf{Eval}_{evk}(C, \tilde{x})$
			\State Delegate $\tilde{C}(\tilde{x})$ to the server using our $\QPIP_0$ protocol.
			\State Get $y$ from the server
			\State \Return $\mathsf{QHE.Dec}(y)$
		\EndProcedure
	\end{algorithmic}
\end{algorithm}

\begin{thm}
    \label{QPIP1thm}
	\protoref{ProtoPrivateDelegation} can evaluate any $L\in\SampBQP$ with negligible completeness and soundness $(O(T^{-c}), O(T^{-c}))$ for any constant $c$.
\end{thm}
\begin{proof}
	The client will receive a sample from $\mathsf{Eval}_{evk}(C, \tilde{x})$. \Ethan{Errors terms?}

	Which by the properties of homomoprhic encryption should decode to $\rho$ that is $\varepsilon$-indistinguishable to $C(x)$ \Ethan{Fill in error terms}

	As decoding commutes with measurements, the verifier requests its desired measurements of $\rho$ and decode it.
\end{proof}

\begin{thm}
	\autoref{ProtoPrivateDelegation} is IND-CPA secure.
\end{thm}
\begin{proof}
	The verifier's first message is encrypted into $\tilde{x}$ in an IND-CPA way as a ciphertext.
	The verifier's second message is the basis choice of Hadamard or test rounds, which can be done using public coins.

	The verifier only sends the prover these two messages, so it follows \Ethan{hopefully?} easily that the protocol itself is also IND-CPA. \Ethan{Maybe not rigorous enough?}
\end{proof}
