\section{Private Delegation}

\Ethan{TODO Do some kinda intro to the section}

\subsection{Quantum homomorphic encryption scheme}

\Ethan{This subsection is copied and pasted directly from the other QMPC paper and is kinda a mess}

We present the interface of quantum full homomorphic encryption (QFHE) scheme as given in \cite{mahadev_qfhe}.

\begin{dfn}
	A homomophic encryption scheme is tuple of algorithms $\mathsf{HE}=(\mathsf{HE.Keygen}, \mathsf{HE.Enc}, \mathsf{HE.Dec}, \mathsf{HE.Eval})$ with the following descriptions:
	\begin{itemize}
		\item $\mathsf{Keygen}(1^\lambda)\rightarrow(pk, evk, sk)$
		\item $\mathsf{Enc}_{pk}(\mu)\rightarrow c$
		\item $\mathsf{Dec}_{sk}(c)\rightarrow \mu^*$
			\Ethan{Urmila didn't specify that $\mathsf{Dec}\circ\mathsf{Enc}=Id$}
		\item $\mathsf{Eval}_{evk}(f, c_1, \ldots, c_l)\rightarrow c_f$
	\end{itemize}
\end{dfn}

Everything also has an implicit dependence on $1^L$, where $L$ is the maximum circuit depth for the homomorphic evaluation.
As this is a homomorphic encryption scheme, it satisfies $$\mathsf{HE.Dec}_{sk}(c_f)=f(\mathsf{HE.Dec}_{sk}(c_0),\ldots,\mathsf{HE.Dec}_{sk}(c_l))$$ with all but negligible probability in $\lambda$

$\mathsf{Keygen}$ and $\mathsf{Enc}$ can be done classically. $\mathsf{Dec}$ commutes with standard basis measurement, and in the event that the result is already a measurement result, can also be done classically.

We also recall the security definition for a FHE scheme.

\begin{dfn}
	A FHE scheme $\mathsf{HE}$ is IND-CPA secure if, for any polynomial time adversary $\cA$, there exists a negligible function $\mu(\cdot)$ such that
	$$\abs{Pr[\cA(pk, evk, \mathsf{HE.Enc}_{pk}(0))=1]-Pr[\cA(pk, evk, \mathsf{HE.Enc}_{pk}(1))=1]}=\mu(\lambda)$$
	where $(pk, evk, sk)\leftarrow\mathsf{QHE.Keygen}(1^\lambda)$
	\Ethan{Urmila didn't specify quantum adversary here? Also dropped the $1^L$.}
\end{dfn}

\subsection{Generalizing our delegation protocol using QFHE}

We construct a verifiable, private, and constant round $\QPIP_0$ protocol by combining our $\QPIP_0$ protocol earlier with \cite{mahadev_qfhe}.

\floatname{algorithm}{Protocol}
\begin{algorithm}
	\caption{Verifiable, private, and constant round delegation}
	\label{ProtoPrivateDelegation}
	\begin{algorithmic}[1]
		\Procedure{Delegation}{C, x}
			\State $\mathsf{Keygen}\rightarrow(pk, evk, sk)$
			\State Let $\tilde{x}=\mathsf{Enc}_{pk}(x)$
			\State Let $\tilde{C}$ to the circuit that takes $\tilde{x}$ as input to evaluate $\mathsf{Eval}_{evk}(C, \tilde{x})$
			\State The client delegates $\tilde{C}(\tilde{x})$ to the server using our $\QPIP_0$ protocol.
		\EndProcedure
	\end{algorithmic}
\end{algorithm}

\begin{thm}
	\autoref{ProtoPrivateDelegation} is IND-CPA secure.
\end{thm}
\begin{proof}
	TODO
\end{proof}

\begin{thm}
	\autoref{ProtoPrivateDelegation} is a $\QPIP$ scheme for... (TODO) \Ethan{Just copy from earlier and hope I don't need to rewrite too much}
\end{thm}
\begin{proof}
	TODO
\end{proof}
