\section{Private Delegation}

\Ethan{TODO Do some kinda intro to the section}

\subsection{Quantum homomorphic encryption scheme}

\Ethan{This subsection is copied and pasted directly from the other QMPC paper and is kinda a mess}

We present the interface of quantum full homomorphic encryption (QFHE) scheme as given in \cite{mahadev_qfhe}.

\begin{dfn}
	A homomophic encryption scheme is tuple of algorithms $\mathsf{HE}=(\mathsf{HE.Keygen}, \mathsf{HE.Enc}, \mathsf{HE.Dec}, \mathsf{HE.Eval})$ with the following descriptions:
	\begin{itemize}
		\item $\mathsf{HE.Keygen}(1^\lambda, 1^L)\rightarrow(pk, evk, sk)$
			\begin{itemize}
				\item $\lambda$: security parameter
				\item $L$: maximum size of circuits that can be homomorphically evaluated
			\end{itemize}
		\item $\mathsf{HE.Enc_{pk}}(\mu)\rightarrow c$
		\item $\mathsf{HE.Dec_{sk}}(c)\rightarrow \mu^*$
			\Ethan{Urmila didn't specify that $\mathsf{Dec}\circ\mathsf{Enc}=Id$}
		\item $\mathsf{HE.Eval_{evk}}(f, c_1, \ldots, c_l)\rightarrow c_f$
			\begin{itemize}
				\item $f$: a circuit of depth at most $L$
				\item Satisfies $$\mathsf{HE.Dec}_{sk}(c_f)=f(\mathsf{HE.Dec}_{sk}(c_1),\ldots,\mathsf{HE.Dec}_{sk}(c_l))$$ with all but negligible probability in $\lambda$
			\end{itemize}
	\end{itemize}
	\Ethan{Figure out what else depends on L}
\end{dfn}

We also recall the security definition for a FHE scheme.

\begin{dfn}
	A FHE scheme $\mathsf{HE}$ is IND-CPA secure if, for any polynomial time adversary $\cA$, there exists a negligible function $\mu(\cdot)$ such that
	$$\abs{Pr[\cA(pk, evk, \mathsf{HE.Enc}_{pk}(0))=1]-Pr[\cA(pk, evk, \mathsf{HE.Enc}_{pk}(1))=1]}=\mu(\lambda)$$
	where $(pk, evk, sk)\leftarrow\mathsf{QHE.Keygen}(1^\lambda)$
	\Ethan{Urmila didn't specify quantum adversary here? Also dropped the $1^L$.}
\end{dfn}

\begin{thm}
	There exists a quantum homomorphic encryption scheme $\mathsf{QHE}$ with the following additional
	\begin{itemize}
		\item $\mathsf{QHE.Keygen}$ additionally takes $1^L$ as parameter
		\item There exists a classical encryption scheme $\mathsf{HE}$ such that $\mathsf{QHE}$'s ciphertexts has the form $TODO$, where $c$ encodes $(x, z)$ under $\mathsf{HE}$.
		\item It allow finer controls of homomorphically applying a circuit on a gate-by-gate basis. There is, there exists a universal gate set $G$ and function $\mathsf{QHE.EvalGate}(TODO)$ that outputs ``as expected" \Ethan{huh?} as long as the corresponding \Ethan{hmm...} circuit depth isn't over $L$.
	\end{itemize}
\end{thm}

\subsection{Generalizing our delegation protocol using QFHE}

\floatname{algorithm}{Protocol}
\begin{algorithm}
	\caption{Verifiable, private, and constant round delegation}
	\label{ProtoQPIP1}
	\begin{algorithmic}[1]
		\State $\mathsf{Keygen}\rightarrow(pk, evk, sk)$
		\State $\mathsf{Enc}_{pk}(x)=c$
		\State The client encrypts its input
		\State The server and the client runs the $\QPIP_0$ protocol homomorphically on this encrypted input \Ethan{Maybe say a bit more since it's not exactly a circuit to begin with... Or I guess it has to be written as one}
	\end{algorithmic}
\end{algorithm}

TODO prove semantic security (shouldn't be hard)

TODO prove this still has verifiability
