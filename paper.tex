\documentclass{article}
\usepackage{authblk}
\usepackage{amsmath, amssymb, amsthm}
\usepackage{braket}
\usepackage{algorithm}
\usepackage{algpseudocode}
\newcommand{\sgn}{\operatorname{sgn}}
\DeclareMathOperator*{\E}{\mathbb{E}}
\DeclareMathOperator*{\spn}{\operatorname{span}}
\DeclareMathOperator*{\poly}{\operatorname{poly}}
\newcommand{\norm}[1]{\left\lVert#1\right\rVert}
\usepackage{etoolbox}
\usepackage{hyperref}
\apptocmd{\thebibliography}{\raggedright}{}{}
\usepackage{cite,color,float}


\newcommand{\KM}[1]{{\footnotesize\color{cyan}[KM: #1]}}


\newtheorem{theorem}{Theorem}[section]
\theoremstyle{definition}
\newtheorem{definition}{Definition}[section]
\newtheorem{lemma}[theorem]{Lemma}

\title{Placeholder Title... Something About Delegating Quantum Computations}

\author[1]{Kai-Min Chung}
\author[1]{Yi Lee}
\author[2]{Han-Hsuan Lin}
\author[3]{Xiaodi Wu}
\affil[1]{Institute of Information Science, Academia Sinica, Taipei, Taiwan}
\affil[2]{Department of Computer Science, University of Texas at Austin}
\affil[3]{
	Department of Computer Science, Institute for Advanced Computer Studies,
	and Joint Center for Quantum Information and Computer Science,
	University of Maryland, USA
}
\usepackage{graphicx,amsmath, amssymb,color,url,booktabs,comment}  %cite
\urlstyle{sf}
%\usepackage[margin=1in]{geometry}
% \usepackage{fancyhdr}
%\usepackage[colorlinks]{hyperref} %pagebackref

\newcommand{\nc}{\newcommand}
\nc{\rnc}{\renewcommand}
%
%\newcommand{\bra}[1]{\langle #1|}
%\newcommand{\ket}[1]{|#1\rangle}
\newcommand{\proj}[1]{|#1\rangle\langle #1|}
% \newcommand{\braket}[2]{\langle #1|#2\rangle}
% \newcommand{\Bra}[1]{\left\langle #1\right|}
% \newcommand{\Ket}[1]{\left|#1\right\rangle}
\newcommand{\Proj}[1]{\left|#1\right\rangle\left\langle #1\right|}
% \newcommand{\Braket}[2]{\left\langle #1\middle|#2\right\rangle}
\nc{\vev}[1]{\langle#1\rangle}
\nc{\grad}{{\vec{\nabla}}}
\nc{\abs}[1]{\lvert#1\rvert}
%\DeclareMathOperator{\abs}{abs}
\DeclareMathOperator{\Bin}{Bin}
\DeclareMathOperator{\conv}{conv}
\DeclareMathOperator{\eig}{eig}
\DeclareMathOperator{\Hist}{Hist}
\DeclareMathOperator{\Hyb}{Hyb}
\DeclareMathOperator{\id}{id}
\DeclareMathOperator{\Img}{Im}
\DeclareMathOperator{\Par}{Par}
% \DeclareMathOperator{\poly}{poly}
\DeclareMathOperator{\negl}{negl}
\DeclareMathOperator{\polylog}{polylog}
\DeclareMathOperator{\tr}{tr}
\DeclareMathOperator{\rank}{rank}
% \DeclareMathOperator{\sgn}{sgn}
\DeclareMathOperator{\Sep}{Sep}
\DeclareMathOperator{\SepSym}{SepSym}
\DeclareMathOperator{\Span}{span}
\DeclareMathOperator{\supp}{supp}
\DeclareMathOperator{\swap}{SWAP}
\DeclareMathOperator{\Sym}{Sym}
\DeclareMathOperator{\ProdSym}{ProdSym}
\DeclareMathOperator{\SEP}{SEP}
\DeclareMathOperator{\PPT}{PPT}
\DeclareMathOperator{\Wg}{Wg}
\DeclareMathOperator{\WMEM}{WMEM}
\DeclareMathOperator{\WOPT}{WOPT}

\DeclareMathOperator{\BPP}{\mathsf{BPP}}
\DeclareMathOperator{\QPIP}{\mathsf{QPIP}}
\DeclareMathOperator{\SampBQP}{\mathsf{SampBQP}}
\DeclareMathOperator{\BQP}{\mathsf{BQP}}
\DeclareMathOperator{\FBQP}{\mathsf{FBQP}}
\DeclareMathOperator{\cnot}{\normalfont\textsc{cnot}}
\DeclareMathOperator{\DTIME}{\mathsf{DTIME}}
\DeclareMathOperator{\NTIME}{\mathsf{NTIME}}
\DeclareMathOperator{\MA}{\mathsf{MA}}
\DeclareMathOperator{\NP}{\mathsf{NP}}
\DeclareMathOperator{\NEXP}{\mathsf{NEXP}}
\DeclareMathOperator{\Ptime}{\mathsf{P}}
\DeclareMathOperator{\QMA}{\mathsf{QMA}}
\DeclareMathOperator{\QCMA}{\mathsf{QCMA}}
\DeclareMathOperator{\BellQMA}{\mathsf{BellQMA}}

\newcommand{\be}{\begin{equation}}
\newcommand{\ee}{\end{equation}}
\newcommand{\bea}{\begin{eqnarray}}
\newcommand{\eea}{\end{eqnarray}}
\newcommand{\nn}{\nonumber}
\newcommand{\bi}{\begin{itemize}}
\newcommand{\ei}{\end{itemize}}
\newcommand{\bn}{\begin{enumerate}}
\newcommand{\en}{\end{enumerate}}
\def\beas#1\eeas{\begin{eqnarray*}#1\end{eqnarray*}}
\def\ba#1\ea{\begin{align}#1\end{align}}
\nc{\bas}{\[\begin{aligned}}
\nc{\eas}{\end{aligned}\]}
\nc{\bpm}{\begin{pmatrix}}
\nc{\epm}{\end{pmatrix}}
\def\non{\nonumber}
\def\nn{\nonumber}
\def\eq#1{(\ref{eq:#1})}
\def\eqs#1#2{(\ref{eq:#1}) and (\ref{eq:#2})}
%\def\eq#1{Eq.~(\ref{eq:#1})}
%\def\eqs#1#2{Eqs.~(\ref{eq:#1}) and (\ref{eq:#2})}
\def\L{\left} 
\def\R{\right}
\def\ra{\rightarrow}
\def\ot{\otimes}

\newtheorem{thm}{Theorem}[section]
\newtheorem{theorem}{Theorem}[section]
%\newtheorem*{thm*}{Theorem}
%\newtheorem{claim}[thm]{Claim}
\newtheorem{cor}{Corollary}[thm]
\newtheorem{lem}{Lemma}[section]
\newtheorem{lemma}{Lemma}[section]
\newtheorem{rmk}{Remark}[thm]
%\newtheorem{prop}[thm]{Proposition}
\newtheorem{dfn}{Definition}[section]
\newtheorem{definition}{Definition}[section]
%\newtheorem{con}[thm]{Conjecture}

\newenvironment{prf}{\begin{proof}}{\end{proof}}

\def\eps{\epsilon}
\def\va{{\vec{a}}}
\def\vb{{\vec{b}}}
\def\vn{{\vec{n}}}
\def\cvs{{\cdot\vec{\sigma}}}
\def\vx{{\vec{x}}}
\def\Usch{U_{\text{Sch}}}

\def\cA{\mathcal{A}}
\def\cB{\mathcal{B}}
\def\cD{\mathcal{D}}
\def\cE{\mathcal{E}}
\def\cF{\mathcal{F}}
\def\cH{\mathcal{H}}
\def\cI{{\cal I}}
\def\cL{{\cal L}}
\def\cM{{\cal M}}
\def\cN{\mathcal{N}}
\def\cO{{\cal O}}
\def\cP{\mathcal{P}}
\def\cQ{\mathcal{Q}}
\def\cS{\mathcal{S}}
\def\cT{{\cal T}}
\def\cU{\mathcal{U}}
\def\cV{\mathcal{V}}
\def\cW{{\cal W}}
\def\cX{{\cal X}}
\def\cY{{\cal Y}}

\def\bp{\mathbf{p}}
\def\bq{\mathbf{q}}
\def\bP{{\bf P}}
\def\bQ{{\bf Q}}
\def\gl{\mathfrak{gl}}

\def\bbC{\mathbb{C}}
% \DeclareMathOperator*{\E}{\mathbb{E}}
\DeclareMathOperator*{\bbE}{\mathbb{E}}
%\DeclareMathOperator*{\Pr}{Pr}
\nc{\Prob}[1]{\ensuremath{\Pr\left[#1\right]}}
\def\bbM{\mathbb{M}}
\def\bbN{\mathbb{N}}
\def\bbR{\mathbb{R}}
\def\bbZ{\mathbb{Z}}
\def\bbP{\mathbb{P}}
\def\bbV{\mathbb{V}}
\newcommand{\Real}{\textrm{Re}}

\def\benum{\begin{enumerate}}
\def\eenum{\end{enumerate}}
% \def\bit{\begin{itemize}}
% \def\eit{\end{itemize}}
\def\bdesc{\begin{description}}
\def\edesc{\end{description}}
\newcommand{\fig}[1]{Fig.~\ref{fig:#1}}
\newcommand{\tab}[1]{Table~\ref{tab:#1}}
\newcommand{\secref}[1]{Section~\ref{sec:#1}}
\newcommand{\appref}[1]{Appendix~\ref{sec:#1}}
\newcommand{\lemref}[1]{Lemma~\ref{lem:#1}}
\newcommand{\thmref}[1]{Theorem~\ref{thm:#1}}
\newcommand{\propref}[1]{Proposition~\ref{prop:#1}}
\newcommand{\protoref}[1]{Protocol~\ref{proto:#1}}
\nc{\myprotoref}[1]{\hyperref[#1]{Protocol~\ref*{#1}}}
\newcommand{\defref}[1]{Definition~\ref{def:#1}}
\newcommand{\corref}[1]{Corollary~\ref{cor:#1}}
\newcommand{\conref}[1]{Conjecture~\ref{con:#1}}

\newcommand{\FIXME}[1]{{\color{red}FIXME: #1}}
\nc{\todo}[1]{\textcolor{red}{todo: #1}}



\newcommand{\boxdfn}[2]{
\begin{figure}[h]
\begin{center}
\noindent \framebox{
\begin{minipage}{0.8\textwidth}
\begin{dfn}[{\bf #1}]
\ \\ \\
#2
\end{dfn}
\end{minipage}
}
\end{center}
\end{figure}
}

\newcommand{\boxproto}[2]{
\begin{figure}[h]
\begin{center}
\noindent \framebox{
\begin{minipage}{0.8\textwidth}
\begin{proto}[{\bf #1}]
\ \\ \\
#2
\end{proto}
\end{minipage}
}
\end{center}
\end{figure}
}

\def\begsub#1#2\endsub{\begin{subequations}\label{eq:#1}\begin{align}#2\end{align}\end{subequations}}
\nc\qand{\qquad\text{and}\qquad}
\nc\mnb[1]{\medskip\noindent{\bf #1}}
\nc\mn{\medskip\noindent}

\renewcommand{\arraystretch}{1.5}
%\nc{\problem}[1]{\item\noindent {\bf #1}}

\setlength{\tabcolsep}{10pt}

%%%%%% Han-Hsuan's commands %%%%%%%%
\nc{\nl}{\nn \\ &=}  %new line
\nc{\nnl}{\nn \\ &}  %new new line
\nc{\fot}{\frac{1}{2}} %frac one two
\nc{\oo}[1]{\frac{1}{#1}} % one over
\newcommand{\ben}{\begin{enumerate}}
\newcommand{\een}{\end{enumerate}}
\nc{\mc}{\mathcal}
\nc{\beq}{\begin{equation}}
\nc{\eeq}{\end{equation}}
% \nc{\norm}[1]{\L\| #1 \R\|}

\nc{\onenorm}[1]{\L\| #1 \R\|_1} %one norm
%\nc{\span}{\ensuremath{\mathrm{span}}}

\DeclareMathOperator*{\argmax}{arg\,max}

%\nc{1}

\newcommand{\hannote}[1]{\textcolor{blue}{\small {\textbf{(Han:} #1\textbf{) }}}}

\newcommand{\Knote}[1]{\textcolor{red}{\small {\textbf{(KM:} #1\textbf{) }}}}

\nc{\Ra}{\Rightarrow}
\nc{\zo}{\{0,1\}}	

%%%%import..
% \newcommand{\secpar}{n}


% %%%Efficient Verifier%
% \newcommand{\setupeff}{\setup_{\mathsf{eff}}}
% \newcommand{\vereff}{V_{\mathsf{eff}}}
% \newcommand{\vereffone}{V_{\mathsf{eff},1}}
% \newcommand{\vereffthree}{V_{\mathsf{eff},3}}
% \newcommand{\vereffout}{V_{\mathsf{eff},\mathsf{out}}}
% \newcommand{\proeff}{P_{\mathsf{eff}}}
% \newcommand{\proefftwo}{P_{\mathsf{eff},2}}
% \newcommand{\proefffour}{P_{\mathsf{eff},4}}
% \newcommand{\advPH}{{P^*}^{H}}
% \newcommand{\setup}{\mathsf{Setup}}
% \newcommand{\re}{\mathsf{re}}
% \newcommand{\crh}{\mathsf{CRH}}
% \newcommand{\transcript}{\mathsf{trans}}

% \newcommand{\setupefffs}{\setup_{\mathsf{eff}\text{-}\mathsf{fs}}}
% \newcommand{\proefffs}{P_{\mathsf{eff}\text{-}\mathsf{fs}}}
% \newcommand{\proefffstwo}{P_{\mathsf{eff}\text{-}\mathsf{fs},2}}
% \newcommand{\verefffs}{V_{\mathsf{eff}\text{-}\mathsf{fs}}}
% \newcommand{\verefffsone}{V_{\mathsf{eff}\text{-}\mathsf{fs},1}}
% \newcommand{\verefffsout}{V_{\mathsf{eff}\text{-}\mathsf{fs},\out}}
% %Games%
% \newcommand{\game}{\mathsf{Game}}

% %\newcommand*{\bra}[1]{\langle#1|}
% %\newcommand*{\ket}[1]{|#1\rangle}
% \newcommand*{\opro}[2]{|#1\rangle\langle#2|}
% \newcommand*{\ipro}[2]{\langle #1|#2\rangle}
% \newcommand{\TD}{\mathsf{TD}}

% %%%%% Registers %%%%%%%
% \newcommand*{\regK}{\mathbf{K}}
% \newcommand*{\regI}{\mathbf{I}}
% \newcommand*{\regR}{\mathbf{R}}
% \newcommand*{\regX}{\mathbf{X}}
% \newcommand*{\regY}{\mathbf{Y}}
% \newcommand*{\regZ}{\mathbf{Z}}
% \newcommand{\regW}{\mathbf{W}}
% \newcommand*{\regC}{\mathbf{C}}
% \newcommand*{\regO}{\mathbf{O}}
% \newcommand*{\regF}{\mathbf{F}}

\begin{document}

\maketitle

\begin{abstract}

This is a placeholder. Lorem ipsum. Lorem ipsum. Lorem ipsum...

\end{abstract}

\section{Introduction}
% \XW{
% \begin{itemize}
%     \item add a lot of references.
%     \item comparison with the most relevant results:
%       \begin{itemize}
%           \item Sampling, the only paper; how about classical sampling?
%           \item the following for BQP  
%           \item blind and verifiable ~\cite{GV19}; we  constant round; technique-wise very different.
%           \item there is a table in~\cite{Grilo19}. Safe to say ~\cite{GV19} only existing blind protocol in the computational setting?
%           \item all previous either quantum clients, or at least 2 provers.
%           \item the following for blindness
%           \item Mahadev in her thesis~\cite{mahadev_2018} discussed a bit about the relation between verifiability and blindness. She hoped to get verifiability out of blindness by designing some non-malleable QFHE but failed.   
%           \item what's the high-level message we can say here?  Use QFHE in a different way? It is correct that not much work in the classical setting either.  Maybe existing classical work employs the principle but with different implementation.
%           \item old approach, first get blindness  (measurement-based, self-testing), and then try to make it verifiable; our approach, first have a verifiable protocol, and upgrade by a QFHE.
%           \item directly QFHE (blindness) won't give verifiability. some thoughts from Mahadev.
%           \item
%       \end{itemize}
% \end{itemize}
% }

Can quantum computation, with potential computational advantages that are intractable for classical computers,
be efficiently verified by classical clients?
This seeming paradox has been one of the central problems in quantum complexity theory and delegation of quantum computation~\cite{web:Aaronson}.
From a philosophical point of view, this question is also known to have a fascinating connection to the \emph{falsifiability} of quantum mechanics in the potential high complexity regime~\cite{survey:AV12}.

A complexity theoretic formulation of this problem by Gottesman in 2004~\cite{web:Aaronson} asks the possibility for an efficient classical verifier/client (a $\BPP$ machine) to verify the output of an
efficient quantum prover (a $\BQP$ machine).
In the absence of techniques for directly tackling this question, earlier feasibility results on this problem have been focusing on two weaker formulations.
The first type of feasibility results (e.g.,~\cite{BFK09,arXiv:ABOEM17,FK17,mf16}) considers the case where the $\BPP$ verifier is equipped with limited quantum power.
The second type of feasibility results (e.g,~\cite{Nat:RUV13, CGJV19, Gheorghiu_2015, HPF15})
considers a $\BPP$ verifier interacting with at least two entangled, non-communicating quantum provers.
In a recent breakthrough, Mahadev~\cite{FOCS:Mahadev18a} proposed the first protocol of classical verification of quantum computation (CVQC) whose soundness is based on a widely recognized computational assumption that the learning with error (LWE) problem~\cite{JACM:Regev09} is hard for $\BQP$ machines.
The technique invented therein has inspired many  subsequent developments of CVQC protocols with improved parameters and functionality (e.g.,~\cite{FOCS:GheVid19,arXiv:AlaChiHun19,arXiv:ChiaChungYam19}).
On the other side, there are known complexity-theoretic limitations on the feasibility of blind CVQC in the information-theoretical setting (e.g.~\cite{aaronson_et_al:LIPIcs:2019:10582}).
%\XW{here some no-go for information theoretical result.}
%We refer curious readers to a slightly outdated  survey~\cite{survey:GKK19} for details.

With the newly developed techniques, we revisit the classical verification of quantum computation problems from both a philosophical and a practical point of view.
We first observe that the \emph{sampling} version of $\BQP$ (e.g., the class $\SampBQP$ formulated by Aaronson~\cite{aaronson_2013}) might be a more appropriate notion to serve the purpose of the original problem.
Philosophically, the outcomes of quantum mechanical experiments are usually samples or statistical information, which is well demonstrated in the famous double-slit experiment.
Moreover, a lot of quantum algorithms from Shor's~\cite{Shor} and Grover's~\cite{Grover} algorithms to some recent developments in machine learning and optimization (e.g.~\cite{brando_et_al:LIPIcs:2019:10603, AGGW17,pmlr-v97-li19b}) contain a significant quantum sampling component.
The fact that almost all quantum supremacy tasks (e.g.,~\cite{Boson, IQP, nature-google}) are sampling ones also suggests the relevance of delegation for quantum sampling problems.

It is worthwhile noting that there is a simple reduction of the delegation of \emph{classical} sampling problems to decision ones: for example, the verifier can fix a seed for pseudo-randomness, send it to the prover, and then the sampling outcome can be computed  bit-by-bit under this pseudo-random seed in a deterministic way. As such, classical literature of delegation of computation primarily focuses on delegation of decision problems. %delegation of classical sampling problem is not explicitly studied in literature. 

Unfortunately, it is unclear about how to make this de-randomization trick work for the delegation of $\SampBQP$, 
as randomness in quantum computation inherently comes from quantum mechanics rather than classical probability theory. 
Thus, it seems that the delegation of $\SampBQP$ needs to take a different technical route.
%delegation of $\SampBQP$ may require a totally different construction from delegation of $\BQP$. 
%
One natural starting point is Mahadev's CVQC protocol~\cite{FOCS:Mahadev18a} for $\BQP$. 
Indeed, some nice features of Mahadev's protocol (e.g., allowing X-Z measurements on any qubit) suggest the feasibility of generating  multi-bit measurement outcomes with the protocol. 
However, as highlighted below in the technical introduction section, there are also important drawbacks of Mahadev's protocol in its current form, which makes it hard to allow a CVQC protocol for $\SampBQP$ with desired performance. As we shall see, such difficulties do not hold for the decision problem, which demonstrates an importance difference between CVQC protocols for the decision and the sampling problems. 

% \XW{Adjust the flow; highlight the key questions}
% \Ethan{This paragraph is new and has to be checked}
% Even with the CVQC protocol for $\BQP$ from \cite{FOCS:Mahadev18a}, however, 
%it is still unclear how to construct a CVQC protocol for $\SampBQP$.
% In the context of delegating \emph{classical} sampling problems,
% there is a simple reduction to that of decision problems.
% Namely, the verifier fixes a seed for pseudorandomness and send it to the prover,
% then the sampling outcome can be computed deterministically bit-by-bit using this pseudorandomness.



Another desirable property of CVQC protocols is the \emph{blindness} where the prover cannot distinguish the particular computation in the protocol from another one of the same size, and hence is blind about the client's input.
Historically, blindness has been achieved in the weaker formulations of CVQC based on various techniques: e.g., the measurement-based quantum computation exploited in~\cite{BFK09}, the quantum authentication scheme exploited in~\cite{arXiv:ABOEM17}, and the self-testing technique exploited in~\cite{Nat:RUV13}.
Moreover, the blindness property is known to be helpful to establish the verifiability of CVQC protocols. However, this is never an easy task.
See for example the significant amount of efforts to add verifiability to blind CVQC protocols in~\cite{FK17}.
It is also known that another important primitive called the Quantum Fully Homomorphic Encryption (QFHE)~\cite{BJ15, DSS16, LC18, NS18, OTF18, mahadev_qfhe} should be helpful for establishing blindness: this is intuitive since QFHE allows fully homomorphic operations on encrypted quantum data. 
Indeed, in another paper~\cite{mahadev_qfhe}, Mahadev constructed the first leveled QFHE based on similar techniques and computational assumptions from~\cite{FOCS:Mahadev18a}.
The constructed QFHE automatically implies a blind CVQC protocol, however, without verifiability.
Extending this protocol with verifiability seems challenging as hinted by failed attempts in Section 2.2.2 of Mahadev's Phd thesis~\cite{mahadev_2018}.

In fact, most existing blind and verifiable protocols for delegation of quantum computation require a notable amount of effort in achieving each property respectively. 
The only successful CVQC protocol~\cite{FOCS:GheVid19} of achieving both so far applies Mahadev's technique~\cite{FOCS:Mahadev18a} to the measurement-based quantum computation, whereas the analysis is still very specific to the construction.
%\Ethan{And it has inverse-poly errors, which makes it not constant-round to achieve negligible errors for $\BQP$, right?}
Could there be a \emph{generic} way to achieve blindness and verifiability for CVQC protocols at the same time, say by leveraging the state-of-the-art CVQC~\cite{FOCS:Mahadev18a} and QFHE~\cite{mahadev_qfhe} protocols? 


%\XW{Mention QFHE as a natural attempt, and highlight the problem}


\vspace{2mm} \noindent \textbf{Contribution.} We provide \emph{affirmative} solutions to both of our questions.
In particular, we demonstrate the feasibility of the classical verification of quantum sampling by
constructing a constant-round CVQC protocol for $\SampBQP$, the sampling version of $\BQP$ formulated by Aaronson~\cite{aaronson_2013}, based on the quantum LWE (QLWE) assumption that the learning-with-error problem is hard for BQP machines. 
Formally, $\SampBQP$ consists of sampling problems $(D_x)_{x\in\zo^*}$ that can be approximately sampled by a $\BQP$ machine with an inverse polynomial accuracy. \KM{add ref to Sec 3}  %Namely, $A(x,1^{1/\eps})$ outputs a sample that is $\eps$-close to the distribution $D_x$ in statistical distance.
%, where given an input $x \in \zo^n$, the goal  
Precisely,
\begin{theorem}[informal]
Assuming the QLWE assumption, there exists a four-message CVQC protocol for all sampling problems in $\SampBQP$ with negligible completeness error and computational soundness.
\end{theorem}

Our second contribution is a simple yet powerful \emph{generic}  compiler that transforms any CVQC protocol to a blind one while preserving verifiability, building on top of QFHE. 
Precisely, we leverage QFHE (especially the one from~\cite{mahadev_qfhe}) to transform any CVQC protocol to \emph{a blind one with the same number of round communication, while preserving completeness and soundness properties}.
As a result, one can \emph{upgrade} every verifiable CVQC protocol with blindness almost for free with the help of QFHE.
Conceptually, we take a very different approach from previous ones  (e.g.,~\cite{FK17} as well as failed attempts in~\cite{mahadev_2018}) which use the blindness as the start point and then work to extend it with verifiability.
Instead, our strategy is to simulate a (verifiable) CVQC protocol under QFHE per each message.
To that end, we do require a special property of QFHE that the classical part of the ciphertext can be operated on separately from the quantum part, which is satisfied by the construction from~\cite{mahadev_qfhe}.
Our construction makes a modular use of QFHE and only requires a minor technicality in the analysis, which will be explained below. 
%As a result, we obtain
\begin{theorem}[informal]
Assuming the QLWE assumption, there exists a protocol compiler that transforms any CVQC protocol $\Pi$ to a CVQC protocol $\Piblind$ that achieves blindness while preserves its round complexity, completeness, and soundness.
\end{theorem}

%\XW{theorem statement for the second contribution and pointer}



As a simple corollary of combining both results above, we achieve a constant-round blind CVQC protocol for $\SampBQP$. %with negligible completeness error and statistical soundness.  
\begin{theorem}[informal]
        Assuming the QLWE assumption, there exists a blind, four-message CVQC protocol for all sampling problems in $\SampBQP$ with negligible completeness error and computational soundness.
\end{theorem}

We also construct the first blind and constant-round CVQC protocol for $\BQP$ by applying our compiler to the parallel repetition of Mahadev's protocol for $\BQP$ from \cite{arXiv:ChiaChungYam19, arXiv:AlaChiHun19}.


\begin{theorem}[informal]
    Assuming the QLWE assumption, there exists a blind, four-message CVQC protocol for all languages in $\BQP$ with negligible completeness and soundness errors.
\end{theorem}



%\XW{here for both $\BQP$ and $\SampBQP$}
%\XW{check the para/terminology here; consider adding a theorem statement,or a pointer to the later section}

To the authors' best knowledge, we are the first to study CVQC protocols for $\SampBQP$ and establish a generic compiler to upgrade CVQC protocols with blindness.
Our result also entails a \emph{constant-round} blind and verifiable CVQC protocol for $\BQP$.
The closest result to ours is by Gheorghiu and Vidick~\cite{FOCS:GheVid19} which shows such a CVQC protocol for $\BQP$, however, with a polynomial number of rounds.
Their protocol was obtained by first constructing a remote state preparation primitive and then combining it with an existing blind and verifiable protocol~\cite{FK17} where the verifier has some limited quantum power.
Our technical approach is quite different and seems incomparable.

% \XW{any more to say about parameters, techniques?}
% \XW{anything we want to say about composability?}
% \Ethan{Last time we checked, Vidick might have better composability since he's got some kind of ideal box and simulator-based proof}

% Related work
% \begin{itemize}
%     \item Comparison with related works here?
% \end{itemize}
%       \begin{itemize}
%           \item Sampling, the only paper; how about classical sampling?
%           \item the following for BQP  
%           \item blind and verifiable ~\cite{GV19}; we  constant round; technique-wise very different.
%           \item there is a table in~\cite{Grilo19}. Safe to say ~\cite{GV19} only existing blind protocol in the computational setting?
%           \item all previous either quantum clients, or at least 2 provers.
%           \item the following for blindness
%           \item Mahadev in her thesis~\cite{mahadev_2018} discussed a bit about the relation between verifiability and blindness. She hoped to get verifiability out of blindness by designing some non-malleable QFHE but failed.   
%           \item what's the high-level message we can say here?  Use QFHE in a different way? It is correct that not much work in the classical setting either.  Maybe existing classical work employs the principle but with different implementation.
%           \item old approach, first get blindness  (measurement-based, self-testing), and then try to make it verifiable; our approach, first have a verifiable protocol, and upgrade by a QFHE.
%           \item directly QFHE (blindness) won't give verifiability. some thoughts from Mahadev.
%
%           technical comparison with the past parallel % repetition.
%           \item

\vspace{2mm} \noindent \textbf{Techniques.} Let us revisit Mahadev's CVQC protocol~\cite{FOCS:Mahadev18a} first for some technical background. 
Following~\cite{FOCS:Mahadev18a}, we formally define $\QPIP_{\tau}$ as classes of CVQC protocols where $\tau$ refers to the size of quantum register in the possession of the classical verifier, or equivalently, the limited quantum computation power of the verifier.
It is known that $\BQP$ can be efficiently verified by a classical verifier that can perform a single qubit $X$ or $Z$ measurement~\cite{PhysRevA.93.022326, mf16}, by reducing any $\BQP$ problem to a local Hamiltonian problem where each term consists of  $X$ and $Z$ only. This leads to a $\QPIP_1$ protocol for $\BQP$.
The main contribution of Mahadev~\cite{FOCS:Mahadev18a} can hence be deemed as a way to compile this $\QPIP_1$ protocol into a $\QPIP_0$ protocol (i.e., with a fully classical verifier).

Precisely, to leverage Mahadev's construction, one needs to start with a $\QPIP_1$ protocol with very small completeness and soundness errors, which will then be compiled into a $\QPIP_0$ protocol with a small completeness error, but a close-to-$3/4$ soundness error under the QLWE assumption.  
This large soundness error is due to the current structure of Mahadev's protocol that consists of the \emph{testing} round and the \emph{Hadamard} round, each of which happens with a half chance. 
At a high level, the protocol verifies the behavior of the prover in the testing round, while assumes that the prover behaves honestly and all the X-Z measurements are correct in the Hadamard round. 
The soundness is obtained by observing that the dishonest prover cannot cheat in both rounds, while cheating in one round alone is possible which leads to a large soundness error. 
This less desirable soundness error, as well as other parameters, has been subsequently improved in \cite{arXiv:AlaChiHun19, arXiv:ChiaChungYam19} by a parallel repetition of Mahadev's original CVQC protocol in the computational setting. 

The first thing when dealing with $\SampBQP$ is to properly define the requirement of CVQC protocols for $\SampBQP$. 
In contrast to the single bit $\mathrm{Accept/Reject}$ information for the decision $\BQP$, any $\SampBQP$ problem needs to always output a sample of the desired distribution.
In the context of CVQC protocols, it means that the protocol should allow the verifier to generate a desired output sample whenever the protocol accepts. 
(Of course, when the prover cheats, the verifier will reject and no further output is required in that case.)
We provide a formal definition for CVQC protocols for $\SampBQP$ in Section~\ref{sec:samp_definition} to capture the above intuition. \KM{say the distribution should be correct conditioned on accepts}

\KM{terminology: testing protocol, test round etc.}

This fundamentally new requirement of any CVQC protocol for $\SampBQP$ exposes one drawback of Mahadev's protocol: 
while it is possible, as we will show, to generate desired output samples in the Hadamard round, 
it is however unclear about how to generate such a sample when the protocol accepts in the testing round, since the sampling information stored in the quantum state will be disturbed by the testing step. 

As a result, in order to construct a $\QPIP_0$ protocol for $\SampBQP$,  one needs to (1) construct a $\QPIP_1$ protocol for $\SampBQP$ with very small completeness and soundness errors, and (2) amend Mahadev's construction so that the protocol will generate the desired sample whenever it accepts. 
Our contribution is a solution to address the above two technical challenges. 

% \XW{revisit Mahadev's approach: (a) QPIP0 to QPIP1; QPIP0 requires very good completeness and soundness. (can say for examples the use of X-Z Hamiltonian, what's the detailed bound transformation?);  the output protocol has poor soundness errors;   (this is still for the decision problem; inefficiency); cite the recent parallel repetition - motivation and result; efficiency- round ? so sequential-repetition is not preferred.}

% \XW{the difference between decision and sampling problems. precision definitions are given in where?}

% \Ethan{Next part needs discussion on context Ie. protocol or local computation}
% Additional challenge comes from modeling and achieving soundness for such a sampling protocol,
% since it is unclear how to apply the usual trick of running many copies of the protocol then applying the Chernoff bound. \XW{too vague to be useful?}


\vspace{2mm} \noindent \textbf{Construction of a $\QPIP_0$ protocol for $\SampBQP$}. 
% \XW{stop here} We will follow the same road map above (i.e., from $\QPIP_1$ to $\QPIP_0$) for $\SampBQP$
% However, since there is no existing $\QPIP_1$ protocol for $\SampBQP$, we make original contributions to both steps as follows:
% \XW{use a different labeling in order to highlight contributions}
Following the aforementioned outline, our construction can be divided into three key technical steps. 

% \XW{(1) a basic step for encoding of the history state in the local Hamiltonian problem, and can be tested (so spectral gap). 
% (2) testing the spectral gap and make sure outputing. QPIP0 protocol. (need and how); here also  
% (3) QPIP1 compilation, and also main the efficiency (round?). But this parallel repetition just makes more complicated because of the sampling requirement! 
% }


\vspace{2mm} \noindent \emph{$\diamond$ Reducing  $\SampBQP$ to the local Hamiltonian problem}: We will continue to employ the local Hamiltonian technique~\cite{kitaev2002classical} and its ground state (known as the history state) as a key technical ingredient to certify the $\SampBQP$ circuits. 
Recall that the $\QPIP_1$ protocol for $\BQP$ in Mahadev's construction also comes from a reduction of $\BQP$ to a $X$-$Z$-only local Hamiltonian problem. 

However, there are important differences between the cases for $\BQP$ and $\SampBQP$. Recall that the original construction of local Hamiltonian $H$ for $\BQP$ (or $\QMA$) contains two parts $H=H_{\mathrm{circuit}}+ H_{\mathrm{out}}$.
Roughly speaking, $H_{\mathrm{circuit}}$ helps guarantee its ground space only contains \emph{valid} history states with correct input and circuit evolution, while $H_{\mathrm{out}}$'s energy encodes the 0/1 output for $\BQP$ circuits.
Thus, the outcome of a $\BQP$ instance can be encoded by the \emph{ground energy} of $H$.

In the case of $\SampBQP$, one needs to output both Accept/Reject information as well as a sample from the desired distribution if the protocol accepts. 
To that end, one hopes to certify the validity of the entire history state which is the ground state of $H_{\mathrm{circuit}}$, rather than the ground energy of $H_{\mathrm{circuit}}$ only. 
% one still uses $H_{\mathrm{circuit}}$ to certify the validity of the history state.
% However, in this case, one needs to measure on the entire final state of the circuit, rather than a single output qubit,
% which can no longer be encoded solely by the ground energy.
Namely, we want to rule out the existence of any state that is far from the history state but its energy (respect to $H_{\mathrm{circuit}}$) is very close to the ground one. 
One nature approach, also ours, is to make the unique valid history state lie in the ground space of a slightly different local Hamiltonian $H'_{\mathrm{circuit}}$ that has a large \emph{spectral} gap between its ground energy and excited ones.
It is hence guaranteed that any state with a close-to-ground energy must also be close to the history state.
In other words, a certification of the energy $H'_{\mathrm{circuit}}$ could lead to a certification of the history state, which in turn helps us generate the desired sample. 
We construct such $H'_{\mathrm{circuit}}$ from $H_{\mathrm{circuit}}$ by using the \emph{perturbation} technique (e.g.,~\cite{kempe_kitaev_regev_2006}) with further restriction to X/Z terms. (\Cref{sec:LHXZ}.)

%\XW{what's good about Mahadev's protocol; one can measure a lot; but of course, one needs to also test it; for the QPIP1 protocol; to get the QPIP0, the story won't be different.}
% \XW{anything useful here: Our protocol leverages the Hamiltonian model and the computational X-Z measurement from~\cite{FOCS:Mahadev18a}.
% However, a significant amount of new techniques have been developed to deal with the difference between $\SampBQP$ and $\BQP$, which will be highlighted in the technical contribution section. }


\vspace{2mm} \noindent \emph{$\diamond$ A $\QPIP_1$ protocol for $\SampBQP$:} A certification protocol for the history state, however, is insufficient to imply a $\QPIP_1$ protocol for $\SampBQP$ directly. 
Intuitively, we face a similar dilemma as we described above about Mahadev's protocol: we can either employ the certification protocol on $H'_{\mathrm{circuit}}$ to do the test, or to measure the history state to generate the desired outcome, but not both at the same time since the testing protocol would disturb the measurement outcome.
% Specifically, we will certify the energy of $H'_{\mathrm{circuit}}$ to guarantee the underlying state is close to the valid history state.
% However, this procedure could be vastly different from outputting a sample by measuring the final state of $\SampBQP$ circuits.
To resolve that, we design a \emph{cut-and-choose} protocol on multiple copies of the history state so as to separate the testing and the outputting parts on separate copies of history states. 

The prover in the real protocol, of course, won't necessarily send copies of history states. Thus, to leverage the aforementioned intuition, one needs to avoid entanglement among these copies to obtain some sort of independence. 
We employ quantum \emph{de Finetti} theorem to address this technical challenge.
Specifically, given any permutation-invariant $k$-register state (where each register could contain many qubits), it is known that the reduced state on many subsets of $k$-register will be close to a separable state. 
Typically, for favorable error bounds, the parameter $k$ could be as large as the dimension of a single register, which is exponential in our context and hence undesirable. 
Fortunately, since there is no entangled operation in the protocol to perform on two copies of history states, one can employ an efficient variant of quantum \emph{de Finetti} theorem~\cite{Brandao2017} whose error bound depends poly-logarithmically on the dimension of the register, which leads to an efficient $\QPIP_1$ protocol for $\SampBQP$. 
We can further achieve very small completeness and soundness errors in this way to satisfy the premise of Mahadev's compilation.  
Note that the established $\QPIP_1$ protocol is information-theoretically sound without any computational assumption.  (\Cref{sec:qpip1}.)

\vspace{2mm} \noindent  \emph{$\diamond$ Compile $\QPIP_1$ into $\QPIP_0$ after parallel repetition}: we are ready to address the aforementioned drawback of Mahadev's protocol. First, we will use Mahadev's compilation to obtain a $\QPIP_0$ protocol $\PiNaive$ for the above $\QPIP_1$ protocol.  As mentioned previously, this protocol $\PiNaive$ will have some trouble outputting the desired sample in the testing round. 
To address this issue, we will separate the testing and the outputting in different runs of the protocol, similarly to what we have done in the design of $\QPIP_1$ protocol. 
Precisely, we will run again a cut-and-choose protocol $\PiSampZ$ on top of many copies of $\PiNaive$, which are executed in parallel to preserve the number of rounds.

However, since $\PiNaive$ only has computational soundness, we need to design a parallel repetition of $\PiNaive$ in the computational setting, which also prevents the use of quantum de Finetti theorem that only holds in the information-theoretical setting. 
Fortunately,  some recent results on computational parallel repetitions of Mahadev's original protocol for $\BQP$~\cite{arXiv:AlaChiHun19, arXiv:ChiaChungYam19} provide a technical starting point for us. 
Precisely, these results open the box of the analysis of Mahadev's protocol and analyze the computational soundness under a parallel repetition. 

\XW{Precisely, comment on the partition lemma~\cite{arXiv:ChiaChungYam19}, and how it is used ...}
% We end up developing a weaker version of  parallel repetition of Mahadev's protocol inspired by the technique from~\cite{arXiv:ChiaChungYam19}. 
\XW{Note that there is also an important difference between us and~\cite{arXiv:ChiaChungYam19}, the difference between decision and sampling problems.}
\Cref{sec:qpip0_all}.

% A naive attempt is to directly apply Mahadev's protocol on the aforementioned $\QPIP_1$ protocol.
% Unfortunately, the plain version of Mahadev's protocol does not yield favorable parameters by itself.
% However, we cannot directly make use of these parallel repetition results due to the subtle difference between protocols for $\BQP$ and $\SampBQP$.
% One of the major difficulties here is still to deal with both the test part and the output part in $\SampBQP$ protocols.
% However, because we are now in the computational setting, there is no longer any available quantum de Finetti theorem that is usually derived in the information-theoretic setting.
% Due to the nature of parallel repetition in the computational setting, our analysis is much less modular and significantly involved for this part.  
% More intuitions and detailed analysis are given in 


It is worthwhile mentioning that we came to notice some online discussion\footnote{\url{https://www.scottaaronson.com/blog/?p=3697}, e.g., comment \#25, \#26, \#42, \#48. } on the possibility of a CVQC protocol for $\SampBQP$ after we developed our own result. 
These comments suggested a possible reduction of $\SampBQP$ to the local Hamiltonian problem following a similar high-level idea as our solution, however, with no detail and a seemingly different technical route. 
However, they failed to identify other important gaps in the protocol design that we have addressed. 

\vspace{2mm} \noindent \textbf{A generic compiler to upgrade $\QPIP_0$ protocols with blindness}. At a high-level, the idea is simple: we run the original protocol under a QFHE with the verifier's key. Intuitively, this allows the prover to compute his next message under encryption without learning the underlying verifier's message, and hence achieves blindness while preserving the properties of the original protocol.
One subtlety with this approach is due to the fact that the verifier is classical while the QFHE cipher text could depend on both quantum and classical data.
In order to make the classical verifier work in this construction, the ciphertext and the encryption/decryption algorithm needs to be classical when the underlying message is classical, which is fortunately satisfied by~\cite{mahadev_qfhe}.

A more subtle issue is to preserve the soundness.
In particular, compiled protocols with only one-time use of QFHE might (1) leak information about the circuit being evaluated during the homomorphic evaluation of QFHE ciphertexts (i.e., no \emph{circuit privacy});
or (2) fail to simulate original protocols upon receiving invalid ciphertexts.
We address these issues by letting the verifier switch to a fresh new key for each round of the protocol.
Details are given in \Cref{sec:BlindBQP2}.
%\XW{To KM: expand the above a bit more?}

\vspace{2mm} \noindent \textbf{Open Questions.} Our main focus is on the feasibility of the desired functionality and properties, which nevertheless leaves a big room for the improvement of efficiency.
Some of our parameter dependence inherits from previous works (e.g.~\cite{FOCS:Mahadev18a}), whereas some is due to our own construction. 
It will be extremely interesting to improve the parameter dependence with potentially new techniques. 

\Ethan{TODO add organization of paper}

% \begin{itemize}
%     \item Open questions, and also explain for some associated high-cost. specifically
%     \item T dependence. In general, improve the efficiency.  
%     \item negligible soundness error.  compare with classical? what's the state-of-the art.
% \end{itemize}

% \Ethan{This section is currently all rough draft. We'll probably rewrite almost all of it.}

% \Ethan{application: verifiable private constant round delegation}


\section{Preliminaries}

\Ethan{Check that the quantifiers are all there}

\subsection{Notations}

Let $\mathcal{B}$ be the Hilbert space corresponding to one qubit. Let $H:\mathcal{B}^{\otimes n}\rightarrow\mathcal{B}^{\otimes n}$ be Hermitian matrices. We use $H\geq0$ to denote $H$ being positive semidefinite. Let $\lambda(H)$ be the smallest eigenvalue of $H$. The ground states of $H$ are the eigenvectors corresponding to $\lambda(H)$. For matrix $H$ and subspace $S$, let $H\big|_S=\Pi_S H \Pi_S$, where $\Pi_S$ is the projector onto the subspace $S$. For a $T$-qubit Hilbert space, let the state $\ket{\widehat{t}}=\ket{1}^{\otimes t}\otimes \ket{0}^{{\otimes (T-t)}}$.
We write $F(\rho_1, \rho_2)=\left(\tr\sqrt{\sqrt{\rho_1}\rho_2\sqrt{\rho_1}}\right)^2$ for the fidelity between $\rho_1$ and $\rho_2$.
We write $\frac{1}{2}\norm{\rho_1-\rho_2}_1$ for the trace distance between $\rho_1$ and $\rho_2$. For all $n$-qubit states $\rho_1, \rho_2\in\cB^{\otimes n}$ we have $\frac{1}{2}\norm{\rho_1-\rho_2}_1\leq\sqrt{1-F(\rho_1, \rho_2)}$.

\begin{definition} [quantum-classical channels]
	\label{def:QCChannel}
	A quantum measurement is given by a set of matrices $\set{M_k}$ such that $M_k\geq0$ and $\sum_k M_k=\id$.
	We associate to any measurement a map $\Lambda(\rho)=\sum_k \tr(M_k\rho)\ket{k}\bra{k}$
	with $\set{\ket{k}}$ an orthonormal basis.
	This map is also called a \emph{quantum-classical channel}.
\end{definition}

The phase gate and Pauli matrices are denoted as follows.

\begin{definition}
	$P(i)=\begin{pmatrix}1&0\\0&i\end{pmatrix}$, $X=\begin{pmatrix}0&1\\1&0\end{pmatrix}$,
	$Y=\begin{pmatrix}0&-i\\i&0\end{pmatrix}$,
	$Z=\begin{pmatrix}1&0\\0&-1\end{pmatrix}$
\end{definition}

\subsection{Relevant complexity classes}

We define a few relevant complexity classes.

\begin{definition} [$\BQP$]
	Definition from Kitaev:
	A \emph{quantum algorithm} for the computation of a function $F:\zo^*\rightarrow\zo^*$ is a classical algorithm (i.e., a Turing machine) that computes a function of the form $x\mapstochar\rightarrow Z(x)$, where $Z(x)$ is a description of a quantum circuit which computes $F(x)$ on empty input. The function $F$ is said to belong to class $\BQP$ if there is a quantum algorithm that computes $F$ in time $\poly(n)$.

	Definition from Complexity Zoo:
	$\BQP$ is the class of languages $L$ for which for all $n\in\bbN$ there exists a quantum circuit constructible in time $\poly(n)$ that, given any $x\in\set{0, 1}^n$ as input, correctly decides whether $x\in L$ at least $\frac{2}{3}$ of the time.
	\Ethan{Just copy this from somewhere... Does anyone even define this?}
\end{definition}

\iffalse

\begin{definition} [$\FBQP$]
	A function $f:\set{0,1}^*\rightarrow\set{0,1}^*$ is in $\FBQP$ if there is a $\BQP$ machine that, $\forall x$, outputs $f(x)$ with overwhelming probability.
	\Ethan{Need to be more formal. Also should be efficient verifiable}
\end{definition}

\fi

We define search and sampling versions of $\BQP$ based on \cite{aaronson_2013}.

\iffalse

\begin{definition} [search problem]
	A search problem $R$ is a collection of nonempty sets $(A_x)_{x\in\set{0, 1}^*}$, one for each input string $x\in\set{0, 1}^*$, where $A_x$... \Ethan{Great, interface doesn't line up correctly}
\end{definition}

\fi

\begin{definition} [sampling problem]
	A sampling problem $S$ is a collection of probability distributions $(D_x)_{x\in\set{0, 1}^*}$, one for each input string $x\in\set{0,1}^n$, where $D_x$ is a distribution over $\set{0,1}^{p(n)}$ for some fixed polynomial $p$.
\end{definition}

\begin{definition} [$\SampBQP$]
	$\SampBQP$ is the class of sampling problems $S=\left(D_x\right)_{x\in\set{0, 1}^*}$ for which there exists a polynomial-time quantum algorithm $B$ that, given $(x, 0^{1/\varepsilon})$ as input, samples from a probability distribution $C_x$ such that $\norm{C_x-D_x}\leq\varepsilon$.
\end{definition}

\subsection{Quantum Prover Interactive Protocol (QPIP)}
We classify the interaction between a (almost classical) client and a quantum server for sampling problems, extending the classification by \cite{FOCS:Mahadev18a}.

\begin{definition}
	We say $\Pi=(P, V)$ is a protocol for the sampling problem $(D_x)_{x\in\zo^*}$ with completeness error $c$ and soundness error $s$ \Ethan{Might need these to be functions of $\abs{x}$} if
	\Ethan{Look up ``interactive protocols for BQP". Right now it's missing quantifier for all x. Also need to mention d is decision bit; maybe do that in next definition and swap locations}
	\begin{itemize}
		\item Let $(d, z)\leftarrow(P, V)(x)$. Then $d=rej$ with probability at most $c$.
		\item For all cheating prover $P^*$, let $(d, z)\leftarrow(P^*, V)(x)$. Let \Ethan{Use display math to make it obvious I'm defining this} $z_{ideal}\leftarrow D_x$ if $d=acc$, else $z_{ideal}=\bot$. Then $\norm{(d, z) - (d, z_{ideal})} \leq s$ \Ethan{make stat. distance notation consistent}.
	\end{itemize} \hannote{need computational?}
\end{definition}

\Ethan{Acc, rej, P, V fonts}

\Ethan{Might need to write our own def. here}

\Ethan{Look at thesis for this}

\begin{definition}
	A sampling problem $S=(D_x)_{x\in\set{0, 1}^*}$ is said to be \Ethan{Try to make this more general; remove mentions of sampling/decision problems} in $\QPIP_\tau$ with completeness $c$ and soundness $s$ \Ethan{Don't tie this with completeness and soundness yet} if there exists a protocol $(\bbP, \bbV)(x)$ for $S$ with the following properties:
	\begin{itemize}
		\item $\bbP$ is run by the prover, a $\BQP$ machine, which also has access to a quantum channel that can transmit $\tau$ qubits to the verifier per use.
		\item $\bbV$ is run by the verifier, which is a hybrid machine of a classical part and a limited quantum part. The classical part is a $\BPP$ machine. The quantum part is a register of $\tau$ qubits, on which the verifier can perform arbitrary quantum operations and which has access to a quantum channel which can transmit $\tau$ qubits. At any given time, the verifier is not allowed to possess more than $\tau$ qubits. The interaction between the quantum and classical parts of the verifier is the usual one: the classical part controls which operations are to be performed on the quantum register, and outcomes of measurements of the quantum register can be used as input to the classical part.
		\item There is also a classical communication channel between the prover and the verifier, which can transmit $\poly(\abs{x})$ many bits to either direction. 
	\end{itemize}
\end{definition}

\Ethan{Two separate definitions for comp and soundness}

\Ethan{Soundness?}

\Ethan{Just call this blindness and put this under interactive protocols}

\Ethan{See Thomas' paper if he defined this. Or actually, see https://www.iis.sinica.edu.tw/~kmchung/download/ConstCZK.pdf}

We present the security definition for interactive protocols:

\begin{definition}
    \Ethan{Define this in terms of views instead; see paper linked on Skype}

	Let $\lambda$ be a security parameter.
	Let $(\bbP, \bbV)$ be an interactive protocol with security parameter $\lambda$.
	Then it is IND-CPA secure if $\forall x\in\set{0,1}^n$ no polynomial time adversary $\cA$ can win \protoref{indcpa} with probability better than $\frac{1}{2}+\negl(\lambda)$
\end{definition}

\begin{protocol}{Attack against semantic security}
	\label{proto:indcpa}
	\begin{enumerate}
		\item The challenge picks $b\in\set{0,1}$ at random
		\item If $b=0$, the challenger runs the protocol with the adversary, acting as the verifier with input $0^n$
		\item Otherwise, the challenger runs the protocol with the adversary, acting as the verifier with input $x$
		\item $\cA$ attempts to guess $b$
	\end{enumerate}
\end{protocol}

\subsection{Chernoff bound}

Taken from \href{http://math.mit.edu/~goemans/18310S15/chernoff-notes.pdf}{here}.

\begin{thm}
\label{thm:Chernoff}
Let $X=\sum_{i=1}^n X_i$ where $X_i$ are i.i.d. \Ethan{Ind. is enough} Bernoulli trials, and $\mu=\E[X]$.
Then for all $0<\delta<1$,
$$P[\abs{X-\mu}\geq\delta\mu]\leq2e^{-\frac{\mu\delta^2}{3}}.$$
\end{thm}

\subsection{Projection Lemma}

We use the projection lemma from \cite{kempe_kitaev_regev_2006}, which describes the conditions under which we can estimate the ground state energy of $H_1 + H_2$ with that of $H_1\big|_{\ker H_2}$.

\begin{thm}
	Let $H=H_1+H_2$ be the sum of two Hamiltonians operating on some Hilbert space $\cH=\cS+\cS^\bot$.
	The Hamiltonian $H_2$ is such that $\cS$ is a zero eigenspace and the eigenvectors in $\cS^\bot$ have eigenvalues at least $J>2\norm{H_1}$. Then,
	$$\lambda\left(H_1\big|_\cS\right)-\frac{\norm{H_1}^2}{J-2\norm{H_1}^2}\leq\lambda(H)\leq\lambda\left(H_1\big|_\cS\right)$$
\end{thm}

We will instead use the following formulation, which can be obtained by relabeling variables from above.

\Ethan{State 2.3; prove from 2.2 in appendix}

\begin{thm}
	\label{thm:projection}
	Let $H_1, H_2$ be local Hamiltonians where $H_2\geq0$. Let $K=\ker H_2$ and
	$$J=\frac{10\norm{H_1}^2}{\lambda\left(H_2\big|_{K^\bot}\right)}$$
	then we have
	$$\lambda(H_1+JH_2)\geq\lambda\left(H_1\big|_K\right)-\frac{1}{8}$$
\end{thm}

\subsection{Quantum de Finetti Theorem under Local Measurements}

De Finetti theorem provides a way to obtain close to independent samples by taking random subsystems of a quantum system.
There are many formulations; we use the one from \cite{Brandão2017} because we need to avoid exponential dependence on number of qubits in each subsystem.
\begin{thm}
	\label{deFinetti}
	Let $\rho^{A_1\ldots A_k}$ be a permutation-invariant state on registers $A_1,\ldots,A_k$ where each register is $s$ qubits,
	then for every $0\leq l\leq k$ there exists states $\set{\rho_i}$ and $\set{p_i}\subset\bbR$ such that
	$$\max_{\Lambda_1,\ldots,\Lambda_l}
	\norm{(\Lambda_1\otimes\ldots\otimes\Lambda_l)\left(\rho^{A_1\ldots A_l}-\sum_ip_i\rho_i^{A_1}\otimes\ldots\otimes\rho_i^{A_l}\right)}_1
	\leq\sqrt{\frac{2l^2s}{k-l}}$$
	where $\Lambda_i$ are quantum-classical channels.
\end{thm}

\subsection{Quantum Homomorphic Encryption Schemes}

\Ethan{Everything gets security param.}

\def\QHE{\mathsf{QHE}}
\def\QGen{\mathsf{QHE.Keygen}}
\def\QEnc{\mathsf{QHE.Enc}}
\def\QEval{\mathsf{QHE.Eval}}
\def\QDec{\mathsf{QHE.Dec}}

We use the quantum fully homomorphic encryption scheme given in \cite{mahadev_qfhe} which is compatible with our use of a classical client. We start by presenting the interface of a homomorphic encryption scheme:
\begin{definition}
	A leveled homomophic encryption scheme is tuple of algorithms \linebreak $\mathsf{HE}=(\mathsf{HE.Keygen}, \mathsf{HE.Enc}, \mathsf{HE.Dec}, \mathsf{HE.Eval})$ with the following descriptions:
	\begin{itemize}
		\item $\mathsf{HE.Keygen}(1^\lambda, 1^L)\rightarrow(pk, sk)$
		\item $\mathsf{HE.Enc}_{pk}(\mu)\rightarrow c$
		\item $\mathsf{HE.Dec}_{sk}(c)\rightarrow \mu^*$
		\item $\mathsf{HE.Eval}_{pk}(f, c_1, \ldots, c_l)\rightarrow c_f$
	\end{itemize}
\end{definition}

$\mathsf{HE}$ also satisfies, with overwhelming probability in $\lambda$, that
$$\mathsf{HE.Dec}_{sk}(\mathsf{HE.Eval}_{pk}(f, c_1, \ldots, c_l))=f(\mathsf{HE.Dec}_{sk}(c_0),\ldots,\mathsf{HE.Dec}_{sk}(c_l))$$
where $f$ is specified by a circuit of depth at most $L$.

\Ethan{To be pedantic, the above doesn't imply Dec undoes Enc even if we sub in $f=\id$.}

We also recall the security definition for a FHE scheme.

\begin{definition}
	A FHE scheme $\mathsf{HE}$ is IND-CPA secure if, for any polynomial time adversary $\cA$, there exists a negligible function $\mu(\cdot)$ such that
	$$\abs{Pr[\cA(pk, \mathsf{HE.Enc}_{pk}(0))=1]-Pr[\cA(pk, \mathsf{HE.Enc}_{pk}(1))=1]}=\mu(\lambda)$$
	where $(pk, sk)\leftarrow\mathsf{QHE.Keygen}(1^\lambda)$
\end{definition}

The quantum homomorphic encryption scheme $\mathsf{QHE}$ from \cite{mahadev_qfhe} has additional properties that facilitate the use of classical clients:
\begin{itemize}
	\item $\QGen$ can be done classically.
	\item In the case where the plaintext is classical, $\QEnc$ can be done classically.
	\item Its ciphertext takes the form $(X^xZ^z\rho Z^zX^x, c_{x, z})$, where $\rho$ is the plaintext and $c_{x, z}$ is a ciphertext that decodes to $(x, z)$ under a certain classical homomorphic encryption scheme.
\end{itemize}

\Ethan{Reword the above? How? 3rd point can be weakened to can be decrypted classically? Maybe call this ``classical friendly". Mention LWE. There exists a classical friendly... etc.}

\section{Quantum Delegation Schemes}

Here we describe schemes that allows a classical client to interact with a quantum server, and various properties that such schemes can have.

\begin{definition}
	A \emph{quantum delegation protocol} is an polynomial-round interactive protocol between a BPP verifier and a BQP prover. The BPP verifier has an arbitrary quantum circuit with classical input that it needs to evaluate with the help of the prover.
\end{definition}

TODO (maybe don't assume universality?)

\begin{definition}
	A quantum delegation protocol has \emph{long output with error $\epsilon$} if the output can have multiple bits. In this case, the verifier should end with a measurement result of the output register. Said result's probability distribution would be within $\epsilon$ of the true distribution.
\end{definition}

\begin{definition}
	A quantum delegation protocol is \emph{secure} if a malicious prover cannot learn the input or the output of the computation. In particular, security notions such as \emph{CPA} can be applied here.
\end{definition}

\begin{definition}
	A quantum delegation protocol is \emph{verifiable} if the verifier can check if the output is correct. TODO need to be more rigorous here.
\end{definition}

In \cite{mahadev_delegation}, a verifiable scheme is proposed. In \cite{mahadev_qfhe}, a secure scheme is proposed. (TODO just make this CPA or whatever it actually is...) We propose a verifiable scheme with long output which can also be made secure.

TODO check where does \cite{1904.06320} fit in.

\subsection{Our contributions}

We start by constructing a verifiable long output scheme. Then, we encrypt the scheme as given in \cite{mahadev_delegation} to achieve security. TODO Does it actually work though? There are multiple steps in the scheme... We encrypting each step separately or what?


\section{Long Output scheme}

Here we describe the scheme between the classical client and an honest quantum server. We mostly follow the logic in \cite{mahadev_delegation}. We first follow \cite{kempe_kitaev_regev_2006} to reduce the problem to an instance of local Hamiltonian.

We pick our universal gate set to be Hadamard gate and controlled phase gate following \cite{quant-ph/0301040}:
$$H=\frac{1}{\sqrt{2}}\begin{pmatrix}1&1\\1&-1\end{pmatrix}$$
$$\Lambda(P(i))=\begin{pmatrix}1&0&0&0\\0&1&0&0\\0&0&1&0\\0&0&0&i\end{pmatrix}$$
This set is proven to be universal in \cite{kitaev_1997}.

\subsection{preprocess the circuit}

Here we first reduce the circuit to real gates and states only. The proof is taken from \cite{quant-ph/0301040}.

We consider the transform $T\ket{\psi}=\ket{0}\otimes\Re\ket{\psi}+\ket{1}\otimes\Im\ket{\psi}$, where $\Re$ and $\Im$ denote real and imaginary parts respectively. Note that after this transformation, the state is real. This transformation preserves measurement results in the standard basis. Under the transform, Hadamard gates are unchanged, but controlled phase gates become a combination of Hadamard and Toffoli gates. Mathematically,
$$T\circ H_f=H_f\circ T$$
$$T\circ\Lambda_f(P(i)_s)=\Lambda_{f,s}(X_0)\Lambda_{f,s}(Z_0)\circ T=\Lambda_{f,s}(X_0)H_0\Lambda_{f,s}(X_0)H_0\circ T$$
Where $\ket{\psi}$ is 1-indexed, and the added qubit is in the 0th position. 

We then double (TODO: maybe multiply by $\gamma>1$ instead of doubling?) the size of the circuit by padding identity matrices at the end. This is so that the final output has a higher chance to be measured in the final result in the following reduction.

After these transformations, let $x$ be the new circuit input, and $U_T...U_1$ be the circuit to evaluate.

\subsection{Constructing a local Hamiltonian instance}

We then attempt to construct a local Hamiltonian with ground state that encodes the history of the computation: $$\phi=\sum_{t=0}^TU_t...U_1\ket{x}\otimes\ket{\hat{t}}$$
We do so by ensuring states perpendicular to it have high eigenvalues. The base construction comes from \cite{kitaev2002classical}. The simplification is taken from \cite{biamonte_love_2008}.

First, we ensure that the invalid clock states have high eigenvalues by applying the following Hamiltonian to the time register.
$$H_{clock}=\sum_{t=1}^{T-1}\ket{01}\bra{01}_{t,t+1}$$
We also set $S_{legal}=\ker H_{clock}$. Note that $S_{legal}$ is precisely the valid clock states; that is, valid unary representations in the time register. Note that $H_{clock}$ can also be written in terms of linear combinations of tensors of $X$, $Z$, and $I$ gates.
$$H_{clock}=\frac{1}{4}((T-1)I + Z_1 - Z_T - \sum_t^{T-1}Z_tZ_{t+1})$$

Then, we ensure that the initial condition is set up correctly. Let $n$ be the number of the qubits in the circuit.
$$H_{in}=\sum_i^n(I-\ket{x_i}\bra{x_i})\otimes\ket{0}\bra{0}_1$$
Where $x_i$ is the $i$th bit of $x$. The kernel of this is precisely where everything is set up consistently with $\phi$ in time step $t=0$. Furthermore, this can be written in terms of linear combinations of tensors of $X$, $Z$, and $I$ gates as
$$H_{in}=\frac{1}{4}\sum_{i=1}^n(I-(-1)^{x_i}Z_i)\otimes(I+Z_1)$$

TODO define $S_{in}$ when projection lemma is used

Then, we ensure that the gates are applied correctly.
$$H_{prop1}=\sum_{t\in T_1}H_{prop,t}$$
$$H_{prop,t}=I\otimes\ket{\widehat{t}}\bra{\widehat{t}}
	+I\otimes\ket{\widehat{t-1}}\bra{\widehat{t-1}}
	-U_t\otimes\ket{\widehat{t}}\bra{\widehat{t-1}}
	-U_t^\dagger\otimes\ket{\widehat{t-1}}\bra{\widehat{t}}$$

I probably need to explain why this works but I'll leave it for another day.

\subsection{2-local ZX Hamiltonians}

We then follow \cite{biamonte_love_2008} in writing each terms of $H$ into only $Z$ and $X$ measurements...

This probably involves some kind of basis arguments on the Hamiltonians and maybe some stuff that's a bit more ugly. This should not change the state. I'll have to write this out later...

\subsection{Checking computation}

We then follow \cite{mahadev_delegation}. That is, the server would commit copies of the ground state of the Hamiltonian for the classical client to measure. The client would check whether the computation is done correctly using a modified Hamiltonian from \cite{kempe_kitaev_regev_2006}. 

And then, this in fact involves some probabilistic analysis shenanigans that we haven't fully figured out yet.

\subsection{Getting the Output}

Thanks to having padded the circuit with identity matrices at the end, if we measured $T>\frac{T}{2}$ on the time register, we would know that the other registers include the output of the required quantum computation. Now I would analyze the chances of getting this right, maybe in terms of fidelity or in terms of distance between this measured distribution and the ideal distribution...




\section{Conclusion}

This is a placeholder. Lorem ipsum...


\bibliographystyle{plain}
\bibliography{refs}

\end{document}
