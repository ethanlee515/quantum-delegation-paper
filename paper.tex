\documentclass[11pt]{article}
\usepackage{authblk}
%\pagestyle{plain}


\usepackage{amsmath, amssymb, amsthm}
\usepackage{braket}
\usepackage{algorithm}
\usepackage{algpseudocode}
\usepackage[margin=1in]{geometry}
\newcommand{\sgn}{\operatorname{sgn}}
\DeclareMathOperator*{\E}{\mathbb{E}}
\DeclareMathOperator*{\spn}{\operatorname{span}}
\DeclareMathOperator*{\poly}{\operatorname{poly}}
\newcommand{\norm}[1]{\left\lVert#1\right\rVert}
\usepackage{etoolbox}
\usepackage{hyperref}
\apptocmd{\thebibliography}{\raggedright}{}{}
\usepackage{cite,color,float}
\usepackage{mdframed} 
\usepackage{cleveref}
\crefname{protocol}{protocol}{protocols}
\Crefname{protocol}{Protocol}{Protocols}
\crefname{thm}{theorem}{theorems}
\Crefname{thm}{Theorem}{Theorems}
\crefname{rmk}{remark}{remarks}
\Crefname{rmk}{Remark}{Remarks}
\crefname{lem}{lemma}{lemmata}
\Crefname{lem}{Lemma}{Lemmata}


%*************************************page layout****************************
\setlength{\textheight}{9in}
\setlength{\columnsep}{2.0pc}
\setlength{\textwidth}{6.5in}
\setlength{\topmargin}{0in}
\setlength{\headheight}{0ex}
\setlength{\hoffset}{0in}
\setlength{\headsep}{0.0in}
\setlength{\oddsidemargin}{0ex}
\setlength{\evensidemargin}{0ex}
\setlength{\parindent}{1pc}

%\pagecolor{black}
%\color{white}

\newcounter{protocol}
\newcommand{\linefill}{\rule{\linewidth}{0.8pt}}

\newenvironment{protocol}[1]{\begingroup\setlength\parindent{0pt}\medskip\noindent\linefill\\
\refstepcounter{protocol}\textbf{Protocol \theprotocol} #1\\\noindent\linefill}
{\vspace{-\topsep}\noindent\linefill\endgroup}


\newcommand{\KM}[1]{{\footnotesize\color{cyan}[KM: #1]}}
\newcommand{\Ethan}[1]{{\footnotesize\color{magenta}[Ethan: #1]}}
\newcommand{\XW}[1]{{\footnotesize\color{red}[XW: #1]}}


\title{Constant-round Blind Classical Verification of Quantum Sampling}

%\iffalse
\author[1]{Kai-Min Chung\thanks{\href{mailto:kmchung@iis.sinica.edu.tw}{kmchung@iis.sinica.edu.tw}. Partially supported by the 2019 Academia Sinica Career Development Award under Grant no. 23-17, and MOST QC project under Grant no. MOST 108-2627-E-002-001.}}
\author[2]{Yi Lee\thanks{\href{mailto:ethanlee515@gmail.com}{ethanlee515@gmail.com}.}}
\author[3]{Han-Hsuan Lin\thanks{\href{mailto:linhh@cs.utexas.edu}{linhh@cs.utexas.edu}.}}
\author[4]{Xiaodi Wu\thanks{\href{mailto:xwu@cs.umd.edu}{xwu@cs.umd.edu}. Partially supported by the U.S. National Science Foundation grant CCF-1755800, CCF-1816695, and CCF-1942837(CAREER).}}
\affil[1, 2]{Institute of Information Science, Academia Sinica, Taiwan}
\affil[3]{Department of Computer Science, University of Texas at Austin, USA}
\affil[4]{
	Department of Computer Science, Institute for Advanced Computer Studies,
	and Joint Center for Quantum Information and Computer Science,
	University of Maryland, USA
}
%\fi
%\author{}
%\institute{}

\usepackage{graphicx,amsmath, amssymb,color,url,booktabs,comment}  %cite
\urlstyle{sf}
%\usepackage[margin=1in]{geometry}
% \usepackage{fancyhdr}
%\usepackage[colorlinks]{hyperref} %pagebackref

\newcommand{\nc}{\newcommand}
\nc{\rnc}{\renewcommand}
%
%\newcommand{\bra}[1]{\langle #1|}
%\newcommand{\ket}[1]{|#1\rangle}
\newcommand{\proj}[1]{|#1\rangle\langle #1|}
% \newcommand{\braket}[2]{\langle #1|#2\rangle}
% \newcommand{\Bra}[1]{\left\langle #1\right|}
% \newcommand{\Ket}[1]{\left|#1\right\rangle}
\newcommand{\Proj}[1]{\left|#1\right\rangle\left\langle #1\right|}
% \newcommand{\Braket}[2]{\left\langle #1\middle|#2\right\rangle}
\nc{\vev}[1]{\langle#1\rangle}
\nc{\grad}{{\vec{\nabla}}}
\nc{\abs}[1]{\lvert#1\rvert}
%\DeclareMathOperator{\abs}{abs}
\DeclareMathOperator{\Bin}{Bin}
\DeclareMathOperator{\conv}{conv}
\DeclareMathOperator{\eig}{eig}
\DeclareMathOperator{\Hist}{Hist}
\DeclareMathOperator{\Hyb}{Hyb}
\DeclareMathOperator{\id}{id}
\DeclareMathOperator{\Img}{Im}
\DeclareMathOperator{\Par}{Par}
% \DeclareMathOperator{\poly}{poly}
\DeclareMathOperator{\negl}{negl}
\DeclareMathOperator{\polylog}{polylog}
\DeclareMathOperator{\tr}{tr}
\DeclareMathOperator{\rank}{rank}
% \DeclareMathOperator{\sgn}{sgn}
\DeclareMathOperator{\Sep}{Sep}
\DeclareMathOperator{\SepSym}{SepSym}
\DeclareMathOperator{\Span}{span}
\DeclareMathOperator{\supp}{supp}
\DeclareMathOperator{\swap}{SWAP}
\DeclareMathOperator{\Sym}{Sym}
\DeclareMathOperator{\ProdSym}{ProdSym}
\DeclareMathOperator{\SEP}{SEP}
\DeclareMathOperator{\PPT}{PPT}
\DeclareMathOperator{\Wg}{Wg}
\DeclareMathOperator{\WMEM}{WMEM}
\DeclareMathOperator{\WOPT}{WOPT}

\DeclareMathOperator{\BPP}{\mathsf{BPP}}
\DeclareMathOperator{\QPIP}{\mathsf{QPIP}}
\DeclareMathOperator{\SampBQP}{\mathsf{SampBQP}}
\DeclareMathOperator{\BQP}{\mathsf{BQP}}
\DeclareMathOperator{\FBQP}{\mathsf{FBQP}}
\DeclareMathOperator{\cnot}{\normalfont\textsc{cnot}}
\DeclareMathOperator{\DTIME}{\mathsf{DTIME}}
\DeclareMathOperator{\NTIME}{\mathsf{NTIME}}
\DeclareMathOperator{\MA}{\mathsf{MA}}
\DeclareMathOperator{\NP}{\mathsf{NP}}
\DeclareMathOperator{\NEXP}{\mathsf{NEXP}}
\DeclareMathOperator{\Ptime}{\mathsf{P}}
\DeclareMathOperator{\QMA}{\mathsf{QMA}}
\DeclareMathOperator{\QCMA}{\mathsf{QCMA}}
\DeclareMathOperator{\BellQMA}{\mathsf{BellQMA}}

\newcommand{\be}{\begin{equation}}
\newcommand{\ee}{\end{equation}}
\newcommand{\bea}{\begin{eqnarray}}
\newcommand{\eea}{\end{eqnarray}}
\newcommand{\nn}{\nonumber}
\newcommand{\bi}{\begin{itemize}}
\newcommand{\ei}{\end{itemize}}
\newcommand{\bn}{\begin{enumerate}}
\newcommand{\en}{\end{enumerate}}
\def\beas#1\eeas{\begin{eqnarray*}#1\end{eqnarray*}}
\def\ba#1\ea{\begin{align}#1\end{align}}
\nc{\bas}{\[\begin{aligned}}
\nc{\eas}{\end{aligned}\]}
\nc{\bpm}{\begin{pmatrix}}
\nc{\epm}{\end{pmatrix}}
\def\non{\nonumber}
\def\nn{\nonumber}
\def\eq#1{(\ref{eq:#1})}
\def\eqs#1#2{(\ref{eq:#1}) and (\ref{eq:#2})}
%\def\eq#1{Eq.~(\ref{eq:#1})}
%\def\eqs#1#2{Eqs.~(\ref{eq:#1}) and (\ref{eq:#2})}
\def\L{\left} 
\def\R{\right}
\def\ra{\rightarrow}
\def\ot{\otimes}

\newtheorem{thm}{Theorem}[section]
\newtheorem{theorem}{Theorem}[section]
%\newtheorem*{thm*}{Theorem}
%\newtheorem{claim}[thm]{Claim}
\newtheorem{cor}{Corollary}[thm]
\newtheorem{lem}{Lemma}[section]
\newtheorem{lemma}{Lemma}[section]
\newtheorem{rmk}{Remark}[thm]
%\newtheorem{prop}[thm]{Proposition}
\newtheorem{dfn}{Definition}[section]
\newtheorem{definition}{Definition}[section]
%\newtheorem{con}[thm]{Conjecture}

\newenvironment{prf}{\begin{proof}}{\end{proof}}

\def\eps{\epsilon}
\def\va{{\vec{a}}}
\def\vb{{\vec{b}}}
\def\vn{{\vec{n}}}
\def\cvs{{\cdot\vec{\sigma}}}
\def\vx{{\vec{x}}}
\def\Usch{U_{\text{Sch}}}

\def\cA{\mathcal{A}}
\def\cB{\mathcal{B}}
\def\cD{\mathcal{D}}
\def\cE{\mathcal{E}}
\def\cF{\mathcal{F}}
\def\cH{\mathcal{H}}
\def\cI{{\cal I}}
\def\cL{{\cal L}}
\def\cM{{\cal M}}
\def\cN{\mathcal{N}}
\def\cO{{\cal O}}
\def\cP{\mathcal{P}}
\def\cQ{\mathcal{Q}}
\def\cS{\mathcal{S}}
\def\cT{{\cal T}}
\def\cU{\mathcal{U}}
\def\cV{\mathcal{V}}
\def\cW{{\cal W}}
\def\cX{{\cal X}}
\def\cY{{\cal Y}}

\def\bp{\mathbf{p}}
\def\bq{\mathbf{q}}
\def\bP{{\bf P}}
\def\bQ{{\bf Q}}
\def\gl{\mathfrak{gl}}

\def\bbC{\mathbb{C}}
% \DeclareMathOperator*{\E}{\mathbb{E}}
\DeclareMathOperator*{\bbE}{\mathbb{E}}
%\DeclareMathOperator*{\Pr}{Pr}
\nc{\Prob}[1]{\ensuremath{\Pr\left[#1\right]}}
\def\bbM{\mathbb{M}}
\def\bbN{\mathbb{N}}
\def\bbR{\mathbb{R}}
\def\bbZ{\mathbb{Z}}
\def\bbP{\mathbb{P}}
\def\bbV{\mathbb{V}}
\newcommand{\Real}{\textrm{Re}}

\def\benum{\begin{enumerate}}
\def\eenum{\end{enumerate}}
% \def\bit{\begin{itemize}}
% \def\eit{\end{itemize}}
\def\bdesc{\begin{description}}
\def\edesc{\end{description}}
\newcommand{\fig}[1]{Fig.~\ref{fig:#1}}
\newcommand{\tab}[1]{Table~\ref{tab:#1}}
\newcommand{\secref}[1]{Section~\ref{sec:#1}}
\newcommand{\appref}[1]{Appendix~\ref{sec:#1}}
\newcommand{\lemref}[1]{Lemma~\ref{lem:#1}}
\newcommand{\thmref}[1]{Theorem~\ref{thm:#1}}
\newcommand{\propref}[1]{Proposition~\ref{prop:#1}}
\newcommand{\protoref}[1]{Protocol~\ref{proto:#1}}
\nc{\myprotoref}[1]{\hyperref[#1]{Protocol~\ref*{#1}}}
\newcommand{\defref}[1]{Definition~\ref{def:#1}}
\newcommand{\corref}[1]{Corollary~\ref{cor:#1}}
\newcommand{\conref}[1]{Conjecture~\ref{con:#1}}

\newcommand{\FIXME}[1]{{\color{red}FIXME: #1}}
\nc{\todo}[1]{\textcolor{red}{todo: #1}}



\newcommand{\boxdfn}[2]{
\begin{figure}[h]
\begin{center}
\noindent \framebox{
\begin{minipage}{0.8\textwidth}
\begin{dfn}[{\bf #1}]
\ \\ \\
#2
\end{dfn}
\end{minipage}
}
\end{center}
\end{figure}
}

\newcommand{\boxproto}[2]{
\begin{figure}[h]
\begin{center}
\noindent \framebox{
\begin{minipage}{0.8\textwidth}
\begin{proto}[{\bf #1}]
\ \\ \\
#2
\end{proto}
\end{minipage}
}
\end{center}
\end{figure}
}

\def\begsub#1#2\endsub{\begin{subequations}\label{eq:#1}\begin{align}#2\end{align}\end{subequations}}
\nc\qand{\qquad\text{and}\qquad}
\nc\mnb[1]{\medskip\noindent{\bf #1}}
\nc\mn{\medskip\noindent}

\renewcommand{\arraystretch}{1.5}
%\nc{\problem}[1]{\item\noindent {\bf #1}}

\setlength{\tabcolsep}{10pt}

%%%%%% Han-Hsuan's commands %%%%%%%%
\nc{\nl}{\nn \\ &=}  %new line
\nc{\nnl}{\nn \\ &}  %new new line
\nc{\fot}{\frac{1}{2}} %frac one two
\nc{\oo}[1]{\frac{1}{#1}} % one over
\newcommand{\ben}{\begin{enumerate}}
\newcommand{\een}{\end{enumerate}}
\nc{\mc}{\mathcal}
\nc{\beq}{\begin{equation}}
\nc{\eeq}{\end{equation}}
% \nc{\norm}[1]{\L\| #1 \R\|}

\nc{\onenorm}[1]{\L\| #1 \R\|_1} %one norm
%\nc{\span}{\ensuremath{\mathrm{span}}}

\DeclareMathOperator*{\argmax}{arg\,max}

%\nc{1}

\newcommand{\hannote}[1]{\textcolor{blue}{\small {\textbf{(Han:} #1\textbf{) }}}}

\newcommand{\Knote}[1]{\textcolor{red}{\small {\textbf{(KM:} #1\textbf{) }}}}

\nc{\Ra}{\Rightarrow}
\nc{\zo}{\{0,1\}}	

%%%%import..
% \newcommand{\secpar}{n}


% %%%Efficient Verifier%
% \newcommand{\setupeff}{\setup_{\mathsf{eff}}}
% \newcommand{\vereff}{V_{\mathsf{eff}}}
% \newcommand{\vereffone}{V_{\mathsf{eff},1}}
% \newcommand{\vereffthree}{V_{\mathsf{eff},3}}
% \newcommand{\vereffout}{V_{\mathsf{eff},\mathsf{out}}}
% \newcommand{\proeff}{P_{\mathsf{eff}}}
% \newcommand{\proefftwo}{P_{\mathsf{eff},2}}
% \newcommand{\proefffour}{P_{\mathsf{eff},4}}
% \newcommand{\advPH}{{P^*}^{H}}
% \newcommand{\setup}{\mathsf{Setup}}
% \newcommand{\re}{\mathsf{re}}
% \newcommand{\crh}{\mathsf{CRH}}
% \newcommand{\transcript}{\mathsf{trans}}

% \newcommand{\setupefffs}{\setup_{\mathsf{eff}\text{-}\mathsf{fs}}}
% \newcommand{\proefffs}{P_{\mathsf{eff}\text{-}\mathsf{fs}}}
% \newcommand{\proefffstwo}{P_{\mathsf{eff}\text{-}\mathsf{fs},2}}
% \newcommand{\verefffs}{V_{\mathsf{eff}\text{-}\mathsf{fs}}}
% \newcommand{\verefffsone}{V_{\mathsf{eff}\text{-}\mathsf{fs},1}}
% \newcommand{\verefffsout}{V_{\mathsf{eff}\text{-}\mathsf{fs},\out}}
% %Games%
% \newcommand{\game}{\mathsf{Game}}

% %\newcommand*{\bra}[1]{\langle#1|}
% %\newcommand*{\ket}[1]{|#1\rangle}
% \newcommand*{\opro}[2]{|#1\rangle\langle#2|}
% \newcommand*{\ipro}[2]{\langle #1|#2\rangle}
% \newcommand{\TD}{\mathsf{TD}}

% %%%%% Registers %%%%%%%
% \newcommand*{\regK}{\mathbf{K}}
% \newcommand*{\regI}{\mathbf{I}}
% \newcommand*{\regR}{\mathbf{R}}
% \newcommand*{\regX}{\mathbf{X}}
% \newcommand*{\regY}{\mathbf{Y}}
% \newcommand*{\regZ}{\mathbf{Z}}
% \newcommand{\regW}{\mathbf{W}}
% \newcommand*{\regC}{\mathbf{C}}
% \newcommand*{\regO}{\mathbf{O}}
% \newcommand*{\regF}{\mathbf{F}}

%\newcommand{\redunderline}[1]{\textcolor{BrickRed}{\underline{\textcolor{black}{#1}}}}
\def \sample { \overset{\hspace{0.1em}\mathsf{\scriptscriptstyle\$}}{\leftarrow} }
\def\lapprox{\overset{<}{\sim}}
% \newcommand{\ra}{\rightarrow}
\newcommand{\la}{\leftarrow}
\newcommand{\pro}{P}
\newcommand{\ver}{V}
\newcommand{\secpar}{n}
% \newcommand{\negl}{\mathsf{negl}}
\newcommand{\lang}{L}
\newcommand{\key}{k}
\newcommand{\comy}{y}
\newcommand{\bfy}{\mathbf{y}}
\newcommand{\bfk}{\mathbf{k}}
\newcommand{\bfc}{\mathbf{c}}
\newcommand{\bfans}{\mathbf{a}}
\newcommand{\td}{\mathsf{td}}
\newcommand{\st}{\mathsf{st}}
\newcommand{\out}{\mathsf{out}}
 \newcommand{\bit}{\{0,1\}}
\newcommand{\ans}{a}
\newcommand{\Ans}{A}
%\newcommand{\poly}{\mathsf{poly}}
%\newcommand{\span}{\mathsf{span}}
%\newcommand{\poly}{\mathsf{poly}}
\newcommand{\hil}{\mathcal{H}}
\newcommand{\work}{W}
\newcommand{\defeq}{:=}
%\newcommand{\BPP}{\mathsf{BPP}}
\newcommand{\Succ}{\mathsf{Succ}}

\newcommand{\A}{\mathcal{A}}
\newcommand{\B}{\mathcal{B}}

\newcommand{\Sgood}{S_{\mathsf{good}}}
\newcommand{\Sbad}{S_{\mathsf{bad}}}
\newcommand{\psigood}{\psi_{\mathsf{good}}}
\newcommand{\psibad}{\psi_{\mathsf{bad}}}

\newcommand{\ext}{\mathsf{Ext}}

\newcommand{\calX}{\mathcal{X}}
\newcommand{\calY}{\mathcal{Y}}
\newcommand{\calS}{\mathcal{S}}
\newcommand{\calD}{\mathcal{D}}

%\newcommand{\ot}{\otimes}
\newcommand{\fail}{\mathsf{fail}}
\newcommand{\QTMtoQC}{\mathsf{QTMtoQC}}
\newcommand{\CRH}{\mathsf{CRH}}
\newcommand{\func}{\mathsf{Func}}
\newcommand{\TT}{\mathtt{T}}
\newcommand{\PRG}{\mathsf{PRG}}
\newcommand{\famCRH}{\mathcal{C}\mathcal{R}\mathcal{H}}
\newcommand{\QTM}{\mathsf{QTM}}
\newcommand{\QTIME}{\mathsf{QTIME}}

\newcommand{\Acc}{\mathsf{Acc}}
%%%Randomized Encoding%
\newcommand{\RE}{\mathsf{RE}}
\newcommand{\crs}{\mathsf{crs}}
\newcommand{\ek}{\mathsf{ek}}
\newcommand{\rsetup}{\mathsf{RE}.\mathsf{Setup}}
\newcommand{\renc}{\mathsf{RE}.\mathsf{Enc}}
\newcommand{\rdec}{\mathsf{RE}.\mathsf{Dec}}
\newcommand{\rsim}{\mathsf{RE}.\mathsf{Sim}}
\newcommand{\inp}{\mathsf{inp}}
\newcommand{\Time}{\mathsf{Time}}
\newcommand{\Menc}{\widehat{M_\inp}}

%%%SNARK%
\newcommand{\SNARK}{\mathsf{SNARK}}
\newcommand{\snark}{\mathsf{snark}}
% \newcommand{\NP}{\mathsf{NP}}
% \newcommand{\NTIME}{\mathsf{NTIME}}
\newcommand{\rela}{\mathcal{R}}

%FHE%%%
\newcommand{\FHE}{\mathsf{FHE}}
\newcommand{\fhe}{\mathsf{fhe}}
\newcommand{\fhekeygen}{\mathsf{FHE}.\mathsf{KeyGen}}
\newcommand{\fheenc}{\mathsf{FHE}.\mathsf{Enc}}
\newcommand{\fhedec}{\mathsf{FHE}.\mathsf{Dec}}
\newcommand{\fheeval}{\mathsf{FHE}.\mathsf{Eval}}
\newcommand{\sk}{\mathsf{sk}}
\newcommand{\pk}{\mathsf{pk}}
\newcommand{\ct}{\mathsf{ct}}

%%%Efficient Verifier%
\newcommand{\setupeff}{\setup_{\mathsf{eff}}}
\newcommand{\vereff}{V_{\mathsf{eff}}}
\newcommand{\vereffone}{V_{\mathsf{eff},1}}
\newcommand{\vereffthree}{V_{\mathsf{eff},3}}
\newcommand{\vereffout}{V_{\mathsf{eff},\mathsf{out}}}
\newcommand{\proeff}{P_{\mathsf{eff}}}
\newcommand{\proefftwo}{P_{\mathsf{eff},2}}
\newcommand{\proefffour}{P_{\mathsf{eff},4}}
\newcommand{\advPH}{{P^*}^{H}}
\newcommand{\setup}{\mathsf{Setup}}
\newcommand{\re}{\mathsf{re}}
\newcommand{\crh}{\mathsf{CRH}}
\newcommand{\transcript}{\mathsf{trans}}

\newcommand{\setupefffs}{\setup_{\mathsf{eff}\text{-}\mathsf{fs}}}
\newcommand{\proefffs}{P_{\mathsf{eff}\text{-}\mathsf{fs}}}
\newcommand{\proefffstwo}{P_{\mathsf{eff}\text{-}\mathsf{fs},2}}
\newcommand{\verefffs}{V_{\mathsf{eff}\text{-}\mathsf{fs}}}
\newcommand{\verefffsone}{V_{\mathsf{eff}\text{-}\mathsf{fs},1}}
\newcommand{\verefffsout}{V_{\mathsf{eff}\text{-}\mathsf{fs},\out}}
%Games%
\newcommand{\game}{\mathsf{Game}}

%\newcommand*{\bra}[1]{\langle#1|}
%\newcommand*{\ket}[1]{|#1\rangle}
\newcommand*{\opro}[2]{|#1\rangle\langle#2|}
\newcommand*{\ipro}[2]{\langle #1|#2\rangle}
\newcommand{\TD}{\mathsf{TD}}

%%%%% Registers %%%%%%%
\newcommand*{\regK}{\mathbf{K}}
\newcommand*{\regI}{\mathbf{I}}
\newcommand*{\regR}{\mathbf{R}}
\newcommand*{\regX}{\mathbf{X}}
\newcommand*{\regY}{\mathbf{Y}}
\newcommand*{\regZ}{\mathbf{Z}}
\newcommand{\regW}{\mathbf{W}}
\newcommand*{\regC}{\mathbf{C}}
\newcommand*{\regO}{\mathbf{O}}
\newcommand*{\regF}{\mathbf{F}}

%%%%% Note %%%%%%%%%%%%%

\newcommand{\nai}[1]{{\color{purple}[Nai: #1]}}
\newcommand{\km}[1]{{\color{brown}[KM: #1]}}
\newcommand{\takashi}[1]{{\color{red}[Takashi: #1]}}



\begin{document}

\maketitle
%\thispagestyle{plain}

\begin{abstract}

In a recent breakthrough, Mahadev constructed a classical verification of quantum computation (CVQC) protocol for a  classical client to delegate decision problems in $\BQP$ to an untrusted quantum prover under computational assumptions. In this work, we explore further the feasibility of CVQC with the more general \emph{sampling} problems in BQP and with the desirable \emph{blindness} property. We contribute affirmative solutions to both as follows. 
\begin{itemize}
\item Motivated by the sampling nature of many quantum applications (e.g., quantum algorithms for machine learning and quantum supremacy tasks), we initiate the study of  CVQC for \emph{quantum sampling problems} (denoted by $\SampBQP$).  Precisely, given an input $x\in \zo^n$ and a quantum circuit $C$, the goal of a classical client is to learn a sample of output $z \leftarrow C(x)$ over $\zo^m$ with the correct distribution up to a small error from an untrusted prover. We demonstrate its feasibility by constructing a four-message CVQC protocol for $\SampBQP$ based on the quantum \emph{Learning With Error} assumption.
\item
The \emph{blindness} of CVQC protocols refers to a property of the protocol where the prover learns nothing, and hence is blind, about the client's input. It is a highly desirable property that has been intensively studied for the delegation of quantum computation. 
Somewhat surprisingly, we provide a simple yet powerful \emph{generic} compiler that transforms any CVQC protocol to a blind one while preserving completeness and soundness errors.  
\end{itemize}
Applying our compiler to (a parallel repetition of) Mahadev's protocol for $\BQP$ and our CVQC protocol for $\SampBQP$ yields the first \emph{constant-round} blind CVQC protocol for $\BQP$ and $\SampBQP$ respectively, with negligible completeness and soundness errors.

\vspace{1mm}
\noindent \textbf{Keywords:} classical delegation of quantum computation, blind quantum computation, quantum sampling problems

\iffalse

\KM{version 2}

In a recent breakthrough, Mahadev constructed a classical verification of quantum computation (CVQC) protocol for a  classical client to delegate \emph{decision problems} in $\BQP$ to an untrusted quantum prover. In this work, motivated by the fact that many applications of quantum computations (e.g., in quantum machine learning) are sampling problems, we initiate the study of CVQC for \emph{quantum sampling problems} (denoted by $\SampBQP$), where given an input $x\in \zo^n$, the goal is to learn a sample an output $z \leftarrow C(x)$ over  $\zo^m$ for a quantum circuit $C$ from an untrusted prover with a correct distribution (up to a small error $\eps$). 

As our main result, we construct a four-message CVQC protocol for $\SampBQP$ with an arbitrarily small inverse polynomial error $\eps$ based on the quantum learning with error assumption.  


\fi


%Following the recent breakthrough of Mahadev, we explore the feasibility of classical verification of quantum computation (CVQC) for a broader class of quantum computation with blindness property. 

%This is a placeholder. Lorem ipsum. Lorem ipsum. Lorem ipsum...

\end{abstract}

\newpage

\section{Introduction}
% \XW{
% \begin{itemize}
%     \item add a lot of references.
%     \item comparison with the most relevant results:
%       \begin{itemize}
%           \item Sampling, the only paper; how about classical sampling?
%           \item the following for BQP  
%           \item blind and verifiable ~\cite{GV19}; we  constant round; technique-wise very different.
%           \item there is a table in~\cite{Grilo19}. Safe to say ~\cite{GV19} only existing blind protocol in the computational setting?
%           \item all previous either quantum clients, or at least 2 provers.
%           \item the following for blindness
%           \item Mahadev in her thesis~\cite{mahadev_2018} discussed a bit about the relation between verifiability and blindness. She hoped to get verifiability out of blindness by designing some non-malleable QFHE but failed.   
%           \item what's the high-level message we can say here?  Use QFHE in a different way? It is correct that not much work in the classical setting either.  Maybe existing classical work employs the principle but with different implementation.
%           \item old approach, first get blindness  (measurement-based, self-testing), and then try to make it verifiable; our approach, first have a verifiable protocol, and upgrade by a QFHE.
%           \item directly QFHE (blindness) won't give verifiability. some thoughts from Mahadev.
%           \item
%       \end{itemize}
% \end{itemize}
% }

Can quantum computation, with potential computational advantages that are intractable for classical computers,
be efficiently verified by classical clients?
This seeming paradox has been one of the central problems in quantum complexity theory and delegation of quantum computation~\cite{web:Aaronson}.
From a philosophical point of view, this question is also known to have a fascinating connection to the \emph{falsifiability} of quantum mechanics in the potential high complexity regime~\cite{survey:AV12}.

A complexity theoretic formulation of this problem by Gottesman in 2004~\cite{web:Aaronson} asks the possibility for an efficient classical verifier/client (a $\BPP$ machine) to verify the output of an
efficient quantum prover (a $\BQP$ machine).
In the absence of techniques for directly tackling this question, earlier feasibility results on this problem have been focusing on two weaker formulations.
The first type of feasibility results (e.g.,~\cite{BFK09,arXiv:ABOEM17,FK17,mf16}) considers the case where the $\BPP$ verifier is equipped with limited quantum power.
The second type of feasibility results (e.g,~\cite{Nat:RUV13, CGJV19, Gheorghiu_2015, HPF15})
considers a $\BPP$ verifier interacting with at least two entangled, non-communicating quantum provers.
In a recent breakthrough, Mahadev~\cite{FOCS:Mahadev18a} proposed the first protocol of classical verification of quantum computation (CVQC) whose soundness is based on a widely recognized computational assumption that the learning with error (LWE) problem~\cite{JACM:Regev09} is hard for $\BQP$ machines.
The technique invented therein has inspired many  subsequent developments of CVQC protocols with improved parameters and functionality (e.g.,~\cite{FOCS:GheVid19,arXiv:AlaChiHun19,arXiv:ChiaChungYam19}).
On the other side, there are known complexity-theoretic limitations on the feasibility of blind CVQC in the information-theoretical setting (e.g.~\cite{aaronson_et_al:LIPIcs:2019:10582}).
%\XW{here some no-go for information theoretical result.}
%We refer curious readers to a slightly outdated  survey~\cite{survey:GKK19} for details.

With the newly developed techniques, we revisit the classical verification of quantum computation problems from both a philosophical and a practical point of view.
We first observe that the \emph{sampling} version of $\BQP$ (e.g., the class $\SampBQP$ formulated by Aaronson~\cite{aaronson_2013}) might be a more appropriate notion to serve the purpose of the original problem.
Philosophically, the outcomes of quantum mechanical experiments are usually samples or statistical information, which is well demonstrated in the famous double-slit experiment.
Moreover, a lot of quantum algorithms from Shor's~\cite{Shor} and Grover's~\cite{Grover} algorithms to some recent developments in machine learning and optimization (e.g.~\cite{brando_et_al:LIPIcs:2019:10603, AGGW17,pmlr-v97-li19b}) contain a significant quantum sampling component.
The fact that almost all quantum supremacy tasks (e.g.,~\cite{Boson, IQP, nature-google}) are sampling ones also suggests the relevance of delegation for quantum sampling problems.

It is worthwhile noting that there is a simple reduction of the delegation of \emph{classical} sampling problems to decision ones: for example, the verifier can fix a seed for pseudo-randomness, send it to the prover, and then the sampling outcome can be computed  bit-by-bit under this pseudo-random seed in a deterministic way. As such, classical literature of delegation of computation primarily focuses on delegation of decision problems. %delegation of classical sampling problem is not explicitly studied in literature. 

Unfortunately, it is unclear about how to make this de-randomization trick work for the delegation of $\SampBQP$, 
as randomness in quantum computation inherently comes from quantum mechanics rather than classical probability theory. 
Thus, it seems that the delegation of $\SampBQP$ needs to take a different technical route.
%delegation of $\SampBQP$ may require a totally different construction from delegation of $\BQP$. 
%
One natural starting point is Mahadev's CVQC protocol~\cite{FOCS:Mahadev18a} for $\BQP$. 
Indeed, some nice features of Mahadev's protocol (e.g., allowing X-Z measurements on any qubit) suggest the feasibility of generating  multi-bit measurement outcomes with the protocol. 
However, as highlighted below in the technical introduction section, there are also important drawbacks of Mahadev's protocol in its current form, which makes it hard to allow a CVQC protocol for $\SampBQP$ with desired performance. As we shall see, such difficulties do not hold for the decision problem, which demonstrates an importance difference between CVQC protocols for the decision and the sampling problems. 

% \XW{Adjust the flow; highlight the key questions}
% \Ethan{This paragraph is new and has to be checked}
% Even with the CVQC protocol for $\BQP$ from \cite{FOCS:Mahadev18a}, however, 
%it is still unclear how to construct a CVQC protocol for $\SampBQP$.
% In the context of delegating \emph{classical} sampling problems,
% there is a simple reduction to that of decision problems.
% Namely, the verifier fixes a seed for pseudorandomness and send it to the prover,
% then the sampling outcome can be computed deterministically bit-by-bit using this pseudorandomness.



Another desirable property of CVQC protocols is the \emph{blindness} where the prover cannot distinguish the particular computation in the protocol from another one of the same size, and hence is blind about the client's input.
Historically, blindness has been achieved in the weaker formulations of CVQC based on various techniques: e.g., the measurement-based quantum computation exploited in~\cite{BFK09}, the quantum authentication scheme exploited in~\cite{arXiv:ABOEM17}, and the self-testing technique exploited in~\cite{Nat:RUV13}.
Moreover, the blindness property is known to be helpful to establish the verifiability of CVQC protocols. However, this is never an easy task.
See for example the significant amount of efforts to add verifiability to blind CVQC protocols in~\cite{FK17}.
It is also known that another important primitive called the Quantum Fully Homomorphic Encryption (QFHE)~\cite{BJ15, DSS16, LC18, NS18, OTF18, mahadev_qfhe} should be helpful for establishing blindness: this is intuitive since QFHE allows fully homomorphic operations on encrypted quantum data. 
Indeed, in another paper~\cite{mahadev_qfhe}, Mahadev constructed the first leveled QFHE based on similar techniques and computational assumptions from~\cite{FOCS:Mahadev18a}.
The constructed QFHE automatically implies a blind CVQC protocol, however, without verifiability.
Extending this protocol with verifiability seems challenging as hinted by failed attempts in Section 2.2.2 of Mahadev's Phd thesis~\cite{mahadev_2018}.

In fact, most existing blind and verifiable protocols for delegation of quantum computation require a notable amount of effort in achieving each property respectively. 
The only successful CVQC protocol~\cite{FOCS:GheVid19} of achieving both so far applies Mahadev's technique~\cite{FOCS:Mahadev18a} to the measurement-based quantum computation, whereas the analysis is still very specific to the construction.
%\Ethan{And it has inverse-poly errors, which makes it not constant-round to achieve negligible errors for $\BQP$, right?}
Could there be a \emph{generic} way to achieve blindness and verifiability for CVQC protocols at the same time, say by leveraging the state-of-the-art CVQC~\cite{FOCS:Mahadev18a} and QFHE~\cite{mahadev_qfhe} protocols? 


%\XW{Mention QFHE as a natural attempt, and highlight the problem}


\vspace{2mm} \noindent \textbf{Contribution.} We provide \emph{affirmative} solutions to both of our questions.
In particular, we demonstrate the feasibility of the classical verification of quantum sampling by
constructing a constant-round CVQC protocol for $\SampBQP$, the sampling version of $\BQP$ formulated by Aaronson~\cite{aaronson_2013}, based on the quantum LWE (QLWE) assumption that the learning-with-error problem is hard for BQP machines. 
Formally, $\SampBQP$ consists of sampling problems $(D_x)_{x\in\zo^*}$ that can be approximately sampled by a $\BQP$ machine with an inverse polynomial accuracy. \KM{add ref to Sec 3}  %Namely, $A(x,1^{1/\eps})$ outputs a sample that is $\eps$-close to the distribution $D_x$ in statistical distance.
%, where given an input $x \in \zo^n$, the goal  
Precisely,
\begin{theorem}[informal]
Assuming the QLWE assumption, there exists a four-message CVQC protocol for all sampling problems in $\SampBQP$ with negligible completeness error and computational soundness.
\end{theorem}

Our second contribution is a simple yet powerful \emph{generic}  compiler that transforms any CVQC protocol to a blind one while preserving verifiability, building on top of QFHE. 
Precisely, we leverage QFHE (especially the one from~\cite{mahadev_qfhe}) to transform any CVQC protocol to \emph{a blind one with the same number of round communication, while preserving completeness and soundness properties}.
As a result, one can \emph{upgrade} every verifiable CVQC protocol with blindness almost for free with the help of QFHE.
Conceptually, we take a very different approach from previous ones  (e.g.,~\cite{FK17} as well as failed attempts in~\cite{mahadev_2018}) which use the blindness as the start point and then work to extend it with verifiability.
Instead, our strategy is to simulate a (verifiable) CVQC protocol under QFHE per each message.
To that end, we do require a special property of QFHE that the classical part of the ciphertext can be operated on separately from the quantum part, which is satisfied by the construction from~\cite{mahadev_qfhe}.
Our construction makes a modular use of QFHE and only requires a minor technicality in the analysis, which will be explained below. 
%As a result, we obtain
\begin{theorem}[informal]
Assuming the QLWE assumption, there exists a protocol compiler that transforms any CVQC protocol $\Pi$ to a CVQC protocol $\Piblind$ that achieves blindness while preserves its round complexity, completeness, and soundness.
\end{theorem}

%\XW{theorem statement for the second contribution and pointer}



As a simple corollary of combining both results above, we achieve a constant-round blind CVQC protocol for $\SampBQP$. %with negligible completeness error and statistical soundness.  
\begin{theorem}[informal]
        Assuming the QLWE assumption, there exists a blind, four-message CVQC protocol for all sampling problems in $\SampBQP$ with negligible completeness error and computational soundness.
\end{theorem}

We also construct the first blind and constant-round CVQC protocol for $\BQP$ by applying our compiler to the parallel repetition of Mahadev's protocol for $\BQP$ from \cite{arXiv:ChiaChungYam19, arXiv:AlaChiHun19}.


\begin{theorem}[informal]
    Assuming the QLWE assumption, there exists a blind, four-message CVQC protocol for all languages in $\BQP$ with negligible completeness and soundness errors.
\end{theorem}



%\XW{here for both $\BQP$ and $\SampBQP$}
%\XW{check the para/terminology here; consider adding a theorem statement,or a pointer to the later section}

To the authors' best knowledge, we are the first to study CVQC protocols for $\SampBQP$ and establish a generic compiler to upgrade CVQC protocols with blindness.
Our result also entails a \emph{constant-round} blind and verifiable CVQC protocol for $\BQP$.
The closest result to ours is by Gheorghiu and Vidick~\cite{FOCS:GheVid19} which shows such a CVQC protocol for $\BQP$, however, with a polynomial number of rounds.
Their protocol was obtained by first constructing a remote state preparation primitive and then combining it with an existing blind and verifiable protocol~\cite{FK17} where the verifier has some limited quantum power.
Our technical approach is quite different and seems incomparable.

% \XW{any more to say about parameters, techniques?}
% \XW{anything we want to say about composability?}
% \Ethan{Last time we checked, Vidick might have better composability since he's got some kind of ideal box and simulator-based proof}

% Related work
% \begin{itemize}
%     \item Comparison with related works here?
% \end{itemize}
%       \begin{itemize}
%           \item Sampling, the only paper; how about classical sampling?
%           \item the following for BQP  
%           \item blind and verifiable ~\cite{GV19}; we  constant round; technique-wise very different.
%           \item there is a table in~\cite{Grilo19}. Safe to say ~\cite{GV19} only existing blind protocol in the computational setting?
%           \item all previous either quantum clients, or at least 2 provers.
%           \item the following for blindness
%           \item Mahadev in her thesis~\cite{mahadev_2018} discussed a bit about the relation between verifiability and blindness. She hoped to get verifiability out of blindness by designing some non-malleable QFHE but failed.   
%           \item what's the high-level message we can say here?  Use QFHE in a different way? It is correct that not much work in the classical setting either.  Maybe existing classical work employs the principle but with different implementation.
%           \item old approach, first get blindness  (measurement-based, self-testing), and then try to make it verifiable; our approach, first have a verifiable protocol, and upgrade by a QFHE.
%           \item directly QFHE (blindness) won't give verifiability. some thoughts from Mahadev.
%
%           technical comparison with the past parallel % repetition.
%           \item

\vspace{2mm} \noindent \textbf{Techniques.} Let us revisit Mahadev's CVQC protocol~\cite{FOCS:Mahadev18a} first for some technical background. 
Following~\cite{FOCS:Mahadev18a}, we formally define $\QPIP_{\tau}$ as classes of CVQC protocols where $\tau$ refers to the size of quantum register in the possession of the classical verifier, or equivalently, the limited quantum computation power of the verifier.
It is known that $\BQP$ can be efficiently verified by a classical verifier that can perform a single qubit $X$ or $Z$ measurement~\cite{PhysRevA.93.022326, mf16}, by reducing any $\BQP$ problem to a local Hamiltonian problem where each term consists of  $X$ and $Z$ only. This leads to a $\QPIP_1$ protocol for $\BQP$.
The main contribution of Mahadev~\cite{FOCS:Mahadev18a} can hence be deemed as a way to compile this $\QPIP_1$ protocol into a $\QPIP_0$ protocol (i.e., with a fully classical verifier).

Precisely, to leverage Mahadev's construction, one needs to start with a $\QPIP_1$ protocol with very small completeness and soundness errors, which will then be compiled into a $\QPIP_0$ protocol with a small completeness error, but a close-to-$3/4$ soundness error under the QLWE assumption.  
This large soundness error is due to the current structure of Mahadev's protocol that consists of the \emph{testing} round and the \emph{Hadamard} round, each of which happens with a half chance. 
At a high level, the protocol verifies the behavior of the prover in the testing round, while assumes that the prover behaves honestly and all the X-Z measurements are correct in the Hadamard round. 
The soundness is obtained by observing that the dishonest prover cannot cheat in both rounds, while cheating in one round alone is possible which leads to a large soundness error. 
This less desirable soundness error, as well as other parameters, has been subsequently improved in \cite{arXiv:AlaChiHun19, arXiv:ChiaChungYam19} by a parallel repetition of Mahadev's original CVQC protocol in the computational setting. 

The first thing when dealing with $\SampBQP$ is to properly define the requirement of CVQC protocols for $\SampBQP$. 
In contrast to the single bit $\mathrm{Accept/Reject}$ information for the decision $\BQP$, any $\SampBQP$ problem needs to always output a sample of the desired distribution.
In the context of CVQC protocols, it means that the protocol should allow the verifier to generate a desired output sample whenever the protocol accepts. 
(Of course, when the prover cheats, the verifier will reject and no further output is required in that case.)
We provide a formal definition for CVQC protocols for $\SampBQP$ in Section~\ref{sec:samp_definition} to capture the above intuition. \KM{say the distribution should be correct conditioned on accepts}

\KM{terminology: testing protocol, test round etc.}

This fundamentally new requirement of any CVQC protocol for $\SampBQP$ exposes one drawback of Mahadev's protocol: 
while it is possible, as we will show, to generate desired output samples in the Hadamard round, 
it is however unclear about how to generate such a sample when the protocol accepts in the testing round, since the sampling information stored in the quantum state will be disturbed by the testing step. 

As a result, in order to construct a $\QPIP_0$ protocol for $\SampBQP$,  one needs to (1) construct a $\QPIP_1$ protocol for $\SampBQP$ with very small completeness and soundness errors, and (2) amend Mahadev's construction so that the protocol will generate the desired sample whenever it accepts. 
Our contribution is a solution to address the above two technical challenges. 

% \XW{revisit Mahadev's approach: (a) QPIP0 to QPIP1; QPIP0 requires very good completeness and soundness. (can say for examples the use of X-Z Hamiltonian, what's the detailed bound transformation?);  the output protocol has poor soundness errors;   (this is still for the decision problem; inefficiency); cite the recent parallel repetition - motivation and result; efficiency- round ? so sequential-repetition is not preferred.}

% \XW{the difference between decision and sampling problems. precision definitions are given in where?}

% \Ethan{Next part needs discussion on context Ie. protocol or local computation}
% Additional challenge comes from modeling and achieving soundness for such a sampling protocol,
% since it is unclear how to apply the usual trick of running many copies of the protocol then applying the Chernoff bound. \XW{too vague to be useful?}


\vspace{2mm} \noindent \textbf{Construction of a $\QPIP_0$ protocol for $\SampBQP$}. 
% \XW{stop here} We will follow the same road map above (i.e., from $\QPIP_1$ to $\QPIP_0$) for $\SampBQP$
% However, since there is no existing $\QPIP_1$ protocol for $\SampBQP$, we make original contributions to both steps as follows:
% \XW{use a different labeling in order to highlight contributions}
Following the aforementioned outline, our construction can be divided into three key technical steps. 

% \XW{(1) a basic step for encoding of the history state in the local Hamiltonian problem, and can be tested (so spectral gap). 
% (2) testing the spectral gap and make sure outputing. QPIP0 protocol. (need and how); here also  
% (3) QPIP1 compilation, and also main the efficiency (round?). But this parallel repetition just makes more complicated because of the sampling requirement! 
% }


\vspace{2mm} \noindent \emph{$\diamond$ Reducing  $\SampBQP$ to the local Hamiltonian problem}: We will continue to employ the local Hamiltonian technique~\cite{kitaev2002classical} and its ground state (known as the history state) as a key technical ingredient to certify the $\SampBQP$ circuits. 
Recall that the $\QPIP_1$ protocol for $\BQP$ in Mahadev's construction also comes from a reduction of $\BQP$ to a $X$-$Z$-only local Hamiltonian problem. 

However, there are important differences between the cases for $\BQP$ and $\SampBQP$. Recall that the original construction of local Hamiltonian $H$ for $\BQP$ (or $\QMA$) contains two parts $H=H_{\mathrm{circuit}}+ H_{\mathrm{out}}$.
Roughly speaking, $H_{\mathrm{circuit}}$ helps guarantee its ground space only contains \emph{valid} history states with correct input and circuit evolution, while $H_{\mathrm{out}}$'s energy encodes the 0/1 output for $\BQP$ circuits.
Thus, the outcome of a $\BQP$ instance can be encoded by the \emph{ground energy} of $H$.

In the case of $\SampBQP$, one needs to output both Accept/Reject information as well as a sample from the desired distribution if the protocol accepts. 
To that end, one hopes to certify the validity of the entire history state which is the ground state of $H_{\mathrm{circuit}}$, rather than the ground energy of $H_{\mathrm{circuit}}$ only. 
% one still uses $H_{\mathrm{circuit}}$ to certify the validity of the history state.
% However, in this case, one needs to measure on the entire final state of the circuit, rather than a single output qubit,
% which can no longer be encoded solely by the ground energy.
Namely, we want to rule out the existence of any state that is far from the history state but its energy (respect to $H_{\mathrm{circuit}}$) is very close to the ground one. 
One nature approach, also ours, is to make the unique valid history state lie in the ground space of a slightly different local Hamiltonian $H'_{\mathrm{circuit}}$ that has a large \emph{spectral} gap between its ground energy and excited ones.
It is hence guaranteed that any state with a close-to-ground energy must also be close to the history state.
In other words, a certification of the energy $H'_{\mathrm{circuit}}$ could lead to a certification of the history state, which in turn helps us generate the desired sample. 
We construct such $H'_{\mathrm{circuit}}$ from $H_{\mathrm{circuit}}$ by using the \emph{perturbation} technique (e.g.,~\cite{kempe_kitaev_regev_2006}) with further restriction to X/Z terms. (\Cref{sec:LHXZ}.)

%\XW{what's good about Mahadev's protocol; one can measure a lot; but of course, one needs to also test it; for the QPIP1 protocol; to get the QPIP0, the story won't be different.}
% \XW{anything useful here: Our protocol leverages the Hamiltonian model and the computational X-Z measurement from~\cite{FOCS:Mahadev18a}.
% However, a significant amount of new techniques have been developed to deal with the difference between $\SampBQP$ and $\BQP$, which will be highlighted in the technical contribution section. }


\vspace{2mm} \noindent \emph{$\diamond$ A $\QPIP_1$ protocol for $\SampBQP$:} A certification protocol for the history state, however, is insufficient to imply a $\QPIP_1$ protocol for $\SampBQP$ directly. 
Intuitively, we face a similar dilemma as we described above about Mahadev's protocol: we can either employ the certification protocol on $H'_{\mathrm{circuit}}$ to do the test, or to measure the history state to generate the desired outcome, but not both at the same time since the testing protocol would disturb the measurement outcome.
% Specifically, we will certify the energy of $H'_{\mathrm{circuit}}$ to guarantee the underlying state is close to the valid history state.
% However, this procedure could be vastly different from outputting a sample by measuring the final state of $\SampBQP$ circuits.
To resolve that, we design a \emph{cut-and-choose} protocol on multiple copies of the history state so as to separate the testing and the outputting parts on separate copies of history states. 

The prover in the real protocol, of course, won't necessarily send copies of history states. Thus, to leverage the aforementioned intuition, one needs to avoid entanglement among these copies to obtain some sort of independence. 
We employ quantum \emph{de Finetti} theorem to address this technical challenge.
Specifically, given any permutation-invariant $k$-register state (where each register could contain many qubits), it is known that the reduced state on many subsets of $k$-register will be close to a separable state. 
Typically, for favorable error bounds, the parameter $k$ could be as large as the dimension of a single register, which is exponential in our context and hence undesirable. 
Fortunately, since there is no entangled operation in the protocol to perform on two copies of history states, one can employ an efficient variant of quantum \emph{de Finetti} theorem~\cite{Brandao2017} whose error bound depends poly-logarithmically on the dimension of the register, which leads to an efficient $\QPIP_1$ protocol for $\SampBQP$. 
We can further achieve very small completeness and soundness errors in this way to satisfy the premise of Mahadev's compilation.  
Note that the established $\QPIP_1$ protocol is information-theoretically sound without any computational assumption.  (\Cref{sec:qpip1}.)

\vspace{2mm} \noindent  \emph{$\diamond$ Compile $\QPIP_1$ into $\QPIP_0$ after parallel repetition}: we are ready to address the aforementioned drawback of Mahadev's protocol. First, we will use Mahadev's compilation to obtain a $\QPIP_0$ protocol $\PiNaive$ for the above $\QPIP_1$ protocol.  As mentioned previously, this protocol $\PiNaive$ will have some trouble outputting the desired sample in the testing round. 
To address this issue, we will separate the testing and the outputting in different runs of the protocol, similarly to what we have done in the design of $\QPIP_1$ protocol. 
Precisely, we will run again a cut-and-choose protocol $\PiSampZ$ on top of many copies of $\PiNaive$, which are executed in parallel to preserve the number of rounds.

However, since $\PiNaive$ only has computational soundness, we need to design a parallel repetition of $\PiNaive$ in the computational setting, which also prevents the use of quantum de Finetti theorem that only holds in the information-theoretical setting. 
Fortunately,  some recent results on computational parallel repetitions of Mahadev's original protocol for $\BQP$~\cite{arXiv:AlaChiHun19, arXiv:ChiaChungYam19} provide a technical starting point for us. 
Precisely, these results open the box of the analysis of Mahadev's protocol and analyze the computational soundness under a parallel repetition. 

\XW{Precisely, comment on the partition lemma~\cite{arXiv:ChiaChungYam19}, and how it is used ...}
% We end up developing a weaker version of  parallel repetition of Mahadev's protocol inspired by the technique from~\cite{arXiv:ChiaChungYam19}. 
\XW{Note that there is also an important difference between us and~\cite{arXiv:ChiaChungYam19}, the difference between decision and sampling problems.}
\Cref{sec:qpip0_all}.

% A naive attempt is to directly apply Mahadev's protocol on the aforementioned $\QPIP_1$ protocol.
% Unfortunately, the plain version of Mahadev's protocol does not yield favorable parameters by itself.
% However, we cannot directly make use of these parallel repetition results due to the subtle difference between protocols for $\BQP$ and $\SampBQP$.
% One of the major difficulties here is still to deal with both the test part and the output part in $\SampBQP$ protocols.
% However, because we are now in the computational setting, there is no longer any available quantum de Finetti theorem that is usually derived in the information-theoretic setting.
% Due to the nature of parallel repetition in the computational setting, our analysis is much less modular and significantly involved for this part.  
% More intuitions and detailed analysis are given in 


It is worthwhile mentioning that we came to notice some online discussion\footnote{\url{https://www.scottaaronson.com/blog/?p=3697}, e.g., comment \#25, \#26, \#42, \#48. } on the possibility of a CVQC protocol for $\SampBQP$ after we developed our own result. 
These comments suggested a possible reduction of $\SampBQP$ to the local Hamiltonian problem following a similar high-level idea as our solution, however, with no detail and a seemingly different technical route. 
However, they failed to identify other important gaps in the protocol design that we have addressed. 

\vspace{2mm} \noindent \textbf{A generic compiler to upgrade $\QPIP_0$ protocols with blindness}. At a high-level, the idea is simple: we run the original protocol under a QFHE with the verifier's key. Intuitively, this allows the prover to compute his next message under encryption without learning the underlying verifier's message, and hence achieves blindness while preserving the properties of the original protocol.
One subtlety with this approach is due to the fact that the verifier is classical while the QFHE cipher text could depend on both quantum and classical data.
In order to make the classical verifier work in this construction, the ciphertext and the encryption/decryption algorithm needs to be classical when the underlying message is classical, which is fortunately satisfied by~\cite{mahadev_qfhe}.

A more subtle issue is to preserve the soundness.
In particular, compiled protocols with only one-time use of QFHE might (1) leak information about the circuit being evaluated during the homomorphic evaluation of QFHE ciphertexts (i.e., no \emph{circuit privacy});
or (2) fail to simulate original protocols upon receiving invalid ciphertexts.
We address these issues by letting the verifier switch to a fresh new key for each round of the protocol.
Details are given in \Cref{sec:BlindBQP2}.
%\XW{To KM: expand the above a bit more?}

\vspace{2mm} \noindent \textbf{Open Questions.} Our main focus is on the feasibility of the desired functionality and properties, which nevertheless leaves a big room for the improvement of efficiency.
Some of our parameter dependence inherits from previous works (e.g.~\cite{FOCS:Mahadev18a}), whereas some is due to our own construction. 
It will be extremely interesting to improve the parameter dependence with potentially new techniques. 

\Ethan{TODO add organization of paper}

% \begin{itemize}
%     \item Open questions, and also explain for some associated high-cost. specifically
%     \item T dependence. In general, improve the efficiency.  
%     \item negligible soundness error.  compare with classical? what's the state-of-the art.
% \end{itemize}

% \Ethan{This section is currently all rough draft. We'll probably rewrite almost all of it.}

% \Ethan{application: verifiable private constant round delegation}


\section{Preliminaries}

\Ethan{Check that the quantifiers are all there}

\subsection{Notations}

Let $\mathcal{B}$ be the Hilbert space corresponding to one qubit. Let $H:\mathcal{B}^{\otimes n}\rightarrow\mathcal{B}^{\otimes n}$ be Hermitian matrices. We use $H\geq0$ to denote $H$ being positive semidefinite. Let $\lambda(H)$ be the smallest eigenvalue of $H$. The ground states of $H$ are the eigenvectors corresponding to $\lambda(H)$. For matrix $H$ and subspace $S$, let $H\big|_S=\Pi_S H \Pi_S$, where $\Pi_S$ is the projector onto the subspace $S$. For a $T$-qubit Hilbert space, let the state $\ket{\widehat{t}}=\ket{1}^{\otimes t}\otimes \ket{0}^{{\otimes (T-t)}}$.
We write $F(\rho_1, \rho_2)=\left(\tr\sqrt{\sqrt{\rho_1}\rho_2\sqrt{\rho_1}}\right)^2$ for the fidelity between $\rho_1$ and $\rho_2$.
We write $\frac{1}{2}\norm{\rho_1-\rho_2}_1$ for the trace distance between $\rho_1$ and $\rho_2$. For all $n$-qubit states $\rho_1, \rho_2\in\cB^{\otimes n}$ we have $\frac{1}{2}\norm{\rho_1-\rho_2}_1\leq\sqrt{1-F(\rho_1, \rho_2)}$.

\begin{definition} [quantum-classical channels]
	\label{def:QCChannel}
	A quantum measurement is given by a set of matrices $\set{M_k}$ such that $M_k\geq0$ and $\sum_k M_k=\id$.
	We associate to any measurement a map $\Lambda(\rho)=\sum_k \tr(M_k\rho)\ket{k}\bra{k}$
	with $\set{\ket{k}}$ an orthonormal basis.
	This map is also called a \emph{quantum-classical channel}.
\end{definition}

The phase gate and Pauli matrices are denoted as follows.

\begin{definition}
	$P(i)=\begin{pmatrix}1&0\\0&i\end{pmatrix}$, $X=\begin{pmatrix}0&1\\1&0\end{pmatrix}$,
	$Y=\begin{pmatrix}0&-i\\i&0\end{pmatrix}$,
	$Z=\begin{pmatrix}1&0\\0&-1\end{pmatrix}$
\end{definition}

\subsection{Relevant complexity classes}

We define a few relevant complexity classes.

\begin{definition} [$\BQP$]
	Definition from Kitaev:
	A \emph{quantum algorithm} for the computation of a function $F:\zo^*\rightarrow\zo^*$ is a classical algorithm (i.e., a Turing machine) that computes a function of the form $x\mapstochar\rightarrow Z(x)$, where $Z(x)$ is a description of a quantum circuit which computes $F(x)$ on empty input. The function $F$ is said to belong to class $\BQP$ if there is a quantum algorithm that computes $F$ in time $\poly(n)$.

	Definition from Complexity Zoo:
	$\BQP$ is the class of languages $L$ for which for all $n\in\bbN$ there exists a quantum circuit constructible in time $\poly(n)$ that, given any $x\in\set{0, 1}^n$ as input, correctly decides whether $x\in L$ at least $\frac{2}{3}$ of the time.
	\Ethan{Just copy this from somewhere... Does anyone even define this?}
\end{definition}

\iffalse

\begin{definition} [$\FBQP$]
	A function $f:\set{0,1}^*\rightarrow\set{0,1}^*$ is in $\FBQP$ if there is a $\BQP$ machine that, $\forall x$, outputs $f(x)$ with overwhelming probability.
	\Ethan{Need to be more formal. Also should be efficient verifiable}
\end{definition}

\fi

We define search and sampling versions of $\BQP$ based on \cite{aaronson_2013}.

\iffalse

\begin{definition} [search problem]
	A search problem $R$ is a collection of nonempty sets $(A_x)_{x\in\set{0, 1}^*}$, one for each input string $x\in\set{0, 1}^*$, where $A_x$... \Ethan{Great, interface doesn't line up correctly}
\end{definition}

\fi

\begin{definition} [sampling problem]
	A sampling problem $S$ is a collection of probability distributions $(D_x)_{x\in\set{0, 1}^*}$, one for each input string $x\in\set{0,1}^n$, where $D_x$ is a distribution over $\set{0,1}^{p(n)}$ for some fixed polynomial $p$.
\end{definition}

\begin{definition} [$\SampBQP$]
	$\SampBQP$ is the class of sampling problems $S=\left(D_x\right)_{x\in\set{0, 1}^*}$ for which there exists a polynomial-time quantum algorithm $B$ that, given $(x, 0^{1/\varepsilon})$ as input, samples from a probability distribution $C_x$ such that $\norm{C_x-D_x}\leq\varepsilon$.
\end{definition}

\subsection{Quantum Prover Interactive Protocol (QPIP)}
We classify the interaction between a (almost classical) client and a quantum server for sampling problems, extending the classification by \cite{FOCS:Mahadev18a}.

\begin{definition}
	We say $\Pi=(P, V)$ is a protocol for the sampling problem $(D_x)_{x\in\zo^*}$ with completeness error $c$ and soundness error $s$ \Ethan{Might need these to be functions of $\abs{x}$} if
	\Ethan{Look up ``interactive protocols for BQP". Right now it's missing quantifier for all x. Also need to mention d is decision bit; maybe do that in next definition and swap locations}
	\begin{itemize}
		\item Let $(d, z)\leftarrow(P, V)(x)$. Then $d=rej$ with probability at most $c$.
		\item For all cheating prover $P^*$, let $(d, z)\leftarrow(P^*, V)(x)$. Let \Ethan{Use display math to make it obvious I'm defining this} $z_{ideal}\leftarrow D_x$ if $d=acc$, else $z_{ideal}=\bot$. Then $\norm{(d, z) - (d, z_{ideal})} \leq s$ \Ethan{make stat. distance notation consistent}.
	\end{itemize} \hannote{need computational?}
\end{definition}

\Ethan{Acc, rej, P, V fonts}

\Ethan{Might need to write our own def. here}

\Ethan{Look at thesis for this}

\begin{definition}
	A sampling problem $S=(D_x)_{x\in\set{0, 1}^*}$ is said to be \Ethan{Try to make this more general; remove mentions of sampling/decision problems} in $\QPIP_\tau$ with completeness $c$ and soundness $s$ \Ethan{Don't tie this with completeness and soundness yet} if there exists a protocol $(\bbP, \bbV)(x)$ for $S$ with the following properties:
	\begin{itemize}
		\item $\bbP$ is run by the prover, a $\BQP$ machine, which also has access to a quantum channel that can transmit $\tau$ qubits to the verifier per use.
		\item $\bbV$ is run by the verifier, which is a hybrid machine of a classical part and a limited quantum part. The classical part is a $\BPP$ machine. The quantum part is a register of $\tau$ qubits, on which the verifier can perform arbitrary quantum operations and which has access to a quantum channel which can transmit $\tau$ qubits. At any given time, the verifier is not allowed to possess more than $\tau$ qubits. The interaction between the quantum and classical parts of the verifier is the usual one: the classical part controls which operations are to be performed on the quantum register, and outcomes of measurements of the quantum register can be used as input to the classical part.
		\item There is also a classical communication channel between the prover and the verifier, which can transmit $\poly(\abs{x})$ many bits to either direction. 
	\end{itemize}
\end{definition}

\Ethan{Two separate definitions for comp and soundness}

\Ethan{Soundness?}

\Ethan{Just call this blindness and put this under interactive protocols}

\Ethan{See Thomas' paper if he defined this. Or actually, see https://www.iis.sinica.edu.tw/~kmchung/download/ConstCZK.pdf}

We present the security definition for interactive protocols:

\begin{definition}
    \Ethan{Define this in terms of views instead; see paper linked on Skype}

	Let $\lambda$ be a security parameter.
	Let $(\bbP, \bbV)$ be an interactive protocol with security parameter $\lambda$.
	Then it is IND-CPA secure if $\forall x\in\set{0,1}^n$ no polynomial time adversary $\cA$ can win \protoref{indcpa} with probability better than $\frac{1}{2}+\negl(\lambda)$
\end{definition}

\begin{protocol}{Attack against semantic security}
	\label{proto:indcpa}
	\begin{enumerate}
		\item The challenge picks $b\in\set{0,1}$ at random
		\item If $b=0$, the challenger runs the protocol with the adversary, acting as the verifier with input $0^n$
		\item Otherwise, the challenger runs the protocol with the adversary, acting as the verifier with input $x$
		\item $\cA$ attempts to guess $b$
	\end{enumerate}
\end{protocol}

\subsection{Chernoff bound}

Taken from \href{http://math.mit.edu/~goemans/18310S15/chernoff-notes.pdf}{here}.

\begin{thm}
\label{thm:Chernoff}
Let $X=\sum_{i=1}^n X_i$ where $X_i$ are i.i.d. \Ethan{Ind. is enough} Bernoulli trials, and $\mu=\E[X]$.
Then for all $0<\delta<1$,
$$P[\abs{X-\mu}\geq\delta\mu]\leq2e^{-\frac{\mu\delta^2}{3}}.$$
\end{thm}

\subsection{Projection Lemma}

We use the projection lemma from \cite{kempe_kitaev_regev_2006}, which describes the conditions under which we can estimate the ground state energy of $H_1 + H_2$ with that of $H_1\big|_{\ker H_2}$.

\begin{thm}
	Let $H=H_1+H_2$ be the sum of two Hamiltonians operating on some Hilbert space $\cH=\cS+\cS^\bot$.
	The Hamiltonian $H_2$ is such that $\cS$ is a zero eigenspace and the eigenvectors in $\cS^\bot$ have eigenvalues at least $J>2\norm{H_1}$. Then,
	$$\lambda\left(H_1\big|_\cS\right)-\frac{\norm{H_1}^2}{J-2\norm{H_1}^2}\leq\lambda(H)\leq\lambda\left(H_1\big|_\cS\right)$$
\end{thm}

We will instead use the following formulation, which can be obtained by relabeling variables from above.

\Ethan{State 2.3; prove from 2.2 in appendix}

\begin{thm}
	\label{thm:projection}
	Let $H_1, H_2$ be local Hamiltonians where $H_2\geq0$. Let $K=\ker H_2$ and
	$$J=\frac{10\norm{H_1}^2}{\lambda\left(H_2\big|_{K^\bot}\right)}$$
	then we have
	$$\lambda(H_1+JH_2)\geq\lambda\left(H_1\big|_K\right)-\frac{1}{8}$$
\end{thm}

\subsection{Quantum de Finetti Theorem under Local Measurements}

De Finetti theorem provides a way to obtain close to independent samples by taking random subsystems of a quantum system.
There are many formulations; we use the one from \cite{Brandão2017} because we need to avoid exponential dependence on number of qubits in each subsystem.
\begin{thm}
	\label{deFinetti}
	Let $\rho^{A_1\ldots A_k}$ be a permutation-invariant state on registers $A_1,\ldots,A_k$ where each register is $s$ qubits,
	then for every $0\leq l\leq k$ there exists states $\set{\rho_i}$ and $\set{p_i}\subset\bbR$ such that
	$$\max_{\Lambda_1,\ldots,\Lambda_l}
	\norm{(\Lambda_1\otimes\ldots\otimes\Lambda_l)\left(\rho^{A_1\ldots A_l}-\sum_ip_i\rho_i^{A_1}\otimes\ldots\otimes\rho_i^{A_l}\right)}_1
	\leq\sqrt{\frac{2l^2s}{k-l}}$$
	where $\Lambda_i$ are quantum-classical channels.
\end{thm}

\subsection{Quantum Homomorphic Encryption Schemes}

\Ethan{Everything gets security param.}

\def\QHE{\mathsf{QHE}}
\def\QGen{\mathsf{QHE.Keygen}}
\def\QEnc{\mathsf{QHE.Enc}}
\def\QEval{\mathsf{QHE.Eval}}
\def\QDec{\mathsf{QHE.Dec}}

We use the quantum fully homomorphic encryption scheme given in \cite{mahadev_qfhe} which is compatible with our use of a classical client. We start by presenting the interface of a homomorphic encryption scheme:
\begin{definition}
	A leveled homomophic encryption scheme is tuple of algorithms \linebreak $\mathsf{HE}=(\mathsf{HE.Keygen}, \mathsf{HE.Enc}, \mathsf{HE.Dec}, \mathsf{HE.Eval})$ with the following descriptions:
	\begin{itemize}
		\item $\mathsf{HE.Keygen}(1^\lambda, 1^L)\rightarrow(pk, sk)$
		\item $\mathsf{HE.Enc}_{pk}(\mu)\rightarrow c$
		\item $\mathsf{HE.Dec}_{sk}(c)\rightarrow \mu^*$
		\item $\mathsf{HE.Eval}_{pk}(f, c_1, \ldots, c_l)\rightarrow c_f$
	\end{itemize}
\end{definition}

$\mathsf{HE}$ also satisfies, with overwhelming probability in $\lambda$, that
$$\mathsf{HE.Dec}_{sk}(\mathsf{HE.Eval}_{pk}(f, c_1, \ldots, c_l))=f(\mathsf{HE.Dec}_{sk}(c_0),\ldots,\mathsf{HE.Dec}_{sk}(c_l))$$
where $f$ is specified by a circuit of depth at most $L$.

\Ethan{To be pedantic, the above doesn't imply Dec undoes Enc even if we sub in $f=\id$.}

We also recall the security definition for a FHE scheme.

\begin{definition}
	A FHE scheme $\mathsf{HE}$ is IND-CPA secure if, for any polynomial time adversary $\cA$, there exists a negligible function $\mu(\cdot)$ such that
	$$\abs{Pr[\cA(pk, \mathsf{HE.Enc}_{pk}(0))=1]-Pr[\cA(pk, \mathsf{HE.Enc}_{pk}(1))=1]}=\mu(\lambda)$$
	where $(pk, sk)\leftarrow\mathsf{QHE.Keygen}(1^\lambda)$
\end{definition}

The quantum homomorphic encryption scheme $\mathsf{QHE}$ from \cite{mahadev_qfhe} has additional properties that facilitate the use of classical clients:
\begin{itemize}
	\item $\QGen$ can be done classically.
	\item In the case where the plaintext is classical, $\QEnc$ can be done classically.
	\item Its ciphertext takes the form $(X^xZ^z\rho Z^zX^x, c_{x, z})$, where $\rho$ is the plaintext and $c_{x, z}$ is a ciphertext that decodes to $(x, z)$ under a certain classical homomorphic encryption scheme.
\end{itemize}

\Ethan{Reword the above? How? 3rd point can be weakened to can be decrypted classically? Maybe call this ``classical friendly". Mention LWE. There exists a classical friendly... etc.}

% \section{Delegation for FBQP}
\label{sec:FBQP}

\def\MF{\mathsf{MF}}
\nc{\PiMF}{\ensuremath{\Pi_\MF}}
\nc{\VMF}{\ensuremath{V_\MF}}
\nc{\PMF}{\ensuremath{P_\MF}}
\nc{\PMFstar}{\ensuremath{P_\MF^*}}
\nc{\cVMF}[1]{\ensuremath{\cV_{\MF,#1}}}
\nc{\cPMF}[1]{\ensuremath{\cP_{\MF,#1}}}

We start with the one-message $\QPIP_1$ protocol $\PiMF$ given in \cite{mf16} for deciding a $\BQP$ language $L$.
Roughly, on input $x\in\set{0,1}^n$, the prover sends a certificate state $\rho$ to the verifier qubit-by-qubit.
The verifier samples a measurement basis choice $\Lambda$ according to a certain distribution, and applies it to $\rho$.
The verifier then, using the measurement results, decides whether to accept or reject.

\begin{protocol}{1-message $\QPIP_1$ protocol $\PiMF = (\PMF, \VMF)$ for $\BQP$ language $L$}
	\label{proto:BQP}
	Common input: $x\in\set{0,1}^n$. 
	\begin{enumerate}
		\item The prover generates a $m$-qubit certificate state $\rho\leftarrow\cPMF{1}(x)$ and sends $\rho$ to the verifier qubit-by-qubit.
		\item The verifier samples $\Lambda \leftarrow \cVMF{2}(x)$, where $\Lambda=\Lambda_1\otimes\ldots\otimes\Lambda_m$;
			each $\Lambda_i$ is a single-qubit $X$ or $Z$ measurement.
			It then applies $\Lambda$ to $\rho$ qubit-by-qubit, obtaining $w\leftarrow\Lambda(\rho)$.
			Finally, it verifies the measurement outcome and produces a decision $o\leftarrow\cVMF{3}(x, \Lambda, w)$ where $o\in\set{acc, rej}$ and $\cVMF{3}$ is a classical polynomial-time algorithm.
	\end{enumerate}
\end{protocol}
\begin{thm}[from \cite{FOCS:Mahadev18a, mf16}]
    $\PiMF$ has negligible completeness and soundness.
\end{thm}

Here we state our delegation protocol for $\FBQP$.
We do so by breaking an $\FBQP$ instance down into $\BQP$ instances.
Note that our $\BQP$ scheme is one-sided; the prover is allowed to cause the verifier to reject a yes-instance. \Ethan{Explain this better... Write more intuitions. Decision vs function problems; exists vs for all}
As a result, we run two copies of $\PiMF$ for every output bit as below:

\begin{protocol}{$\QPIP_1$ protocol $\Pi_{\FBQP}$ for $f\in\FBQP$}
	\label{proto:QPIP0FBQP}
	Verifier's input: $x\in\set{0, 1}^n$

	Denote the $i$-th bit of $f(x)$ as $f_i(x)$.
	For $i\in\bbN, b\in\set{0,1}$, define the $\BQP$ language $L_{i, b}=\set{s\in\set{0,1}^* : f_i(s)=b}$.
	Let $m=\abs{f(x)}$; we assume without loss of generality that it depends only on $n$ and can be efficiently computed classically.
	\Ethan{Move this assumption to the definition and make a remark there}
	\begin{enumerate}
		\item For all $i\in[m], b\in\set{0, 1}$ in parallel, the verifier and the prover run $\PiMF(x)$ for $L_{i, b}$.
		\item The verifier gets outputs $v_{i, b}\in\set{acc, rej}$.
			If $\exists i$ s.t. $v_{i,0}=v_{i,1}$, reject.
			Otherwise, define $y\in\set{0,1}^m$ s.t. $y_i = b$ where $v_{i, b}=acc$,
			The verifier accepts and outputs $y$.
	\end{enumerate}
\end{protocol}

\begin{theorem}
	$\Pi_{\FBQP}$ has negligible completeness and soundness errors.
\end{theorem}
\begin{prf}
	Completeness follows by inspection.

	By the soundness of $\PiMF$, $v_{i, b}=acc$ implies that $x\in L_{i, b}$ with overwhelming probability.
	In other words, $y_i$ has probability $\varepsilon_i=\negl(n)$ to be incorrect.
	By union bound over all output bits, the soundness of $\Pi_{\FBQP}$ is
	$\sum_{i=1}^m \varepsilon_i=\negl(n)$.
\end{prf}


\section{Construction of the $\QPIP_1$ Protocol for $\SampBQP$}
\label{sec:sampbqp}

\subsection{Reducing $\SampBQP$ to the X-Z Local Hamiltonian} \label{sec:LHXZ}

Recall the definition of the history state which serves as a transcript of the circuit evolution~\cite{kitaev2002classical}:

\begin{dfn}[History-state]
    \label{dfn:groundstate}    
    Given any quantum circuit $C=U_T\ldots U_1$ of $T$ elementary gates and input $x\in\{0,1\}^n$, the \emph{history}-state $\histpsi{C(x)}$ is defined by
    \begin{equation}
        \histpsi{C(x)} \equiv \frac{1}{\sqrt{T}}\sum_{t=0}^{T-1}U_t\ldots U_1\ket{x,0}\otimes\ket{\hat{t}},
    \end{equation}
    where the first register of $n$-qubit refers to the input, the second of $m$-qubit refers to the work space ($\mathrm{poly}(n)$, w.l.o.g, $\leq T$ size) which is initialized to $\ket{0}$, and the last refers to the clock space which encodes the time information. Note that $\hat{t}$ could be some representation of $t$=0,..., $T-1$.
\end{dfn}

We define the X-Z local Hamiltonian of interest as follows:

\begin{dfn}[$k$-local X-Z Hamiltonian] For any $n$-qubit system, the set of $k$-local X-Z terms, denoted by $\LHXZ{k}$, contains Hermitian matrices that apply non-trivially on at most $k$ qubits as a product of Pauli $X$ and $Z$ terms. Namely,
\begin{equation}
  \LHXZ{k} = \left\{h_1 \otimes h_2 \otimes \ldots \otimes h_n: \forall i \in [n], h_i \in \{I, X, Z\}, \text{and}, \abs{\set{i: h_i=X \text{ or }Z}} \leq k. \right \}.
\end{equation}
A $k$-local X-Z Hamiltonian $H$ is a linear combination of terms from $\LHXZ{k}$. Namely,
\begin{equation}
  H = \sum_i \alpha_{i} H_i,  \quad \forall i, \alpha_i \in R,  H_i \in \LHXZ{k}.
\end{equation}
\end{dfn}

As our starting point, we will include the existing result of constructing X-Z local Hamitonians for general $\BQP$ computation.
First, we note the fact Toffoli and Hadamard gates form a universal gate set for quantum computation~\cite{Shi03, quant-ph/0301040}.
It is easy to see that both Toffoli and Hadamard gates can be represented as linear combinations of terms from $\LHXZ{}$. For example,
 \begin{equation}
     \mathrm{(Hadamard)} \quad H \equiv \frac{1}{\sqrt{2}} \begin{pmatrix}1&1\\1&-1\end{pmatrix} = \frac{1}{\sqrt{2}}\left (X+Z\right).
 \end{equation}
 The Toffoli gate maps bits $(a,b,c)$ to $(a,b, c \oplus (a \text{ and } b))$, which can be decomposed as $\ket{11}\bra{11}\otimes X+(I-\ket{11}\bra{11})\otimes I$ where $\ket{11}\bra{11}=\frac{1}{4}(I\otimes I+Z\otimes Z-I\otimes Z-Z\otimes I)$.
 
We will follow the unary clock register design from Kitaev's original 5-local Hamiltonian construction~\cite{kitaev2002classical}. Namely, valid unary clock states ($T$-qubit) are $\ket{00\ldots0}$, $\ket{10\dots0}$, $\ket{110\ldots0}$, etc, which span the ground energy space of the following local Hamiltonian:
\begin{equation}
    \Hclock= \sum_{t=1}^{T-1}\proj{01}_{t,t+1},
\end{equation}
where $\proj{01}_{t,t+1}$ stands for a projection on the $t$th and $(t+1)$th qubit in the clock register.
It is observed in~\cite{PhysRevA.78.012352} that $\Hclock$ can be reformulated as a linear combination of terms from $\LHXZ{}$ as follows,
\begin{equation} \label{eqn:Hclock}
   \Hclock= \frac{1}{4}(Z_1 - Z_T) + \frac{1}{4}\sum_{t=1}^{T-1}(I-Z_t\otimes Z_{t+1}),
\end{equation}
where $Z_t$ refers to Pauli $Z$ operated on the $t$th qubit in the clock register.

One can achieve so similarly for $\Hin$ and $\Hprop$. For the input condition, we want to make sure $x=(x_1, \ldots, x_n) \in \{0,1\}^n$ is in the input space and the workspace is initialized to $\ket{0}$ for all qubits at the time $0$. Namely, one can set $\Hin$ to be
\begin{equation}
    \Hin = \sum_{i=1}^n(I- \proj{x_i}_i)\otimes \proj{0}_1 + \sum_{i=1}^m \proj{1}_{n+i} \otimes \proj{0}_1,
\end{equation}
where the last part $\proj{0}_1$ applies on to the first qubit in the clock register. One can rewrite $\Hin$ as
\begin{equation}\label{eqn:Hin}
 \Hin=\frac{1}{4}\sum_{i=1}^n(I-(-1)^{x_i}Z_i)\otimes(I+Z_1) + \frac{1}{4} \sum_{i=1}^m (I - Z_{n+i}) \otimes (I + Z_1).
\end{equation}

For the propagation of the quantum state through the circuit, one uses the $\Hprop$ as follows:
\begin{equation*} \label{eqn:Hprop}
    \Hprop=\sum_{t=1}^T \Hprop^t,
\end{equation*}
where
\begin{equation}
    \Hprop^t=\frac{1}{2}I\otimes\proj{\widehat{t}}
    +\frac{1}{2}I\otimes\proj{\widehat{t-1}}
    -\frac{1}{2} U_t\otimes\ket{\widehat{t}}\bra{\widehat{t-1}}
    -\frac{1}{2}U_t^\dagger\otimes\ket{\widehat{t-1}}\bra{\widehat{t}}.
\end{equation}
Note that $\ket{\widehat{t}}\bra{\widehat{t-1}}=\ket{110}\bra{100}_{(t-1,t,t+1)}$ and similarly for $\ket{\widehat{t-1}}\bra{\widehat{t}}$.
Note that $U_t^\dagger=U_t$, since our gates are either Hadamard or Toffoli. It is observed in~\cite{PhysRevA.78.012352} that
\begin{equation}
   \Hprop^t=\frac{I}{4}\otimes(I-Z_{t-1})\otimes (I+Z_{t+1})-\frac{U_t}{4}\otimes(I-Z_{t-1})\otimes X_t\otimes (I+Z_{t+1}), \forall t=2, \ldots, T-1,
\end{equation}
and
\begin{eqnarray}
  \Hprop^1 &= & \frac{1}{2}(I+Z_2)-U_1\otimes\frac{1}{2}(X_1+X_1\otimes Z_2), \\
  \Hprop^T &= & \frac{1}{2}(I-Z_{t-1})-U_T\otimes\frac{1}{2}(X_T-Z_{T-1}\otimes X_T).
\end{eqnarray}
Combining with the fact that each $U_t$ can be written as a linear combination of terms from $\LHXZ{}$, we conclude that $\Hclock$, $\Hin$, $\Hprop$ are 6-local X-Z Hamiltonian.

We will employ the perturbation technique to amplify the spectral gap of $\Hclock + \Hin + \Hprop$.
Let $\ground{H}$ denote the ground energy of any Hamiltonian $H$.
The projection lemma from \cite{kempe_kitaev_regev_2006} approximates $\ground{H_1 + H_2}$ in terms of $\ground{H_1\big|_{\ker H_2}}$, where $\ker H_2$ denotes the \emph{kernel} space of $H_2$.

\begin{lem}[Lemma 1 in \cite{kempe_kitaev_regev_2006}]
    \label{thm:proj1}
    Let $H=H_1+H_2$ be the sum of two Hamiltonians operating on Hilbert space $\cH=\cS+\cS^\bot$.
    The Hamiltonian $H_2$ is such that $\cS$ is a zero eigenspace and the eigenvectors in $\cS^\bot$ have eigenvalues at least $J>2\norm{H_1}$. Then,
    $$\lambda\left(H_1\big|_\cS\right)-\frac{\norm{H_1}^2}{J-2\norm{H_1}}\leq\lambda(H)\leq\lambda\left(H_1\big|_\cS\right).$$
\end{lem}

We will use the following simple reformulation instead.

\begin{lem}
    \label{lem:projection}
    Let $H_1, H_2$ be local Hamiltonians where $H_2\geq0$. Let $K=\ker H_2$ and
    $$J=\frac{8\norm{H_1}^2+ 2\norm{H_1}}{\lambda\left(H_2\big|_{K^\bot}\right)},$$
    then we have
    $$\lambda(H_1+JH_2)\geq\lambda\left(H_1\big|_K\right)-\frac{1}{8}.$$
\end{lem}
\begin{proof}
    Apply \Cref{thm:proj1} to $H=H_1+JH_2$. Note that the least non zero eigenvalue of $JH_2$ is greater than $2\norm{H_1}$.
\end{proof}

\begin{thm}
    \label{thm:LHReduction}
    Given any quantum circuit $C = U_T \ldots U_1$ of  $T$ elementary gates and input $x \in \{0,1\}^n$, one can construct a 6-local X-Z Hamiltonian $H_{C(x)}$ in polynomial time such that
    \begin{enumerate}
        \item[(1)] $H_{C(x)} = \sum_i \alpha_i H_i$ where
        each $H_i \in \LHXZ{6}$ and $|\alpha_i| \in O(T^9)$. Moreover, there are at most $O(T)$ non-zero terms.
        \item[(2)] $H_{C(x)}$ has $\histpsi{C(x)}$ as the unique ground state with eigenvalue $0$ and has a spectral gap at least $\frac{3}{4}$. Namely,  for any state $\ket\phi$ that is orthogonal to $\histpsi{C(x)}$, we have $\braket{\phi|H_{C(x)}|\phi}\geq \frac{3}{4}$.
    \end{enumerate}
\end{thm}

\begin{proof}
We will use the above construction $\Hclock$ (\cref{eqn:Hclock}), $\Hin$ (\cref{eqn:Hin}), $\Hprop$ (\cref{eqn:Hprop}) as our starting point, which are already 6-local X-Z Hamiltonian constructable in polynomial time. However, $H_{\mathrm{old}}=\Hin + \Hclock + \Hprop$ does not have the desired spectral gap. To that end, our construction will be a weighted sum of $\Hin$, $\Hclock$, and $\Hprop$ as follows,
\begin{equation}
    H_{\mathrm{new}}= \Hin + \Jclock \Hclock + \Jprop \Hprop,
\end{equation}
where $\Jclock$ and $\Jprop$ will be obtained using \Cref{lem:projection}.

Let $\Kin=\ker \Hin$, $\Kclock=\ker \Hclock$, and $\Kprop=\ker \Hprop$. It is known from e.g.,~\cite{kitaev2002classical}, that
\[
   \Kin \cap \Kclock \cap \Kprop = \spn\set{\histpsi{C(x)}}.
\]
Thus $\histpsi{C(x)}$ remains in the ground space of $H_{\mathrm{new}}$. Let $S$ denote its orthogonal space. Namely, $S=(\spn\set{\ket{\psi_{C(x)}}})^\bot$.
Denote by $\Hin\big|_S$ the restriction of $\Hin$ on space $S$ and similarly for others.

Consider $\Hin + \Jclock\Hclock$ first. According to \cref{lem:projection}, by choosing
\[
  \Jclock = \frac{8\norm{\Hin\big|_S}^2 + 2 \norm{\Hin\big|_S}}{\ground{\Hclock\big|_{S\cap \Kclock^\bot}}} = O(T^2),
\]
where we use the fact $\norm{\Hin|_S}\leq T$ and $\ground{\Hclock\big|_{S\cap \Kclock^\bot}}\geq \ground{\Hclock\big|_{\Kclock^\bot}}=1$,
we have
\[
 \ground{\Hin\big|_S + \Jclock\Hclock\big|_S}\geq \ground{\Hin\big|_{S\cap \Kclock}} - \frac{1}{8}.
\]
Consider further adding $\Jprop\Hprop$ term.  By choosing
\[
 \Jprop= \frac{O(\norm{\Hin\big|_S + \Jclock\Hclock\big|_S}^2)}{\ground{\Hprop\big|_{S \cap \Kprop^\bot}}}= O(T^8),
\]
where we use the fact $\ground{\Hprop\big|_{S\cap \Kprop^\bot}}\geq \ground{\Hprop\big|_{\Kprop^\bot}}=\Omega(T^{-2})$~\cite{kitaev2002classical}, we have  
\[
 \ground{\Hin\big|_S + \Jclock\Hclock\big|_S + \Jprop\Hprop\big|_S} \geq \ground{\Hin\big|_{S \cap \Kclock \cap \Kprop}} - \frac{1}{4}.
\]
A simple observation here is that $S \cap \Kclock \cap \Kprop$ is the span of history states with different inputs or different initialization of the work space. Namely, $\ground{\Hin\big|_{S \cap \Kclock \cap \Kprop}} \geq 1$. Thus,
\[
  \ground{(\Hin + \Jclock\Hclock + \Jprop\Hprop)\big|_S} \geq 1- \frac{1}{4}=\frac{3}{4}.
\]
Given that $\histpsi{C(x)}$ is the ground state of $H_{\mathrm{new}}$ with eigenvalue 0, and any orthogonal state to $\histpsi{C(x)}$ has eigenvalue at least $\ground{H_{\mathrm{new}}}\geq 3/4$, the spectral gap of $H_{\mathrm{new}}$ at at least $3/4$.

Note that $H_{\mathrm{new}}$ is a 6-local X-Z Hamiltonian by construction. It suffices to check the bound of $\abs{\alpha_i}$ and the number of terms. The former is one more than the order of $\Jprop$ since each $\Hprop^t$ contributes $\frac{1}{4}$ to the $I$ term, creating an extra factor of $T$. The latter is by counting the number of terms from $\Hin, \Hclock, \Hprop$, each of which is bounded by $O(T)$.
\end{proof}

\noindent \textbf{Remark.} We believe that the specific parameter dependence above can be tightened by a more careful analysis. However, as our focus is on the feasibility, we keep the above slightly loose analysis which might be more intuitive. 


\subsection{Delegation Protocol for $\QPIP_1$ client}
\label{sec:qpip1}
In this subsection, we construct a one-message $\QPIP_1$ delegation protocol for $\SampBQP$. By definition of $\QPIP_1$, we assume the client has limited quantum power, e.g., performing single qubit $X$ or $Z$ measurement one by one.
Intuitively, one should expect the one-message from the server to the client is something like the history state so that the client can measure to sample.  

At a high-level, the design of such protocols should consist of at least two components: (1) the first component should test whether the message is indeed a valid history state; (2) the second component should simulate the last step of any $\SampBQP$ computation by measuring the final state in the computational basis.

Our construction of X-Z local Hamiltonian $H$ from \Cref{thm:LHReduction} will help serve the first purpose.
In particular, we adopt a variant of the energy verification protocol for local Hamiltonian (e.g., ~\cite{mf16, PhysRevA.93.022326}) to certify the energy of $H$ with only $X$ or $Z$ measurements.
Moreover, because of the large spectral gap, when the energy is small, the underlying state must also be close to the history state.
Precisely, consider the following protocol $\cVGS$:

\begin{protocol}{Energy verification for X-Z local Hamiltonian $\cVGS$} \label{AlgGroundStateCheck}
Given a $k$-local X-Z Hamiltonian
$H=\sum_i \alpha_{i} H_i$ (i.e., $\forall i, H_i \in \LHXZ{k}$) and any state $\ket{\phi}$.
%Let $\ket\phi$ be the potential ground state to check.

\begin{itemize}
\item Let $p_i= \abs{\alpha_i}/\sum_i \abs{\alpha_i}$ for each $i$. Sample $i^*$ according to $p_{i^*}$.
\item Pick $H_{i^*}$ which acts non-trivially on at most $k$ qubits of $\ket{\phi}$. Measure the corresponding single-qubit Pauli X or Z operator.
Record the list of the results $x_j=\pm 1$ for $j=1, \ldots k$.
\item Let $r=x_1x_2\cdots x_k$. The protocol \emph{accepts} if $r$ and $\alpha_{i^*}$ have different signs, i.e., $\sgn(\alpha_{i^*})r=-1$. Otherwise, the protocol \emph{rejects}.
\end{itemize}
\end{protocol}
% In this subsection, we prove \cref{ThmXZCheck}.
% That is, we present and analyze an algorithm that checks whether a given state is the ground state of some fixed $H_{C(x)}$, following~\cite{PhysRevA.93.022326}.

\begin{lem}[\cite{PhysRevA.93.022326}]
    \label{thm:HamCheck}
    For any $k$-local X-Z Hamiltonian $H=\sum_i \alpha_{i} H_i$ and any state $\ket{\phi}$,
    the protocol $\cVGS$ in Protocol~\ref{AlgGroundStateCheck} accepts with
    probability
\begin{equation}
 \mathrm{Prob}[ \cVGS \text{ accepts } \ket{\psi}] = \frac{1}{2} - \frac{1}{2 \sum_i \abs{\alpha_i}}\braket{\phi|H|\phi}.
\end{equation}
\end{lem}

\begin{theorem} \label{thm:HamCheckClose}
Given any quantum circuit $C$ and input $x$, consider using $H_{C(x)}$ from \Cref{thm:LHReduction} in Protocol~\ref{AlgGroundStateCheck} ($\cVGS$).
For any state $\rho$, and $0< \epsilon < 1$, if $\cVGS$ accepts $\rho$ with probability,
\[
 \mathrm{Prob}[\cVGS \text{ accepts } \rho] \geq \frac{1}{2} - \frac{\epsilon}{2 \sum_i \abs{\alpha_i}},
\]
then the trace distance between $\rho$ and $\histpsi{C(x)}$ is at most $\frac{2}{\sqrt{3}}\sqrt{\epsilon}$.
\end{theorem}

\begin{prf} Consider the pure state case $\rho=\proj{\phi}$ first. By \Cref{thm:HamCheck} and our assumption, we have $\braket{\phi|H|\phi} \leq \epsilon$.
Decompose $\ket{\phi}= \alpha \histpsi{C(x)} + \beta \histpsi{C(x)}^\bot$. Note that $\histpsi{C(x)}$ is an eigenvector $H_{C(x)}$ of eigenvalue 0 and all other eigenvalues are at least $3/4$. Thus, we have $\abs{\alpha}^2 \geq 1-\frac{4}{3}\epsilon$. Thus,
\[
   \norm{\proj{\psi^{\mathrm{hist}}_{C(x)}}- \proj{\phi}}_{\tr} = \sqrt{1- |\braket{\psi^{\mathrm{hist}}_{C(x)}|\phi}|^2}
   \leq \frac{2}{\sqrt{3}}\sqrt{\epsilon}.
\]
For any mixed state $\rho=\sum_i p_i \proj{\phi_i}$, by the triangle inequality, we have
\[
    \norm{\proj{\psi^{\mathrm{hist}}_{C(x)}}- \rho}_{\tr} \leq \sum_i p_i    \norm{\proj{\psi^{\mathrm{hist}}_{C(x)}}- \proj{\phi_i}}_{\tr} \leq \sum_i p_i \frac{2}{\sqrt{3}} \sqrt{\epsilon} =\frac{2}{\sqrt{3}}\sqrt{\epsilon}.
\]
\end{prf}

To serve the second purpose, one needs to combine the test and the output on multiple copies of the history states, where we construct the following cut-and-choose protocol.  
The challenge comes from the fact that a cheating prover might send something rather than copies of the history state.
In particular, the prover can entangle between different copies in order to cheat.
In the case of certifying a $\BQP$ computation, the goal is to verify the ground energy of any local Hamiltonian.
A cheating strategy with potential entanglement won't create any witness state with an energy lower than the actual ground energy.
Thus, this kind of attack won't work for $\BQP$ computation.

However, in the context of $\SampBQP$, one needs to certify the ground energy (known to be zero in this case) and to output a good copy.
While the statistical test as before can be used to certify the ground energy,
it has less control on the shape of the output copy.
In fact, the prover can always prepare a bad copy among with many good copies as a plain attack.
This attack will succeed when the bad copy is chosen to output, the probability of which is non-negligible in terms of the total number of copies assuming some symmetry of the protocol.
The potential entanglement among different copies could further complicate the analysis.
We employ the quantum \emph{de Finetti}'s theorem to address this technical challenge.
Specifically, given any permutation-invariant $k$-register state, it is known that the reduced state on many subsets of $k$-register will be close to a separable state.
This helps establish some sort of independence between different copies in the analysis.
To serve our purpose, we adopt the following version of quantum de Finetti's theorem from~\cite{Brandao2017} where the error depends nicely  on the number of qubits, rather than the dimension of quantum systems.

\begin{thm}[\cite{Brandao2017}]
    \label{deFinetti}
    Let $\rho^{A_1\ldots A_k}$ be a permutation-invariant state on registers $A_1,\ldots,A_k$ where each register contains $s$ qubits.
    For any $0\leq l\leq k$,  there exists states $\set{\rho_i}$ and $\set{p_i}\subset\bbR$ such that
    $$\max_{\Lambda_1,\ldots,\Lambda_l}
    \norm{(\Lambda_1\otimes\ldots\otimes\Lambda_l)\left(\rho^{A_1\ldots A_l}-\sum_ip_i\rho_i^{A_1}\otimes\ldots\otimes\rho_i^{A_l}\right)}_1
    \leq\sqrt{\frac{2l^2s}{k-l}}$$
    where $\Lambda_i$ are quantum-classical channels.
\end{thm}

\section{Delegation Protocol for Hybrid Client}

In this section, we construct a one-message $\QPIP_1$ delegation protocol for $\SampBQP$.
At a high level, it is a cut-and-choose protocol.
The server constructs multiple copies of the ground state as certificates,
then the client randomly chooses a copy to output and checks the rest.
Unfortunately our approach incurs inverse polynomial soundness errors.

There are two main challenges to this cut-and-choose approach.
First, the client needs to reliably extract the circuit output from its corresponding ground state.
We accomplish this by padding the circuit with identity gates at the end.
By doing so, the clock register collapses to after the last non-identity gate with high probability.

The other challenge is that a cheating prover's certificates can be arbitrarily entangled,
so common techniques such as Chernoff bounds can't be applied as-is.
This wasn't a challenge for $\Piblind$ for $\BQP$ earlier because...
\Ethan{TODO; need brief and intuitive explanation}
In our case with $\SampBQP$, we overcome this challenge instead by using de Finetti's theorem
to approximate our measurement results with that of some unentangled copies,
within inverse polynomial errors.

\def\GS{\mathsf{GS}}
\nc{\PiGS}{\ensuremath{\Pi_\GS}}
\nc{\VGS}{\ensuremath{V_\GS}}
\nc{\PGS}{\ensuremath{P_\GS}}
\nc{\PGSstar}{\ensuremath{P_\GS^*}}
\nc{\cVGS}[1]{\ensuremath{\cV_{\GS,#1}}}
\nc{\cPGS}[1]{\ensuremath{\cP_{\GS,#1}}}

\def\Samp{\mathsf{Samp}}
\nc{\PiSamp}{\ensuremath{\Pi_\Samp}}
\nc{\VSamp}{\ensuremath{V_\Samp}}
\nc{\PSamp}{\ensuremath{P_\Samp}}
\nc{\PSampstar}{\ensuremath{P_\Samp^*}}
\nc{\cVSamp}[1]{\ensuremath{\cV_{\Samp,#1}}}
\nc{\cPSamp}[1]{\ensuremath{\cP_{\Samp,#1}}}

\def\GS{\mathsf{GS}}

Now we present our $\QPIP_1$ protocol. It is parametrized by $\lambda$, the degree of its inverse polynomial soundness.

\begin{protocol}{$\QPIP_1$ protocol $\PiSamp$ for $\SampBQP$}\label{ProtoQPIP1}
	Let $C'$ be $C$ padded with $T^{\lambda + 1}$ identity gates at the end.

	Pick $\varepsilon$ small enough so it holds for all $\ket\phi$ that:
	\begin{equation}
		\label{QPIP1eps}
		P[\cA_\GS(\ket{\phi})=acc]>\frac{1}{2}-2\varepsilon\Rightarrow\braket{\phi|H_{C'(x)}|\phi}<\frac{1}{T^\lambda}
	\end{equation}

	Pick $n$ large enough so by Chernoff bound:
	\begin{equation}
		\label{QPIP1Chernoff1}
		P\left[Bin(n, \frac{1}{2}-2\varepsilon)\geq\left(\frac{1}{2}-\varepsilon\right)n\right]<2^{-\lambda}
	\end{equation}
	\begin{equation}
		\label{QPIP1Chernoff2}
		P\left[Bin(n, \frac{1}{2})\leq\left(\frac{1}{2}-\varepsilon\right)n\right]<2^{-\lambda}
	\end{equation}

	Let $N$ be large enough so that de Finetti's theorem can be applied at the cost of $T^{-\lambda}$ errors if one randomly picks $n+1$ subsystems out of $N$ permutation-invariant subsystems.
		That is,
		$$\sqrt{\frac{2(n+1)^2\ln\abs{A}}{N-(n+1)}}<T^{-\lambda}$$
		where $\abs{A}$ is the number of qubits in $\ket{\psi_{C'(x)}}$

	\begin{enumerate}
		\item The verifier privately samples $I\subset[N]$ s.t. $\abs{I}=n$.
			It then privately samples $k\xleftarrow{\$}[N]\setminus I$.
		\item The honest prover prepares $N$ copies of $\ket{\psi_{C'(x)}}$ and sends all of them to the verifier qubit-by-qubit.
		\item For $i$ from $1$ to $N$, the verifier chooses what to do to the $i$-th copy, $\rho_i$, as follows:
		\begin{enumerate}
			\item If $i\in I$, run $\cA_{\GS}(\rho_i)$.
			\item Otherwise, if $i=k$, measure the data register of $\rho_i$ and save the outcome as $y$.
			\item Otherwise, discard $\rho_i$.
		\end{enumerate}
		\item If the proportion of copies accepted by $\cA_{\GS}$ is greater than $\frac{1}{2}-\varepsilon$ then the verifier accepts and outputs $y$. Otherwise, it rejects.
	\end{enumerate}
\end{protocol}

Note that $\VSamp$ only needs to apply $X$ and $Z$ measurements, and is classical otherwise. We now show the completeness and soundness of $\PiSamp$.

\begin{thm}
    \label{QPIP1thm}
	$\PiSamp$ has negligible completeness error and $O(T^-\lambda)$ soundness error.
\end{thm}
\begin{prf}
	The completeness follows from \cref{QPIP1Chernoff2} and inspection.

	Now we show soundness.
	Suppose $\PSampstar$ is a cheating prover that sends some $\sigma$ to the verifier.

	We first show that de Finetti's theorem can indeed be used here to achieve independence.
	Randomly picking $n+1$ out of $N$ registers is equivalent to first applying a random permutation then taking the first $n+1$ registers.
	A random permutation, in turn, is a classical mix over all possible permutations:
	$$\sigma'=\frac{1}{\abs{\Sym(N)}}\sum_{\Pi\in\Sym(N)}\Pi\sigma\Pi^\dagger$$
	We then verify that $\sigma'$ is permutation-invariant.
	Fix $\tilde{\Pi}\in\Sym(N)$, then
	$$\tilde{\Pi}\sigma'\tilde{\Pi}^\dagger
	=\frac{1}{\abs{\Sym(N)}}\sum_{\Pi\in\Sym(N)}\tilde{\Pi}\Pi\sigma\Pi^\dagger\tilde{\Pi}^\dagger
	=\frac{1}{\abs{\Sym(N)}}\sum_{\hat{\Pi}\in\Sym(N)}\hat{\Pi}\sigma\hat{\Pi}^\dagger
	=\sigma'$$
	where the second equality is by relabeling $\tilde{\Pi}\Pi=\hat{\Pi}$, which is allowed since $\Sym(N)$ is a group.

	Now we apply \cref{deFinetti} to approximate $\sigma'$ with a classical mix over tensors of independent states.
	That is, $\exists\rho=\sum_i w_i\rho_i^{\otimes n+1}$ such that:
	$$\max_{\Lambda_i}\norm{\Lambda_1\otimes\ldots\otimes\Lambda_{n+1}(\sigma'-\rho)}=O(\varepsilon)$$

	Let $p_i$ be the accept probability of at least $(\frac{1}{2}-\varepsilon)n$ copies from $\rho_i^{\otimes n}$ accept under $\cA_{\GS}$.
	We now partition the terms $\rho_i$ into two categories according to the Chernoff bound in \cref{QPIP1Chernoff1}:
	$$I=\set{i:p_i\geq 2^{-\lambda}}$$

	For $i\notin I$ we have
	$$\sum_{i\in I} w_i p_i<2^{-\lambda}=\negl(\lambda)$$
	so we can approximate with negligible error that all $\rho_i$ where $i\notin I$ are rejected.

	For $i\in I$ we have
	$$P[\cA_\GS(\rho_i)=acc]>\frac{1}{2}-2\varepsilon$$
	and hence
	$$\braket{\rho_i|H_{C'(x)}|\rho_i}<\frac{1}{T^\lambda}$$
	by \cref{QPIP1eps}.
	We also know that the least nonzero eigenvalue of $H_{C'(x)}$ is lower-bounded by $\frac{3}{4}$, so we obtain... \Ethan{TODO not done yet. Probably some squares or square roots missing here.}
	$$\braket{\rho_i|\phi_{C'(x)}}>1-O(T^{-\lambda})$$
	\Ethan{Now we need some kinda standard fidelity argument to show that the measurement results will be close to as if the ground state is measured.}

	The probability of measuring $t<T$ on the clock register of $\ket{\psi_{C'(x)}}$ is $\frac{T}{T+T^{\lambda+1}}<T^{-\lambda}$,
	so the data register has $1-\varepsilon$ probability to be $C(x)$ at this point.

	The soundness errors incurred at each step is at most $O(T^{-\lambda})$, so the conclusion follows.
\end{prf}






\section{Delegation Protocol for Hybrid Client}

In this section, we construct a one-message $\QPIP_1$ delegation protocol for $\SampBQP$.
At a high level, it is a cut-and-choose protocol.
The server constructs multiple copies of the ground state as certificates,
then the client randomly chooses a copy to output and checks the rest.
Unfortunately our approach incurs inverse polynomial soundness errors.

There are two main challenges to this cut-and-choose approach.
First, the client needs to reliably extract the circuit output from its corresponding ground state.
We accomplish this by padding the circuit with identity gates at the end.
By doing so, the clock register collapses to after the last non-identity gate with high probability.

The other challenge is that a cheating prover's certificates can be arbitrarily entangled,
so common techniques such as Chernoff bounds can't be applied as-is.
This wasn't a challenge for $\Piblind$ for $\BQP$ earlier because...
\Ethan{TODO; need brief and intuitive explanation}
In our case with $\SampBQP$, we overcome this challenge instead by using de Finetti's theorem
to approximate our measurement results with that of some unentangled copies,
within inverse polynomial errors.

\def\GS{\mathsf{GS}}
\nc{\PiGS}{\ensuremath{\Pi_\GS}}
\nc{\VGS}{\ensuremath{V_\GS}}
\nc{\PGS}{\ensuremath{P_\GS}}
\nc{\PGSstar}{\ensuremath{P_\GS^*}}
\nc{\cVGS}[1]{\ensuremath{\cV_{\GS,#1}}}
\nc{\cPGS}[1]{\ensuremath{\cP_{\GS,#1}}}

\def\Samp{\mathsf{Samp}}
\nc{\PiSamp}{\ensuremath{\Pi_\Samp}}
\nc{\VSamp}{\ensuremath{V_\Samp}}
\nc{\PSamp}{\ensuremath{P_\Samp}}
\nc{\PSampstar}{\ensuremath{P_\Samp^*}}
\nc{\cVSamp}[1]{\ensuremath{\cV_{\Samp,#1}}}
\nc{\cPSamp}[1]{\ensuremath{\cP_{\Samp,#1}}}

\def\GS{\mathsf{GS}}

Now we present our $\QPIP_1$ protocol. It is parametrized by $\lambda$, the degree of its inverse polynomial soundness.

\begin{protocol}{$\QPIP_1$ protocol $\PiSamp$ for $\SampBQP$}\label{ProtoQPIP1}
	Let $C'$ be $C$ padded with $T^{\lambda + 1}$ identity gates at the end.

	Pick $\varepsilon$ small enough so it holds for all $\ket\phi$ that:
	\begin{equation}
		\label{QPIP1eps}
		P[\cA_\GS(\ket{\phi})=acc]>\frac{1}{2}-2\varepsilon\Rightarrow\braket{\phi|H_{C'(x)}|\phi}<\frac{1}{T^\lambda}
	\end{equation}

	Pick $n$ large enough so by Chernoff bound:
	\begin{equation}
		\label{QPIP1Chernoff1}
		P\left[Bin(n, \frac{1}{2}-2\varepsilon)\geq\left(\frac{1}{2}-\varepsilon\right)n\right]<2^{-\lambda}
	\end{equation}
	\begin{equation}
		\label{QPIP1Chernoff2}
		P\left[Bin(n, \frac{1}{2})\leq\left(\frac{1}{2}-\varepsilon\right)n\right]<2^{-\lambda}
	\end{equation}

	Let $N$ be large enough so that de Finetti's theorem can be applied at the cost of $T^{-\lambda}$ errors if one randomly picks $n+1$ subsystems out of $N$ permutation-invariant subsystems.
		That is,
		$$\sqrt{\frac{2(n+1)^2\ln\abs{A}}{N-(n+1)}}<T^{-\lambda}$$
		where $\abs{A}$ is the number of qubits in $\ket{\psi_{C'(x)}}$

	\begin{enumerate}
		\item The verifier privately samples $I\subset[N]$ s.t. $\abs{I}=n$.
			It then privately samples $k\xleftarrow{\$}[N]\setminus I$.
		\item The honest prover prepares $N$ copies of $\ket{\psi_{C'(x)}}$ and sends all of them to the verifier qubit-by-qubit.
		\item For $i$ from $1$ to $N$, the verifier chooses what to do to the $i$-th copy, $\rho_i$, as follows:
		\begin{enumerate}
			\item If $i\in I$, run $\cA_{\GS}(\rho_i)$.
			\item Otherwise, if $i=k$, measure the data register of $\rho_i$ and save the outcome as $y$.
			\item Otherwise, discard $\rho_i$.
		\end{enumerate}
		\item If the proportion of copies accepted by $\cA_{\GS}$ is greater than $\frac{1}{2}-\varepsilon$ then the verifier accepts and outputs $y$. Otherwise, it rejects.
	\end{enumerate}
\end{protocol}

Note that $\VSamp$ only needs to apply $X$ and $Z$ measurements, and is classical otherwise. We now show the completeness and soundness of $\PiSamp$.

\begin{thm}
    \label{QPIP1thm}
	$\PiSamp$ has negligible completeness error and $O(T^-\lambda)$ soundness error.
\end{thm}
\begin{prf}
	The completeness follows from \cref{QPIP1Chernoff2} and inspection.

	Now we show soundness.
	Suppose $\PSampstar$ is a cheating prover that sends some $\sigma$ to the verifier.

	We first show that de Finetti's theorem can indeed be used here to achieve independence.
	Randomly picking $n+1$ out of $N$ registers is equivalent to first applying a random permutation then taking the first $n+1$ registers.
	A random permutation, in turn, is a classical mix over all possible permutations:
	$$\sigma'=\frac{1}{\abs{\Sym(N)}}\sum_{\Pi\in\Sym(N)}\Pi\sigma\Pi^\dagger$$
	We then verify that $\sigma'$ is permutation-invariant.
	Fix $\tilde{\Pi}\in\Sym(N)$, then
	$$\tilde{\Pi}\sigma'\tilde{\Pi}^\dagger
	=\frac{1}{\abs{\Sym(N)}}\sum_{\Pi\in\Sym(N)}\tilde{\Pi}\Pi\sigma\Pi^\dagger\tilde{\Pi}^\dagger
	=\frac{1}{\abs{\Sym(N)}}\sum_{\hat{\Pi}\in\Sym(N)}\hat{\Pi}\sigma\hat{\Pi}^\dagger
	=\sigma'$$
	where the second equality is by relabeling $\tilde{\Pi}\Pi=\hat{\Pi}$, which is allowed since $\Sym(N)$ is a group.

	Now we apply \cref{deFinetti} to approximate $\sigma'$ with a classical mix over tensors of independent states.
	That is, $\exists\rho=\sum_i w_i\rho_i^{\otimes n+1}$ such that:
	$$\max_{\Lambda_i}\norm{\Lambda_1\otimes\ldots\otimes\Lambda_{n+1}(\sigma'-\rho)}=O(\varepsilon)$$

	Let $p_i$ be the accept probability of at least $(\frac{1}{2}-\varepsilon)n$ copies from $\rho_i^{\otimes n}$ accept under $\cA_{\GS}$.
	We now partition the terms $\rho_i$ into two categories according to the Chernoff bound in \cref{QPIP1Chernoff1}:
	$$I=\set{i:p_i\geq 2^{-\lambda}}$$

	For $i\notin I$ we have
	$$\sum_{i\in I} w_i p_i<2^{-\lambda}=\negl(\lambda)$$
	so we can approximate with negligible error that all $\rho_i$ where $i\notin I$ are rejected.

	For $i\in I$ we have
	$$P[\cA_\GS(\rho_i)=acc]>\frac{1}{2}-2\varepsilon$$
	and hence
	$$\braket{\rho_i|H_{C'(x)}|\rho_i}<\frac{1}{T^\lambda}$$
	by \cref{QPIP1eps}.
	We also know that the least nonzero eigenvalue of $H_{C'(x)}$ is lower-bounded by $\frac{3}{4}$, so we obtain... \Ethan{TODO not done yet. Probably some squares or square roots missing here.}
	$$\braket{\rho_i|\phi_{C'(x)}}>1-O(T^{-\lambda})$$
	\Ethan{Now we need some kinda standard fidelity argument to show that the measurement results will be close to as if the ground state is measured.}

	The probability of measuring $t<T$ on the clock register of $\ket{\psi_{C'(x)}}$ is $\frac{T}{T+T^{\lambda+1}}<T^{-\lambda}$,
	so the data register has $1-\varepsilon$ probability to be $C(x)$ at this point.

	The soundness errors incurred at each step is at most $O(T^{-\lambda})$, so the conclusion follows.
\end{prf}



\section{$\SampBQP$ Delegation Protocol for Fully Classical Client}
\label{sec:qpip0_all}

In this section, we create a delegation protocol for $\SampBQP$ with fully classical clients by adapting the approach taken in \cite{FOCS:Mahadev18a}. In \cite{FOCS:Mahadev18a}, 
Mahadev designed a protocol $\PiMeasure$ (\Cref{proto:urmila4}) that allows a $\BQP$ prover to ``commit a state" for a classical verifier to choose a $X$ or $Z$ measurement and obtain corresponding measurement results.
Composing it with the $\QPIP_1$ protocol for $\BQP$ from \cite{mf16} results in a $\QPIP_0$ protocol for $\BQP$.
In this work, we will compose $\PiMeasure$ with our $\QPIP_1$ protocol $\PiSamp$ (\Cref{ProtoQPIP1}) for $\SampBQP$ in order to obtain a $\QPIP_0$ protocol for $\SampBQP$. 

A direct composition of $\PiSamp$ and $\PiMeasure$, however, results in $\PiNaive$ (\Cref{proto:qpip0_naive}) which does not provide reasonable completeness or accuracy guarantees.
As we will see, this is due to $\PiMeasure$ itself having peculiar and weak guarantees:
the client doesn't always obtain measurement outcomes even if the server were honest.
When that happens under the $\BQP$ context, the verifier can simply accept the prover at the cost of some soundness error;
under our $\SampBQP$ context, however, we must run many copies of $\PiNaive$ in parallel so the verifier can generate its outputs from some copy.
We will spend the majority of this section analyzing the soundness of this parallel repetition.

\subsection{Mahadev's measurement protocol}\label{sec:urmila4}

$\PiMeasure$ is a 4-round protocol between a verifier (which corresponds to our client) and a prover (which corresponds to our server).
The verifier (secretly) chooses a string $h$ specifying the measurements he wants to make, and generates keys $pk, sk$ from $h$. It sends $pk$ to the prover. The prover ``commits" to a state $\rho$ of its choice using $pk$ and replies with its commitment $y$.
The verifier must then choose between two options: do a \emph{testing round} or a \emph{Hadamard round}.
In a testing round the verifier can catch cheating provers,
and in a Hadamard round the verifier receives some measurement outcome.
He sends his choice to the prover, and the prover replies accordingly. If the verifier chose testing round, he checks the prover's reply against the previous commitment, and rejects if he sees an inconsistency. If the verifier chose Hadamard round, he calculates $M_{XZ}(\rho, h)$ based on the reply.
We now formally describe the interface of $\PiMeasure$ while omitting the implementation details.

\begin{protocol}{Mahadev's measurement protocol $\PiMeasure=(\PMeasure, \VMeasure)$}
	\label{proto:urmila4}

	Inputs:
	\begin{itemize}
		\item Common input: Security parameter $1^\lambda$ where $\lambda\in\bbN$.
		\item Prover's input: a state $\rho\in\cB^{\otimes n}$ for the verifier to measure.
		\item Verifier's input: the measurement basis choice $h \in \zo^n$
	\end{itemize}

	Protocol:
	\begin{enumerate}
		\item \label{step:measure1} The verifier generates a public and secret key pair $(pk, sk)\leftarrow\cVMeasure{1}($ $1^\lambda, h)$. It sends $pk$ to the prover.
		\item \label{step:measure2} The prover generates $(y, \sigma)\leftarrow\cPMeasure{2}(pk, \rho)$.
			$y$ is a classical ``commitment", and $\sigma$ is some internal state.
			He sends $y$ to the verifier.
		\item \label{step:measure3} The verifier samples $c\xleftarrow{\$}\zo$ uniformly at random and sends it to the prover. $c=0$ indicates a \emph{testing round}, while $c=1$ indicates a \emph{Hadamard round}.
		\item \label{step:measure4} The prover generates a classical string $a\leftarrow\cPMeasure{4}(pk, c, \sigma)$ and sends it back to the verifier.
		\item \label{step:output} If it is a testing round ($c=0$), then the verifier generates and outputs $o\leftarrow\cVMeasure{T}(pk, y, a)$ where $o\in\set{\Acc, \Rej}$.
			If it is a Hadamard round ($c=1$), then the verifier generates and outputs $v\leftarrow\cVMeasure{H}($ $sk, h, y, a)$.
	\end{enumerate}
\end{protocol}

$\PiMeasure$ has negligible completeness errors, i.e. if both the prover and verifier are honest, the verifier accepts with overwhelming probability and his output on Hadamard round is computationally indistinguishable from $M_{XZ}(\rho, h)$. As for soundness,
it gives the following \emph{binding property} against cheating provers:
if a prover would always succeed on the testing round, then there exists some $\rho$ so that for any $h$ the verifier obtains $M_{XZ}(\rho, h)$ if he had chosen the Hadamard round.

\begin{lemma}[binding property of $\PiMeasure$; special case of Claim 7.1 in \cite{FOCS:Mahadev18a}]
	\label{lem:urmila-binding}
	Let $\PMeasureStar$ be a $\BQP$ cheating  prover for $\PiMeasure$ and $\lambda$ be the security parameter. Let $1-p_{h,T}$ be the  probability that the verifier accepts $\PMeasureStar$ in the testing round on basis choice $h$.\footnote{Compared to Claim 7.1 of \cite{FOCS:Mahadev18a}, we don't have a $p_{h,H}$ term here. This is because on rejecting a Hadamard round, the verifier can output a uniformly random string, and that is same as the result of measuring $h$ on the totally mixed state.} Under the QLWE assumption, there exists some $\rho^*$ so that for all verifier's input $h \in \zo^n$, the verifier's outputs on the Hadamard round is $\sqrt{p_{h,T}}+\negl(n)$-computationally indistinguishable from $M_{XZ}(\rho^*, h)$.
\end{lemma}

We now combine $\PiMeasure$ with our $\QPIP_1$ Protocol for $\SampBQP$, $\PiSamp=(\PSamp, \VSamp)$ (\Cref{ProtoQPIP1}), to get a corresponding $\QPIP_0$ protocol $\PiNaive$.
Recall that in $\PiSamp$ the verifier takes $X$ and $Z$ measurements on the prover's message.
In $\PiNaive$ we let the verifier use $\PiMeasure$ to learn those measurement outcomes instead.

\begin{protocol}{Intermediate $\QPIP_0$ protocol $\PiNaive$ for the $\SampBQP$ problem $(D_x)_{x\in\set{0, 1}^*}$}
	\label{proto:qpip0_naive}

	Inputs:
	\begin{itemize}
		\item Security parameter $1^\lambda$ where $\lambda\in\bbN$
		\item Error parameter $\eps\in(0, 1)$
		\item Classical input $x\in\zo^n$ to the $\SampBQP$ instance
	\end{itemize}

	Protocol:
	\begin{enumerate}
		\item \label{step:naive1} The verifier chooses a $XZ$-measurement $h$ from the distribution specified in \stepref{qpip1-verify} of $\PiSamp$.
		\item \label{step:naive2} The prover prepares $\rho$ by running \stepref{qpip1-state-gen} of $\PiSamp$.
		\item \label{step:urmila-in-naive}
			The verifier and prover run $(\PMeasure(\rho), \VMeasure(h))(1^\lambda)$.
			\begin{enumerate}
				\item The verifier samples $(pk, sk)\leftarrow\cVNaive{1}(1^\lambda, h)$ and sends $pk$ to the prover, where $\cVNaive{1}$ is the same as $\cVMeasure{1}$ of \Cref{proto:urmila4}. 
				\item The prover runs $(y, \sigma)\leftarrow\cPNaive{2}(pk, \rho)$ and sends $y$ to the verifier, where $\cPNaive{2}$ is the same as $\cPMeasure{2}$.
					Here we allow the prover to abort by sending $y=\bot$, which does not benefit cheating provers but simplifies our analysis of parallel repetition later.
				\item\label{step:c-urmila-in-naive} The verifier samples $c\xleftarrow{\$}\zo$ and sends it to the prover.
				\item The prover replies $a\leftarrow\cPNaive{4}(pk, c, \sigma)$.
				\item
					If it is a testing round, the verifier accepts or rejects based on the outcome of $\PiMeasure$.
					If it is a Hadamard round, the verifier obtains $v$.
			\end{enumerate}
		\item \label{step:naive-output} If it's a Hadamard round, the verifier finishes the verification step of Protocol~\ref{ProtoQPIP1} by generating and outputting $(d, z)$
	\end{enumerate}
\end{protocol}

There are several problems with using $\PiNaive$ as a $\SampBQP$ protocol. First, since the verifier doesn't get a sample if he had chosen the testing round in Step~\ref{step:c-urmila-in-naive}, the protocol has completeness error at least $1/2$. Moreover, since $\PiMeasure$ does not check anything on the Hadamard round, a cheating prover can give up passing the testing round and breaks the commitment on the Hadamard round, with only a constant $1/2$ probability of being caught.
However, we can show that $\PiNaive$ has a binding property similar to $\PiMeasure$:
if a cheating prover $\PNaiveStar$ passes the testing round with overwhelming probability whenever it doesn't abort on the second message,
then the corresponding output $(d, z)\leftarrow(\PNaiveStar, \VNaive)$ is close to $(d, z_{ideal})$.
Recall the ideal output is
$$\begin{cases}
	z_{ideal}=\bot & \text{if } d=\Rej\\
	z_{ideal}\leftarrow D_x & \text{if } d=\Acc.
\end{cases}$$
This binding property is formalized in \Cref{lem:naive-qpip0-binding}.
Intuitively,  the proof of \Cref{lem:naive-qpip0-binding}  combines the binding property of \Cref{proto:qpip0_naive} (\Cref{lem:urmila-binding}) and $\PiSamp$'s soundness (\Cref{QPIP1thm}). There is a technical issue that \Cref{proto:qpip0_naive} allows the prover to abort while \Cref{proto:urmila4} does not. This issue is solved by constructing another $\BQP$ prover $\Pstar$ for every cheating prover $\PNaiveStar$. 
Specifically, $\Pstar$ uses $\PNaiveStar$'s strategy when it doesn't abort, otherwise honestly chooses the totally mixed state for the verifier to measure.

\begin{theorem}[binding property of $\PiNaive$]
	\label{lem:naive-qpip0-binding}
	Let $\PNaiveStar$ be a cheating $\BQP$ prover for $\PiNaive$ and $\lambda$ be the security parameter.
	Suppose that $\Prob{d=\Acc\mid y\ne\bot, c=0}$ is overwhelming, 
	under the QLWE assumption, then the verifier's output in the Hadamard round is $O(\eps)$-computationally indistinguishable from $(d, z_{ideal})$.
\end{theorem}
\begin{proof}[\Cref{lem:naive-qpip0-binding}]
	We first introduce the \emph{dummy strategy} for $\PiMeasure$, where the prover chooses $\rho$ as the maximally mixed state and executes the rest of the protocol honestly.
	It is straightforward to verify that this prover would be accepted in the testing round with probability $1-\negl(\lambda)$,
	but has negligible probability passing the verification  after the Hadamard round.

	Now we construct a cheating $\BQP$ prover for \Cref{proto:qpip0_naive}, $\Pstar$, that does the same thing as $\PNaiveStar$ except at Step~\ref{step:urmila-in-naive}, where the prover and verifier runs \Cref{proto:urmila4}. $\Pstar$ does the following in Step~\ref{step:urmila-in-naive}:
	for the second message, run $(y, \sigma)\leftarrow\cPNaiveStar{2}(pk, \rho)$.
	If $y\ne\bot$, then reply $y$;
	else, run the corresponding step of the dummy strategy and reply with its results.
	For the fourth message, if $y\ne\bot$, run and reply with $a\leftarrow\cPNaiveStar{4}(pk, c, \sigma)$;
	else, continue the dummy strategy.

	 In the following we fix an $x$. Let the distribution on $h$ specified in Step~\ref{step:naive1} of the protocol be $p_x(h)$. Define $\Pstarsub(x)$ as $\Pstar$'s response in Step~\ref{step:urmila-in-naive}. Note that we can view $\Pstarsub(x)$ as a prover strategy for \Cref{proto:urmila4}. By construction $\Pstarsub(x)$ passes testing round with overwhelming probability over $p_x(h)$, i.e. $\sum_h p_x(h) p_{h,T} =\negl(\lambda)$, where $p_{h,T}$ is $\Pstar$'s probability of getting accepted by the prover on the testing round on basis choice $h$. By \Cref{lem:urmila-binding} and Cauchy's inequality, there exists some $\rho$ such that  $\sum_h p_x(h) \norm{v_h -M_{XZ}(\rho, h)}_c = \negl(\lambda)$, where we use $\norm{A-B}_c=\alpha$ to denote that $A$ is $\alpha$-computational indistinguishable to $B$. Therefore $v= \sum_h p_x(h) v_h$ is computationally indistinguishable to $\sum_h p_x(h) M_{XZ}(\rho, h)$. Combining it with $\PiSamp$'s soundness (\Cref{QPIP1thm}), 
	we see that $(d', z')\leftarrow(\Pstar, \VNaive)(1^\lambda, 1^{1/\epsilon}, x)$  is $\eps$-computationally indistinguishable to $(d', z_{ideal}')$.

	Now we relate $(d', z')$ back to $(d, z)$.
	First, conditioned on that $\PNaiveStar$ aborts, since dummy strategy will be rejected with overwhelming probability in Hadamard round,
	we have $(d', z')$ is computationally indistinguishable to $(\Rej, \bot)=(d, z)$.
	On the other hand, conditioned on $\PNaiveStar$ not aborting, clearly $(d, z)=(d', z')$.
	So $(d, z)$ is computationally indistinguishable to $(d', z')$,
	which in turn is $O(\eps)$-computationally indistinguishable to $(d', z_{ideal}')$.
	Since $\norm{d-d'}_{tr}= O(\eps)$,
	 $(d, z_{ideal})$ is $O(\eps)$-computationally indistinguishable to $(d', z_{ideal}')$.
	Combining everything, we conclude that $(d, z)$ is $O(\eps)$-computationally indistinguishable to $(d, z_{ideal})$.
\end{proof}

\subsection{$\QPIP_0$ protocol for $\SampBQP$} \label{sec:qpip0}

We now introduce our $\QPIP_0$ protocol $\PiSampZ$ for $\SampBQP$.
It is essentially a $m$-fold parallel repetition of $\PiNaive$,
from which we uniformly randomly pick one copy to run Hadamard round to get our samples and run testing round on all other $m-1$ copies.
Intuitively, if the server wants to cheat by sending something not binding on some copy,
he will be caught when that copy is a testing round, which is with probability $1-1/m$.
This over-simplified analysis does not take into account that the server might create entanglement between the copies. Therefore, a more technically involved analysis is required.

In the description of our protocol below, we describe $\PiNaive$ and $\PiMeasure$ in details in order to introduce notations that we need in our analysis.

\begin{protocol}{$\QPIP_0$ protocol $\PiSampZ$ for the $\SampBQP$ problem $(D_x)_{x\in\set{0, 1}^*}$}
	\label{proto:QPIP0samp}

	Inputs:
	\begin{itemize}
		\item Security parameter $1^\lambda$ for $\lambda\in\bbN$.
		\item Accuracy parameter $1^{1/\eps}$ for the $\SampBQP$ problem.
		\item Input $x\in\zo^{\poly(\lambda)}$ for the $\SampBQP$ instance.
	\end{itemize}

	Ingredient: Let $m=O(1/\eps^2)$ be the number of parallel repetitions to run.
\bigskip 

	Protocol:
	\begin{enumerate}
		\item  The verifier  generates $m$ independent copies of basis choices $\vec{h}=(h_1,\ldots,h_m)$, where each copy is generated as in \stepref{naive1} of $\PiNaive$.
		\item The prover prepares $\rho^{\otimes m}$; each copy of $\rho$ is prepared as in \stepref{naive2} of $\PiNaive$.
		\item \label{step:urmila-in-qpip0-1} The verifier generates $m$ key pairs for $\PiMeasure$, $\vec{pk}=(pk_1,\ldots,pk_m)$ and $\vec{sk}=(sk_1,\ldots,sk_m)$, as in \stepref{measure1} of $\PiMeasure$.
			It sends $\vec{pk}$ to the prover.
		\item \label{step:urmila-in-qpip0-2}The prover generates $\vec{y}=(y_1,\ldots,y_m)$ and $\sigma$ as in \stepref{measure2} of $\PiMeasure$.
			It sends $\vec{y}$ to the verifier.
		\item \label{step:urmila-in-qpip0-3}The verifier samples $r\xleftarrow{\$}[m]$ which is the copy to run Hadamard round for.
			For $1\leq i\leq m$, if $i\ne r$ then set $c_i\leftarrow 0$, else set $c_i\leftarrow 1$.
			It sends $\vec{c}=(c_1,\ldots,c_m)$ to the prover.
		\item \label{step:urmila-in-qpip0-4}The prover generates $\vec{a}$ as in \stepref{measure4} of $\PiMeasure$, and sends it back to the verifier.
		\item The verifier computes the outcome for each round as in \stepref{naive-output} of $\PiNaive$.
			If any of the testing round copies are rejected, the verifier outputs $(\Rej, \bot)$.
			Else, it outputs the result from the Hadamard round copy.
	\end{enumerate}
\end{protocol}
By inspection, $\PiSampZ$ is a $\QPIP_0$ protocol for $\SampBQP$ with negligible completeness error.
To show that it is computationally sound, we first use the partition lemma from \cite{arXiv:ChiaChungYam19}.

Intuitively, the partition lemma says that for any cheating prover and for each copy $i\in[m]$, there exist two efficient ``projectors" \footnote{Actually they are not projectors, but for the simplicity of this discussion let's assume they are.} $G_{0,i}$ and $G_{1,i}$ in the prover's internal space with $G_{0,i}+G_{1,i} \approx Id$. $G_{0,i}$ and $G_{1,i}$ splits up the prover's residual internal state after sending back his first message.
$G_{0,i}$ intuitively represents the subspace where the prover does not knows the answer to the testing round on the $i$-th copy, while $G_{1,i}$ represents the subspace where the prover does. Note that the prover is using a single internal space for all copies, and every $G_{0,i}$ and every $G_{1,i}$ is acting on this single internal space. 
By using this partition lemma iteratively, we can decompose the prover's internal state $\ket{\psi}$ into sum of subnormalized states.
First we apply it to the first copy, writing $\ket{\psi}=G_{0,1}\ket{\psi}+G_{1,1}\ket{\psi} \equiv \ket{\psi_0}+\ket{\psi_1}$.
The component $\ket{\psi_0}$ would then get rejected as long as the first copy is chosen as a testing round,
which occurs with pretty high probability.
More precisely, the output corresponding to $\ket{\psi_0}$ is $1/m$-close to the ideal distribution that just rejects all the time.
On the other hand, $\ket{\psi_1}$ is now binding on the first copy;
we now similarly apply the partition lemma of the second copy to $\ket{\psi_1}$.
We write $\ket{\psi_1}=G_{0,2}\ket{\psi_1}+G_{1,2}\ket{\psi_1}\equiv \ket{\psi_{10}}+\ket{\psi_{11}}$, and apply the same argument about $\ket{\psi_{10}}$ and $\ket{\psi_{11}}$.
We then continue to decompose $\ket{\psi_{11}}=\ket{\psi_{110}}+\ket{\psi_{111}}$ and so on, until we reach the last copy and obtain $\ket{\psi_{1^m}}$.
Intuitively, the $\ket{\psi_{1^m}}$ term represents the ``good" component where the prover knows the answer to every testing round and therefore has high accept probability. Therefore, $\ket{\psi_{1^m}}$ also satisfies some binding property,
so the verifier should obtain a measurement result of some state on the Hadamard round copy,
and the analysis from the $\QPIP_1$ protocol $\PiSamp$ follows.

However, the intuition that $\ket{\psi_{1^m}}$ is binding to every Hadamard round is incorrect. As $G_{1,i}$ does not commute with $G_{1,j}$, $\ket{\psi_{1^m}}$ is unfortunately only binding for the $m$-th copy.
To solve this problem, we start with a pointwise argument and fix the Hadamard round on the $i$-th copy where $\ket{\psi_{1^i}}$ is binding,
and show that the corresponding output is $O(\norm{\ket{\psi_{1^{i-1}0}}})$-close to ideal.
We can later average out this error over the different choices of $i$, since not all $\norm{\ket{\psi_{1^{i-1}0}}}$ can be large at the same time. Another way to see this issue is to notice that we are partitioning a quantum state, not probability events, so there are some inconsistencies between our intuition and calculation. Indeed, the error we get in the end is $O(\sqrt{1/m})$ instead of the $O(1/m)$ we expected. 



Also a careful reader might have noticed that the prover's space don't always decompose cleanly into parts that the verifier either rejects or accepts with high probability, as there might be some states that is accepted with mediocre probability. As in \cite{arXiv:ChiaChungYam19}, we solve this by splitting the space into parts that are accepted with probability higher or lower than a small threshold $\gamma$ and applying Marriott-Watrous~\cite{marriott2005quantum} amplification to boost the accept probability if it is bigger than $\gamma$, getting a corresponding amplified prover action $\ext$. However, states with accept probability really close to the threshold $\gamma$ can not be classified, so we average over randomly chosen $\gamma$ to have $G_{0,i}+G_{1,i} \approx Id$. Now we give a formal description of the partition lemma.

\begin{lemma}[partition lemma; revision of Lemma 3.5 of \cite{arXiv:ChiaChungYam19}\footnote{$G_{0}$ and $G_{1}$ of this version are created from doing $G$ of \cite{arXiv:ChiaChungYam19} and post-selecting on the $ph,th,in$ register being $0^t01$ or $0^t11$ then discard $ph,th,in$. Property~\ref{property:partition-err} corresponds to Property~1. Property~\ref{property:partition-testing} corresponds to Property~4, with $2^{m-1}$ changes to $m-1$ because we only have $m$ possible choices of $\vec{c}$. Property~\ref{property:partition-binding} corresponds to Property~5. Property~\ref{property-partition-norm-sum} comes from the fact that $G_0$ and $G_1$ are post-selections of orthogonal results of the same $G$.}]\label{lem:partition2}
	Let $\lambda$ be the security parameter, and $\gamma_0 \in[0,1]$ and $T\in \mathbb{N}$ be parameters that will be related to the randomly-chosen threshold $\gamma$.
	Let $(U_0,U)$ be a prover's strategy in a $m$-fold parallel repetition of $\PiMeasure$\footnote{A $m$-fold parallel repetition of $\PiMeasure$ is running step~\ref{step:urmila-in-qpip0-1}~\ref{step:urmila-in-qpip0-2}~\ref{step:urmila-in-qpip0-3}~\ref{step:urmila-in-qpip0-4} of \Cref{proto:QPIP0samp} with verifier input $\vec{h}$ and prover input $\rho^{\otimes n}$, followed by an output step where the verifier rejects if any of the $m-1$ testing round copies is rejected, otherwise outputs the result of the Hadamard round copy.}, where $U_0$ is how the prover generates $\vec{y}$ on the second message, and $U$ is how the prover generates $\vec{a}$ on the fourth message. Let $H_{\regX,\regZ}$ be the Hilbert space of the prover's internal calculation.
	Denote the string $0^{i-1}10^{m-i} \in \zo^m $ as $e_i$, which corresponds to doing Hadamard round on the $i$-th copy and testing round on all others.

	For all $i\in[m]$, $\gamma \in \L\{\frac{\gamma_0}{T},\frac{2\gamma_0}{T},\dots,\frac{T\gamma_0}{T}\R\}$, there exist two $\poly(1/\gamma_0,T,\lambda)$-time quantum circuit with post selection\footnote{A quantum circuit with post selection is composed of unitary gates followed by a post selection on some measurement outcome on ancilla qubits, so it produces a subnormalized state, where the amplitude square of the output state is the probability of post selection.} $G_{0,i,\gamma}$ and $G_{1,i,\gamma}$ such that for all (possibly sub-normalized)  quantum states $\ket{\psi}_{\regX,\regZ}\in  H_{\regX,\regZ}$,  properties \ref{property:partition-err}~\ref{property:partition-testing}~\ref{property:partition-binding}~\ref{property-partition-norm-sum}, to be described later, are satisfied. Before we describe the properties, we introduce the following notations:  

	\begin{align}
	\label{eq:psi0}
	\ket{\psi_{0,i,\gamma}}_{\regX,\regZ}
	\defeq&
	G_{0,i,\gamma}\ket{\psi}_{\regX,\regZ}, \\ 
	\label{eq:psi1}	
	\ket{\psi_{1,i,\gamma}}_{\regX,\regZ}
	 \defeq&
	G_{1,i,\gamma}\ket{\psi}_{\regX,\regZ}
	,  \\
	\label{eq:psierr}
	\ket{\psi_{err,i,\gamma}}_{\regX,\regZ}
	\defeq&
	\ket{\psi}_{\regX,\regZ} -\ket{\psi_{0,i,\gamma}}_{\regX,\regZ}- \ket{\psi_{1,i,\gamma}}_{\regX,\regZ}
	.
	\end{align}

	Note that $G_{0,i,\gamma}$ and $G_{1,i,\gamma}$ has failure probabilities, and this is reflected by the fact that $\ket{\psi_{0,i,\gamma}}_{\regX,\regZ}$ and $\ket{\psi_{1,i,\gamma}}_{\regX,\regZ}$ are  sub-normalized. $G_{0,i,\gamma}$ and $G_{1,i,\gamma}$ depend on $(U_0,U)$ and $\vec{pk},\vec{y}$.

	The following properties are satisfied for all $i\in[m]$:
	\begin{enumerate}
		\item \label{property:partition-err}  $$\E_{\gamma}\|\ket{\psi_{err,i,\gamma}}_{\regX,\regZ}\|^2 \leq \frac{6}{T}+\negl(\lambda),$$

			where the averaged is over uniformly sampled $\gamma$. This also implies
			\begin{align}
				\E_{\gamma}\|\ket{\psi_{err,i,\gamma}}_{\regX,\regZ}\| \leq \sqrt{\frac{6}{T}}+\negl(\lambda)
			\end{align}
			by Cauchy's inequality.

		\item \label{property:partition-testing}
			For all $\vec{pk}$, $\vec{y}$, $\gamma$, and  $j\neq i$, we have
			\begin{align}
				\norm{ P_{acc, i} \circ U\frac{\ket{e_j}_{\regC}\ket{\psi_{0,i,\gamma}}_{\regX,\regZ}}{\|\ket{\psi_{0,i,\gamma}}_{\regX,\regZ}\|}}^2 \leq (m-1)\gamma_0+\negl(\lambda),
			\end{align}
			where $P_{acc, i}$ are projector to the states that $i$-th testing round accepts with $pk_i,y_i$, including the last measurement the prover did before sending $\vec{a}$.  This means that $\ket{\psi_{0,i,\gamma}}$ is rejected by the $i$-th testing round with high probability.
		\item \label{property:partition-binding}
			For all $\vec{pk}$, $\vec{y}$, $\gamma$, and $j\neq i$, there exists an efficient quantum algorithm $\ext_i$ such that
			\begin{align}
				\norm{P_{acc, i} \circ \ext_i\left(\frac{\ket{e_j}_{\regC}\ket{\psi_{1,i,\gamma}}_{\regX,\regZ}}{\|\ket{\psi_{1,i,\gamma}}_{\regX,\regZ}\|}\right)}^2 =1-\negl(\lambda).
			\end{align}

			This will imply that $\ket{\psi_{1,i,\gamma}}$ is binding to the $i$-th Hadamard round.

		\item \label{property-partition-norm-sum}
			For all $\gamma$,
			\begin{align}
				\norm{\ket{\psi_{0,i,\gamma}}_{\regX,\regZ}}^2+ \norm{\ket{\psi_{1,i,\gamma}}_{\regX,\regZ}}^2 \leq  \norm{\ket{\psi}_{\regX,\regZ}}^2.
			\end{align}
	\end{enumerate}
\end{lemma}

Note that in property~\ref{property:partition-binding}, we are using $\ext_i$ instead of $U$ because we use amplitude amplification to boost the success probability. 

We now decompose the prover's internal state by using \Cref{lem:partition2} iteratively.
Let $\ket{\psi}$ be the state the prover holds before he receives $\vec{c}$;
we denote the corresponding Hilbert space as $H_{\regX,\regZ}$.
For all $k \in [m]$, $d\in \zo^k$, $\gamma=(\gamma_1, \ldots, \gamma_k)$ where each $\gamma_j\in\set{\frac{\gamma_0}{T},\frac{2\gamma_0}{T},\dots,\frac{T\gamma_0}{T}}$,  
and $\ket{\psi} \in H_{\regX,\regZ}$, define $$\ket{\psi_{d,\gamma}}\defeq G_{d_k,k,\gamma_k}\ldots G_{d_2,2,\gamma_2} G_{d_1,1,\gamma_1} \ket{\psi}.$$
For all $i\in[m]$, we then decompose $\ket{\psi}$ into
\begin{equation}
	\label{eq:partition-string}
	\ket{\psi}=\sum_{j=0}^{i-1} \ket{\psi_{1^j0,\gamma}} +\ket{\psi_{1^i,\gamma}} +\sum_{j=1}^{i}\ket{\psi_{err,j,\gamma}}
\end{equation}
by using \Cref{eq:psi0,eq:psi1,eq:psierr} repeatedly,  where $\ket{\psi_{err,i,\gamma}}$ denotes the error state from decomposing $\ket{\psi_{1^{i-1},\gamma}}$.

We denote the projector in $H_{\regX,\regZ}$ corresponding to outputting string $z$ when doing Hadamard on $i$-th copy as $P_{acc,-i,z}$.
Note that $P_{acc,-i,z}$ also depends on $\vec{pk}, \vec{y}$, and $(sk_i, h_i)$ since it includes the measurement the prover did before sending $\vec{a}$,  verifier's checking on $(m-1)$ copies of testing rounds, and  the verifier's final computation from $(sk_i,h_i,y_i,a_i)$. $P_{acc,-i,z}$ is a projector because it only involves the standard basis measurements to get $a$ and classical post-processing of the verifiers. Also note that  $P_{acc,-i,z} P_{acc,-i,z'}=0$ for all $z\neq z'$, and $\sum_z P_{acc,-i,z} =\Pi_{j \neq i} P_{acc,j}\leq Id$.

We denote the string $0^{i-1}10^{m-i} \in \zo^m$ as $e_i$. The output string corresponding to $\ket{\psi} \in H_{\regX,\regZ}$ when $c=e_i$ is then
\begin{equation}
	\label{eq:zi-def}
	z_i\defeq \E_{pk,y} \sum_z \norm{P_{acc,-i,z} U\ket{e_i,\psi}}^2\proj{z},
\end{equation}
 where $\ket{e_i,\psi}=\ket{e_i}_\regC\ket{\psi}_{\regX,\regZ}$ and $U$ is the unitary the prover applies on the last round.
Note that we have averaged over $\vec{pk}, \vec{y}$ where as previously everything has fixed $\vec{pk}$ and $\vec{y}$.

By Property~\ref{property:partition-testing} of \Cref{lem:partition2},
it clearly follows that 
\begin{cor}
	\label{lem:partition-testing}
	For all $\gamma\in\set{\frac{\gamma_0}{T},\frac{2\gamma_0}{T},\dots,\frac{T\gamma_0}{T}}$, and all $i,j\in[m]$ such that $j<i-1$, we have
	$$\norm{\sum_z P_{acc,-i,z} U \ket{e_i, \psi_{1^j0,\gamma}}}^2\leq (m-1)\gamma_0+\negl(n).$$
\end{cor}

Now we define
\begin{equation}
	\label{eq:zgoodi-def}
	z_{good, i}=\E_{\gamma, pk, y} \sum_z \norm{P_{acc,-i,z} U\ket{e_i,\psi_{1^{i-1}1,\gamma}}}^2\proj{z}
\end{equation}
as the output corresponding to a component that would pass the $i$-th testing rounds.
We will show that it is $O(\norm{\ket{\psi_{1^{i-1}0}}})$-close to $z_i$.
Before doing so, we present a technical lemma.

\begin{lemma}\label{lem:samp-tech-2}
	For any state $\ket{\psi}$,  $\ket{\phi}$ and projectors $\{P_z\}$ such that $P_z P_{z'} =0 $ for all $z\neq z'$, we have
	$$  \sum_z |\vev{\psi|P_z|\phi}| \leq \sqrt{\norm{\sum_z P_z\ket{\psi}}^2 } \sqrt{\norm{\sum_z P_z\ket{\phi}}^2 }. $$
\end{lemma}
\begin{proof}
	\begin{align}
		\sum_z |\vev{\psi|P_z|\phi}| =&\sum_z|\vev{\psi|P_zP_z|\phi}| \nn \\
		\leq& \sum_z \norm{\bra{\psi}P_z} \norm{ P_z\ket{\phi}} \nn \\
		\leq&  \sqrt{\sum_z \norm{P_z\ket{\psi}}^2} \sqrt{\sum_z\norm{P_z\ket{\phi}}^2} \nn \\
		\leq& \sqrt{\norm{\sum_z P_z\ket{\psi}}^2 } \sqrt{\norm{\sum_z P_z\ket{\phi}}^2 } \nn,
	\end{align}
	where we used Cauchy's inequality on the second and third line and $P_z P_{z'} =0 $ on the fourth line.
\end{proof}

\begin{cor}\label{lem:samp-tech}
	For any state $\ket{\psi}$,  $\ket{\phi}$ and projectors $\{P_z\}$ such that $\sum_z P_z \leq Id$ and $P_z P_{z'} =0 $ for all $z\neq z'$, we have
	$$  \sum_z |\vev{\psi|P_z|\phi}| \leq \norm{\psi}\norm{\phi}. $$
\end{cor}

Now we can estimate $z_i$ using $z_{good, i}$, with errors on the orders of $\norm{\ket{\psi_{1^{i-1}0}}}$.
This error might not be small in general,
but we can average it out later by considering uniformly random $i\in[m]$.
The analysis is tedious but straightforward;
we simply expand $z_i$ and bound the terms that are not $z_{good, i}$.

\begin{lemma}
	\label{thm:zi-zgoodi}
	\begin{align*}
	\tr\abs{z_i-z_{good, i}}\leq&\E_{pk, y, \gamma}\L[\norm{\ket{\psi_{1^{i-1}0,\gamma}}}^2+2\norm{\ket{\psi_{1^{i-1}0,\gamma}}}\R]\\
	&+O\L(\frac{m^2}{\sqrt T}+m\sqrt{(m-1)\gamma_0}\R).
	\end{align*}
\end{lemma}

\begin{proof}[\Cref{thm:zi-zgoodi}]
	We take expectation of \Cref{eq:partition-string} over $\gamma$
	$$\ket{\psi}=\E_{\gamma}\left[
		\sum_{j=0}^{i-1} \ket{\psi_{1^j0,\gamma}} +\ket{\psi_{1^i,\gamma}} +\sum_{j=1}^{i}\ket{\psi_{err,j,\gamma}}
	\right],$$
	and expand $z_i$ from \Cref{eq:zi-def} as
	\begin{align}
		z_i &= z_{good,i}+ \E_{pk, y, \gamma} \sum_z \L[\sum_{k=0}^{i-1} \bra{\psi_{1^k0,\gamma}}U^\dag  P_{acc,-i,z}U   \sum_{j=0}^{i-1} \ket{\psi_{1^j0,\gamma}} \R. \nn \\
		&+
		\sum_{k=0}^{i-1} \bra{\psi_{1^k0,\gamma}}U^\dag  P_{acc,-i,z}U \ket{\psi_{1^i,\gamma}} +\sum_{k=0}^{i-1} \bra{\psi_{1^k0,\gamma}}U^\dag  P_{acc,-i,z}U\sum_{j=1}^{i}\ket{\psi_{err,j,\gamma}} \nn \\
		&+\bra{\psi_{1^i,\gamma}} U^\dag  P_{acc,-i,z}U \sum_{j=0}^{i-1} \ket{\psi_{1^j0,\gamma}} +\bra{\psi_{1^i,\gamma}} U^\dag  P_{acc,-i,z}U \sum_{j=1}^{i}\ket{\psi_{err,j,\gamma}}
		\nn \\
		&+ \sum_{k=1}^{i}\bra{\psi_{err,k,\gamma}} U^\dag  P_{acc,-i,z}U  \sum_{j=0}^{i-1} \ket{\psi_{1^j0,\gamma}} + \sum_{k=1}^{i}\bra{\psi_{err,k,\gamma}} U^\dag  P_{acc,-i,z}U \ket{\psi_{1^i,\gamma}}
		\nn \\
		&\L.    +\sum_{k=1}^{i}\bra{\psi_{err,k,\gamma}} U^\dag  P_{acc,-i,z}U \sum_{j=1}^{i}\ket{\psi_{err,j,\gamma}} \R] \proj{z} , \nn     
	\end{align}
	where we omitted writing out $e_i$.
	Therefore we have
	\begin{align*}
		\tr|z_i-z_{good,i}|\leq \E_{pk, y, \gamma} \sum_z &\L[ \sum_{k=0}^{i-1} \sum_{j=0}^{i-1} \L| \bra{\psi_{1^k0,\gamma}}U^\dag  P_{acc,-i,z}U \ket{\psi_{1^j0,\gamma}} \R|\R.\\
		&+
		2 \sum_{k=0}^{i-1} \L|\bra{\psi_{1^k0,\gamma}}U^\dag  P_{acc,-i,z}U \ket{\psi_{1^i,\gamma}} \R| \\
		&+ 2 \sum_{k=0}^{i-1}\sum_{j=1}^{i}\L| \bra{\psi_{1^k0,\gamma}}U^\dag  P_{acc,-i,z}U\ket{\psi_{err,j,\gamma}}\R| \\   
		&+2 \sum_{j=1}^{i}\L|\bra{\psi_{1^i,\gamma}} U^\dag  P_{acc,-i,z}U \ket{\psi_{err,j,\gamma}}\R| \\
		&+\L. \sum_{k=1}^{i}\sum_{j=1}^{i}\L| \bra{\psi_{err,k,\gamma}} U^\dag  P_{acc,-i,z}U \ket{\psi_{err,j,\gamma}}\R| \R]
	\end{align*}
	by the triangle inequality.
	The last three error terms sum to $O\L(\frac{m^2}{\sqrt{T}}\R)$ by \Cref{lem:samp-tech} and property~\ref{property:partition-err} of \Cref{lem:partition2}.
	As for the first two terms, by \Cref{lem:samp-tech-2} and \Cref{lem:partition-testing}, we see that
	\begin{align*}
		\sum_z \sum_{k=0}^{i-1}\sum_{j=0}^{i-1}
		&\abs{\bra{\psi_{1^k0,\gamma}}U^\dag  P_{acc,-i,z}U \ket{\psi_{1^j0,\gamma}}} \\
		&\leq\sum_z \abs{\bra{\psi_{1^{i-1}0,\gamma}}U^\dag  P_{acc,-i,z}U \ket{\psi_{1^{i-1}0,\gamma}}} + O\L(m^2(m-1)\gamma_0\R) \\
		&\leq\norm{\ket{\psi_{1^{i-1}0,\gamma}}}^2 + O\L(m^2(m-1)\gamma_0\R)
	\end{align*}
	and similarly
	\begin{align*}
		\sum_z\sum_{k=0}^{i-1}
		&\abs{\bra{\psi_{1^k0,\gamma}}U^\dag  P_{acc,-i,z}U \ket{\psi_{1^i,\gamma}}}\\
		&\leq\sum_z\abs{\bra{\psi_{1^{i-1}0,\gamma}}U^\dag  P_{acc,-i,z}U \ket{\psi_{1^i,\gamma}}}+O\L(m\sqrt{(m-1)\gamma_0}\R)\\
		&\leq\norm{\ket{\psi_{1^i,\gamma}}}+O\L(m\sqrt{(m-1)\gamma_0}\R).
	\end{align*}
\end{proof}

Now let $z_{true}$, as a mixed state, be the correct sample of the $\SampBQP$ instance $D_x$,
and let $z_{ideal, i}=\tr(z_{good, i})z_{true}$.
We show that $z_{ideal, i}$ is close to $z_{good, i}$.
\begin{lemma}
	\label{thm:zgood-zideal}
	$z_{good, i}$ is $O(\eps)$-computationally indistinguishable to $z_{ideal, i}$,
	where $\eps\in\bbR$ is the accuracy parameter picked earlier in $\PiSampZ$.
\end{lemma}
\begin{proof}[\Cref{thm:zgood-zideal}]
	For every $i\in [m]$ and every prover strategy $(U_0,U)$ for $\PiSampZ$, consider the following composite strategy, $\Picomp{i}$, as a prover for $\PiNaive$. Note that a prover only interacts with the verifier in Step~\ref{step:urmila-in-naive} of $\PiNaive$ where $\PiMeasure$ is run, so we describe a prover's action in terms of the four rounds of communication in $\PiMeasure$. 

	$\Picomp{i}$ tries to run $U_0$ by taking the verifier's input as the input to the $i$-th copy of $\PiMeasure$ in $\PiSampZ$ and simulating other $m-1$ copies by himself. The prover then picks a uniformly random $\gamma$ and  tries to generate $\ket{\psi_{1^{i-1}1,\gamma}}$ by applying $G_{i,1,\gamma}G_{i-1,1,\gamma} \cdots G_{2,1,\gamma}G_{1,1,\gamma}$. This can be efficiently done because of \Cref{lem:partition2} and our choice of $\gamma_0$ and $T$ in \Cref{thm:qpip0}. If the prover fails to generate $\ket{\psi_{1^{i-1}1,\gamma}}$, he throws out everything and aborts by sending $\bot$ back.   On the fourth round,  If it's a testing round the prover reply with the $i$-th register of $\ext_i\left(\frac{\ket{e_j}_{\regC}\ket{\psi_{1,i,\gamma}}_{\regX,\regZ}}{\|\ket{\psi_1}_{\regX,\regZ}\|}\right)$, where $\ext_i$ is specified in property~\ref{property:partition-binding} of Lemma~\ref{lem:partition2}. If it's the Hadamard round  the prover  runs $U$ and checks whether every copy except the $i$-th copy would be accepted. If all $m-1$ copies are accepted, he replies with the $i$-th copy, otherwise reply $\bot$.
	
	Denote the result we would get in the Hadamard round by $z_{composite,i}$. By construction, when $G_{i,1,\gamma}\ldots G_{1,1,\gamma}$ succeeded, the corresponding output would be $z_{good,i}$. Also note that this is the only case where the verifier won't reject, so $z_{composite,i}=z_{good,i}$.

	In the testing round, by property~\ref{property:partition-binding} of ~\Cref{lem:partition2}, the above strategy is accepted with probability $1-\negl(n)$ when the prover didn't abort.
	Since the prover's strategy is also efficient, by ~\Cref{lem:naive-qpip0-binding},
	$z_{composite,i}$ is $O(\eps)$-computationally indistinguishable to $z_{ideal, i}$. 
\end{proof}

Now we try to put together all $i\in [m]$. First let
$$z=\frac{1}{m} \sum_i z_i= \frac{1}{m} \sum_i \sum_z \proj{z} \cdot \vev{e_i,\psi|U^\dag P_{acc,-i,z} U|e_i,\psi},$$
which is the output distribution of $\PiSampZ$.
We also define the following accordingly:
$$z_{good}\defeq \frac{1}{m}\sum_i z_{good,i,}$$
$$z_{ideal}\defeq \frac{1}{m}\sum_i z_{ideal,i}.$$
Notice that $z_{ideal}$ is some ideal output distribution, which might not have the same accept probability as $z$.

\begin{theorem}\label{thm:qpip0} 
	Under the QLWE assumption, $\PiSampZ$ is a protocol for the $\SampBQP$ problem $(D_x)_{x\in\set{0,1}^*}$  with negligible completeness error and is computationally sound.\footnote{The soundness and completeness of a $\SampBQP$ protocol is defined in \Cref{dfn:stats-secure-proto-sampbqp}}
	
\end{theorem}
\begin{proof}
	Completeness is trivial. In the following we prove the soundness.
	
	By Property~\ref{property-partition-norm-sum} of Lemma~\ref{lem:partition2}, we have
	\begin{align} \label{eq:bad-term-sum}
		\norm{\ket{\psi}}^2 \geq& \norm{\ket{\psi_{0,\gamma}}}^2+\norm{\ket{\psi_{1,\gamma}}}^2 \nn \\
		\geq& \norm{\ket{\psi_{0,\gamma}}}^2+
		\norm{\ket{\psi_{10,\gamma}}}^2+ \norm{\ket{\psi_{11,\gamma}}}^2 \nn \\
		\geq& \norm{\ket{\psi_{0,\gamma}}}^2+
		\norm{\ket{\psi_{10,\gamma}}}^2+ \norm{\ket{\psi_{110,\gamma}}}^2 +\cdots  \nn \\
		&+ \norm{\ket{\psi_{1^{m-1}0,\gamma}}}^2+ \norm{\ket{\psi_{1^{m-1}1,\gamma}}}^2.
	\end{align}

	We have
	\begin{align} \label{eq:z-z-good}
		\tr|z-z_{good}| =& \tr\L|\frac{1}{m}\sum_i (z_i-z_{good,i})\R| \nn \\
		\leq&  \frac{1}{m}\sum_i\tr| (z_i-z_{good,i})| \nn \\
		\leq&  \frac{1}{m}\sum_i\L[\E_{pk, y, \gamma}\L[\norm{\ket{\psi_{1^{i-1}0,\gamma}}}^2+ 2\norm{\ket{\psi_{1^{i-1}0,\gamma}}}\R] \R.\nn \\
		&+ \L. O\L(\frac{m^2}{\sqrt T}+m\sqrt{(m-1)\gamma_0}\R)\R] \nn \\%%%%%%%%
		\leq&  \frac{1}{m}+ 2\frac{1}{\sqrt m}+O\L(\frac{m^2}{\sqrt T}+m\sqrt{(m-1)\gamma_0}\R) \nn \\ %%%%%
		=&O\L( \frac{1}{\sqrt m}+\frac{m^2}{\sqrt T}+m\sqrt{(m-1)\gamma_0}\R),  
	\end{align}
	where we used triangle inequality on the second line, \Cref{thm:zi-zgoodi} on the third line, Equation~\ref{eq:bad-term-sum} and Cauchy's inequality on the fourth line.
	Set $m=O(1/\eps^2), T=O(1/\eps^2),\gamma_0=\eps^8$. Combining \Cref{thm:zgood-zideal} and \Cref{eq:z-z-good} by triangle inequality, we have $z$ is $O(\epsilon)$-computationally indistinguishable to $z_{ideal}$. Therefore, $(d,z)$  $O(\epsilon)$-computationally indistinguishable to $(d,z_{ideal})$.
\end{proof}

~\Cref{thm:qpip0-informal} follows as a corollary.


\section{Constant-Round and Blind $\QPIP_1$ protocol for $\SampBQP$}

We now present a constant-round, blind, and verifiable delegation scheme for $\SampBQP$ as an application of our $\QPIP_0$ construction for $\SampBQP$,
which we achieve by using the homomorphic encryption scheme from \cite{mahadev_qfhe} together with our construction.

\subsection{Our Delegation Protocol}

Now we compile our $\QPIP_1$ protocol \myprotoref{protoQPIP1} using the quantum homomorphic encryption $\mathsf{QHE}$ \Ethan{TODO maybe wrap this as a def} in a similar way as what we did for our blind $\BQP$ delegation protocol in \cref{sec:BlindQBP}

The construction is obvious, \Ethan{Maybe still write it out...? Or maybe not?} so we directly go into analysis. \Ethan{Need to remember security param if we write this out}

\begin{thm}
	Compiling \cref{proto:QPIP0samp} using $\mathsf{QHE}$ preserves its completeness and soundness.
\end{thm}
\begin{proof}
	A honest prover still gets accepted with overwhelming probability by the correctness of $\mathsf{QHE}$. \Ethan{Not sure if too hand-wavy}

	As for soundness, a cheating prover cannot do better under this compiliation, since each message still decodes to some plaintext.

	In other words, anything a cheating prover can do after the compiliation, there exists a corresponding strategy in the version before.
	\Ethan {Ahhhh this is not rigorous at all}
\end{proof}

\begin{thm}
	\myprotoref{ProtoPriv} is IND-CPA secure.
\end{thm}
\begin{proof}
	The verifier's first message is encrypted into $\tilde{x}$ in an IND-CPA way as a ciphertext.
	The verifier's second message is the basis choice of Hadamard or test rounds, which can be done using public coins.

	The verifier only sends the prover these two messages, so it follows \Ethan{hopefully?} easily that the protocol itself is also IND-CPA. \Ethan{Maybe not rigorous enough?}
\end{proof}


\bibliographystyle{plain}
\bibliography{refs}

\end{document}
