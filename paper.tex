\documentclass[runningheads,a4paper]{llncs}
\pagestyle{plain}
%\documentclass[11pt]{article}
%\usepackage[margin=1in]{geometry}
\usepackage{amsmath, amssymb}
\usepackage{stmaryrd}
\usepackage{braket}
\usepackage{algorithm}
\usepackage{enumitem}
\usepackage{algpseudocode}
\newcommand{\sgn}{\operatorname{sgn}}
\DeclareMathOperator*{\E}{\mathbb{E}}
\DeclareMathOperator*{\spn}{\operatorname{span}}
\DeclareMathOperator*{\poly}{\operatorname{poly}}
\newcommand{\norm}[1]{\left\lVert#1\right\rVert}
\usepackage{etoolbox}
\usepackage{hyperref}
\apptocmd{\thebibliography}{\raggedright}{}{}
\usepackage{cite,color,float}
\usepackage{mdframed} 
\usepackage{cleveref}
\crefname{protocol}{protocol}{protocols}
\Crefname{protocol}{Protocol}{Protocols}
\crefname{thm}{theorem}{theorems}
\Crefname{thm}{Theorem}{Theorems}
\crefname{rmk}{remark}{remarks}
\Crefname{rmk}{Remark}{Remarks}
\crefname{lem}{lemma}{lemmata}
\Crefname{lem}{Lemma}{Lemmata}

\numberwithin{equation}{section}
\newcounter{protocol}
\newcommand{\linefill}{\rule{\linewidth}{0.8pt}}

\newenvironment{protocol}[1]{\begingroup\setlength\parindent{0pt}\medskip\noindent\linefill\\
\refstepcounter{protocol}\textbf{Protocol \theprotocol} #1\\\noindent\linefill}
{\vspace{-\topsep}\noindent\linefill\endgroup}

\def \Hin {H_{\mathrm{in}}}
\def \Hout {H_{\mathrm{out}}}
\def \Hprop {H_{\mathrm{prop}}}
\def \Hclock {H_{\mathrm{clock}}}
\def \Jclock {J_{\mathrm{clock}}}
\def \Jprop {J_{\mathrm{prop}}}
\def \Kin {K_{\mathrm{in}}}
\def \Kclock {K_{\mathrm{clock}}}
\def \Kprop {K_{\mathrm{prop}}}

\newcommand{\histpsi}[1]{\ket{\psi_{#1}^{\mathrm{hist}}}}
\newcommand{\LHXZ}[1]{\mathrm{LH}_{\mathrm{XZ}}^{#1}}
\newcommand{\ground}[1]{{\lambda \left (#1 \right)}}

\title{Constant-round Blind Classical Verification of Quantum Sampling}

%\iffalse
\author{Kai-Min Chung
\inst{1}\orcidID{0000-0002-3356-369X}
\thanks{\href{mailto:kmchung@iis.sinica.edu.tw}{kmchung@iis.sinica.edu.tw}. Partially supported by the 2019 Academia Sinica Career Development Award under Grant no. 23-17, and MOST QC project under Grant no. MOST 108-2627-E-002-001.}
\and
Yi Lee
\inst{2}\orcidID{0000-0003-3742-3296}
\thanks{\href{mailto:ylee1228@umd.edu}{ylee1228@umd.edu}. This work was done while affiliated to Academia Sinica and to National Taiwan University.}
\and
Han-Hsuan Lin
\inst{3}\orcidID{???}
\thanks{\href{mailto:linhh@cs.nthu.edu.tw}{linhh@cs.nthu.edu.tw}. Part of this work was done while supported by Scott Aaronson's Vannevar Bush Faculty Fellowship from the US Department of Defense. Partially funded by MOST Grant no. 110-2222-E-007-002-MY3}
\and
Xiaodi Wu
\inst{2,4}\orcidID{0000-0001-8877-9802}
\thanks{\href{mailto:xwu@cs.umd.edu}{xwu@cs.umd.edu}. Partially supported by the U.S. National Science Foundation grant CCF-1755800, CCF-1816695, and CCF-1942837 (CAREER). }}

\index{Chung, Kai-Min}
\index{Lee, Yi}
\index{Lin, Han-Hsuan}
\index{Wu, Xiaodi}

%\fi
%\author{}
\institute{
Institute of Information Science, Academia Sinica, Taiwan
\and
Department of Computer Science, University of Maryland, USA
\and
Department of Computer Science, National Tsing Hua University, Taiwan
\and
Joint Center for Quantum Information and Computer Science, University of Maryland, USA
}

\usepackage{graphicx,amsmath, amssymb,color,url,booktabs,comment}  %cite
\urlstyle{sf}
%\usepackage[margin=1in]{geometry}
% \usepackage{fancyhdr}
%\usepackage[colorlinks]{hyperref} %pagebackref

\newcommand{\nc}{\newcommand}
\nc{\rnc}{\renewcommand}

\def\View{\mathsf{View}}

\def\GS{\mathsf{Ham}}
\nc{\cVGS}{\ensuremath{\cV_\GS}}

\def\Samp{\mathsf{Samp}}
\nc{\PiSamp}{\ensuremath{\Pi_\Samp}}
\nc{\VSamp}{\ensuremath{V_\Samp}}
\nc{\PSamp}{\ensuremath{P_\Samp}}
\nc{\PSampstar}{\ensuremath{P_\Samp^*}}
\nc{\cVSamp}[1]{\ensuremath{\cV_{\Samp,#1}}}
\nc{\cPSamp}[1]{\ensuremath{\cP_{\Samp,#1}}}

\def\HE{\mathsf{HE}}
\def\HGen{\mathsf{HE.Keygen}}
\def\HEnc{\mathsf{HE.Enc}}
\def\HEval{\mathsf{HE.Eval}}
\def\HDec{\mathsf{HE.Dec}}
\def\Rej{\mathsf{Rej}}

\def\blind{\mathsf{blind}}
\nc{\Piblind}{\ensuremath{\Pi_\blind}}
\nc{\Vblind}{\ensuremath{V_\blind}}
\nc{\Pblind}{\ensuremath{P_\blind}}
\nc{\Pblindstar}{\ensuremath{P_\blind^*}}
\nc{\cVblind}[1]{\ensuremath{\cV_{\blind,#1}}}
\nc{\cPblind}[1]{\ensuremath{\cP_{\blind,#1}}}
\def\Pstar{P^*}
\nc{\cPstar}[1]{\ensuremath{\cP^*_{#1}}}
\nc{\ctx}[3]{\ensuremath{{{\widehat{#1}}_{#2}^{(#3)}}}}

%
%\newcommand{\bra}[1]{\langle #1|}
%\newcommand{\ket}[1]{|#1\rangle}
\newcommand{\proj}[1]{|#1\rangle\langle #1|}
% \newcommand{\braket}[2]{\langle #1|#2\rangle}
% \newcommand{\Bra}[1]{\left\langle #1\right|}
% \newcommand{\Ket}[1]{\left|#1\right\rangle}
\newcommand{\Proj}[1]{\left|#1\right\rangle\left\langle #1\right|}
% \newcommand{\Braket}[2]{\left\langle #1\middle|#2\right\rangle}
\nc{\vev}[1]{\langle#1\rangle}
\nc{\grad}{{\vec{\nabla}}}
\nc{\abs}[1]{\lvert#1\rvert}
%\DeclareMathOperator{\abs}{abs}
\DeclareMathOperator{\Bin}{Bin}
\DeclareMathOperator{\conv}{conv}
\DeclareMathOperator{\eig}{eig}
\DeclareMathOperator{\Hist}{Hist}
\DeclareMathOperator{\Hyb}{Hyb}
\DeclareMathOperator{\id}{id}
\DeclareMathOperator{\Img}{Im}
\DeclareMathOperator{\Par}{Par}
% \DeclareMathOperator{\poly}{poly}
\DeclareMathOperator{\negl}{negl}
\DeclareMathOperator{\polylog}{polylog}
\DeclareMathOperator{\tr}{tr}
\DeclareMathOperator{\rank}{rank}
% \DeclareMathOperator{\sgn}{sgn}
\DeclareMathOperator{\Sep}{Sep}
\DeclareMathOperator{\SepSym}{SepSym}
\DeclareMathOperator{\Span}{span}
\DeclareMathOperator{\supp}{supp}
\DeclareMathOperator{\swap}{SWAP}
\DeclareMathOperator{\Sym}{Sym}
\DeclareMathOperator{\ProdSym}{ProdSym}
\DeclareMathOperator{\SEP}{SEP}
\DeclareMathOperator{\PPT}{PPT}
\DeclareMathOperator{\Wg}{Wg}
\DeclareMathOperator{\WMEM}{WMEM}
\DeclareMathOperator{\WOPT}{WOPT}

\DeclareMathOperator{\BPP}{\mathsf{BPP}}
\DeclareMathOperator{\QPIP}{\mathsf{QPIP}}
\DeclareMathOperator{\SampBQP}{\mathsf{SampBQP}}
\DeclareMathOperator{\BQP}{\mathsf{BQP}}
\DeclareMathOperator{\FBQP}{\mathsf{FBQP}}
\DeclareMathOperator{\cnot}{\normalfont\textsc{cnot}}
\DeclareMathOperator{\DTIME}{\mathsf{DTIME}}
\DeclareMathOperator{\NTIME}{\mathsf{NTIME}}
\DeclareMathOperator{\MA}{\mathsf{MA}}
\DeclareMathOperator{\NP}{\mathsf{NP}}
\DeclareMathOperator{\NEXP}{\mathsf{NEXP}}
\DeclareMathOperator{\Ptime}{\mathsf{P}}
\DeclareMathOperator{\QMA}{\mathsf{QMA}}
\DeclareMathOperator{\QCMA}{\mathsf{QCMA}}
\DeclareMathOperator{\BellQMA}{\mathsf{BellQMA}}

\newcommand{\be}{\begin{equation}}
\newcommand{\ee}{\end{equation}}
\newcommand{\bea}{\begin{eqnarray}}
\newcommand{\eea}{\end{eqnarray}}
\newcommand{\nn}{\nonumber}
\newcommand{\bi}{\begin{itemize}}
\newcommand{\ei}{\end{itemize}}
\newcommand{\bn}{\begin{enumerate}}
\newcommand{\en}{\end{enumerate}}
\def\beas#1\eeas{\begin{eqnarray*}#1\end{eqnarray*}}
\def\ba#1\ea{\begin{align}#1\end{align}}
\nc{\bas}{\[\begin{aligned}}
\nc{\eas}{\end{aligned}\]}
\nc{\bpm}{\begin{pmatrix}}
\nc{\epm}{\end{pmatrix}}
\def\non{\nonumber}
\def\nn{\nonumber}
\def\eq#1{(\ref{eq:#1})}
\def\eqs#1#2{(\ref{eq:#1}) and (\ref{eq:#2})}
%\def\eq#1{Eq.~(\ref{eq:#1})}
%\def\eqs#1#2{Eqs.~(\ref{eq:#1}) and (\ref{eq:#2})}
\def\L{\left} 
\def\R{\right}
\def\ra{\rightarrow}
\def\ot{\otimes}
\nc{\given}{\ensuremath{\;\middle|\;}}

\newtheorem{thm}{Theorem}[section]
\newtheorem{theorem}{Theorem}[section]
%\newtheorem*{thm*}{Theorem}
%\newtheorem{claim}[thm]{Claim}
\newtheorem{cor}{Corollary}[thm]
\newtheorem{lem}{Lemma}[section]
\newtheorem{lemma}{Lemma}[section]
\newtheorem{rmk}{Remark}[thm]
%\newtheorem{prop}[thm]{Proposition}
\newtheorem{dfn}{Definition}[section]
\newtheorem{definition}{Definition}[section]
%\newtheorem{con}[thm]{Conjecture}

\newenvironment{prf}{\begin{proof}}{\end{proof}}

\def\eps{\epsilon}
\def\va{{\vec{a}}}
\def\vb{{\vec{b}}}
\def\vn{{\vec{n}}}
\def\cvs{{\cdot\vec{\sigma}}}
\def\vx{{\vec{x}}}
\def\Usch{U_{\text{Sch}}}

\def\cA{\mathcal{A}}
\def\cB{\mathcal{B}}
\def\cD{\mathcal{D}}
\def\cE{\mathcal{E}}
\def\cF{\mathcal{F}}
\def\cH{\mathcal{H}}
\def\cI{{\cal I}}
\def\cL{{\cal L}}
\def\cM{{\cal M}}
\def\cN{\mathcal{N}}
\def\cO{{\cal O}}
\def\cP{\mathcal{P}}
\def\cQ{\mathcal{Q}}
\def\cS{\mathcal{S}}
\def\cT{{\cal T}}
\def\cU{\mathcal{U}}
\def\cV{\mathcal{V}}
\def\cW{{\cal W}}
\def\cX{{\cal X}}
\def\cY{{\cal Y}}

\def\bp{\mathbf{p}}
\def\bq{\mathbf{q}}
\def\bP{{\bf P}}
\def\bQ{{\bf Q}}
\def\gl{\mathfrak{gl}}

\def\bbC{\mathbb{C}}
% \DeclareMathOperator*{\E}{\mathbb{E}}
\DeclareMathOperator*{\bbE}{\mathbb{E}}
%\DeclareMathOperator*{\Pr}{Pr}
\nc{\Prob}[1]{\ensuremath{\Pr\left[#1\right]}}
\def\bbM{\mathbb{M}}
\def\bbN{\mathbb{N}}
\def\bbR{\mathbb{R}}
\def\bbZ{\mathbb{Z}}
\def\bbP{\mathbb{P}}
\def\bbV{\mathbb{V}}
\newcommand{\Real}{\textrm{Re}}

\def\benum{\begin{enumerate}}
\def\eenum{\end{enumerate}}
% \def\bit{\begin{itemize}}
% \def\eit{\end{itemize}}
\def\bdesc{\begin{description}}
\def\edesc{\end{description}}
\newcommand{\fig}[1]{Fig.~\ref{fig:#1}}
\newcommand{\tab}[1]{Table~\ref{tab:#1}}
\newcommand{\secref}[1]{Section~\ref{sec:#1}}
\newcommand{\appref}[1]{Appendix~\ref{sec:#1}}
\newcommand{\lemref}[1]{Lemma~\ref{lem:#1}}
\newcommand{\thmref}[1]{Theorem~\ref{thm:#1}}
\newcommand{\propref}[1]{Proposition~\ref{prop:#1}}
\newcommand{\protoref}[1]{Protocol~\ref{proto:#1}}
\nc{\myprotoref}[1]{\hyperref[#1]{Protocol~\ref*{#1}}}
\newcommand{\defref}[1]{Definition~\ref{def:#1}}
\newcommand{\corref}[1]{Corollary~\ref{cor:#1}}
\newcommand{\conref}[1]{Conjecture~\ref{con:#1}}

\newcommand{\FIXME}[1]{{\color{red}FIXME: #1}}
\nc{\todo}[1]{\textcolor{red}{todo: #1}}



\newcommand{\boxdfn}[2]{
\begin{figure}[h]
\begin{center}
\noindent \framebox{
\begin{minipage}{0.8\textwidth}
\begin{dfn}[{\bf #1}]
\ \\ \\
#2
\end{dfn}
\end{minipage}
}
\end{center}
\end{figure}
}

\newcommand{\boxproto}[2]{
\begin{figure}[h]
\begin{center}
\noindent \framebox{
\begin{minipage}{0.8\textwidth}
\begin{proto}[{\bf #1}]
\ \\ \\
#2
\end{proto}
\end{minipage}
}
\end{center}
\end{figure}
}

\def\begsub#1#2\endsub{\begin{subequations}\label{eq:#1}\begin{align}#2\end{align}\end{subequations}}
\nc\qand{\qquad\text{and}\qquad}
\nc\mnb[1]{\medskip\noindent{\bf #1}}
\nc\mn{\medskip\noindent}

\renewcommand{\arraystretch}{1.5}
%\nc{\problem}[1]{\item\noindent {\bf #1}}

\setlength{\tabcolsep}{10pt}

%%%%%% Han-Hsuan's commands %%%%%%%%
\nc{\nl}{\nn \\ &=}  %new line
\nc{\nnl}{\nn \\ &}  %new new line
\nc{\fot}{\frac{1}{2}} %frac one two
\nc{\oo}[1]{\frac{1}{#1}} % one over
\newcommand{\ben}{\begin{enumerate}}
\newcommand{\een}{\end{enumerate}}
\nc{\mc}{\mathcal}
\nc{\beq}{\begin{equation}}
\nc{\eeq}{\end{equation}}
% \nc{\norm}[1]{\L\| #1 \R\|}

\nc{\onenorm}[1]{\L\| #1 \R\|_1} %one norm
%\nc{\span}{\ensuremath{\mathrm{span}}}

\DeclareMathOperator*{\argmax}{arg\,max}

%\nc{1}

\newcommand{\hannote}[1]{\textcolor{blue}{\small {\textbf{(Han:} #1\textbf{) }}}}

\newcommand{\Knote}[1]{\textcolor{red}{\small {\textbf{(KM:} #1\textbf{) }}}}

\nc{\Ra}{\Rightarrow}
\nc{\zo}{\{0,1\}}	

%%%%import..
% \newcommand{\secpar}{n}


% %%%Efficient Verifier%
% \newcommand{\setupeff}{\setup_{\mathsf{eff}}}
% \newcommand{\vereff}{V_{\mathsf{eff}}}
% \newcommand{\vereffone}{V_{\mathsf{eff},1}}
% \newcommand{\vereffthree}{V_{\mathsf{eff},3}}
% \newcommand{\vereffout}{V_{\mathsf{eff},\mathsf{out}}}
% \newcommand{\proeff}{P_{\mathsf{eff}}}
% \newcommand{\proefftwo}{P_{\mathsf{eff},2}}
% \newcommand{\proefffour}{P_{\mathsf{eff},4}}
% \newcommand{\advPH}{{P^*}^{H}}
% \newcommand{\setup}{\mathsf{Setup}}
% \newcommand{\re}{\mathsf{re}}
% \newcommand{\crh}{\mathsf{CRH}}
% \newcommand{\transcript}{\mathsf{trans}}

% \newcommand{\setupefffs}{\setup_{\mathsf{eff}\text{-}\mathsf{fs}}}
% \newcommand{\proefffs}{P_{\mathsf{eff}\text{-}\mathsf{fs}}}
% \newcommand{\proefffstwo}{P_{\mathsf{eff}\text{-}\mathsf{fs},2}}
% \newcommand{\verefffs}{V_{\mathsf{eff}\text{-}\mathsf{fs}}}
% \newcommand{\verefffsone}{V_{\mathsf{eff}\text{-}\mathsf{fs},1}}
% \newcommand{\verefffsout}{V_{\mathsf{eff}\text{-}\mathsf{fs},\out}}
% %Games%
% \newcommand{\game}{\mathsf{Game}}

% %\newcommand*{\bra}[1]{\langle#1|}
% %\newcommand*{\ket}[1]{|#1\rangle}
% \newcommand*{\opro}[2]{|#1\rangle\langle#2|}
% \newcommand*{\ipro}[2]{\langle #1|#2\rangle}
% \newcommand{\TD}{\mathsf{TD}}

% %%%%% Registers %%%%%%%
% \newcommand*{\regK}{\mathbf{K}}
% \newcommand*{\regI}{\mathbf{I}}
% \newcommand*{\regR}{\mathbf{R}}
% \newcommand*{\regX}{\mathbf{X}}
% \newcommand*{\regY}{\mathbf{Y}}
% \newcommand*{\regZ}{\mathbf{Z}}
% \newcommand{\regW}{\mathbf{W}}
% \newcommand*{\regC}{\mathbf{C}}
% \newcommand*{\regO}{\mathbf{O}}
% \newcommand*{\regF}{\mathbf{F}}

%\newcommand{\redunderline}[1]{\textcolor{BrickRed}{\underline{\textcolor{black}{#1}}}}
\def \sample { \overset{\hspace{0.1em}\mathsf{\scriptscriptstyle\$}}{\leftarrow} }
\def\lapprox{\overset{<}{\sim}}
% \newcommand{\ra}{\rightarrow}
\newcommand{\la}{\leftarrow}
\newcommand{\pro}{P}
\newcommand{\ver}{V}
\newcommand{\secpar}{n}
% \newcommand{\negl}{\mathsf{negl}}
\newcommand{\lang}{L}
\newcommand{\key}{k}
\newcommand{\comy}{y}
\newcommand{\bfy}{\mathbf{y}}
\newcommand{\bfk}{\mathbf{k}}
\newcommand{\bfc}{\mathbf{c}}
\newcommand{\bfans}{\mathbf{a}}
\newcommand{\td}{\mathsf{td}}
\newcommand{\st}{\mathsf{st}}
\newcommand{\out}{\mathsf{out}}
 \newcommand{\bit}{\{0,1\}}
\newcommand{\ans}{a}
\newcommand{\Ans}{A}
%\newcommand{\poly}{\mathsf{poly}}
%\newcommand{\span}{\mathsf{span}}
%\newcommand{\poly}{\mathsf{poly}}
\newcommand{\hil}{\mathcal{H}}
\newcommand{\work}{W}
\newcommand{\defeq}{:=}
%\newcommand{\BPP}{\mathsf{BPP}}
\newcommand{\Succ}{\mathsf{Succ}}

\newcommand{\A}{\mathcal{A}}
\newcommand{\B}{\mathcal{B}}

\newcommand{\Sgood}{S_{\mathsf{good}}}
\newcommand{\Sbad}{S_{\mathsf{bad}}}
\newcommand{\psigood}{\psi_{\mathsf{good}}}
\newcommand{\psibad}{\psi_{\mathsf{bad}}}

\newcommand{\ext}{\mathsf{Ext}}

\newcommand{\calX}{\mathcal{X}}
\newcommand{\calY}{\mathcal{Y}}
\newcommand{\calS}{\mathcal{S}}
\newcommand{\calD}{\mathcal{D}}

%\newcommand{\ot}{\otimes}
\newcommand{\fail}{\mathsf{fail}}
\newcommand{\QTMtoQC}{\mathsf{QTMtoQC}}
\newcommand{\CRH}{\mathsf{CRH}}
\newcommand{\func}{\mathsf{Func}}
\newcommand{\TT}{\mathtt{T}}
\newcommand{\PRG}{\mathsf{PRG}}
\newcommand{\famCRH}{\mathcal{C}\mathcal{R}\mathcal{H}}
\newcommand{\QTM}{\mathsf{QTM}}
\newcommand{\QTIME}{\mathsf{QTIME}}

\newcommand{\Acc}{\mathsf{Acc}}
%%%Randomized Encoding%
\newcommand{\RE}{\mathsf{RE}}
\newcommand{\crs}{\mathsf{crs}}
\newcommand{\ek}{\mathsf{ek}}
\newcommand{\rsetup}{\mathsf{RE}.\mathsf{Setup}}
\newcommand{\renc}{\mathsf{RE}.\mathsf{Enc}}
\newcommand{\rdec}{\mathsf{RE}.\mathsf{Dec}}
\newcommand{\rsim}{\mathsf{RE}.\mathsf{Sim}}
\newcommand{\inp}{\mathsf{inp}}
\newcommand{\Time}{\mathsf{Time}}
\newcommand{\Menc}{\widehat{M_\inp}}

%%%SNARK%
\newcommand{\SNARK}{\mathsf{SNARK}}
\newcommand{\snark}{\mathsf{snark}}
% \newcommand{\NP}{\mathsf{NP}}
% \newcommand{\NTIME}{\mathsf{NTIME}}
\newcommand{\rela}{\mathcal{R}}

%FHE%%%
\newcommand{\FHE}{\mathsf{FHE}}
\newcommand{\fhe}{\mathsf{fhe}}
\newcommand{\fhekeygen}{\mathsf{FHE}.\mathsf{KeyGen}}
\newcommand{\fheenc}{\mathsf{FHE}.\mathsf{Enc}}
\newcommand{\fhedec}{\mathsf{FHE}.\mathsf{Dec}}
\newcommand{\fheeval}{\mathsf{FHE}.\mathsf{Eval}}
\newcommand{\sk}{\mathsf{sk}}
\newcommand{\pk}{\mathsf{pk}}
\newcommand{\ct}{\mathsf{ct}}

%%%Efficient Verifier%
\newcommand{\setupeff}{\setup_{\mathsf{eff}}}
\newcommand{\vereff}{V_{\mathsf{eff}}}
\newcommand{\vereffone}{V_{\mathsf{eff},1}}
\newcommand{\vereffthree}{V_{\mathsf{eff},3}}
\newcommand{\vereffout}{V_{\mathsf{eff},\mathsf{out}}}
\newcommand{\proeff}{P_{\mathsf{eff}}}
\newcommand{\proefftwo}{P_{\mathsf{eff},2}}
\newcommand{\proefffour}{P_{\mathsf{eff},4}}
\newcommand{\advPH}{{P^*}^{H}}
\newcommand{\setup}{\mathsf{Setup}}
\newcommand{\re}{\mathsf{re}}
\newcommand{\crh}{\mathsf{CRH}}
\newcommand{\transcript}{\mathsf{trans}}

\newcommand{\setupefffs}{\setup_{\mathsf{eff}\text{-}\mathsf{fs}}}
\newcommand{\proefffs}{P_{\mathsf{eff}\text{-}\mathsf{fs}}}
\newcommand{\proefffstwo}{P_{\mathsf{eff}\text{-}\mathsf{fs},2}}
\newcommand{\verefffs}{V_{\mathsf{eff}\text{-}\mathsf{fs}}}
\newcommand{\verefffsone}{V_{\mathsf{eff}\text{-}\mathsf{fs},1}}
\newcommand{\verefffsout}{V_{\mathsf{eff}\text{-}\mathsf{fs},\out}}
%Games%
\newcommand{\game}{\mathsf{Game}}

%\newcommand*{\bra}[1]{\langle#1|}
%\newcommand*{\ket}[1]{|#1\rangle}
\newcommand*{\opro}[2]{|#1\rangle\langle#2|}
\newcommand*{\ipro}[2]{\langle #1|#2\rangle}
\newcommand{\TD}{\mathsf{TD}}

%%%%% Registers %%%%%%%
\newcommand*{\regK}{\mathbf{K}}
\newcommand*{\regI}{\mathbf{I}}
\newcommand*{\regR}{\mathbf{R}}
\newcommand*{\regX}{\mathbf{X}}
\newcommand*{\regY}{\mathbf{Y}}
\newcommand*{\regZ}{\mathbf{Z}}
\newcommand{\regW}{\mathbf{W}}
\newcommand*{\regC}{\mathbf{C}}
\newcommand*{\regO}{\mathbf{O}}
\newcommand*{\regF}{\mathbf{F}}

%%%%% Note %%%%%%%%%%%%%

\newcommand{\nai}[1]{{\color{purple}[Nai: #1]}}
\newcommand{\km}[1]{{\color{brown}[KM: #1]}}
\newcommand{\takashi}[1]{{\color{red}[Takashi: #1]}}



\begin{document}

\maketitle

\begin{abstract}

In a recent breakthrough, Mahadev constructed a classical verification of quantum computation (CVQC)  protocol for a  classical client to delegate decision problems in $\BQP$ to an untrusted quantum prover under computational assumptions. In this work, we explore further the feasibility of CVQC with the more general \emph{sampling} problems in BQP and with the desirable \emph{blindness} property. We contribute affirmative solutions to both as follows. 
\begin{itemize}
\item  Motivated by the sampling nature of many quantum applications (e.g., quantum algorithms for machine learning and quantum supremacy tasks), we initiate the study of  CVQC for \emph{quantum sampling problems} (denoted by $\SampBQP$).  More precisely, in a CVQC protocol for a $\SampBQP$ problem, the prover and the verifier are given an input $x\in \zo^n$ and a quantum circuit $C$, and the goal of the classical client is to learn a sample from the output $z \leftarrow C(x)$ up to a small error, from its interaction with an untrusted prover. We demonstrate its feasibility by constructing a four-message CVQC protocol for $\SampBQP$ based on the quantum \emph{Learning With Errors} assumption.

\item
The \emph{blindness} of CVQC protocols refers to a property of the protocol where the prover learns nothing, and hence is blind, about the client's input. It is a highly desirable property that has been intensively studied for the delegation of quantum computation. 
We provide a simple yet powerful \emph{generic} compiler that transforms any CVQC protocol to a blind one while preserving its completeness and soundness errors as well as the number of rounds.  
\end{itemize}
Applying our compiler to (a parallel repetition of) Mahadev's CVQC protocol for $\BQP$ and our CVQC protocol for $\SampBQP$ yields the first \emph{constant-round} blind CVQC protocol for $\BQP$ and $\SampBQP$ respectively, with negligible and inverse polynomial soundness errors respectively, and negligible completeness errors. 

\vspace{1mm}
\noindent \textbf{Keywords:} classical delegation of quantum computation, blind quantum computation, quantum sampling problems

\end{abstract}

\newpage

\section{Introduction}
% \XW{
% \begin{itemize}
%     \item add a lot of references.
%     \item comparison with the most relevant results:
%       \begin{itemize}
%           \item Sampling, the only paper; how about classical sampling?
%           \item the following for BQP  
%           \item blind and verifiable ~\cite{GV19}; we  constant round; technique-wise very different.
%           \item there is a table in~\cite{Grilo19}. Safe to say ~\cite{GV19} only existing blind protocol in the computational setting?
%           \item all previous either quantum clients, or at least 2 provers.
%           \item the following for blindness
%           \item Mahadev in her thesis~\cite{mahadev_2018} discussed a bit about the relation between verifiability and blindness. She hoped to get verifiability out of blindness by designing some non-malleable QFHE but failed.   
%           \item what's the high-level message we can say here?  Use QFHE in a different way? It is correct that not much work in the classical setting either.  Maybe existing classical work employs the principle but with different implementation.
%           \item old approach, first get blindness  (measurement-based, self-testing), and then try to make it verifiable; our approach, first have a verifiable protocol, and upgrade by a QFHE.
%           \item directly QFHE (blindness) won't give verifiability. some thoughts from Mahadev.
%           \item
%       \end{itemize}
% \end{itemize}
% }
Can quantum computation, with potential computational advantages that are intractable for classical computers,
be efficiently verified by classical human beings?
This seeming paradox has been one of the central problems in quantum complexity theory and delegation of quantum computation~\cite{web:Aaronson}.
From a philosophical point of view, this question is also known to have a fascinating connection to the \emph{falsifiability} of quantum mechanics in the potential high complexity regime~\cite{survey:AV12}.

A complexity theoretic formulation of this problem by Gottesman in 2004~\cite{web:Aaronson} asks the possibility for an efficient classical verifier/client (a $\BPP$ machine) to verify the output of an
efficient quantum prover (a $\BQP$ machine).
In the absence of techniques for directly tackling this question, earlier feasibility results on this problem have been focusing on two weaker formulations.
The first type of feasibility results (e.g.,~\cite{BFK09,arXiv:ABOEM17,FK17,mf16}) considers the case where the $\BPP$ verifier is equipped with limited quantum power.
The second type of feasibility results (e.g,~\cite{Nat:RUV13, CGJV19, Gheorghiu_2015, HPF15})
considers a $\BPP$ verifier interacting with at least two entangled, non-communicating quantum provers.
In a recent breakthrough, Mahadev~\cite{FOCS:Mahadev18a} proposed the first protocol of classical verification of quantum computation (CVQC) whose soundness is based on a widely recognized computational assumption that the learning with error (LWE) problem~\cite{JACM:Regev09} is hard for $\BQP$ machines.
The technique invented therein has inspired many  subsequent developments of CVQC protocols with improved parameters and functionality (e.g.~\cite{FOCS:GheVid19,arXiv:AlaChiHun19,arXiv:ChiaChungYam19}).
We refer curious readers to the survey~\cite{survey:GKK19} for details.

With the newly developed techniques, we revisit the classical verification of quantum computation problems from both a philosophical and a practical point of view.
We first observe that the \emph{sampling} version of $\BQP$ (e.g., the class $\SampBQP$ formulated by Aaronson~\cite{aaronson_2013}) might be a more appropriate notion to serve the purpose of the original problem.
Philosophically, the outcomes of quantum mechanical experiments are usually samples or statistical information, which is well demonstrated in the famous double-slit experiment.
Moreover, a lot of quantum algorithms from Shor's~\cite{Shor} and Grover's~\cite{Grover} algorithms to some recent developments in machine learning and optimization (e.g.~\cite{brando_et_al:LIPIcs:2019:10603, AGGW17,pmlr-v97-li19b}) contain a significant quantum sampling component.
The fact that almost all quantum supremacy tasks (e.g.,~\cite{Boson, IQP, nature-google}) are sampling ones strengthens the importance of delegation for quantum sampling problems.
 %Even though the relation between $\BQP$ and $\SampBQP$ is relatively understood in the plain model,
However, it is far from clear whether the subtle difference between $\BQP$ and its sampling version could lead to
technical barrier in the context of CVQC, or
 %what are the potential technical challenges raised by their subtle differences in the context of CVQC and how to resolve them is however far from clear.
whether one can develop a CVQC protocol for $\SampBQP$ based on Mahadev's technique~\cite{FOCS:Mahadev18a}.

Another desirable property of CVQC protocols is the \emph{blindness} where the prover cannot distinguish the particular computation in the protocol from another one of the same size, and hence is blind about the client's input.
Historically, blindness has been achieved in the weaker formulations of CVQC based on various techniques: e.g., the measurement-based quantum computation exploited in~\cite{BFK09}, the quantum authentication scheme exploited in~\cite{arXiv:ABOEM17}, and the self-testing technique exploited in~\cite{Nat:RUV13}.
Moreover, the blindness property is known to be helpful to establish the verifiability of CVQC protocols. However, this is never an easy task.
See for example the significant amount of efforts to add verifiability to blind CVQC protocols in~\cite{FK17}.
Achieving both blindness and verifiability on top of Mahadev's technique~\cite{FOCS:Mahadev18a} is a conceivably much more challenging task.
The only successful attempt~\cite{FOCS:GheVid19} so far applies Mahadev's technique to the measurement-based quantum computation,
whereas the analysis is still very specific to the construction.
Could there be a \emph{generic} way to achieve blindness and verifiability for CVQC protocols at the same time?


\vspace{2mm} \noindent \textbf{Contribution.} We provide \emph{affirmative} solutions to both of our questions.
In particular, we demonstrate the feasibility of the classical verification of quantum sampling by
constructing a constant-round CVQC protocol for $\SampBQP$, the sampling version of $\BQP$ formulated by Aaronson~\cite{aaronson_2013}. Formally, $\SampBQP$ consists of sampling problems $(D_x)_{x\in\zo^*}$ that can be approximately sampled by a $\BQP$ machine with an inverse polynomial accurate. %Namely, $A(x,1^{1/\eps})$ outputs a sample that is $\eps$-close to the distribution $D_x$ in statistical distance.
%, where given an input $x \in \zo^n$, the goal  
Our protocol leverages the Hamiltonian model and the computational X-Z measurement from~\cite{FOCS:Mahadev18a}.
However, a significant amount of new techniques have been developed to deal with the difference between $\SampBQP$ and $\BQP$, which will be highlighted in the technical contribution section. Precisely,
\begin{theorem}[informal]
Assuming the QLWE assumption, there exists a four-message CVQC protocol for all sampling problems in $\SampBQP$ with negligible completeness error and computational soundness.
\end{theorem}
%\XW{Insert the theorem statement for the first result here! and a pointer!}

Somewhat surprisingly, our second contribution is a simple yet powerful generic compiler that transforms any CVQC protocol to a blind one while preserving completeness and soundness errors.
Our construction builds upon another important primitive called the Quantum Fully Homomorphic Encryption (QFHE)~\cite{BJ15, DSS16, LC18, NS18, OTF18, mahadev_qfhe}.
Intuitively, QFHE allows fully homomorphic operations on encrypted quantum data and thus could be an ideal technical candidate for achieving blindness.
Indeed, in another paper~\cite{mahadev_qfhe}, Mahadev constructed the first leveled QFHE based on similar techniques and computational assumptions from~\cite{FOCS:Mahadev18a}.
The constructed QFHE automatically implies a blind CVQC protocol, however, without verifiability.
Extending this protocol with verifiability seems challenging as hinted by failed attempts in Section 2.2.2 of~\cite{mahadev_2018}.
In fact, most existing blind and verifiable CVQC protocols require a notable amount of effort in achieving each property respectively.

We observe that QFHE, especially the one from~\cite{mahadev_qfhe}, can be used to transform any CVQC protocol to a blind one with the same number of round communication, while preserving completeness and soundness error.
As a result, one can \emph{upgrade} every verifiable CVQC protocol with blindness almost for free with the help of QFHE.
Conceptually, we take a very different approach from previous results (e.g.,~\cite{FK17}) which use the blindness as the start point and then work to extend it with verifiability.
At a high level, our strategy is to simulate the original CVQC protocol under QFHE per each message.
To that end, we do require a special property of QFHE that the classical part of the ciphertext can be operated on separately from the quantum part, which is satisfied by the construction from~\cite{mahadev_qfhe}.
Our construction makes a modular use of QFHE and only requires a minor technicality in the analysis, which will be explained below. As a result, we obtain
\begin{theorem}[informal]
Assuming the QLWE assumption, there exists a protocol compiler that transforms any CVQC protocol $\Pi$ to a CVQC protocol $\Piblind$ that achieves blindness while preserves its round complexity, completeness, and soundness.
\end{theorem}

%\XW{theorem statement for the second contribution and pointer}



As a simple corollary of combining both results above, we achieve a constant-round blind CVQC protocol for $\SampBQP$. %with negligible completeness error and statistical soundness.  
\begin{theorem}[informal]
        Assuming the QLWE assumption, there exists a blind, four-message CVQC protocol for all sampling problems in $\SampBQP$ with negligible completeness error and computational soundness.
\end{theorem}

We can also the first blind and constant-round CVQC protocol for $\BQP$ by applying our compiler to the parallel repetition of Mahadev's protocol for $\BQP$ from \cite{arXiv:ChiaChungYam19, arXiv:AlaChiHun19}.


\begin{theorem}[informal]
    Assuming the QLWE assumption, there exists a blind, four-message CVQC protocol for all languages in $\BQP$ with negligible completeness and soundness errors.
\end{theorem}



%\XW{here for both $\BQP$ and $\SampBQP$}
%\XW{check the para/terminology here; consider adding a theorem statement,or a pointer to the later section}

To the authors' best knowledge, we are the first to study CVQC protocols for $\SampBQP$ and establish a generic compiler to upgrade CVQC protocols with blindness.
Our result also entails a \emph{constant-round} blind and verifiable CVQC protocol for $\BQP$.
The closest result to ours is by Gheorghiu and Vidick~\cite{FOCS:GheVid19} which shows such a CVQC protocol for $\BQP$, however, with a polynomial number of rounds.
Their protocol was obtained by first constructing a remote state preparation primitive and then combining it with an existing blind and verifiable protocol~\cite{FK17} where the verifier has some limited quantum power.
Our technical approach is quite different and seems incomparable.

% \XW{any more to say about parameters, techniques?}
% \XW{anything we want to say about composability?}
% \Ethan{Last time we checked, Vidick might have better composability since he's got some kind of ideal box and simulator-based proof}

% Related work
% \begin{itemize}
%     \item Comparison with related works here?
% \end{itemize}
%       \begin{itemize}
%           \item Sampling, the only paper; how about classical sampling?
%           \item the following for BQP  
%           \item blind and verifiable ~\cite{GV19}; we  constant round; technique-wise very different.
%           \item there is a table in~\cite{Grilo19}. Safe to say ~\cite{GV19} only existing blind protocol in the computational setting?
%           \item all previous either quantum clients, or at least 2 provers.
%           \item the following for blindness
%           \item Mahadev in her thesis~\cite{mahadev_2018} discussed a bit about the relation between verifiability and blindness. She hoped to get verifiability out of blindness by designing some non-malleable QFHE but failed.   
%           \item what's the high-level message we can say here?  Use QFHE in a different way? It is correct that not much work in the classical setting either.  Maybe existing classical work employs the principle but with different implementation.
%           \item old approach, first get blindness  (measurement-based, self-testing), and then try to make it verifiable; our approach, first have a verifiable protocol, and upgrade by a QFHE.
%           \item directly QFHE (blindness) won't give verifiability. some thoughts from Mahadev.
%
%           technical comparison with the past parallel % repetition.
%           \item

\vspace{2mm} \noindent \textbf{Techniques.} Following~\cite{FOCS:Mahadev18a}, we formally define $\QPIP_{\tau}$ as classes of CVQC protocols where $\tau$ refers to the size of quantum register in the possession of the classical verifier, or equivalently, the limited quantum computation power of the verifier.
It is known that $\BQP$ can be efficiently verified by a classical verifier that can perform a single qubit X or Z measurement~\cite{PhysRevA.93.022326, mf16}.
Namely, there is a $\QPIP_1$ protocol for $\BQP$.
The main contribution of Mahadev~\cite{FOCS:Mahadev18a} can hence be deemed as a way to compile this $\QPIP_1$ protocol into a $\QPIP_0$ protocol (i.e., with a fully classical verifier).

\vspace{2mm} \noindent \textbf{(1) Construction of a $\QPIP_0$ protocol for $\SampBQP$}.
We will follow the same road map above (i.e., from $\QPIP_1$ to $\QPIP_0$) for $\SampBQP$. However, since there is no existing $\QPIP_1$ protocol for $\SampBQP$, we make original contributions to both steps as follows:

\vspace{2mm} \noindent \emph{$\diamond \, \QPIP_1$ protocol for $\SampBQP$}: We will employ the local Hamiltonian technique~\cite{kitaev2002classical} and its ground state (known as the history state) as a key technical ingredient to certify the $\SampBQP$ circuits.
However, there are important differences between the cases for $\BQP$ and $\SampBQP$.
Recall that the original construction of local Hamiltonian $H$ for $\BQP$ (or $\QMA$) contains two parts $H=H_{\mathrm{circuit}}+ H_{\mathrm{out}}$.
Roughly speaking, $H_{\mathrm{circuit}}$ helps guarantee its ground space only contains \emph{valid} history states with correct input and circuit evolution, while $H_{\mathrm{out}}$'s energy encodes the 0/1 output for $\BQP$ circuits.
Thus, its outcome can be encoded by the \emph{ground energy} of $H$.
For $\SampBQP$, one still uses $H_{\mathrm{circuit}}$ to certify the validity of the history state.
However, in this case, one needs to measure on the entire final state of the circuit, rather than a single output qubit,
which can no longer be encoded solely by the ground energy.
Our approach is to have the valid history state lie in the ground space of a different local Hamiltonian $H'_{\mathrm{circuit}}$  that has a large \emph{spectral} gap between its ground energy and excited ones.
It is hence guaranteed that any state with close-to-ground energy must also be close to the history state.
In other words, a certification of the energy $H'_{\mathrm{circuit}}$ could lead to a certification of the history state.
We construct such $H'_{\mathrm{circuit}}$ from $H_{\mathrm{circuit}}$ by using the \emph{perturbation} technique (e.g.,~\cite{kempe_kitaev_regev_2006}) with further restriction to X/Z terms. (\Cref{sec:LHXZ}.)


Another high-level difficulty in constructing a $\QPIP_1$ protocol for $\SampBQP$ is due to the distinction between the test part and the output part.
Specifically, we will certify the energy of $H'_{\mathrm{circuit}}$ to guarantee the underlying state is close to the valid history state.
However, this procedure could be vastly different from outputting a sample by measuring the final state of $\SampBQP$ circuits.
We design a \emph{cut-and-choose} protocol on multiple copies of the history state for both testing and outputting.
We also employ a variant of quantum \emph{de Finetti} theorem~\cite{Brandão2017}
to prevent potential cheating strategies by entangling different copies of history states.
(\Cref{sec:qpip1}.)

\vspace{2mm} \noindent  \emph{$\diamond$ Compile $\QPIP_1$ into $\QPIP_0$}:
A naive attempt is to directly apply Mahadev's protocol on the aforementioned $\QPIP_1$ protocol.
Unfortunately, the plain version of Mahadev's protocol does not yield favorable parameters by itself.
In fact, there are some recent results~\cite{arXiv:AlaChiHun19, arXiv:ChiaChungYam19} that provide a parallel repetition of Mahadev's original protocol for $\BQP$ with
much more favorable parameters.

However, we cannot directly make use of these parallel repetition results due to the subtle difference between protocols for $\BQP$ and $\SampBQP$.
One of the major difficulties here is still to deal with both the test part and the output part in $\SampBQP$ protocols.
However, because we are now in the computational setting, there is no longer any available quantum de Finetti theorem that is usually derived in the information-theoretic setting.
We end up developing a weaker version of  parallel repetition of Mahadev's protocol inspired by the technique from~\cite{arXiv:ChiaChungYam19}.
Due to the nature of parallel repetition in the computational setting, our analysis is much less modular and significantly involved for this part.  
More intuitions and detailed analysis are given in \Cref{sec:qpip0_all}.

\vspace{2mm} \noindent \textbf{(2) A generic compiler to upgrade $\QPIP_0$ protocols with blindness}. At a high-level, the idea is simple: we run the original protocol under a QFHE with the verifier's key. Intuitively, this allows the prover to compute his next message under encryption without learning the underlying verifier's message, and hence achieves blindness while preserving the properties of the original protocol.
One subtlety with this approach is due to the fact that the verifier is classical while the QFHE cipher text could depend on both quantum and classical data.
In order to make the classical verifier work in this construction, the ciphertext and the encryption/decryption algorithm needs to be classical when the underlying message is classical, which is fortunately satisfied by~\cite{mahadev_qfhe}.

A more subtle issue is to preserve the soundness.
In particular, compiled protocols with only one-time use of QFHE might (1) leak information about the circuit being evaluated during the homomorphic evaluation of QFHE ciphertexts (i.e., no \emph{circuit privacy});
or (2) fail to simulate original protocols upon receiving invalid ciphertexts.
We address these issues by letting the verifier switch to a fresh new key for each round of the protocol.
Details are given in \Cref{sec:BlindBQP2}.

\vspace{2mm} \noindent \textbf{Open Questions.} Our main focus is on the feasibility of the desired functionality and properties, which nevertheless leaves a big room for the improvement of efficiency.
Some of our parameter dependence inherits from previous works (e.g.~\cite{FOCS:Mahadev18a}), whereas some is due to our own construction. 
It will be extremely interesting to improve the parameter dependence with potentially new techniques. 

\Ethan{TODO add organization of paper}

% \begin{itemize}
%     \item Open questions, and also explain for some associated high-cost. specifically
%     \item T dependence. In general, improve the efficiency.  
%     \item negligible soundness error.  compare with classical? what's the state-of-the art.
% \end{itemize}

% \Ethan{This section is currently all rough draft. We'll probably rewrite almost all of it.}

% \Ethan{application: verifiable private constant round delegation}

% \Ethan{Below is my attempt to talk about it in my SoP}

% It was proven that BQP=BQIP\hannote{who  when and cite}. That is, if a quantum computer can efficiently solve a given decision problem, then it can also efficiently convince a classical machine of its solution. I'm generalizing this to arbitrary efficient quantum computations. The proof for decision problems involves the classical verifier reducing the problem to a local Hamiltonian instance; the quantum prover would then commit its certificate and act as the verifier’s trusted measurement device as put forth in ``Classical Verification of Quantum Computations" by Mahadev. It isn't as trivial as it may seem. Repeating the scheme for each qubit loses the information carried by entanglements and throws off the joint distribution between qubits. Simply measuring the entire output register instead is difficult to analyze. For decision problems, it’s not hard to argue that a malicious prover cannot do better than sending identical copies of some pure state unentangled with each other. That same reasoning doesn't apply here a priori. I've been trying to get a grasp on the particular structure of the local Hamiltonian reduction in order to better analyze it.

% \Ethan{Below is my attempt to talk about it in my research proposal}

% We are interested in delegating quantum computations from a classical client to an untrusted quantum server. Under this setting, the client would send the server a quantum circuit and an initial state. Then, through interaction with the honest server, the client obtains a measurement result as if he measured the true output of the circuit. If the server attempts to deceive the client, the client should reject it. The case where the circuit encodes a decision problem has been well-studied, and we're now trying to generalize those results to circuits with possibly many bits of output.

% If the circuit encodes a decision problem, then by considering adiabatic quantum computation, there exists a reduction to local Hamiltonian. Local Hamiltonian is QMA-complete, so there's a certificate for every yes-instance, and no valid certificates for any no-instances. An introduction can be found in Kitaev, Shen, and Vyalyi's "Classical and Quantum Computation". Furthermore, Biamonte and Love's "Realizable Hamiltonians for Universal Adiabatic Quantum Computers" states that these local Hamiltonians have very simple forms, which in turn implies that in order to check such certificates one only requires abilities to receive qubits and perform X/Z measurements. Based on this observation, Mahadev constructed a protocol in "Classical Verification of Quantum Computations" which, under the LWE assumption (a widely believed conjecture in quantum cryptography), allows the prover to commit qubits and act as the verfier's trusted X/Z measurement device. This solves the delegation of decision problem from a fully classical client to a quantum server.

% To generalize delegation of quantum computations to allow long output, simply repeating the known protocol for every output qubit doesn't work. The joint probability distribution between qubits would be incorrect due to entanglements. In fact, generally the output qubits encode sampling problems rather than decision problems. Furthermore, for decision problems one can argue that the prover's optimal strategy is to send identical copies of a certificate, so Chernoff bound can be applied, but said argument doesn't generalize to sampling problems either. To overcome these challenges, we start by modifying the local Hamiltonian construction so it is compatible with long output. We then analyze our protocol's soundness more carefully, before using Mahadev's result as a black box to solve the long output case too for fully classical clients.

% A possible application for our long output protocol is to make the computation not only verifiable, but also private in the sense of homomorphic encryptions. That is, the input is encrypted before being sent to the server. The server computes on the encrypted input, obtaining an encrypted output. The client then receives and decrypts the output. Here we can combine results from Mahadev's "Classical Homomorphic Encryption for Quantum Circuits" with our long output scheme. The client can simply send the homomorphic evaluation circuit to the server with the encrypted input.




\section{Delegation of Quantum Sampling Problems} \label{sec:samp_definition}

In this section, we formally introduce the task of delegation for quantum sampling problems. We start by recalling the complexity class $\SampBQP$ defined by Aaronson~\cite{aaronson_2013, Boson}, which captures the class of sampling problems that
are approximately solvable by polynomial-time quantum algorithms.


\begin{definition} [Sampling Problem]
    \label{dfn:sampling-problem}
    A \emph{sampling problem} is a collection of probability distributions $(D_x)_{x\in\set{0, 1}^*}$, one for each input string $x\in\set{0,1}^n$, where $D_x$ is a distribution over $\set{0,1}^{m(n)}$ for some fixed polynomial $m$.
\end{definition}

\begin{definition} [$\SampBQP$]
    $\SampBQP$ is the class of sampling problems $\left(D_x\right)_{x\in\set{0, 1}^*}$ that can be (approximately) sampled by polynomial-size uniform quantum circuits. Namely, there exists a Turing machine $M$ such that for every $n \in \bbN$ and $\eps \in (0,1)$, $M(1^n, 1^{1/\eps})$ outputs a quantum circuit $C$ in $\poly(n, 1/\eps)$ time such that for every $x \in \zo^n$, the output of $C(x)$ (measured in standard basis) is $\eps$-close to $D_x$ in the total variation distance. 
\end{definition}

Note that in the above definition, there is an accuracy parameter $\eps$ and the quantum sampling algorithm only requires to output a sample that is $\eps$-close to the correct distribution in time $\poly(n,1/\eps)$.
\cite{aaronson_2013, Boson} discussed multiple reasons for allowing the inverse polynomial error, such as to take into account the inherent noise in conceivable physical realizations of quantum computer.
On the other hand, it is also meaningful to require negligible error. As discussed, it is an intriguing open question to delegate quantum sampling problem with negligible error.

We next define what it means for a $\QPIP_\tau$ protocol\footnote{See Appendix~\ref{sec:qpip_def} for a formal definition of $\QPIP_\tau$.} to solve a $\SampBQP$ problem $\left(D_x\right)_{x\in\set{0, 1}^*}$.
Since sampling problems come with an accuracy parameter $\eps$, we let the prover $P$ and the verifier $V$ receive the input $x$ and $1^{1/\eps}$ as common inputs. 
Completeness is straightforward to define, which requires that when the prover $P$ is honest, the verifier $V$ should accept with high probability and output a sample $z$ distributed close to $D_x$ on input $x$. Defining soundness is more subtle. Intuitively, it requires that the verifier $V$ should never be ``cheated'' to accept and output an incorrect sample even when interacting with a malicious prover. We formalize this by a strong simulation-based definition, where we require that the joint distribution of the decision bit $d \in \set{\Acc, \Rej}$ and the output $z$ (which is $\bot$ when $d = \Rej$) is $\eps$-close (in either statistical or computational sense) to an ``ideal distribution'' $(d,z_{ideal})$, where $z_{ideal}$ is sampled from $D_x$ when $d = \Acc$ and set to $\bot$ when $d = \Rej$. Since the protocol receives the accuracy parameter $1^{1/\eps}$ as input to specify the allowed error, we do not need to introduce an additional soundness error parameter in the definition.

\begin{definition}
    \label{dfn:stats-secure-proto-sampbqp}
    Let $\Pi=(P, V)$ be a $\QPIP_\tau$ protocol.
    We say it is a protocol for the $\SampBQP$ instance $(D_x)_{x\in\zo^*}$ with completeness error $c(\cdot)$ and statistical (resp., computational) soundness if the following holds:
    \begin{itemize}
        \item On public inputs $1^\lambda$, $1^{1/\eps}$, and $x\in\zo^{\poly(\lambda)}$, $V$ outputs $(d, z)$ where $d\in\set{\Acc, \Rej}$.
            If $d=\Acc$ then $z\in\zo^{m(\abs{x})}$ where $m$ is given in \Cref{dfn:sampling-problem}, otherwise $z=\bot$.
        \item (Completeness):
            For all accuracy parameters $\eps(\lambda)=\frac{1}{\poly(\lambda)}$,
            security parameters $\lambda\in\bbN$,
            and $x\in\zo^{\poly(\lambda)}$,
            let $(d, z)\leftarrow(P, V)(1^\lambda, 1^{1/\eps}, x)$, then $d=\Rej$ with probability at most $c(\lambda)$.
        \item (Statistical soundness): For all cheating provers $P^*$,
            accuracy parameters $\eps(\lambda)=\frac{1}{\poly(\lambda)}$,
            sufficiently large $\lambda\in\bbN$, and $x\in\zo^{\poly(\lambda)}$,
            consider the following experiment:
            \begin{itemize}
                \item Let $(d, z)\leftarrow(P^*, V)(1^\lambda, 1^{1/\eps}, x)$.
                \item Define $z_{ideal}$ by
                $$\begin{cases}
                    z_{ideal}=\bot & \text{if } d=\Rej\\
                    z_{ideal}\leftarrow D_x & \text{if } d=\Acc
                \end{cases}$$.
            \end{itemize}
            It holds that $\norm{(d,z)-(d,z_{ideal})}_{\mathrm{TV}}\leq\eps$.
		\item (Computational soundness):
        For all cheating $\BQP$ provers $P^*$, $\BQP$ distinguishers $\mathsf{D}$, accuracy parameters $\eps(\lambda)=\frac{1}{\poly(\lambda)}$,
            sufficiently large $\lambda\in\bbN$, and all $x\in\zo^{\poly(\lambda)}$,
            let us define $d, z, z_{ideal}$ by the same experiment as above.
            It holds that $(d, z)$ is $\eps$-computationally indistinguishable to $(d, z_{ideal})$ over $\lambda$.
    \end{itemize}
\end{definition}

As in the case of $\BQP$, we are particularly interested in the case that $\tau = 0$, i.e., when the verifier $V$ is classical. In this case, we say that $\Pi$ is a CVQC protocol for the $\SampBQP$ problem $(D_x)_{x\in\zo^*}$.


\section{Construction of the $\QPIP_1$ Protocol for $\SampBQP$}
\label{sec:sampbqp_short}

As we mentioned in this introduction, we will employ the circuit \emph{history} state in the original construction of the Local Hamiltonian problem~\cite{kitaev2002classical} to encode the circuit information for $\SampBQP$.
However, there are distinct requirements between certifying the computation for $\BQP$ and $\SampBQP$ based on the history state.
For any quantum circuit $C$ on input $x$, the original construction for certifying $\BQP$\footnote{The original construction is for the purpose of certifying problems in QMA. We consider its simple restriction to problems inside BQP.} consists of local Hamiltonian $\Hin, \Hclock, \Hprop$, $\Hout$ where $\Hin$ is used to certify the initial input $x$, $\Hclock$ to certify the validness of the clock register,  $\Hprop$ to certify the gate-by-gate evolution according to the circuit description, and $\Hout$ to certify the final output.
In particular, the corresponding history state is in the ground space of $\Hin$, $\Hclock$, and $\Hprop$. Note that $\BQP$ is a decision problem and its outcome (0/1) can be easily encoded into the energy $\Hout$ on the single output qubit.
As a result, the outcome of $\BQP$ can simply be encoded by the \emph{ground energy} of $\Hin + \Hclock+\Hprop + \Hout$.

To deal with $\SampBQP$, we will still employ $\Hin, \Hclock$, and $\Hprop$ to certify the circuit's input, the clock register, and gate-by-gate evolution. However, in $\SampBQP$, we care about the entire final state of the circuit, rather than the energy on the output qubit.
%The history state remains in the ground space of $\Hin + \Hprop$.  
Our approach to certify the entire final state (which is encoded inside the history state) is to make sure that the history state is the unique ground state of $\Hin + \Hclock+ \Hprop$ and all other orthogonal states will have much higher energies.
Namely, we need to construct some $\Hin'+ \Hclock'+ \Hprop'$ with the history state as the unique ground state and with a large \emph{spectral} gap between the ground energy and excited energies.
It is hence guaranteed that any state with close-to-ground energy must also be close to the history state.
We remark that this is a different requirement from most local Hamiltonian constructions that focus on the ground energy.
We achieve so by using the \emph{perturbation} technique developed in~\cite{kempe_kitaev_regev_2006} for reducing the locality of Hamiltonian.
Another example of local Hamiltonian construction with a focus on the spectral gap can be found in~\cite{adiabatic}, where the purpose is to simulate quantum circuits by adiabatic quantum computation.

We need two more twists for our purpose.
First, as we will eventually measure the final state in order to obtain classical samples, we need that the final state occupies a large fraction of the history state. We can simply add dummy identity gates.
Second, as we are only able to perform $X$ or $Z$ measurement by techniques from~\cite{FOCS:Mahadev18a},
we need to construct X-Z only local Hamiltonians.
Indeed, this has been shown possible in, e.g.,~\cite{PhysRevA.78.012352}, which serves as the starting point of our construction.


We present the formal construction of our $\QPIP_1$ protocol $\PiSamp$ for $\SampBQP$ in our full version \cite{full-version}. The soundness and completeness of \Cref{ProtoQPIP1} is stated in the following theorem, whose proof is also deferred to \cite{full-version}.% The proof of 


 %For more details of the construction of our $\QPIP_1$ Protocol for $\SampBQP$, \Cref{ProtoQPIP1}, see \Cref{sec:sampbqp}. The soundness and completeness of \Cref{ProtoQPIP1} is stated in the following theorem.

\begin{thm}
    \label{QPIP1thm}
	$\PiSamp$ is a $\QPIP_1$ protocol for the $\SampBQP$ problem  $(D_x)_{x\in\set{0,1}^*}$ with negligible completeness error and is statistically sound\footnote{The soundness and completeness of a $\SampBQP$ protocol is defined in \Cref{dfn:stats-secure-proto-sampbqp}.} where the verifier only needs to do non-adaptive $X/Z$ measurements.
	%\XW{do we want precise error bound?}	
\end{thm}

\vspace{-3pt}

\section{$\SampBQP$ Delegation Protocol for Fully Classical Client}
\label{sec:qpip0_all}

%\hannote{In this section, we modify the construction of \cite{FOCS:Mahadev18a} to show that a $\QPIP_1$ protocol for $\SampBQP$ implies a $\QPIP_0$ for $\SampBQP$.}

% We then extend our scheme for $\QPIP_0$ using results from \cite{FOCS:Mahadev18a}.

In this section, we combine the core protocol from \cite{FOCS:Mahadev18a}, which we restate as Protocol~\ref{proto:urmila4}, with \Cref{QPIP1thm} to create a delegation protocol for $\SampBQP$ for fully classical clients. A direct composition results in Protocol~\ref{proto:qpip0_naive}. However, as we will see in \Cref{sec:urmila4}, Protocol~\ref{proto:urmila4} has a peculiar and weak guarantee, so \Cref{proto:qpip0_naive} don't have any reasonable completeness and soundness. To boost the soundness of Protocol~\ref{proto:qpip0_naive}, we run $m=\poly(\lambda)$ copies of Protocol~\ref{proto:qpip0_naive} in parallel and test on $m-1$ copies of them, resulting in Protocol~\ref{proto:QPIP0samp}. We use techniques from \cite{arXiv:ChiaChungYam19} to prove that Protocol~\ref{proto:QPIP0samp} has soundness error $1/\sqrt{m}$ in \Cref{thm:qpip0}.



% since Protocol~\ref{proto:urmila4} has a constant soundness error\hannote{too imprecise}\footnote{Protocol~\ref{proto:urmila4} doesn't have a formally defined soundness error. But intuitively, the verifier would accept a non-binding result will constant probability.}, Protocol~\ref{proto:qpip0_naive}  also has a constant soundness error, which means the outputted samples have constant computational distance to the ideal distribution, making the result not very useful.

%Subprotocol for Quantum Measurements
% \subsection{$\QPIP_0$ for BQP?}
% \hannote{Ethan old description before parallel rep.}
% As a warm-up, we  restate the construction in \cite{FOCS:Mahadev18a}, which shows that $\QPIP_1$ protocol for $\BQP$ implies a $\QPIP_0$ for $\BQP$.

% Let $\rho$ be an $n$-qubit state. Let $h$ be an n-bits string called the \emph{basis choice}. That is, $h_i=0$ indicates that the $i$-th qubit of $\rho$ is to be measured in the standard basis, while $i=1$ indicates Hadamard basis measurement instead. Let $D_{\rho, h}$ be the distribution of the corresponding measurement results.



\subsection{Mahadev's measurement protocol}\label{sec:urmila4}

Before introducing Mahadev's protocol, we introduce a notation.

\begin{definition}[$M_{XZ}(\rho,h)$]

	For any natural number $n$, $n$-bit string $h$, and $n$-qubit quantum state $\rho$, consider the following measurement procedure: measure the first qubit of $\rho$ in $X$ basis if $h_1=0$; measure the first qubit of $\rho$ in $Z$ basis if $h_1=1$.  Measure the second qubit of $\rho$ in $X$ basis if $h_2=0$; measure the second qubit of $\rho$ in $Z$ basis if $h_2=1$. Continue qubit-by-qubit until all $n$-qubits of $\rho$ are measured, where $i$-th qubit is measured in $X$ basis if $h_i=0$ and  $i$-th qubit is measured in the $Z$ basis if $h_i=1$.

	We denote the $n$-bit random variable corresponding to the measurement results as $M_{XZ}(\rho,h)$.

\end{definition}

In her groundbreaking work, Mahadev~\cite{FOCS:Mahadev18a} gives a $\QPIP_0$ protocol for $\BQP$ languages.
The core of this work is a 4-round $\QPIP_0$ protocol $\PiMeasure$ that lets a $\BQP$ machine ``commit a $XZ$ measurement" to a classical machine.
Intuitively, the verifier chooses a string $h$ specifying  the measurement he wants to make, and sends encoded $h$ \Ethan{what?} to the prover. The prover ``commits" to a state $\rho$ with the encoded $h$ and replies. The verifier then uniformly chooses between two options: do a \emph{testing round} or do a \emph{Hadamard round}. He sends his choice between the two to the prover and has the  prover reply. If the verifier chose testing, he checks the prover's two replies and rejects if he sees an inconsistency. If the verifier chose the Hadamard round, he calculates $M_{XZ}(\rho,h)$ based on the replies. Mahadev~\cite{FOCS:Mahadev18a} proves a ``binding" property of the above protocol: if a prover's strategy would always succeeds on the testing round, then that strategy must binds to a $\rho$, meaning that no matter what $h$ the verifier has picked at the beginning, the verifier can get $M_{XZ}(\rho,h)$  with the same $\rho$ if he had chosen the Hadamard round.

We now formally describe the interface of $\PiMeasure$ while omitting the implementation details.

%  that : if $h_i=0$, he wants a $X$ measurement on $i$-th qubit; if $h_i=1$, he wants a $Z$ measurement on $i$-th qubit.
% At the end of $\PiMeasure$, the verifier learns a measurement outcome $M_{XZ}(\rho, h)$.

\begin{protocol}{Mahadev's measurement protocol $\PiMeasure=(\PMeasure, \VMeasure)$}
	\label{proto:urmila4}

	Inputs:
	\begin{itemize}
		\item Common input: Security parameter $1^\lambda$ where $\lambda\in\bbN$.
		\item Prover's input: a state $\rho\in\cB^{\otimes n}$ for the verifier to measure.
		\item Verifier's input:
			the measurement basis choice $h \in \{0,1\}^n$
	\end{itemize}

	Protocol:
	\begin{enumerate}
		\item \label{step:measure1} The verifier generates a public and secret key pair $(pk, sk)\leftarrow\cVMeasure{1}(1^\lambda, h)$. It sends $pk$ to the prover.
		\item \label{step:measure2} The prover generates $(y, \sigma)\leftarrow\cPMeasure{2}(pk, \rho)$.
			$y$ is a classical ``commitment", and $\sigma$ is some internal state.
			He sends $y$ to the verifier.
		\item \label{step:measure3} The verifier samples $c\xleftarrow{\$}\zo$ uniformly at random and sends it to the prover. $c=0$ indicates a \emph{testing round}, while $c=1$ indicates a \emph{Hadamard round}.
		\item \label{step:measure4} The prover generates a classical string $a\leftarrow\cPMeasure{4}(pk, c, \sigma)$ and sends it back to the verifier.
		\item \label{step:output} If it is a testing round ($c=0$), then the verifier generates and outputs $o\leftarrow\cVMeasure{T}(pk, y, a)$ where $o\in\set{\Acc, \Rej}$.
			If it is a Hadamard round ($c=1$), then the verifier generates and outputs $v\leftarrow\cVMeasure{H}(sk, h, y, a)$.
	\end{enumerate}
\end{protocol}

The verifier only learns the measurement outcome on a Hadamard round.
The protocol achieves a ``binding" property that gives guarantees against cheating provers.

% \begin{definition}
% With loss of generality,  we can assume that the testing round verification is deterministic. We denote the set of strings $a$ \Ethan{Font makes this $a$ look screwy} accepted by the verifier in the testing round with $(pk,y)$ as $\Acc_{pk,y}$. \hannote{do  i actually use this}
% \end{definition}

\begin{lemma}[binding property of $\PiMeasure$]
	\label{lem:urmila-binding}
	Let $\PMeasureStar$ be a $\BQP$ cheating  prover for $\PiMeasure$ and $\lambda$ be the security parameter. Under the QLWE assumption, if $\PMeasureStar$ passes the testing round with probability $1-\negl(\lambda)$, then there exists some $\rho$ so that for all verifier's input $h \in \zo^n$, the verifier's outputs on the Hadamard round is computationally indistinguishable from $M_{XZ}(\rho, h)$.
	\iffalse    
	Suppose that for all $\lambda\in\bbN$ and $h\in\zo^*$ \Ethan{or $\zo^n$?},
	$\PMeasureStar$ passes the testing round with probability $1-\negl(\lambda)$.
	Then, under the QLWE assumption, there exists some $\rho$ so that for all $h$,
	The verifier's output on the Hadamard round is $\negl(\lambda)$-computationally indistinguishable from $M_{XZ}(\rho, h)$.
	\fi
\end{lemma}

We also mention a fact that will be useful later.
\begin{fact}
	\label{lem:trivial-4-round-strategy}
	There exist a prover strategy for $\PiMeasure$ that is accepted with probability $1-\negl(\lambda)$
\end{fact}

We now use $\PiMeasure$ to transform our $\QPIP_1$ Protocol for $\SampBQP$, $\PiSamp=(\PSamp, \VSamp)$, to a corresponding $\QPIP_0$ protocol $\PiNaive$.
Recall that in $\PiSamp$ the verifier takes $X$ and $Z$ measurements on the prover's message.
In $\PiNaive$ we let the verifier use $\PiMeasure$ to learn those measurement outcomes instead.

\begin{protocol}{Intermediate $\QPIP_0$ protocol $\PiNaive$ for a $\SampBQP$ problem $(D_x)_{x\in\set{0, 1}^*}$}
	\label{proto:qpip0_naive}

	Inputs:
	\begin{itemize}
		\item Security parameter $1^\lambda$ where $\lambda\in\bbN$
		\item Error parameter $\eps\in(0, 1)$
		\item Classical input $x\in\zo^n$ to the $\SampBQP$ instance
	\end{itemize}

	Protocol:
	\begin{enumerate}
		\item \label{step:naive1} The verifier chooses a $XZ$-measurement $h$ from the distribution specified in \stepref{qpip1-verify} of $\PiSamp$.
		\item \label{step:naive2} The prover prepares $\rho$ by running \stepref{qpip1-state-gen} of $\PiSamp$.
		\item \label{step:urmila-in-naive}
			The verifier and prover run $(\PMeasure(\rho), \VMeasure(h))(1^\lambda)$.
			\begin{enumerate}
				\item The verifier samples $(pk, sk)\leftarrow\cVNaive{1}(1^\lambda, h)$ and sends $pk$ to the prover.
				\item The prover runs $(y, \sigma)\leftarrow\cPNaive{2}(pk, \rho)$ and sends $y$ to the verifier.
					Here we allow the prover to abort by sending $y=\bot$, which does not benefit cheating provers but is useful for analysis.
				\item\label{step:c-urmila-in-naive} The verifier samples $c\leftarrow\xleftarrow{\$}\zo$ and sends it to the prover.
				\item The prover replies $a\leftarrow\cPNaive{4}(pk, c, \sigma)$.
				\item
					If it is a testing round, the verifier accepts or rejects based on the outcome of $\PiMeasure$.
					If it is a Hadamard round, the verifier obtains $v$.
			\end{enumerate}
		\item \label{step:naive-output} If it's a Hadamard round, the verifier finishes the verification step of Protocol~\ref{ProtoQPIP1} by generating and outputting $(d, z)$
	\end{enumerate}
\end{protocol}

There are several problems with using $\PiNaive$ as a $\SampBQP$ protocol. first, since the verifier doesn't get a sample if he had chosen the testing round in Step~\ref{step:c-urmila-in-naive}, so the protocol has  completeness error at least $1/2$. And since Protocol~\ref{proto:urmila4} does not check anything on the Hadamard round, a cheating prover can give up passing the testing round and breaks the commitment on the Hadamard round. Such cheating prover only has a constant $1/2$ probability of being caught, resulting in constant soundness error.


% Note that in $\PiNaive$ the verifier only learns the output in the Hadamard round.
% So $\PiNaive$ is not a $\QPIP_0$ with good completeness or soundness.


On the other hand, \Cref{proto:qpip0_naive} has a binding property like \Cref{proto:urmila4}, which we will use to achieve completeness and soundness in \Cref{sec:qpip0}.

\begin{lemma}[binding property of $\PiNaive$]
	\label{lem:naive-qpip0-binding}
	Let $\PNaiveStar$ be a cheating $\BQP$ prover for $\PiNaive$ and $\lambda$ be the security parameter.
	Under the QLWE assumption, if conditioned on $\PNaiveStar$ not aborting, $\PNaiveStar$ passes the testing round with probability $1-\negl(\lambda)$,
	then the verifier's output in the Hadamard round is $O(\eps)$-computationally indistinguishable from $(d, z_{ideal})$.
\end{lemma}
\begin{proof}
	Consider the following reduction to another cheating $\BQP$ prover for $\PiNaive$ that is perfect and non-aborting.

	We define $\Pstar$ as follows.
	For the second message, run $(y, \sigma)\leftarrow\cPNaiveStar{2}(pk, \rho)$.
	If $y\ne\bot$, then reply $y$;
	else, run the corresponding step of the dummy strategy as in \Cref{lem:trivial-4-round-strategy} and reply with its results.
	For the fourth message, if $y\ne\bot$, run and reply with $a\leftarrow\cPNaiveStar{4}(pk, c, \sigma)$;
	else, continue the dummy strategy.

	Observe that this $\Pstar$ passes testing round with overwhelming probability by construction,
	so we can apply \Cref{lem:urmila-binding} to the $\PiMeasure$ call to use its binding property (\Cref{lem:urmila-binding}).
	That is, there exists some $\rho$ such that $v=M_{XZ}(\rho, h)$.
	Combining it with $\PiSamp$'s soundness (\Cref{QPIP1thm}),
	we see that $(d', z')\leftarrow(\Pstar, \VNaive)(1^\lambda, 1^{1/\varepsilon}, x)$ is $\eps$-computationally indistinguishable to $(d', z_{ideal}')$.

	Now we relate $(d', z')$ back to $(d, z)$.
	First, for the case that $\PNaiveStar$ aborts, since dummy strategy will be rejected with high probability in Hadamard round,
	we have $(d', z')\approx_{c, O(\eps)}(\Rej, \bot)=(d, z)$ conditioned on $\PNaiveStar$ aborts.
	For the other case, conditioned on $\PNaiveStar$ not aborts, clearly $(d, z)=(d', z')$.
	So we have $(d, z)\approx_{c, O(\varepsilon)}(d', z')\approx_{c, \negl(\lambda)}(d', z_{ideal}')$,
	which in turn implies $\abs{d-d'}\leq O(\eps)$
	and $(d, z_{ideal})\approx_{c, O(\eps)}(d', z_{ideal}')$.
	Combining everything, we conclude that $(d, z)\approx_{c, O(\eps)}(d, z_{ideal})$.
\end{proof}

\subsection{$\QPIP_0$ protocol for $\SampBQP$} \label{sec:qpip0}
%Parallel Repetition of the Measurement Subprotocol

\iffalse
The following protocol is a $\QPIP_0$ protocol for $\SampBQP$

$\forall c\in\bbN$ Soundness = $O(T^{-c})$

Given inverse poly p(T), we can parameterize the protocol to have soundness p(T)
\fi

%    \label{ProtoQPIP1}

We now introduce our $\QPIP_0$ protocol $\PiSampZ$ for $\SampBQP$.
It is essentially a $m$-fold parallel repetition of $\PiNaive$.
Instead of having each copy running testing round or Hadamard round uniformly at random, we randomly pick one copy to run Hadamard round to get our samples, and run testing round on all other $m-1$ copies.
In the description of our protocol below, we describe $\PiNaive$ and $\PiMeasure$ in details  in order to introduce notations that we need in our analysis.

\begin{protocol}{$\QPIP_0$ protocol $\PiSampZ$ for $\SampBQP$}
	\label{proto:QPIP0samp}

	Inputs:
	\begin{itemize}
		\item Security parameter $1^\lambda$ for $\lambda\in\bbN$.
		\item Accuracy parameter $1^{1/\eps}$ for the $\SampBQP$ instance
		\item Input $x\in\zo^{\poly(\lambda)}$ for the $\SampBQP$ instance
	\end{itemize}

	Ingredient: Let $m=O(1/\eps^2)$ be the number of parallel repetitions to run.

	Protocol:
	\begin{enumerate}
		\item The verifier generates $m$ independently copies of basis choices $\vec{h}=(h_1,\ldots,h_m)$ as in \stepref{naive1} of $\PiNaive$.
		\item The prover prepares $\rho^{\otimes m}$; each copy of $\rho$ is prepared as in \stepref{naive2} of $\PiNaive$.
		\item The verifier generates $m$ key pairs for $\PiMeasure$, $\vec{pk}=(pk_1,\ldots,pk_m)$ and $\vec{sk}=(sk_1,\ldots,sk_m)$, as in \stepref{measure1} of $\PiMeasure$.
			It sends $\vec{pk}$ to the prover.
		\item The prover generates $\vec{y}=(y_1,\ldots,y_m)$ and $\sigma$ as in \stepref{measure2} of $\PiMeasure$.
			It sends $\vec{y}$ to the verifier.
		\item The verifier samples $r\xleftarrow{\$}[m]$ which is the copy to run Hadamard round for.
			For $1\leq i\leq m$, if $i\ne r$ then set $c_i\leftarrow 0$, else set $c_i\leftarrow 1$.
			It sends $\vec{c}=(c_1,\ldots,c_m)$ to the prover.
		\item The prover generates $\vec{a}$ as in \stepref{measure4} of $\PiMeasure$, and sends it back to the verifier.
		\item \label{step:multi-testing}
			The verifier computes the outcome for each round as in \stepref{naive-output} of $\PiNaive$.
			If any of the testing round copies are rejected, the verifier outputs $(\Rej, \bot)$.
			Else, it outputs the result from the Hadamard round copy.
	\end{enumerate}
\end{protocol}

\begin{theorem}\label{thm:qpip0}
	Assuming the QLWE assumption, $\PiSampZ$ is a $\QPIP_0$ protocol for all problems in $\SampBQP$ with negligible  completeness error and computational soundness.
\end{theorem}


The intuition behind this theorem is that the we send out $m$ independent copies of the naive $\QPIP_0$ and do testing on randomly chosen $m-1$ copies of them, and do Hadamard round and calculate results from the remaining one copy. Therefore if the prover want to cheat, i.e. send something not binding in the Hadamard round copy, he would be caught with probability $1-1/m$. However, the prover might create correlations between copies that let him cheat with probability much higher than $1/m$. This is a common problem in trying to create parallel repetition protocols. We are able to bound the soundness error as $O(1/\sqrt{m})$. To prove that, we use the following partition lemma from \cite{arXiv:ChiaChungYam19}, which intuitively says that for each $i\in[m]$, there exist two efficient ``projectors" \footnote{Actually they are not projectors, but for the simplicity of this discussion let's assume they are.} $G_{0,i}$ and $G_{1,i}$ in the prover's internal space with $G_{0,i}+G_{1,i} \approx Id$.  $G_{0,i}$ represents the subspace that are rejected with high probability in the testing round of $i$-th copy; $G_{1,i}$  represents the subspace accepted with ``high probability"\footnote{This is also not technically correct as we will discuss later.} in the testing round of $i$-th copy.  $G_{0,i}$ and $G_{1,i}$ need to be efficient because $\PiMeasure$ only gives computational security. Once we have $G_{0,i}$ and $G_{1,i}$, we can split up the prover's internal state $\ket{\psi}$ like conditional probabilistic events.  We  split up  $\ket{\psi}$ into $G_{0,1}\ket{\psi}+ G_{0,2}G_{1,1}\ket{\psi} + G_{0,3}G_{1,2}G_{1,1}\ket{\psi} + \dots +  G_{0,m} G_{1,m-1}\dots G_{1,2}G_{1,1}\ket{\psi} + G_{1,m} G_{1,m-1}\dots G_{1,2}G_{1,1}\ket{\psi}$, where for example, $G_{0,2}G_{1,1}\ket{\psi}$ intuitively corresponds to passing the testing round of the first copy then getting rejected on the testing round of the second copy. Since each of the  substates in  $\{G_{0,1}\ket{\psi}, G_{0,2}G_{1,1}\ket{\psi} , G_{0,3}G_{1,2}G_{1,1}\ket{\psi} , \dots ,  G_{0,m} G_{1,m-1}\dots G_{1,2}G_{1,1}\ket{\psi} \}$ all have  probability $1-1/m$ of being caught by one of the $m-1$ testing rounds, they have well-bounded 
 probability of being accepted in Step~\ref{step:multi-testing} of Protocol~\ref{proto:QPIP0samp}, so the output in  Step~\ref{step:multi-testing} would mostly corresponds to $G_{1,m} G_{1,m-1}\dots G_{1,2}G_{1,1}\ket{\psi}$. However, since $G_{1,i}$ don't commute with $G_{1,j}$, $G_{1,m} G_{1,m-1}\dots G_{1,2}G_{1,1}\ket{\psi}$ is only binding for the $m$-th copy, so this simplified strategy is not quite correct. We were able to fix this issue with a more careful split up of $\ket{\psi}$. Also as a careful reader might have noticed, the prover's space don't happen to split nicely into parts of very high accept probability in the testing round and parts with very low accept probability, as there will always be something in the middle. In \cite{arXiv:ChiaChungYam19} this is solved by doing eigenvalue estimation to calculate the accept probability, then split the space into parts that are accepted with probability higher or lower than a small threshold $\gamma$. However, states with accept probability really close to the threshold $\gamma$ can not be classified, so we need to average over randomly chosen $\gamma$ to have $G_{0,i}+G_{1,i} \approx Id$.

 %but there will always be some states  therefore we introduce  an extra parameter $\gamma$ that corresponds to the probability cut-off line that to distinguish the accepted and rejected part, and we need to average over $\gamma$.

 % We split the state sub-normalized states corresponding to the prover cheating on different copies and show that the sum of probabilities corresponding to bad states can't be more than $1/m$. {The actual proof is more complicated than this sketch because the $G$'s don't commute with each other, so we need to handle the splitting carefully.}

 %For example, $G_{0,3} G_{1,5} \ket{\psi}$ corresponds to the events of passing the testing round on the fifth copy then fails the testing round on the third copy.

 %Let $S_m$ be sets of $\{c\}$ such that only one of the $c_i=1$.
\begin{lemma}[partition lemma; revision of Lemma 3.5 of \cite{arXiv:ChiaChungYam19}\footnote{$G_{0}$ and $G_{1}$ of this version are created from doing $G$ of \cite{arXiv:ChiaChungYam19} and post-select on the $th$ register being $0$ or $1$ then discard $ph,th,in$. Property~\ref{property:partition-err} corresponds to Property~1. Property~\ref{property:partition-testing} corresponds to Property~4, with $2^{m-1}$ changes to $m-1$ because we only have $m$ possible choices of $\{c\}$. Property~\ref{property:partition-binding} corresponds to Property~5. Property~\ref{property-partition-norm-sum} comes from the fact that $G_0$ and $G_1$ are post-selections of orthogonal results of the same $G$.}]\label{lem:partition2}
	Let $\lambda$ be the security parameter and  $\eps$ be the accuracy parameter. Let $(U_0,U)$ be a prover's strategy in a $m$-fold parallel repetition of $\PiMeasure$, where $U_0$ is how the prover generates $\vec{y}$ on the second message, and $U$ is how the prover generates $\vec{a}$ on the fourth message. Denote the string $0^{i-1}10^{m-i} \in \zo^m $ as $e_i$, which corresponds to Hadamard round on the $i$-th copy and testing round on all others. Let $\gamma_0 \in[0,1]$, and $T\in \mathbb{N}$ such that $\gamma_0=\poly(\eps)$ and $T=1/\poly(\eps)$.

	For all $i\in[m]$, $\gamma \in \L\{\frac{\gamma_0}{T},\frac{2\gamma_0}{T},\dots,\frac{T\gamma_0}{T}\R\}$, there exist two efficient quantum circuit $G_{0,i,\gamma}$ and $G_{1,i,\gamma}$ such that for all (possibly sub-normalized) $\poly(\lambda)$-qubits quantum state $\ket{\psi}_{\regX,\regZ}$,  

	\begin{align}
		G_{0,i,\gamma}\ket{\psi}_{\regX,\regZ} \defeq& \ket{\psi_{0,i,\gamma}}_{\regX,\regZ} \\ G_{1,i,\gamma}\ket{\psi}_{\regX,\regZ} \defeq& \ket{\psi_{1,i,\gamma}}_{\regX,\regZ}  \\
		\ket{\psi}_{\regX,\regZ} =&   \ket{\psi_{0,i,\gamma}}_{\regX,\regZ}+ \ket{\psi_{1,i,\gamma}}_{\regX,\regZ}+\ket{\psi_{err,i,\gamma}}_{\regX,\regZ}
	\end{align}

	% $$ \ket{\psi}_{\regX,\regZ} =  G_{0,i,\gamma}\ket{\psi}_{\regX,\regZ}+ G_{1,i,\gamma}\ket{\psi}_{\regX,\regZ}+\ket{\psi_{err}}_{\regX,\regZ},$$




	Note that $G_{0,i,\gamma}$ and $G_{1,i,\gamma}$ has failure probabilities, and this is reflected by the fact that $\ket{\psi_{0,i,\gamma}}_{\regX,\regZ}$ and $\ket{\psi_{1,i,\gamma}}_{\regX,\regZ}$ are  sub-normalized. $G_{0,i,\gamma}$ and $G_{1,i,\gamma}$ depend on $(U_0,U)$ and $\vec{pk},\vec{y}$.



	Furthermore, the following properties are satisfied for all $i\in[m]$.
	%
	\begin{enumerate}
		\item \label{property:partition-err}  $$\E_{\gamma}\|\ket{\psi_{err,i,\gamma}}_{\regX,\regZ}\|^2 \leq \frac{6}{T}+\negl(\lambda),$$

			where the averaged is over uniformly sampled $\gamma$. This also implies
			\begin{align}
				\E_{\gamma}\|\ket{\psi_{err,i,\gamma}}_{\regX,\regZ}\| \leq \sqrt{\frac{6}{T}}+\negl(\lambda)
			\end{align}
			by Cauchy's inequality.

		\item \label{property:partition-testing}
			For all $\vec{pk}$, $\vec{y}$, fixed $\gamma$, and  $j\neq i$, we have

			%      \begin{align*}
			%  \Pr\left[M_{\regX_i}\circ U\frac{\ket{\{c\}}_{\regC}\ket{\psi_{0,i,\gamma}}_{\regX,\regZ}}{\|\ket{\psi_0}_{\regX,\regZ}\|}\in \Acc_{pk_i,y_i}\right]\leq (m-1)\gamma+\negl(\secpar).
			%  \end{align*}

			%  Define
			%  $$\ket{\widetilde{\psi_{0,i,\gamma}}}\defeq U\frac{\ket{\{c\}}_{\regC}\ket{\psi_{0,i,\gamma}}_{\regX,\regZ}}{\|\ket{\psi_0}_{\regX,\regZ}\|}.$$
			%  We have
			%  \begin{align}
			%      \vev{\widetilde{\psi_{0,i,\gamma}}|P_{i,pk_i,y_i,acc}|\widetilde{\psi_{0,i,\gamma}}} \leq (m-1)\gamma+\negl(\secpar),
			%  \end{align}
			\begin{align}
				\norm{ P_{i,pk_i,y_i,acc} \circ U\frac{\ket{e_j}_{\regC}\ket{\psi_{0,i,\gamma}}_{\regX,\regZ}}{\|\ket{\psi_0}_{\regX,\regZ}\|}}^2 \leq (m-1)\gamma_0+\negl(\lambda),
			\end{align}


			where $P_{i,pk_i,y_i,acc}$ are projector to the states that $i$-th testing round accepts with $pk_i,y_i$, including the last measurement the prover did before sending $\vec{a}$.  This means that $\ket{\psi_{0,i,\gamma}}$ is rejected by the $i$-th testing round with high probability.


		\item \label{property:partition-binding}

			% $\{c\}\in S_m$ such that $c_i = 0$
			For all $\vec{pk}$, $\vec{y}$, fixed $\gamma$, and $j\neq i$, there exists an efficient quantum algorithm $\ext_i$ such that

			\begin{align}
				\norm{P_{i,pk_i,y_i,acc} \circ \ext_i\left(\frac{\ket{e_j}_{\regC}\ket{\psi_{1,i,\gamma}}_{\regX,\regZ}}{\|\ket{\psi_1}_{\regX,\regZ}\|}\right)}^2 =1-\negl(\lambda).
			\end{align}

			% \begin{align*}  
			%   \Pr\left[M_{\regX_i}\circ \ext_i\left(\frac{\ket{\{c\}}_{\regC}\ket{\psi_{1,i,\gamma}}_{\regX,\regZ}}{\|\ket{\psi_1}_{\regX,\regZ}\|}\right)\in \Acc_{pk_i,y_i}\right]=1-\negl(\secpar).
			%   \end{align*}
			This will imply that    $\ket{\psi_{1,i,\gamma}}$ is binding to the $i$-th Hadamard round.

		\item \label{property-partition-norm-sum}
			For all $\gamma$,
			\begin{align}
				\norm{\ket{\psi_{0,i,\gamma}}}^2+ \norm{\ket{\psi_{1,i,\gamma}}}^2 \leq  \norm{\ket{\psi}}^2
			\end{align}


			% \item \hannote{added by me..not needed?}

			% $$\vev{\psi_{0,i,\gamma}|\psi_{1,i,\gamma}} = \negl(n)$$



	\end{enumerate}
\end{lemma}







% \begin{proof}
% Step~\ref{step:sum-ob}:
% \begin{align}
%   ( G_{0,i,\gamma}+G_{1,i,\gamma}) \ket{\psi}_{\regX,\regZ}
%   &= \ket{\psi_{0,i,\gamma}}_{\regX,\regZ}+\ket{\psi_{1,i,\gamma}}_{\regX,\regZ}
% \end{align}
% \end{proof}

We also need the following technical lemma.
\begin{lemma}\label{lem:samp-tech}
	For any state $\ket{\psi}$,  $\ket{\phi}$ and projectors $\{P_z\}$ such that $\sum_z P_z \leq Id$ and $P_z P_{z'} =0 $ for all $z\neq z'$, we have
	$$  \sum_z |\vev{\psi|P_z|\phi}| \leq \norm{\psi}\norm{\phi} $$
\end{lemma}
\begin{proof}
	\begin{align}
		\sum_z |\vev{\psi|P_z|\phi}| =&\sum_z|\vev{\psi|P_zP_z|\phi}| \nn \\
		\leq& \sum_z \norm{\bra{\psi}P_z} \norm{ P_z\ket{\phi}} \nn \\
		\leq&  \sqrt{\sum_z \norm{P_z\ket{\psi}}^2} \sqrt{\sum_z\norm{P_z\ket{\phi}}^2} \nn \\
		\leq& \sqrt{\norm{\sum_z P_z\ket{\psi}}^2 } \sqrt{\norm{\sum_z P_z\ket{\phi}}^2 } \nn \\
		\leq & \norm{\ket{\psi}}\norm{\ket{\phi}},
	\end{align}
	where we used Cauchy's inequality on the second and third line and $P_z P_{z'} =0 $ on the fourth line.
\end{proof}

Now we are ready to prove Theorem~\ref{thm:qpip0}

\Ethan{Try to cut the proof into two: point-wise and average arguments. Figure out what does z-good mean intuitively.

Also more comments/remarks/discussions. Subnormalized state and stuff.}

\begin{proof}

	Completeness is trivial.


	For soundness, We begin by considering the state $\ket{\psi}$ the prover in $\PiSampZ$ holds before he receives $\vec{c}$. We denote the corresponding Hilbert space as $H_{\regX,\regZ}$.

	For all $k\leq m$, $d\in \zo^k$, \Ethan{and for all $\gamma$?} and $\ket{\psi} \in H_{\regX,\regZ}$, define

	$$\ket{\psi_{d,\gamma}}\defeq G_{d_k,k,\gamma}G_{d_{k-1},k-1,\gamma}\cdots G_{d_2,2,\gamma} G_{d_1,1,\gamma} \ket{\psi}$$

	By Lemma~\ref{lem:partition2}, we have  

	\begin{align} \label{eq:partition-string}
		\ket{\psi} =& \ket{\psi_{0,\gamma}}+\ket{\psi_{1,\gamma}}+\ket{\psi_{err,1,\gamma}} \nn \\
		=& \ket{\psi_{0,\gamma}}+\ket{\psi_{10,\gamma}}+\ket{\psi_{11,\gamma}}+\ket{\psi_{err,1,\gamma}}+\ket{\psi_{err,2,\gamma}} \nn \\
		=& \ket{\psi_{0,\gamma}}+\ket{\psi_{10,\gamma}}+\ket{\psi_{110,\gamma}}+\cdots+\ket{\psi_{1^{m-1}0,\gamma}}+\ket{\psi_{1^{m-1}1,\gamma}} \nn \\
		&+\ket{\psi_{err,1,\gamma}}+\ket{\psi_{err,2,\gamma}}+\cdots+\ket{\psi_{err,m,\gamma}},
	\end{align}

	where we abuse the notation and use $\ket{\psi_{err,i,\gamma}}$ to denote the error state we get from decomposing $\ket{\psi_{1^{i-1},\gamma}}$. %\hannote{Note that the prover's state is actually a mixed state. In particular, $\ket{\psi}$ depends on $pk$ and we are considering the mixed state averaged over the verifier's choice of $pk$.}

	By Property~\ref{property-partition-norm-sum} of Lemma~\ref{lem:partition2}, we have
	\begin{align} \label{eq:bad-term-sum}
		\norm{\ket{\psi}}^2 \geq& \norm{\ket{\psi_{0,\gamma}}}^2+\norm{\ket{\psi_{1,\gamma}}}^2 \nn \\
		\geq& \norm{\ket{\psi_{0,\gamma}}}^2+
		\norm{\ket{\psi_{10,\gamma}}}^2+ \norm{\ket{\psi_{11,\gamma}}}^2 \nn \\
		\geq& \norm{\ket{\psi_{0,\gamma}}}^2+
		\norm{\ket{\psi_{10,\gamma}}}^2+ \norm{\ket{\psi_{110,\gamma}}}^2 +\cdots  \nn \\
		&+ \norm{\ket{\psi_{1^{m-1}0,\gamma}}}^2+ \norm{\ket{\psi_{1^{m-1}1,\gamma}}}^2
	\end{align}

	Denote the projector in $H_{\regX,\regZ}$ corresponding to outputting string $z$ when doing Hadamard on $i$-th copy as

	$$P_{acc,i,z}.$$
	Note that $P_{acc,i,z}$ also depends on $\vec{pk}, \vec{y}$, and $(sk_i, h_i)$ since it includes the measurement the prover did before sending $\vec{a}$,  verifier's checking on $(m-1)$ copies of testing rounds, and  the verifier's final computation from $(sk_i,h_i,y_i,a_i)$.

	Since the verifier only accepts if all $(m-1)$ copies of testing rounds accepts, for all $j\neq i$,

	$$P_{acc,i,z}=P_{acc,i,z}P_{j,pk_j,y_j,acc}.$$

	And therefore by Property~\ref{property:partition-testing} of Lemma~\ref{lem:partition2}, we have that for all $j <i-1$   %\hannote{problem with $e_j$ or $e_i$}

	\begin{align} \label{eq:rejected-d}
		\norm{P_{acc,i,z} U \ket{e_i, \psi_{1^j0,\gamma}}}^2
		=& \norm{P_{acc,i,z}P_{j,pk_j,y_j,acc} U \ket{e_i}\, G_{0,j+1,\gamma}\ket{\psi_{1^j,\gamma}}  }^2 \nn \\
		\leq& \norm{P_{j,pk_j,y_j,acc} U \ket{e_i}\, G_{0,j+1,\gamma}\ket{\psi_{1^j,\gamma}}  }^2 \nn \\
		\leq& (m-1)\gamma_0+\negl(n)
	\end{align}


	Denote the string $0^{i-1}10^{m-i} \in \zo^m $ as $e_i$. The output string corresponding to $\ket{\psi} \in H_{\regX,\regZ}$ when $c=e_i$ is then




	$$z_i\defeq \E_{pk,y} \sum_z \proj{z} \cdot \vev{e_i,\psi|U^\dag P_{acc,i,z} U|e_i,\psi},$$
	where $\ket{e_i,\psi}=\ket{e_i}_\regC\ket{\psi}_{\regX,\regZ}$ and $U$ is the unitary the prover applies on the last round. Note that we have averaged over $\vec{pk}, \vec{y}$ where as previously everything has fixed $\vec{pk}$ and $\vec{y}$.
	\Ethan{Here $\psi$ and $y$ are implicitly dependent on $pk$. Might need to clarify what the expected value over $y$ means.}

	Since $\vec{c}$ is drawn  from $e_i$ with uniformly random $i\in [m]$, we have
$$ z=\frac{1}{m} \sum_i z_i= \frac{1}{m} \sum_i \sum_z \proj{z} \cdot \vev{e_i,\psi|U^\dag P_{acc,i,z} U|e_i,\psi},$$
	where we represent the random variable $z$ as a real non-negative diagonal matrix, with the matrix entries begin \Ethan{typo?} probabilities. Note that $z$ is ``sub-normalized", i.e. $\tr (z) \leq 1 \text{}$, and $\tr(z)$ equals to the probability of getting accepted.






Define

\begin{align}
	z_{good,i}=\E_\gamma \sum_z \proj{z} \cdot \vev{e_i,\psi_{1^{i-1}1,\gamma}|U^\dag P_{acc,i,z} U|e_i,\psi_{1^{i-1}1,\gamma}}
\end{align}

Splitting $\ket{\psi}$ with Equation~\ref{eq:partition-string},
%\hannote{gamma..}

\begin{align}
	\ket{\psi}=& \L. \ket{\psi_{0,\gamma}}+\ket{\psi_{10,\gamma}}+\ket{\psi_{110,\gamma}}+\cdots+\ket{\psi_{1^{i-1}0,\gamma}}+
	\ket{\psi_{1^{i-1}1,\gamma}} \R. \nn \\
	 +& \L.\ket{\psi_{err,1,\gamma}}+\ket{\psi_{err,2,\gamma}}+\cdots+\ket{\psi_{err,i,\gamma}}\R. \nn \\
	 =& \sum_{j=0}^{i-1} \ket{\psi_{1^j0,\gamma}} +\ket{\psi_{1^i,\gamma}} +\sum_{j=1}^{i}\ket{\psi_{err,j,\gamma}}
\end{align}


we have

\begin{align}
	z_i =& \sum_z \proj{z} \cdot \vev{e_i,\psi|U^\dag P_{acc,i,z} U|e_i,\psi} \nn \\
	=& \sum_z \proj{z} \L[\sum_{k=0}^{i-1} \bra{\psi_{1^k0,\gamma}} +\bra{\psi_{1^i,\gamma}} +\sum_{k=1}^{i}\bra{\psi_{err,k,\gamma}} \R]U^\dag  P_{acc,i,z} U\nn \\
	&\L[ \sum_{j=0}^{i-1} \ket{\psi_{1^j0,\gamma}} +\ket{\psi_{1^i,\gamma}} +\sum_{j=1}^{i}\ket{\psi_{err,j,\gamma}}\R]  \nn \\
	=& \E_\gamma \sum_z \proj{z} \L[\sum_{k=0}^{i-1} \bra{\psi_{1^k0,\gamma}} +\bra{\psi_{1^i,\gamma}} +\sum_{k=1}^{i}\bra{\psi_{err,k,\gamma}} \R]U^\dag  P_{acc,i,z} U\nn \\
	&\L[ \sum_{j=0}^{i-1} \ket{\psi_{1^j0,\gamma}} +\ket{\psi_{1^i,\gamma}} +\sum_{j=1}^{i}\ket{\psi_{err,j,\gamma}}\R]  \nn \\
	=& z_{good,i}+ \E_\gamma \sum_z \proj{z} \L[\sum_{k=0}^{i-1} \bra{\psi_{1^k0,\gamma}}U^\dag  P_{acc,i,z}U   \sum_{j=0}^{i-1} \ket{\psi_{1^j0,\gamma}}+
	\sum_{k=0}^{i-1} \bra{\psi_{1^k0,\gamma}}U^\dag  P_{acc,i,z}U \ket{\psi_{1^i,\gamma}}  \R. \nn \\
	  +&  \sum_{k=0}^{i-1} \bra{\psi_{1^k0,\gamma}}U^\dag  P_{acc,i,z}U\sum_{j=1}^{i}\ket{\psi_{err,j,\gamma}}
	+\bra{\psi_{1^i,\gamma}} U^\dag  P_{acc,i,z}U \sum_{j=0}^{i-1} \ket{\psi_{1^j0,\gamma}}
	\nn \\
	+&  \bra{\psi_{1^i,\gamma}} U^\dag  P_{acc,i,z}U \sum_{j=1}^{i}\ket{\psi_{err,j,\gamma}}
	+\sum_{k=1}^{i}\bra{\psi_{err,k,\gamma}} U^\dag  P_{acc,i,z}U  \sum_{j=0}^{i-1} \ket{\psi_{1^j0,\gamma}}
	\nn \\
	+&\L.   \sum_{k=1}^{i}\bra{\psi_{err,k,\gamma}} U^\dag  P_{acc,i,z}U \ket{\psi_{1^i,\gamma}} +\sum_{k=1}^{i}\bra{\psi_{err,k,\gamma}} U^\dag  P_{acc,i,z}U \sum_{j=1}^{i}\ket{\psi_{err,j,\gamma}} \R] , \nn     
	%=& z_{good,i} +(\text{terms with } \psi_{1^j0},\, j\neq i ) + (\text{terms with } \psi_{1^{i-1}0}) +(\text{terms with }err )
\end{align}

%  U^\dag  P_{acc,i,z}U

% \sum_{j=0}^{i-1} \ket{\psi_{1^j0,\gamma}}
% \ket{\psi_{1^i,\gamma}}
% \sum_{j=1}^{i}\ket{\psi_{err,j,\gamma}}  

 where we omitted writing out $e_i$ starting the second line. We have
\begin{align} \label{eq:zi-zgoodi}
	&\tr|z_i-z_{good,i}|   \nn \\
	\leq&  \sum_z  \L| \E_\gamma \L[\sum_{k=0}^{i-1} \bra{\psi_{1^k0,\gamma}}U^\dag  P_{acc,i,z}U   \sum_{j=0}^{i-1} \ket{\psi_{1^j0,\gamma}}+
	\sum_{k=0}^{i-1} \bra{\psi_{1^k0,\gamma}}U^\dag  P_{acc,i,z}U \ket{\psi_{1^i,\gamma}}  \R. \R. \nn \\
	  +&  \sum_{k=0}^{i-1} \bra{\psi_{1^k0,\gamma}}U^\dag  P_{acc,i,z}U\sum_{j=1}^{i}\ket{\psi_{err,j,\gamma}}
	+\bra{\psi_{1^i,\gamma}} U^\dag  P_{acc,i,z}U \sum_{j=0}^{i-1} \ket{\psi_{1^j0,\gamma}}
	\nn \\
	+&  \bra{\psi_{1^i,\gamma}} U^\dag  P_{acc,i,z}U \sum_{j=1}^{i}\ket{\psi_{err,j,\gamma}}
	+\sum_{k=1}^{i}\bra{\psi_{err,k,\gamma}} U^\dag  P_{acc,i,z}U  \sum_{j=0}^{i-1} \ket{\psi_{1^j0,\gamma}}
	 \nn \\
	+&\L.\L. \sum_{k=1}^{i}\bra{\psi_{err,k,\gamma}} U^\dag  P_{acc,i,z}U \ket{\psi_{1^i,\gamma}}
	+\sum_{k=1}^{i}\bra{\psi_{err,k,\gamma}} U^\dag  P_{acc,i,z}U \sum_{j=1}^{i}\ket{\psi_{err,j,\gamma}} \R]\R| \nn \\  %%%%%%%%%%%
	\leq&  \sum_z   \E_\gamma \L[\sum_{k=0}^{i-1} \sum_{j=0}^{i-1} \L| \bra{\psi_{1^k0,\gamma}}U^\dag  P_{acc,i,z}U    \ket{\psi_{1^j0,\gamma}} \R|+
	2 \sum_{k=0}^{i-1} \L|\bra{\psi_{1^k0,\gamma}}U^\dag  P_{acc,i,z}U \ket{\psi_{1^i,\gamma}} \R|  \R.  \nn \\
	  +&  2 \sum_{k=0}^{i-1}\sum_{j=1}^{i}\L| \bra{\psi_{1^k0,\gamma}}U^\dag  P_{acc,i,z}U\ket{\psi_{err,j,\gamma}}\R|    
	+2 \sum_{j=1}^{i}\L|\bra{\psi_{1^i,\gamma}} U^\dag  P_{acc,i,z}U \ket{\psi_{err,j,\gamma}}\R| \nn \\
	+&\L. \sum_{k=1}^{i}\sum_{j=1}^{i}\L| \bra{\psi_{err,k,\gamma}} U^\dag  P_{acc,i,z}U \ket{\psi_{err,j,\gamma}}\R| \R] \nn \\ %%%%%%%%%
	  \leq&  \sum_z   \L[\sum_{k=0}^{i-1} \sum_{j=0}^{i-1} \L| \bra{e_i,\psi_{1^k0,\gamma}}U^\dag  P_{acc,i,z}U    \ket{e_i,\psi_{1^j0,\gamma}} \R|+
	2 \sum_{k=0}^{i-1} \L|\bra{e_i,\psi_{1^k0,\gamma}}U^\dag  P_{acc,i,z}U \ket{e_i,\psi_{1^i,\gamma}} \R|  \R]\nn \\
	+& O\L(\frac{m^2}{\sqrt T}\R)\nn \\ %%%%%%%%%
	\leq&  \sum_z   \L[\L| \bra{\psi_{1^{i-1}0,\gamma}}U^\dag  P_{acc,i,z}U    \ket{\psi_{1^{i-1}0,\gamma}} \R|+
	2  \L|\bra{\psi_{1^{i-1}0,\gamma}}U^\dag  P_{acc,i,z}U \ket{\psi_{1^i,\gamma}} \R|  \R]    \nn \\ 
	+&O\L(\frac{m^2}{\sqrt T}+m^2{(m-1)\gamma_0}+m\sqrt{(m-1)\gamma_0}\R)\nn \\ %%%%%%%%%
	\leq& \norm{\ket{\psi_{1^{i-1}0,\gamma}}}^2+ 2\norm{\ket{\psi_{1^{i-1}0,\gamma}}}+O\L(\frac{m^2}{\sqrt T}+m\sqrt{(m-1)\gamma_0}\R),
\end{align}
where on the second inequality we used triangle inequality, on the third inequality we used  Lemma~\ref{lem:samp-tech} and property~\ref{property:partition-err} of Lemma~\ref{lem:partition2}, on the fourth inequality we used Lemma~\ref{lem:samp-tech} and Equation~\ref{eq:rejected-d}, and on the last inequality we used Lemma~\ref{lem:samp-tech}. Once again, we omit $e_i$ when it is not relevant.






% \begin{align}
%     &\vev{e_i,\psi|U^\dag P_{acc,i,z} U|e_i,\psi} \nn \\
%     =& \bra{e_i, \psi_{1^{i-1}1,\gamma}} U^\dag P_{acc,i,z} U \ket{e_i\psi_{1^{i-1}1,\gamma}} +\hannote{poly terms?}+ \negl(n)
% \end{align}

Now we try to  put together all $i\in [m]$. Define
$$z_{good}\defeq \frac{1}{m}\sum_i z_{good,i}.$$

We have
\begin{align} \label{eq:z-z-good}
	\tr|z-z_{good}| =& \tr\L|\frac{1}{m}\sum_i (z_i-z_{good,i})\R| \nn \\
	\leq&  \frac{1}{m}\sum_i\tr| (z_i-z_{good,i})| \nn \\
	\leq&  \frac{1}{m}\sum_i\L[\norm{\ket{\psi_{1^{i-1}0,\gamma}}}^2+ 2\norm{\ket{\psi_{1^{i-1}0,\gamma}}}+O\L(\frac{m^2}{\sqrt T}+m\sqrt{(m-1)\gamma_0}\R)\R] \nn \\%%%%%%%%
	\leq&  \frac{1}{m}+ 2\frac{1}{\sqrt m}+O\L(\frac{m^2}{\sqrt T}+m\sqrt{(m-1)\gamma_0}\R) \nn \\ %%%%%
	=&O\L( \frac{1}{\sqrt m}+\frac{m^2}{\sqrt T}+m\sqrt{(m-1)\gamma_0}\R)
\end{align}

where we used triangle inequality on the second line, Equation~\ref{eq:zi-zgoodi} on the third line, Equation~\ref{eq:bad-term-sum} and Cauchy's inequality on the fourth line.





We now proceed to prove that $z_{good}$ is close to the ideal distribution.

For every $i\in [m]$ and every prover strategy $(U_0,U)$ for Protocol~\ref{proto:QPIP0samp}, consider the following composite strategy of the prover for the naive $\QPIP_0$ Protocol, Protocol~\ref{proto:qpip0_naive}. Note that the prover only interact with the verifier in Step~\ref{step:urmila-in-naive} of Protocol~\ref{proto:qpip0_naive} where Protocol~\ref{proto:urmila4} is run, so we describe the prover's action in turns of the four rounds of communication in Protocol~\ref{proto:urmila4}.

The prover tries to run $U_0$ by taking the verifier's input as $i$-th copy of Protocol~\ref{proto:urmila4} in Protocol~\ref{proto:QPIP0samp} and simulating other $m-1$ copies by himself. The prover then picks a uniformly random $\gamma$ and  tries to generate $\ket{\psi_{1^{i-1}1,\gamma}}$ by applying $G_{i,1,\gamma}G_{i-1,1,\gamma} \cdots G_{2,1,\gamma}G_{1,1,\gamma}$. If the prover fails to generate $\ket{\psi_{1^{i-1}1,\gamma}}$, he throws out everything and aborts by sending $\bot$ back.   On the fourth round,  If it's a testing round the prover reply with the $i$-th register of $\ext_i\left(\frac{\ket{e_j}_{\regC}\ket{\psi_{1,i,\gamma}}_{\regX,\regZ}}{\|\ket{\psi_1}_{\regX,\regZ}\|}\right)$, where $\ext_i$ is specified in property~\ref{property:partition-binding} of Lemma~\ref{lem:partition2}. If it's the Hadamard round  the prover  runs $U$ and checks whether every copy except the $i$-th copy would be accepted. If all $m-1$ copies are accepted, he replies with the $i$-th copy, otherwise reply $\bot$.

%old protocol when jerection on 4-th round
% The prover tries to run $U_0$ by taking the verifier's input as $i$-th copy of Protocol~\ref{proto:urmila4} in Protocol~\ref{proto:QPIP0samp} and simulating other $m-1$ copies by himself. The prover then picks a uniformly random $\gamma$ and  tries to generate $\ket{\psi_{1^{i-1}1,\gamma}}$ by applying $G_{i,1,\gamma}G_{i-1,1,\gamma} \cdots G_{2,1,\gamma}G_{1,1,\gamma}$. If the prover fails to generate $\ket{\psi_{1^{i-1}1,\gamma}}$, he throws out everything and run $U_{trivial,0}$ described  in  Lemma~\ref{lem:trivial-4-round-strategy} instead. This is possible because all the inputs are classical.  On the fourth round, there are four possibilities depending on $c$ and whether the prover generated $\ket{\psi_{1^{i-1}1,\gamma}}$. If $c=0$ and the prover generated $\ket{\psi_{1^{i-1}1,\gamma}}$, he reply with the $i$-th register of $\ext_i\left(\frac{\ket{e_j}_{\regC}\ket{\psi_{1,i,\gamma}}_{\regX,\regZ}}{\|\ket{\psi_1}_{\regX,\regZ}\|}\right)$, where $\ext_i$ is specified in property~\ref{property:partition-binding} of Lemma~\ref{lem:partition2}. If $c=1$ and the prover generated $\ket{\psi_{1^{i-1}1,\gamma}}$, he runs $U$ and check whether every copy except the $i$-th copy would be accepted. If all $m-1$ copies accepts, he reply with the $i$-th copy, otherwise reply $\bot$. If $c=0$ and the prover failed to generated $\ket{\psi_{1^{i-1}1,\gamma}}$, he runs $U_{trivial}$ and repies. If $c=1$ and the prover failed to generated $\ket{\psi_{1^{i-1}1,\gamma}}$, he replies with $\bot$. \footnote{Recall that $c=0$ is the testing round and $c=1$ is the Hadamard round.}

%We denote the random variable the verifier would get on the Hadamard round as $z_{composite}$. Also define $(d,z_{composite}) \defeq (\Acc, z_{composite}) + (1-|z_{composite}|)(\Rej,\bot)$ \hannote{correct notation?}


Denote the result we would get when $c=1$ by $(d,z_{composite,i})$.

By  property~\ref{property:partition-binding} of Lemma~\ref{lem:partition2}, the above strategy is accepted with probability $1-\negl(n)$ when the prover didn't abort. Since the prover's strategy is also efficient, by Lemma~\ref{lem:naive-qpip0-binding},
\begin{align}
	(d,z_{composite,i})\approx_{c, O(\eps)}(d,z_{ideal,i})
\end{align}



By construction, when $G_{i,1,\gamma}G_{i-1,1,\gamma} \cdots G_{2,1,\gamma}G_{1,1,\gamma}$ succeeded, the corresponding output would be $z_{good,i}$. Also note that this is the only case where the verifier won't reject, so $(d,z_{composite,i})=(d,z_{good,i})$, where $(d,z_{good,i})$ is defined as:

$$\begin{cases}
	(d,z_{good,i}) \la (\Acc, z_{good,i}/|z_{good,i}|) & \text{with probability } |z_{good,i}|\\
					(d,z_{good,i}) = (\Rej,\bot)  & \text{otherwise }
\end{cases}$$.


% where  $(d,z)_{good,i} \defeq (\Acc, z_{good,i}) + (1-|z_{good,i}|)(\Rej,\bot)$ \hannote{correct notation? operational def?} .


Therefore

\begin{align}
	(d,z_{good,i})\approx_{c, O(\eps)}(d,z_{ideal,i}).
\end{align}

Average over $i\in[m]$ we get

\begin{align}
	(d,z_{good})\approx_{c, O(\eps)}(d,z_{ideal}).
\end{align}
% \hannote{do we lose a factor of $m$?}
Recall that by Eq~\ref{eq:z-z-good} we have $\norm{z-z_{good}}_1 \leq O(\eps)$ if we pick $m=O(1/\eps^2), T=O(1/\eps^2),\gamma_0=\eps^8$, which implies
\begin{align}
	(d,z)\approx_{c, O(\eps)}(d,z_{good}).
\end{align}

Therefore  we have  
\begin{align}
	(d,z)\approx_{c, O(\eps)}(d,z_{ideal}).
\end{align}
%\hannote{the composition of z ideal and triangle ineq is sketchy lol. Also need to specify different z ideal}


\end{proof}


\section{Constant-Round, Blind, and Verifiable Delegation}

\Ethan{TODO Do some kinda intro to the section}

We now present a constant-round, blind, and verifiable delegation scheme for $\SampBQP$ as an application of our $\QPIP_0$ construction for $\SampBQP$,
which we achieve by using the homomorphic encryption scheme from \cite{mahadev_qfhe} together with our construction.

\subsection{Homomorphic Encryption with Classical Client}

\Ethan{This subsection is copied and pasted directly from the other QMPC paper and is kinda a mess}

We present the quantum full homomorphic encryption (QFHE) scheme, $\mathsf{QHE}=(\mathsf{QHE.keygen}, \mathsf{QHE.Enc}, \mathsf{QHE.Dec}, \mathsf{QHE.Eval})$ given in \cite{mahadev_qfhe} which allows the use of a classical client. Specifically, it has the following extra properties:
\begin{itemize}
	\item $\mathsf{QHE.Keygen}$ can be done classically.
	\item In the case where the plaintext is classical, $\mathsf{QHE.Enc}$ can be done classically.
\end{itemize}

Furthermore, one could either decode or measure first, in the following sense:
\begin{lemma}
	\label{decodeMeasureOrder}
	Let $\Lambda$ be a product of $X$ and $Z$ measurements. Then there exists $\Lambda'$ product of $X$ and $Z$ measurements and $\widehat{\mathsf{QHE.Dec}}$ a BPP algorithm so that
		$$\Lambda\circ\mathsf{QHE.Dec}=\widehat{\mathsf{QHE.Dec}}\circ\Lambda'$$
\end{lemma}

\subsection{Generalizing our delegation protocol using QFHE}

\Ethan{TODO intro}

Here's what the client should do.

\begin{algorithm}
	\caption{Verifiable, secure, and constant round delegation}
	\label{ProtoPriv}
	\begin{algorithmic}[1]
		\Procedure{Delegation}{C, x}
			\State Compute $\mathsf{QHE.Keygen}\rightarrow(pk, evk, sk)$
			\State Compute $\tilde{x}=\mathsf{Enc}_{pk}(x)$
			\State Let $\tilde{C}$ be the quantum circuit that takes $\tilde{x}$ as input to evaluate $\mathsf{QHE.Eval}_{evk}(C, \tilde{x})$
			\State Delegate $\tilde{C}(\tilde{x})$ to the server using our $\QPIP_0$ protocol.
			\State Request $\Lambda'$ based on \autoref{decodeMeasureOrder}, where $\Lambda$ is the basis choice if $C(x)$ were delegated.
			\State Let $y$ be the measurement results from the server
			\State \Return $\widehat{\mathsf{QHE.Dec}_{sk}}(y)$
		\EndProcedure
	\end{algorithmic}
\end{algorithm}

\begin{thm}
    \label{QPIP1thm}
	\myprotoref{ProtoPriv} can evaluate any $L\in\SampBQP$ with negligible completeness and soundness $(O(T^{-c}), O(T^{-c}))$ for any constant $c$.
\end{thm}
\begin{proof}
	Suppose $\bbV$ accepts $\bbP'$ at least $\delta$ of the time.

	Then the $\QPIP_0$ delegation must accept at least that often too,
	so the client will receive a sample from $\mathsf{Eval}_{evk}(C, \tilde{x})$ with at most inverse poly error. \Ethan{security param?}

	The verifier ends up with a state within inverse poly distance to
		$$\widehat{\mathsf{QHE.Dec}_{sk}}\circ\Lambda'(\tilde{C}(\tilde{x}))$$
		$$=\Lambda\circ\mathsf{QHE.Dec}_{sk}(\tilde{C}(\tilde{x}))$$
		$$=\Lambda\circ\mathsf{QHE.Dec}_{sk}(\mathsf{QHE.Eval}_{evk}(C, \tilde{x}))$$

	By the properties of homomoprhic encryption, with overwhelming probability this is indistinguishable to \Ethan{Check if it's true} $\Lambda(C(x))$
\end{proof}

\begin{thm}
	\myprotoref{ProtoPriv} is IND-CPA secure.
\end{thm}
\begin{proof}
	The verifier's first message is encrypted into $\tilde{x}$ in an IND-CPA way as a ciphertext.
	The verifier's second message is the basis choice of Hadamard or test rounds, which can be done using public coins.

	The verifier only sends the prover these two messages, so it follows \Ethan{hopefully?} easily that the protocol itself is also IND-CPA. \Ethan{Maybe not rigorous enough?}
\end{proof}


\section*{Acknowledgments}

The authors would like to thank Tomoyuki Morimae for his valuable feedback that helped improve the paper and for pointing out the related works \cite{takeuchi2018verification, hayashi2015verifiable}.
We are also thankful to anonymous reviewers for various useful comments.


\bibliographystyle{plain}
\bibliography{refs}

\newpage

\begin{center}
\LARGE
SUPPLEMENTARY MATERIAL
\end{center}

\appendix

\section{Preliminaries}

\subsection{Notations}

Let $\mathcal{B}$ be the Hilbert space corresponding to one qubit. Let $H:\mathcal{B}^{\otimes n}\rightarrow\mathcal{B}^{\otimes n}$ be Hermitian matrices. We use $H\geq0$ to denote $H$ being positive semidefinite. Let $\lambda(H)$ be the smallest eigenvalue of $H$. The ground states of $H$ are the eigenvectors corresponding to $\lambda(H)$. For matrix $H$ and subspace $S$, let $H\big|_S=\Pi_S H \Pi_S$, where $\Pi_S$ is the projector onto the subspace $S$. For a $T$-qubit Hilbert space, let the state $\ket{\widehat{t}}=\ket{1}^{\otimes t}\otimes \ket{0}^{{\otimes (T-t)}}$.
We write $F(\rho_1, \rho_2)=\left(\tr\sqrt{\sqrt{\rho_1}\rho_2\sqrt{\rho_1}}\right)^2$ for the fidelity between $\rho_1$ and $\rho_2$.
We write $\frac{1}{2}\norm{\rho_1-\rho_2}_1$ for the trace distance between $\rho_1$ and $\rho_2$. For all $n$-qubit states $\rho_1, \rho_2\in\cB^{\otimes n}$ we have $\frac{1}{2}\norm{\rho_1-\rho_2}_1\leq\sqrt{1-F(\rho_1, \rho_2)}$.

\begin{definition} [quantum-classical channels]
	\label{def:QCChannel}
	A quantum measurement is given by a set of matrices $\set{M_k}$ such that $M_k\geq0$ and $\sum_k M_k=\id$.
	We associate to any measurement a map $\Lambda(\rho)=\sum_k \tr(M_k\rho)\ket{k}\bra{k}$
	with $\set{\ket{k}}$ an orthonormal basis.
	This map is also called a \emph{quantum-classical channel}.
\end{definition}

The phase gate and Pauli matrices are denoted as follows.

\begin{definition}
	$P(i)=\begin{pmatrix}1&0\\0&i\end{pmatrix}$, $X=\begin{pmatrix}0&1\\1&0\end{pmatrix}$,
	$Y=\begin{pmatrix}0&-i\\i&0\end{pmatrix}$,
	$Z=\begin{pmatrix}1&0\\0&-1\end{pmatrix}$
\end{definition}

\subsection{Relevant complexity classes}

We define a few relevant complexity classes.

\begin{definition} [$\BQP$]
	Definition from Kitaev:
	A \emph{quantum algorithm} for the computation of a function $F:\zo^*\rightarrow\zo^*$ is a classical algorithm (i.e., a Turing machine) that computes a function of the form $x\mapstochar\rightarrow Z(x)$, where $Z(x)$ is a description of a quantum circuit which computes $F(x)$ on empty input. The function $F$ is said to belong to class $\BQP$ if there is a quantum algorithm that computes $F$ in time $\poly(n)$.

	Definition from Complexity Zoo:
	$\BQP$ is the class of languages $L$ for which for all $n\in\bbN$ there exists a quantum circuit constructible in time $\poly(n)$ that, given any $x\in\set{0, 1}^n$ as input, correctly decides whether $x\in L$ at least $\frac{2}{3}$ of the time.
	\Ethan{Just copy this from somewhere... Does anyone even define this?}
\end{definition}

\begin{definition} [$\FBQP$]
	A function $f:\set{0,1}^*\rightarrow\set{0,1}^*$ is in $\FBQP$ if there is a $\BQP$ machine that, $\forall x$, outputs $f(x)$ with overwhelming probability.
	\Ethan{Need to be more formal. Also should be efficient verifiable}
\end{definition}

We define search and sampling versions of $\BQP$ based on \cite{aaronson_2013}.

\begin{definition} [search problem]
	A search problem $R$ is a collection of nonempty sets $(A_x)_{x\in\set{0, 1}^*}$, one for each input string $x\in\set{0, 1}^*$, where $A_x$... \Ethan{Great, interface doesn't line up correctly}
\end{definition}

\begin{definition} [sampling problem]
	A sampling problem $S$ is a collection of probability distributions $(D_x)_{x\in\set{0, 1}^*}$, one for each input string $x\in\set{0,1}^n$, where $D_x$ is a distribution over $\set{0,1}^{p(n)}$ for some fixed polynomial $p$.
\end{definition}

\begin{definition} [$\SampBQP$]
	$\SampBQP$ is the class of sampling problems $S=\left(D_x\right)_{x\in\set{0, 1}^*}$ for which there exists a polynomial-time quantum algorithm $B$ that, given $(x, 0^{1/\varepsilon})$ as input, samples from a probability distribution $C_x$ such that $\norm{C_x-D_x}\leq\varepsilon$.
\end{definition}

\subsection{Quantum Prover Interactive Protocol (QPIP)}
We classify the interaction between a (almost classical) client and a quantum server for sampling problems, extending the classification by \cite{FOCS:Mahadev18a}.

\begin{definition}
	We say $\Pi=(P, V)(x)$ is a protocol for the sampling problem $(D_x)_{x\in\zo^*}$ with completeness error $c$ and soundness error $s$ \Ethan{Might need these to be functions of $\abs{x}$} if
	\Ethan{Look up ``interactive protocols for BQP". Right now it's missing quantifier for all x. Also need to mention d is decision bit; maybe do that in next definition and swap locations}
	\begin{itemize}
		\item Let $(d, z)\leftarrow(P, V)(x)$. Then $d=rej$ with probability at most $c$.
		\item For all cheating prover $P^*$, let $(d, z)\leftarrow(P^*, V)(x)$. Let \Ethan{Use display math to make it obvious I'm defining this} $z_{ideal}\leftarrow D_x$ if $d=acc$, else $z_{ideal}=\bot$. Then $\norm{(d, z) - (d, z_{ideal})} \leq s$ \Ethan{make stat. distance notation consistent}.
	\end{itemize}
\end{definition}

\Ethan{Acc, rej, P, V fonts}

\Ethan{Might need to write our own def. here}

\Ethan{Look at thesis for this}

\begin{definition}
	A sampling problem $S=(D_x)_{x\in\set{0, 1}^*}$ is said to be \Ethan{Try to make this more general; remove mentions of sampling/decision problems} in $\QPIP_\tau$ with completeness $c$ and soundness $s$ \Ethan{Don't tie this with completeness and soundness yet} if there exists a protocol $(\bbP, \bbV)(x)$ for $S$ with the following properties:
	\begin{itemize}
		\item $\bbP$ is run by the prover, a $\BQP$ machine, which also has access to a quantum channel that can transmit $\tau$ qubits to the verifier per use.
		\item $\bbV$ is run by the verifier, which is a hybrid machine of a classical part and a limited quantum part. The classical part is a $\BPP$ machine. The quantum part is a register of $\tau$ qubits, on which the verifier can perform arbitrary quantum operations and which has access to a quantum channel which can transmit $\tau$ qubits. At any given time, the verifier is not allowed to possess more than $\tau$ qubits. The interaction between the quantum and classical parts of the verifier is the usual one: the classical part controls which operations are to be performed on the quantum register, and outcomes of measurements of the quantum register can be used as input to the classical part.
		\item There is also a classical communication channel between the prover and the verifier, which can transmit $\poly(\abs{x})$ many bits to either direction. 
	\end{itemize}
\end{definition}

\Ethan{Two separate definitions for comp and soundness}

\Ethan{Soundness?}

\subsection{Semantic security for interactive protocols}
\Ethan{Just call this blindness and put this under interactive protocols}

\Ethan{See Thomas' paper if he defined this}

We present the security definition for interactive protocols:

\begin{definition}
	Let $\lambda$ be a security parameter.
	Let $(\bbP, \bbV)$ be an interactive protocol with security parameter $\lambda$.
	Then it is IND-CPA secure if $\forall x\in\set{0,1}^n$ no polynomial time adversary $\cA$ can win \protoref{indcpa} with probability better than $\frac{1}{2}+\negl(\lambda)$
\end{definition}

\begin{protocol}{Attack against semantic security}
	\label{proto:indcpa}
	\begin{enumerate}
		\item The challenge picks $b\in\set{0,1}$ at random
		\item If $b=0$, the challenger runs the protocol with the adversary, acting as the verifier with input $0^n$
		\item Otherwise, the challenger runs the protocol with the adversary, acting as the verifier with input $x$
		\item $\cA$ attempts to guess $b$
	\end{enumerate}
\end{protocol}

\subsection{Chernoff bound}

Taken from \href{http://math.mit.edu/~goemans/18310S15/chernoff-notes.pdf}{here}.

\begin{thm}
\label{thm:Chernoff}
Let $X=\sum_{i=1}^n X_i$ where $X_i$ are i.i.d. Bernoulli trials, and $\mu=\E[X]$.
Then for all $0<\delta<1$,
$$P[\abs{X-\mu}\geq\delta\mu]\leq2e^{-\frac{\mu\delta^2}{3}}$$
\end{thm}

\subsection{Projection Lemma}

We use the projection lemma from \cite{kempe_kitaev_regev_2006}, which describes the conditions under which we can estimate the ground state energy of $H_1 + H_2$ with that of $H_1\big|_{\ker H_2}$.

\begin{thm}
	Let $H=H_1+H_2$ be the sum of two Hamiltonians operating on some Hilbert space $\cH=\cS+\cS^\bot$.
	The Hamiltonian $H_2$ is such that $\cS$ is a zero eigenspace and the eigenvectors in $\cS^\bot$ have eigenvalues at least $J>2\norm{H_1}$. Then,
	$$\lambda\left(H_1\big|_\cS\right)-\frac{\norm{H_1}^2}{J-2\norm{H_1}^2}\leq\lambda(H)\leq\lambda\left(H_1\big|_\cS\right)$$
\end{thm}

We will instead use the following formulation, which can be obtained by relabeling variables from above.

\begin{thm}
	\label{thm:projection}
	Let $H_1, H_2$ be local Hamiltonians where $H_2\geq0$. Let $K=\ker H_2$ and
	$$J=\frac{10\norm{H_1}^2}{\lambda\left(H_2\big|_{K^\bot}\right)}$$
	then we have
	$$\lambda(H_1+JH_2)\geq\lambda\left(H_1\big|_K\right)-\frac{1}{8}$$
\end{thm}

\subsection{Quantum de Finetti Theorem under Local Measurements}

De Finetti theorem provides a way to obtain close to independent samples by taking random subsystems of a quantum system.
There are many formulations; we use the one from \cite{Brandão2017} because we need to avoid exponential dependence on number of qubits in each subsystem.
\begin{thm}
	\label{deFinetti}
	Let $\rho^{A_1\ldots A_k}$ be a permutation-invariant state on registers $A_1,\ldots,A_k$ where each register is $s$ qubits,
	then for every $0\leq l\leq k$ there exists states $\set{\rho_i}$ and $\set{p_i}\subset\bbR$ such that
	$$\max_{\Lambda_1,\ldots,\Lambda_l}
	\norm{(\Lambda_1\otimes\ldots\otimes\Lambda_l)\left(\rho^{A_1\ldots A_l}-\sum_ip_i\rho_i^{A_1}\otimes\ldots\otimes\rho_i^{A_l}\right)}_1
	\leq\sqrt{\frac{2l^2s}{k-l}}$$
	where $\Lambda_i$ are quantum-classical channels.
\end{thm}

\subsection{Quantum Homomorphic Encryption Schemes}

\def\QHE{\mathsf{QHE}}
\def\QGen{\mathsf{QHE.Keygen}}
\def\QEnc{\mathsf{QHE.Enc}}
\def\QEval{\mathsf{QHE.Eval}}
\def\QDec{\mathsf{QHE.Dec}}

We use the quantum fully homomorphic encryption scheme given in \cite{mahadev_qfhe} which is compatible with our use of a classical client. We start by presenting the interface of a homomorphic encryption scheme:
\begin{definition}
	A leveled homomophic encryption scheme is tuple of algorithms \linebreak $\mathsf{HE}=(\mathsf{HE.Keygen}, \mathsf{HE.Enc}, \mathsf{HE.Dec}, \mathsf{HE.Eval})$ with the following descriptions:
	\begin{itemize}
		\item $\mathsf{HE.Keygen}(1^\lambda, 1^L)\rightarrow(pk, sk)$
		\item $\mathsf{HE.Enc}_{pk}(\mu)\rightarrow c$
		\item $\mathsf{HE.Dec}_{sk}(c)\rightarrow \mu^*$
		\item $\mathsf{HE.Eval}_{pk}(f, c_1, \ldots, c_l)\rightarrow c_f$
	\end{itemize}
\end{definition}

$\mathsf{HE}$ also satisfies, with overwhelming probability in $\lambda$, that
$$\mathsf{HE.Dec}_{sk}(\mathsf{HE.Eval}_{pk}(f, c_1, \ldots, c_l)=f(\mathsf{HE.Dec}_{sk}(c_0),\ldots,\mathsf{HE.Dec}_{sk}(c_l))$$
where $f$ is specified by a circuit of depth at most $L$.

\Ethan{To be pedantic, the above doesn't imply Dec undoes Enc even if we sub in $f=\id$.}

We also recall the security definition for a FHE scheme.

\begin{definition}
	A FHE scheme $\mathsf{HE}$ is IND-CPA secure if, for any polynomial time adversary $\cA$, there exists a negligible function $\mu(\cdot)$ such that
	$$\abs{Pr[\cA(pk, \mathsf{HE.Enc}_{pk}(0))=1]-Pr[\cA(pk, \mathsf{HE.Enc}_{pk}(1))=1]}=\mu(\lambda)$$
	where $(pk, sk)\leftarrow\mathsf{QHE.Keygen}(1^\lambda)$
\end{definition}

The quantum homomorphic encryption scheme $\mathsf{QHE}$ from \cite{mahadev_qfhe} has additional properties that facilitates the use of classical clients:
\begin{itemize}
	\item $\QGen$ can be done classically.
	\item In the case where the plaintext is classical, $\QEnc$ can be done classically.
	\item Its ciphertext takes the form $(X^xZ^z\rho Z^zX^x, c_{x, z})$, where $\rho$ is the plaintext and $c_{x, z}$ is a ciphertext that decodes to $(x, z)$ under a certain classical homomorphic encryption scheme.
\end{itemize}


\section{Construction of the XZ Local Hamiltonian}
\label{sec:Hamiltonian}

\XW{highlight the different requirement from the normal Local Hamiltonian reduction; and highlight a bit about the solution idea}

In this section, we give a reduction from $\SampBQP$ to a local Hamiltonian instance.

The local Hamiltonian problem is $\QMA$-complete, which means it's $\BQP$-hard.
Protocols to delegate $\BQP$ computations such as \cite{FOCS:Mahadev18a} \Ethan{Are there any more?} take advantage of this fact and require the prover to send the corresponding $\QMA$ certificate.
We take a similar approach to delegate $\SampBQP$ computations; however, we need to extract more information from the certificates.
Local Hamiltonians have been extensively studied in contexts of adiabatic quantum computations, so we will summarize known results from sources such as \cite{adiabatic}.
\Ethan{This source was given to us by Yu-Ching; I haven't read it. Also I probably should cite more sources here.}

We begin by recalling the form that local Hamiltonian certificates:

\begin{dfn}
	\label{dfn:groundstate}	
	Let $C=U_T\ldots U_1$ be a quantum circuit consisting of $T$ gates, and let $x\in\set{0,1}^n$. Then we denote
	$$\ket{\psi_{C(x)}}=\sum_{t=0}^TU_t\ldots U_1\ket{x}\otimes\ket{\hat{t}}$$
	We call the first register of $\ket{\psi_{C(x)}}$ the data register, and the second register of $\ket{\psi_{C(x)}}$ the time register.
\end{dfn}

\begin{rmk}
	\label{idpadding}
	One can first pad $C$ with identity gates at the end, doubling its number of gates and obtaining another circuit $C'$.
	Then by measuring the time register of $\ket{\psi_{C'(x)}}$ one obtains with probability $\frac{1}{2}$ the state $C(x)$.
\end{rmk}

We denote the set of $5$-local tensor products of $X$ and $Z$ gates as $\mathcal{G}_{XZ}$ as below; note that this spans a subspace of $5$-local Hamiltonians.
\begin{dfn}
	$$\mathcal{G}_{XZ}=\set{U_0\otimes U_1\otimes\ldots\otimes U_n: U_i\in\set{I,X,Z}, \abs{\set{i: U_i\ne I}}\leq 5}$$
\end{dfn}

Now we state the main theorems that we'll work towards in this section.
The first one is the reduction from $\SampBQP$ to a local Hamiltonian instance, with some additional properties that will be convenient for us:
\begin{thm}
	\label{thm:LHReduction}
	Let $C$ be a circuit of size $T$. For any input string $x$, there exists a local Hamiltonian $H_{C(x)}$  with the following properties:
	\begin{itemize}
		\item $H_{C(x)}$ can be written as $\sum_{S\in\mathcal{G}_{XZ}} d_S S$ with at most $O(T)$ terms
		\item $H_{C(x)}$ has $\ket{\psi_{C(x)}}$ as the unique ground state with eigenvalue $0$.
		\item For any state $\ket\phi$ such that $\braket{\phi|\psi_{C(x)}}=0$,  $\braket{\phi|H_{C(x)}|\phi}>\frac{3}{4}$.
		\item Its descriptions can be constructed by a $\BPP$ machine in polynomial time.
	\end{itemize}
\end{thm}

The second one is a way to verify ground states of local Hamiltonians, again with some extra properties:
\begin{thm}
	\label{ThmXZCheck}
	Given circuit $C$ and an input string $x\in\set{0,1}^*$, there exists a $\BQP$ algorithm that accepts $\ket{\phi}$ with probability $\frac{1}{2}-\Omega(\frac{1}{\poly(T)})\braket{\phi|H_{C(x)}|\phi}$. (In particular, it accepts the $\ket{\psi_{C(x)}}$ with probability $\fot$.) Furthermore, this algorithm doesn't apply any quantum gates, using only $X$ and $Z$ measurements.
\end{thm}

\subsection{Circuits of Toffoli and Hadamard gates}

Old notation: Let $\Lambda_c(U)$ denote the gate $U$ controlled on qubit $c$, and $\Lambda^2_{f, s}(U)$ denote the gate $U$ controlled by both $f$ and $s$. I.e. $\Lambda^2_{1, 2}(X_3)$ would be a Toffoli gate.
\Ethan{TODO: replace with something that looks better and doesn't conflict with the use of $\Lambda$ as measurement.}

Here we restate the result from from \cite{quant-ph/0301040} showing that $\set{H, \Lambda^2(X)}$ is a universal gate set.
Because of this result, we can assume without loss of generality that our quantum circuits consists of only Toffoli and Hadamard gates.
Which is significant because this gate set is a linear combinations of tensor products of $X$ and $Z$ operators:
\begin{thm}
	$H,\Lambda^2(X)\in\spn\mathcal{G}_{XZ}$ with $O(1)$ nonzero components.
\end{thm}
\begin{prf}
	$$H=\frac{1}{\sqrt{2}}(X+Z)$$
	$$\Lambda^2_{1,2}(X_3)=\ket{11}\bra{11}\otimes X+(I-\ket{11}\bra{11})\otimes I$$
	$$\ket{11}\bra{11}=\frac{1}{4}(I\otimes I+Z\otimes Z-I\otimes Z-Z\otimes I)$$
\end{prf}

Now we present the proof that $\set{H, \Lambda^2(X)}$ is indeed a universal gate set from \cite{quant-ph/0301040}: \Ethan{This definitely belongs in the appendix instead...}
\begin{thm}
	Let $C$ be a circuit that:
	\begin{itemize}
		\item consists of $T$ gates, each either $H$ and $\Lambda(P(i))$
		\item uses $n$ qubits
	\end{itemize}
	Then a classical machine given $C$ can compute a circuit $C'$ that:
	\begin{itemize}
		\item consists of at most $4T$ gates, each either $H$ or $\Lambda^2(X)$ \hannote{don't like this notation}.
		\item uses $n+1$ qubits
		\item Let $x\in\set{0,1}^n$. Let $x|| 0\in\set{0,1}^{n+1}$ be $x$ concatenated with $0$.
		The measurement result of a tensor product of $X$ and $Z$ operators on the first $n$ qubits of $C'(x||0)$ has the same distribution as that on $C(x)$.
	\end{itemize}
\end{thm}
\begin{prf}
	We know that Hadamard gate and controlled phase gate is universal from \cite{kitaev_1997}
	$$H=\frac{1}{\sqrt{2}}\begin{pmatrix}1&1\\1&-1\end{pmatrix}$$
		$$\Lambda(P(i))=\begin{pmatrix}1&0&0&0\\0&1&0&0\\0&0&1&0\\0&0&0&i\end{pmatrix}$$

	Now consider the transform on quantum states
	$$\mathcal{F}(\ket{\phi})=(\Re\ket{\phi})\otimes\ket{0}+(\Im\ket{\phi})\otimes\ket{1}$$
	where $\Re$ and $\Im$ denote real and imaginary parts respectively.

	This transform commutes with Hadamard gates on the respective qubit. On the other hand, exchanging $\mathcal{F}$ with a controlled phase gate turn it into a combination of Hadamard and Toffoli gates. Mathematically,
	$$\mathcal{F}\circ H_s=H_s\circ\mathcal{F}$$
	$$\mathcal{F}\circ\Lambda_f(P(i)_s)=\Lambda^2_{f,s}(X_0Z_0)\circ\mathcal{F}=\Lambda^2_{f,s}(X_0)H_0\Lambda^2_{f,s}(X_0)H_0\circ\mathcal{F}$$

	We construct $C'$ so that $\mathcal{F}\circ C=C'\circ\mathcal{F}$ by following the computation above. It satisfies the required properties by construction.
	\begin{itemize}
		\item $C'$ uses only Hadamard and Toffoli gates.
		\item Exchanging $\mathcal{F}$ with $H$ doesn't change the circuit size. Exchanging $\mathcal{F}$ with controlled phase gate turns it into $4$ gates. So the final result is at most $4T$ gates.
		\item When $x$ is classical, $\mathcal{F}(x)=x||0$. So $F\circ C(x)=C'(x||0)$. It is simple \hannote{need a little more work?}to verify that $\mathcal{F}$ preserves $X$ and $Z$ measurement results on the first $n$ qubits.
	\end{itemize}
\end{prf}

\subsection{Reducing quantum circuit to local Hamiltonian}

In this subsection, we prove \cref{thm:LHReduction} by showing how to reduce a $\SampBQP$ circuit to a local-$XZ$-Hamiltonian.

	The base construction comes from \cite{kitaev2002classical}. The simplification to $\spn\mathcal{G}_{XZ}$ is taken from \cite{PhysRevA.78.012352}.

	Let $x_i$ denote the $i$-th bit of $x$, and let $n$ be the number of qubits in $C$.

	We need to ensure the excited states of $H_{C(x)}$ have high eigenvalues.

	First, we ensure that the invalid clock states have high eigenvalues by applying the following Hamiltonian to the time register.
	$$H_{clock}=\sum_{t=1}^{T-1}\ket{01}\bra{01}_{t,t+1}$$
	As a sum of projections, clearly $H_{clock}\geq 0$. We shall also show that $H_{clock}\in\spn\mathcal{G}_{XZ}$.
	$$H_{clock}=\frac{1}{4}(Z_1 - Z_T) + \frac{1}{4}\sum_{t=1}^{T-1}(I-Z_tZ_{t+1}) $$
	This can be checked by fixing the first and last qubit, then doing induction on the number of switches.

	Then, we ensure that the initial condition is set up correctly.
	$$H_{in}=\sum_{i=1}^n(I-\ket{x_i}\bra{x_i})\otimes\ket{0}\bra{0}_1$$
	The kernel of this is precisely where everything is set up consistently with $\phi$ in time step $t=0$. Furthermore, $H_{in}\in\spn\mathcal{G}_{XZ}$
	$$H_{in}=\frac{1}{4}\sum_{i=1}^n(I-(-1)^{x_i}Z_i)\otimes(I+Z_1)$$

	Then, we ensure that the gates are applied correctly.
	$$H_{prop}=\sum_{t\in T_1}H_{prop,t}$$
	$$H_{prop,t}=I\otimes\ket{\widehat{t}}\bra{\widehat{t}}
	+I\otimes\ket{\widehat{t-1}}\bra{\widehat{t-1}}
	-U_t\otimes\ket{\widehat{t}}\bra{\widehat{t-1}}
	-U_t^\dagger\otimes\ket{\widehat{t-1}}\bra{\widehat{t}}$$

	We have $H_{prop}\geq 0$ according to \cite{2002quant.ph.10077A}; in fact, the least nonzero eigenvalue of $H_{prop}$ is lower bounded by $\frac{1}{2(T+1)^2}$.

	We can also write $H_{prop}\in\spn\mathcal{G}_{XZ}$. Note that $U^\dagger=U$, since our gates are either Hadamard or Toffoli.
	Additionally, $\frac{1}{2}(I-Z_{t-1})$ annihilates time steps before $t-1$. $\frac{1}{2}(I+Z_{t+1})$ similarly annihilates steps $t+1$ and after.
	$$H_{prop,t}=\frac{I}{4}\otimes(I-Z_{t-1})(I+Z_{t+1})-\frac{U}{4}\otimes(I-Z_{t-1})X_t(I+Z_{t+1})$$
	Extra care must be taken for boundary cases.
	$$H_{prop,1}=\frac{1}{2}(I+Z_2)-U_1\otimes\frac{1}{2}(X_1+X_1Z_2)$$
	$$H_{prop,T}=\frac{1}{2}(I-Z_{t-1})-U_T\otimes\frac{1}{2}(X_T-Z_{T-1}X_T)$$

	To properly combine the three Hamiltonians defined and analyze it using the projection lemma (\cref{thm:projection}), we consider the kernels of the Hamiltonians we defined.
	$$K_{clock}=\ker H_{clock}$$
	$$K_{in}=\ker H_{in}$$
	$$K_{prop}=\ker H_{prop}$$

	Clearly,
	$$K_{clock}=\set{\sum_{t=1}^T \ket{\phi_t}\otimes\ket{\hat{t}}:\ket{\phi_t}\in\cB^{\otimes n}}$$
	Let $\ket\phi\in K_{clock}\cap K_{prop}$.
	Consider some $\widetilde{t}$ such that $\braket{\widetilde{t}|\phi}\ne0$.
	As $\ket\phi\in K_{prop}$, we can then do induction using $\widetilde{t}$ as base case to get the relations between the $\ket{\phi_t}$s and conclude that $\ket\phi$ must have the form:
	$$\ket\phi=\sum_{t=0}^TU_t\ldots U_1\ket{y}\otimes\ket{\hat{t}}$$
	$$\Rightarrow K_{clock}\cap K_{prop}=\set{\sum_{t=0}^TU_t\ldots U_1\ket{y}\otimes\ket{\hat{t}}: \ket{y}\in\mathcal{B}^{\otimes n}}$$

	As a direct result, we get
	$$K_{clock}\cap K_{in}\cap K_{prop}=\spn\set{\ket{\psi_{C(x)}}}$$

We now consider the space outside the desired state.
$$S=(\spn\set{\ket{\psi_{C(x)}}})^\bot$$
$$H_{clock}\big|_S,H_{in}\big|_S,H_{prop}\big|_S$$
To combine the Hamiltonians, we apply the projection lemma twice.
$$\exists J_{clock}
=\frac{\poly\left(\norm{H_{in}\big|_S}\right)}{\lambda\left(H_{clock}\big|_{S\cap K^\bot_{clock}}\right)}
=O(n)=O(T)$$
$$\lambda(H_{in}\big|_S+J_{clock}H_{clock}\big|_S)\geq
\lambda(H_{in}\big|_{S\cap K_{clock}})-\frac{1}{8}$$
$$\exists J_{prop}=\frac{\poly\left(\norm{H_{in}\big|_S+J_{clock}H_{clock}\big|_S}\right)}{\lambda\left(H_{prop}\big|_{S\cap K^\bot_{prop}}\right)}
=\frac{O(n+T)}{\Omega(T^{-2})}=O(T^3)$$
$$\lambda(H_{in}\big|_S+J_{clock}H_{clock}\big|_S+J_{prop}H_{prop}\big|_S)\geq
\lambda(H_{in}\big|_{S\cap K_{clock}\cap K_{prop}})-\frac{1}{4}$$
\begin{align*}
	S\cap K_{clock}\cap K_{prop}&=S\cap\set{\sum_{t=0}^TU_t\ldots U_1\ket{y}\otimes\ket{\hat{t}}|\ket{y}\in\mathcal{B}^{\otimes n}}\\
	&=\set{\sum_{t=0}^TU_t\ldots U_1\ket{y}\otimes\ket{\hat{t}}:\braket{\psi_{C(x)}|y}=0}
\end{align*}
$$\Rightarrow\lambda((H_{in}+J_{clock}H_{clock}+J_{prop}H_{prop})\big|_S)\geq\frac{3}{4}$$
So we set $H_{C(x)}=H_{in}+J_{clock}H_{clock}+J_{prop}H_{prop}$, which satisfies the required properties by construction.

\subsection{Checking the ground state of the local Hamiltonian}

In this subsection, we prove \cref{ThmXZCheck}.
That is, we present and analyze an algorithm that checks whether a given state is the ground state of some fixed $H_{C(x)}$, following~\cite{PhysRevA.93.022326}.

\begin{algorithm}
	\caption{Check for ground state}
	\label{AlgGroundStateCheck}
		Let $H=\sum_{S\in\mathcal{G}_{XZ}} d_S S$.
		Let $\ket\phi$ be the potential ground state to check.
		\begin{itemize}
			\item Set $D = \sum_{S\in\mathcal{G}_{XZ}}|d_S|$
			\item Set $p_S = \frac{|d_S|}{D}$
			\item Sample $\widetilde{S}$ from $\mathcal{G}_{XZ}$, weighted by $p_S$.
			\item Measure $\ket\phi$ in the $\widetilde{S}$ basis, recording the result as $\lambda_{\widetilde{S}}$.
			\item If $\sgn(d_{\widetilde{S}})\lambda_{\widetilde{S}}=-1$, accept. Otherwise, reject.
		\end{itemize}
\end{algorithm}

\begin{thm}
	Let $H$ have $O(\poly(T))$ nonzero components whose coefficients at most $O(\poly(T))$.
	Then \cref{AlgGroundStateCheck} accepts $\ket{\phi}$ with probability $\frac{1}{2}-\Omega(\frac{1}{\poly(T)})\braket{\phi|H_{C(x)}|\phi}$.
\end{thm}
\begin{prf}
	Here we follow \cite{PhysRevA.93.022326}.
	\begin{align*}
		\frac{1}{D}\braket{\phi|H|\phi}&=\sum_{S\in\mathcal{G}_{XZ}} p_S\sgn(d_S)\braket{\phi|S|\phi}\\
		&=\sum_{S\in\mathcal{G}_{XZ}} p_S\sgn(d_S)\E[\lambda_S]\\
		&=\E_{\widetilde{S}}[\sgn(d_{\widetilde{S}})\E[\lambda_{\widetilde{S}}]]\\
		&=\E_{\widetilde{S}}[\sgn(d_{\widetilde{S}})\lambda_{\widetilde{S}}]
	\end{align*}

	Note that $\sgn(d_{\widetilde{S}})\lambda_{\widetilde{S}}=\pm1$. Let $p$ be the probability that $\sgn(d_{\widetilde{S}})\lambda_{\widetilde{S}}=-1$.
	$$\Rightarrow \frac{1}{D}\braket{\phi|H|\phi}=\E_{\widetilde{S}}[\sgn(d_{\widetilde{S}})\lambda_{\widetilde{S}}]=-p+(1-p)$$
	\begin{align*}
		\Rightarrow p&=\frac{1}{2}-\frac{1}{2D}\braket{\phi|H|\phi}\\
		&=\frac{1}{2}-\Omega\left(\frac{1}{\poly(T)}\right)\braket{\phi|H|\phi}
	\end{align*}
\end{prf}

%%%%
%%%% Keep the old version here for the record. 
%%%%

% \section{The History XZ Local Hamiltonian}
% \label{sec:Hamiltonian}

% In this section, we give a reduction from $\SampBQP$ to a local Hamiltonian instance.

% The local Hamiltonian problem is $\QMA$-complete, which means it's $\BQP$-hard.
% Protocols to delegate $\BQP$ computations such as \cite{FOCS:Mahadev18a} \Ethan{Are there any more?} take advantage of this fact and require the prover to send the corresponding $\QMA$ certificate.
% We take a similar approach to delegate $\SampBQP$ computations; however, we need to extract more information from the certificates.
% Local Hamiltonians have been extensively studied in contexts of adiabatic quantum computations, so we will summarize known results from sources such as \cite{adiabatic}.
% \Ethan{This source was given to us by Yu-Ching; I haven't read it. Also I probably should cite more sources here.}

% We begin by recalling the form that local Hamiltonian certificates:

% \begin{dfn}
% 	\label{dfn:groundstate}	
% 	Let $C=U_T\ldots U_1$ be a quantum circuit consisting of $T$ gates, and let $x\in\set{0,1}^n$. Then we denote
% 	$$\ket{\psi_{C(x)}}=\sum_{t=0}^TU_t\ldots U_1\ket{x}\otimes\ket{\hat{t}}$$
% 	We call the first register of $\ket{\psi_{C(x)}}$ the data register, and the second register of $\ket{\psi_{C(x)}}$ the time register.
% \end{dfn}

% \begin{rmk}
% 	\label{idpadding}
% 	One can first pad $C$ with identity gates at the end, doubling its number of gates and obtaining another circuit $C'$.
% 	Then by measuring the time register of $\ket{\psi_{C'(x)}}$ one obtains with probability $\frac{1}{2}$ the state $C(x)$.
% \end{rmk}

% We denote the set of $5$-local tensor products of $X$ and $Z$ gates as $\mathcal{G}_{XZ}$ as below; note that this spans a subspace of $5$-local Hamiltonians.
% \begin{dfn}
% 	$$\mathcal{G}_{XZ}=\set{U_0\otimes U_1\otimes\ldots\otimes U_n: U_i\in\set{I,X,Z}, \abs{\set{i: U_i\ne I}}\leq 5}$$
% \end{dfn}

% Now we state the main theorems that we'll work towards in this section.
% The first one is the reduction from $\SampBQP$ to a local Hamiltonian instance, with some additional properties that will be convenient for us:
% \begin{thm}
% 	\label{thm:LHReduction}
% 	Let $C$ be a circuit of size $T$. For any input string $x$, there exists a local Hamiltonian $H_{C(x)}$  with the following properties:
% 	\begin{itemize}
% 		\item $H_{C(x)}$ can be written as $\sum_{S\in\mathcal{G}_{XZ}} d_S S$ with at most $O(T)$ terms
% 		\item $H_{C(x)}$ has $\ket{\psi_{C(x)}}$ as the unique ground state with eigenvalue $0$.
% 		\item For any state $\ket\phi$ such that $\braket{\phi|\psi_{C(x)}}=0$,  $\braket{\phi|H_{C(x)}|\phi}>\frac{3}{4}$.
% 		\item Its descriptions can be constructed by a $\BPP$ machine in polynomial time.
% 	\end{itemize}
% \end{thm}

% The second one is a way to verify ground states of local Hamiltonians, again with some extra properties:
% \begin{thm}
% 	\label{ThmXZCheck}
% 	Given circuit $C$ and an input string $x\in\set{0,1}^*$, there exists a $\BQP$ algorithm that accepts $\ket{\phi}$ with probability $\frac{1}{2}-\Omega(\frac{1}{\poly(T)})\braket{\phi|H_{C(x)}|\phi}$. (In particular, it accepts the $\ket{\psi_{C(x)}}$ with probability $\fot$.) Furthermore, this algorithm doesn't apply any quantum gates, using only $X$ and $Z$ measurements.
% \end{thm}

% \subsection{Circuits of Toffoli and Hadamard gates}

% Here we restate the result from from \cite{quant-ph/0301040} showing that $\set{H, \Lambda^2(X)}$ is a universal gate set.
% Because of this result, we can assume without loss of generality that our quantum circuits consists of only Toffoli and Hadamard gates.
% Which is significant because this gate set is a linear combinations of tensor products of $X$ and $Z$ operators:
% \begin{thm}
% 	$H,\Lambda^2(X)\in\spn\mathcal{G}_{XZ}$ with $O(1)$ nonzero components.
% \end{thm}
% \begin{prf}
% 	$$H=\frac{1}{\sqrt{2}}(X+Z)$$
% 	$$\Lambda^2_{1,2}(X_3)=\ket{11}\bra{11}\otimes X+(I-\ket{11}\bra{11})\otimes I$$
% 	$$\ket{11}\bra{11}=\frac{1}{4}(I\otimes I+Z\otimes Z-I\otimes Z-Z\otimes I)$$
% \end{prf}

% Now we present the proof that $\set{H, \Lambda^2(X)}$ is indeed a universal gate set from \cite{quant-ph/0301040}: \Ethan{This definitely belongs in the appendix instead...}
% \begin{thm}
% 	Let $C$ be a circuit that:
% 	\begin{itemize}
% 		\item consists of $T$ gates, each either $H$ and $\Lambda(P(i))$
% 		\item uses $n$ qubits
% 	\end{itemize}
% 	Then a classical machine given $C$ can compute a circuit $C'$ that:
% 	\begin{itemize}
% 		\item consists of at most $4T$ gates, each either $H$ or $\Lambda^2(X)$ \hannote{don't like this notation}.
% 		\item uses $n+1$ qubits
% 		\item Let $x\in\set{0,1}^n$. Let $x|| 0\in\set{0,1}^{n+1}$ be $x$ concatenated with $0$.
% 		The measurement result of a tensor product of $X$ and $Z$ operators on the first $n$ qubits of $C'(x||0)$ has the same distribution as that on $C(x)$.
% 	\end{itemize}
% \end{thm}
% \begin{prf}
% 	We know that Hadamard gate and controlled phase gate is universal from \cite{kitaev_1997}
% 	$$H=\frac{1}{\sqrt{2}}\begin{pmatrix}1&1\\1&-1\end{pmatrix}$$
% 		$$\Lambda(P(i))=\begin{pmatrix}1&0&0&0\\0&1&0&0\\0&0&1&0\\0&0&0&i\end{pmatrix}$$

% 	Now consider the transform on quantum states
% 	$$\mathcal{F}(\ket{\phi})=(\Re\ket{\phi})\otimes\ket{0}+(\Im\ket{\phi})\otimes\ket{1}$$
% 	where $\Re$ and $\Im$ denote real and imaginary parts respectively.

% 	This transform commutes with Hadamard gates on the respective qubit. On the other hand, exchanging $\mathcal{F}$ with a controlled phase gate turn it into a combination of Hadamard and Toffoli gates. Mathematically,
% 	$$\mathcal{F}\circ H_s=H_s\circ\mathcal{F}$$
% 	$$\mathcal{F}\circ\Lambda_f(P(i)_s)=\Lambda^2_{f,s}(X_0Z_0)\circ\mathcal{F}=\Lambda^2_{f,s}(X_0)H_0\Lambda^2_{f,s}(X_0)H_0\circ\mathcal{F}$$

% 	We construct $C'$ so that $\mathcal{F}\circ C=C'\circ\mathcal{F}$ by following the computation above. It satisfies the required properties by construction.
% 	\begin{itemize}
% 		\item $C'$ uses only Hadamard and Toffoli gates.
% 		\item Exchanging $\mathcal{F}$ with $H$ doesn't change the circuit size. Exchanging $\mathcal{F}$ with controlled phase gate turns it into $4$ gates. So the final result is at most $4T$ gates.
% 		\item When $x$ is classical, $\mathcal{F}(x)=x||0$. So $F\circ C(x)=C'(x||0)$. It is simple \hannote{need a little more work?}to verify that $\mathcal{F}$ preserves $X$ and $Z$ measurement results on the first $n$ qubits.
% 	\end{itemize}
% \end{prf}

% \subsection{Reducing quantum circuit to local Hamiltonian}

% In this subsection, we prove \cref{thm:LHReduction} by showing how to reduce a $\SampBQP$ circuit to a local-$XZ$-Hamiltonian.

% 	The base construction comes from \cite{kitaev2002classical}. The simplification to $\spn\mathcal{G}_{XZ}$ is taken from \cite{PhysRevA.78.012352}.

% 	Let $x_i$ denote the $i$-th bit of $x$, and let $n$ be the number of qubits in $C$.

% 	We need to ensure the excited states of $H_{C(x)}$ have high eigenvalues.

% 	First, we ensure that the invalid clock states have high eigenvalues by applying the following Hamiltonian to the time register.
% 	$$H_{clock}=\sum_{t=1}^{T-1}\ket{01}\bra{01}_{t,t+1}$$
% 	As a sum of projections, clearly $H_{clock}\geq 0$. We shall also show that $H_{clock}\in\spn\mathcal{G}_{XZ}$.
% 	$$H_{clock}=\frac{1}{4}(Z_1 - Z_T) + \frac{1}{4}\sum_{t=1}^{T-1}(I-Z_tZ_{t+1}) $$
% 	This can be checked by fixing the first and last qubit, then doing induction on the number of switches.

% 	Then, we ensure that the initial condition is set up correctly.
% 	$$H_{in}=\sum_{i=1}^n(I-\ket{x_i}\bra{x_i})\otimes\ket{0}\bra{0}_1$$
% 	The kernel of this is precisely where everything is set up consistently with $\phi$ in time step $t=0$. Furthermore, $H_{in}\in\spn\mathcal{G}_{XZ}$
% 	$$H_{in}=\frac{1}{4}\sum_{i=1}^n(I-(-1)^{x_i}Z_i)\otimes(I+Z_1)$$

% 	Then, we ensure that the gates are applied correctly.
% 	$$H_{prop}=\sum_{t\in T_1}H_{prop,t}$$
% 	$$H_{prop,t}=I\otimes\ket{\widehat{t}}\bra{\widehat{t}}
% 	+I\otimes\ket{\widehat{t-1}}\bra{\widehat{t-1}}
% 	-U_t\otimes\ket{\widehat{t}}\bra{\widehat{t-1}}
% 	-U_t^\dagger\otimes\ket{\widehat{t-1}}\bra{\widehat{t}}$$

% 	We have $H_{prop}\geq 0$ according to \cite{2002quant.ph.10077A}; in fact, the least nonzero eigenvalue of $H_{prop}$ is lower bounded by $\frac{1}{2(T+1)^2}$.

% 	We can also write $H_{prop}\in\spn\mathcal{G}_{XZ}$. Note that $U^\dagger=U$, since our gates are either Hadamard or Toffoli.
% 	Additionally, $\frac{1}{2}(I-Z_{t-1})$ annihilates time steps before $t-1$. $\frac{1}{2}(I+Z_{t+1})$ similarly annihilates steps $t+1$ and after.
% 	$$H_{prop,t}=\frac{I}{4}\otimes(I-Z_{t-1})(I+Z_{t+1})-\frac{U}{4}\otimes(I-Z_{t-1})X_t(I+Z_{t+1})$$
% 	Extra care must be taken for boundary cases.
% 	$$H_{prop,1}=\frac{1}{2}(I+Z_2)-U_1\otimes\frac{1}{2}(X_1+X_1Z_2)$$
% 	$$H_{prop,T}=\frac{1}{2}(I-Z_{t-1})-U_T\otimes\frac{1}{2}(X_T-Z_{T-1}X_T)$$

% 	To properly combine the three Hamiltonians defined and analyze it using the projection lemma (\cref{thm:projection}), we consider the kernels of the Hamiltonians we defined.
% 	$$K_{clock}=\ker H_{clock}$$
% 	$$K_{in}=\ker H_{in}$$
% 	$$K_{prop}=\ker H_{prop}$$

% 	Clearly,
% 	$$K_{clock}=\set{\sum_{t=1}^T \ket{\phi_t}\otimes\ket{\hat{t}}:\ket{\phi_t}\in\cB^{\otimes n}}$$
% 	Let $\ket\phi\in K_{clock}\cap K_{prop}$.
% 	Consider some $\widetilde{t}$ such that $\braket{\widetilde{t}|\phi}\ne0$.
% 	As $\ket\phi\in K_{prop}$, we can then do induction using $\widetilde{t}$ as base case to get the relations between the $\ket{\phi_t}$s and conclude that $\ket\phi$ must have the form:
% 	$$\ket\phi=\sum_{t=0}^TU_t\ldots U_1\ket{y}\otimes\ket{\hat{t}}$$
% 	$$\Rightarrow K_{clock}\cap K_{prop}=\set{\sum_{t=0}^TU_t\ldots U_1\ket{y}\otimes\ket{\hat{t}}: \ket{y}\in\mathcal{B}^{\otimes n}}$$

% 	As a direct result, we get
% 	$$K_{clock}\cap K_{in}\cap K_{prop}=\spn\set{\ket{\psi_{C(x)}}}$$

% We now consider the space outside the desired state.
% $$S=(\spn\set{\ket{\psi_{C(x)}}})^\bot$$
% $$H_{clock}\big|_S,H_{in}\big|_S,H_{prop}\big|_S$$
% To combine the Hamiltonians, we apply the projection lemma twice.
% $$\exists J_{clock}
% =\frac{\poly\left(\norm{H_{in}\big|_S}\right)}{\lambda\left(H_{clock}\big|_{S\cap K^\bot_{clock}}\right)}
% =O(n)=O(T)$$
% $$\lambda(H_{in}\big|_S+J_{clock}H_{clock}\big|_S)\geq
% \lambda(H_{in}\big|_{S\cap K_{clock}})-\frac{1}{8}$$
% $$\exists J_{prop}=\frac{\poly\left(\norm{H_{in}\big|_S+J_{clock}H_{clock}\big|_S}\right)}{\lambda\left(H_{prop}\big|_{S\cap K^\bot_{prop}}\right)}
% =\frac{O(n+T)}{\Omega(T^{-2})}=O(T^3)$$
% $$\lambda(H_{in}\big|_S+J_{clock}H_{clock}\big|_S+J_{prop}H_{prop}\big|_S)\geq
% \lambda(H_{in}\big|_{S\cap K_{clock}\cap K_{prop}})-\frac{1}{4}$$
% \begin{align*}
% 	S\cap K_{clock}\cap K_{prop}&=S\cap\set{\sum_{t=0}^TU_t\ldots U_1\ket{y}\otimes\ket{\hat{t}}|\ket{y}\in\mathcal{B}^{\otimes n}}\\
% 	&=\set{\sum_{t=0}^TU_t\ldots U_1\ket{y}\otimes\ket{\hat{t}}:\braket{\psi_{C(x)}|y}=0}
% \end{align*}
% $$\Rightarrow\lambda((H_{in}+J_{clock}H_{clock}+J_{prop}H_{prop})\big|_S)\geq\frac{3}{4}$$
% So we set $H_{C(x)}=H_{in}+J_{clock}H_{clock}+J_{prop}H_{prop}$, which satisfies the required properties by construction.

% \subsection{Checking the ground state of the local Hamiltonian}

% In this subsection, we prove \cref{ThmXZCheck}.
% That is, we present and analyze an algorithm that checks whether a given state is the ground state of some fixed $H_{C(x)}$, following~\cite{PhysRevA.93.022326}.

% \begin{algorithm}
% 	\caption{Check for ground state}
% 	\label{AlgGroundStateCheck}
% 		Let $H=\sum_{S\in\mathcal{G}_{XZ}} d_S S$.
% 		Let $\ket\phi$ be the potential ground state to check.
% 		\begin{itemize}
% 			\item Set $D = \sum_{S\in\mathcal{G}_{XZ}}|d_S|$
% 			\item Set $p_S = \frac{|d_S|}{D}$
% 			\item Sample $\widetilde{S}$ from $\mathcal{G}_{XZ}$, weighted by $p_S$.
% 			\item Measure $\ket\phi$ in the $\widetilde{S}$ basis, recording the result as $\lambda_{\widetilde{S}}$.
% 			\item If $\sgn(d_{\widetilde{S}})\lambda_{\widetilde{S}}=-1$, accept. Otherwise, reject.
% 		\end{itemize}
% \end{algorithm}

% \begin{thm}
% 	Let $H$ have $O(\poly(T))$ nonzero components whose coefficients at most $O(\poly(T))$.
% 	Then \cref{AlgGroundStateCheck} accepts $\ket{\phi}$ with probability $\frac{1}{2}-\Omega(\frac{1}{\poly(T)})\braket{\phi|H_{C(x)}|\phi}$.
% \end{thm}
% \begin{prf}
% 	Here we follow \cite{PhysRevA.93.022326}.
% 	\begin{align*}
% 		\frac{1}{D}\braket{\phi|H|\phi}&=\sum_{S\in\mathcal{G}_{XZ}} p_S\sgn(d_S)\braket{\phi|S|\phi}\\
% 		&=\sum_{S\in\mathcal{G}_{XZ}} p_S\sgn(d_S)\E[\lambda_S]\\
% 		&=\E_{\widetilde{S}}[\sgn(d_{\widetilde{S}})\E[\lambda_{\widetilde{S}}]]\\
% 		&=\E_{\widetilde{S}}[\sgn(d_{\widetilde{S}})\lambda_{\widetilde{S}}]
% 	\end{align*}

% 	Note that $\sgn(d_{\widetilde{S}})\lambda_{\widetilde{S}}=\pm1$. Let $p$ be the probability that $\sgn(d_{\widetilde{S}})\lambda_{\widetilde{S}}=-1$.
% 	$$\Rightarrow \frac{1}{D}\braket{\phi|H|\phi}=\E_{\widetilde{S}}[\sgn(d_{\widetilde{S}})\lambda_{\widetilde{S}}]=-p+(1-p)$$
% 	\begin{align*}
% 		\Rightarrow p&=\frac{1}{2}-\frac{1}{2D}\braket{\phi|H|\phi}\\
% 		&=\frac{1}{2}-\Omega\left(\frac{1}{\poly(T)}\right)\braket{\phi|H|\phi}
% 	\end{align*}
% \end{prf}



\section{Proofs for \Cref{sec:qpip0_all}}
\label{sec:qpip0_proof}

% \begin{theorem}[binding property of $\PiNaive$]
% 	\label{lem:naive-qpip0-binding}
% 	Let $\PNaiveStar$ be a cheating $\BQP$ prover for $\PiNaive$ and $\lambda$ be the security parameter.
% 	Suppose that $\Prob{d=\Acc\mid y\ne\bot, c=0}$ is overwhelming, 
% 	under the QLWE assumption, then the verifier's output in the Hadamard round is $O(\eps)$-computationally indistinguishable from $(d, z_{ideal})$.
% \end{theorem}
\begin{proof}[\Cref{lem:naive-qpip0-binding}]
	We first introduce the \emph{dummy strategy} for ~\Cref{proto:urmila4}, where the prover chooses $\rho$ as the maximally mixed state and executes the rest of the protocol honestly.
	It is straightforward to verify that this prover would be accepted in the testing round with probability $1-\negl(\lambda)$,
	but has negligible probability passing the verification  after the Hadamard round.

  
   %for $\PiMeasure$ $\Pstarsub$ that is almost perfect as follows. 
	Now we construct a cheating $\BQP$ prover for \Cref{proto:qpip0_naive}, $\Pstar$, that does the same thing as $\PNaiveStar$ except at Step~\ref{step:urmila-in-naive}, where the prover and verifier runs \Cref{proto:urmila4}. $\Pstar$ does the following in Step~\ref{step:urmila-in-naive}:
	for the second message, run $(y, \sigma)\leftarrow\cPNaiveStar{2}(pk, \rho)$.
	If $y\ne\bot$, then reply $y$;
	else, run the corresponding step of the dummy strategy and reply with its results.
	For the fourth message, if $y\ne\bot$, run and reply with $a\leftarrow\cPNaiveStar{4}(pk, c, \sigma)$;
	else, continue the dummy strategy.


%	so we can apply \Cref{lem:urmila-binding} to the $\PiMeasure$ call to use its binding property (\Cref{lem:urmila-binding}).
%That is, there exists some $\rho$ such that $v=M_{XZ}(\rho, h)$.\hannote{only comp}


	 In the following we fix an $x$. Let the distribution on $h$ specified in Step~\ref{step:naive1} of the protocol be $p_x(h)$. Define $\Pstarsub(x)$ as $\Pstar$'s responds in Step~\ref{step:urmila-in-naive}. Note that we can view $\Pstarsub(x)$ as a prover strategy for \Cref{proto:urmila4}. By construction $\Pstarsub(x)$ passes testing round with overwhelming probability over $p_x(h)$, i.e. $\sum_h p_x(h) p_{h,T} =\negl(\lambda)$, where $p_{h,T}$ is $\Pstar$'s probability of getting accepted by the prover on the testing round on basis choice $h$. By \Cref{lem:urmila-binding} and Cauchy's inequality, there exists some $\rho$ such that  $\sum_h p_x(h) \norm{v_h -M_{XZ}(\rho, h)}_c = \negl(\lambda)$, where we use $\norm{A-B}_c=\alpha$ to denote that $A$ is $\alpha$-computational indistinguishable to $B$. Therefore $v= \sum_h p_x(h) v_h$ is computationally indistinguishable to $\sum_h p_x(h) M_{XZ}(\rho, h)$. Combining it with $\PiSamp$'s soundness (\Cref{QPIP1thm}), 
	we see that $(d', z')\leftarrow(\Pstar, \VNaive)(1^\lambda, 1^{1/\epsilon}, x)$  is $\eps$-computationally indistinguishable to $(d', z_{ideal}')$.

	Now we relate $(d', z')$ back to $(d, z)$.
	First, conditioned on that $\PNaiveStar$ aborts, since dummy strategy will be rejected with overwhelming probability in Hadamard round,
	we have $(d', z')$ is computationally indistinguishable to $(\Rej, \bot)=(d, z)$.
	On the other hand, conditioned on $\PNaiveStar$ not aborting, clearly $(d, z)=(d', z')$.
	So $(d, z)$ is computationally indistinguishable to $(d', z')$,
	which in turn is $O(\eps)$-computationally indistinguishable to $(d', z_{ideal}')$.
	Since $\norm{d-d'}_{tr}= O(\eps)$,
	 $(d, z_{ideal})$ is $O(\eps)$-computationally indistinguishable to $(d', z_{ideal}')$.
	Combining everything, we conclude that $(d, z)$ is $O(\eps)$-computationally indistinguishable to $(d, z_{ideal})$.
\end{proof}


\begin{proof}[\Cref{thm:zi-zgoodi}]
	We take expectation of \Cref{eq:partition-string} over $\gamma$
	$$\ket{\psi}=\E_{\gamma}\left[
		\sum_{j=0}^{i-1} \ket{\psi_{1^j0,\gamma}} +\ket{\psi_{1^i,\gamma}} +\sum_{j=1}^{i}\ket{\psi_{err,j,\gamma}}
	\right],$$
	and expand $z_i$ from \Cref{eq:zi-def} as
	\begin{align}
		z_i &= z_{good,i}+ \E_{pk, y, \gamma} \sum_z \L[\sum_{k=0}^{i-1} \bra{\psi_{1^k0,\gamma}}U^\dag  P_{acc,i,z}U   \sum_{j=0}^{i-1} \ket{\psi_{1^j0,\gamma}} \R. \nn \\
		&+
		\sum_{k=0}^{i-1} \bra{\psi_{1^k0,\gamma}}U^\dag  P_{acc,i,z}U \ket{\psi_{1^i,\gamma}} +\sum_{k=0}^{i-1} \bra{\psi_{1^k0,\gamma}}U^\dag  P_{acc,i,z}U\sum_{j=1}^{i}\ket{\psi_{err,j,\gamma}} \nn \\
		&+\bra{\psi_{1^i,\gamma}} U^\dag  P_{acc,i,z}U \sum_{j=0}^{i-1} \ket{\psi_{1^j0,\gamma}} +\bra{\psi_{1^i,\gamma}} U^\dag  P_{acc,i,z}U \sum_{j=1}^{i}\ket{\psi_{err,j,\gamma}}
		\nn \\
		&+ \sum_{k=1}^{i}\bra{\psi_{err,k,\gamma}} U^\dag  P_{acc,i,z}U  \sum_{j=0}^{i-1} \ket{\psi_{1^j0,\gamma}} + \sum_{k=1}^{i}\bra{\psi_{err,k,\gamma}} U^\dag  P_{acc,i,z}U \ket{\psi_{1^i,\gamma}}
		\nn \\
		&\L.    +\sum_{k=1}^{i}\bra{\psi_{err,k,\gamma}} U^\dag  P_{acc,i,z}U \sum_{j=1}^{i}\ket{\psi_{err,j,\gamma}} \R] \proj{z} , \nn     
		%=& z_{good,i} +(\text{terms with } \psi_{1^j0},\, j\neq i ) + (\text{terms with } \psi_{1^{i-1}0}) +(\text{terms with }err )
	\end{align}
	where we omitted writing out $e_i$.
	Therefore we have
	\begin{align*}
		\tr|z_i-z_{good,i}|\leq \E_{pk, y, \gamma} \sum_z &\L[ \sum_{k=0}^{i-1} \sum_{j=0}^{i-1} \L| \bra{\psi_{1^k0,\gamma}}U^\dag  P_{acc,i,z}U \ket{\psi_{1^j0,\gamma}} \R|\R.\\
		&+
		2 \sum_{k=0}^{i-1} \L|\bra{\psi_{1^k0,\gamma}}U^\dag  P_{acc,i,z}U \ket{\psi_{1^i,\gamma}} \R| \\
		&+ 2 \sum_{k=0}^{i-1}\sum_{j=1}^{i}\L| \bra{\psi_{1^k0,\gamma}}U^\dag  P_{acc,i,z}U\ket{\psi_{err,j,\gamma}}\R| \\   
		&+2 \sum_{j=1}^{i}\L|\bra{\psi_{1^i,\gamma}} U^\dag  P_{acc,i,z}U \ket{\psi_{err,j,\gamma}}\R| \\
		&+\L. \sum_{k=1}^{i}\sum_{j=1}^{i}\L| \bra{\psi_{err,k,\gamma}} U^\dag  P_{acc,i,z}U \ket{\psi_{err,j,\gamma}}\R| \R] \\ %%%%%%%%%
	\end{align*}
	by the triangle inequality.
	The last three error terms sum to $O\L(\frac{m^2}{\sqrt{T}}\R)$ by \Cref{lem:samp-tech} and property~\ref{property:partition-err} of \Cref{lem:partition2}.
	As for the first two terms, by \Cref{lem:samp-tech} and \Cref{lem:partition-testing}, we see that
	\begin{align*}
		\sum_z \sum_{k=0}^{i-1}\sum_{j=0}^{i-1}
		&\abs{\bra{\psi_{1^k0,\gamma}}U^\dag  P_{acc,i,z}U \ket{\psi_{1^j0,\gamma}}} \\
		&\leq\sum_z \abs{\bra{\psi_{1^{i-1}0,\gamma}}U^\dag  P_{acc,i,z}U \ket{\psi_{1^{i-1}0,\gamma}}} + O\L(m^2(m-1)\gamma_0\R) \\
		&\leq\norm{\ket{\psi_{1^{i-1}0,\gamma}}}^2 + O\L(m^2(m-1)\gamma_0\R)
	\end{align*}
	and similarly
	\begin{align*}
		\sum_z\sum_{k=0}^{i-1}
		&\abs{\bra{\psi_{1^k0,\gamma}}U^\dag  P_{acc,i,z}U \ket{\psi_{1^i,\gamma}}}\\
		&\leq\sum_z\abs{\bra{\psi_{1^{i-1}0,\gamma}}U^\dag  P_{acc,i,z}U \ket{\psi_{1^i,\gamma}}}+O\L(m\sqrt{(m-1)\gamma_0}\R)\\
		&\leq\norm{\ket{\psi_{1^i,\gamma}}}+O\L(m\sqrt{(m-1)\gamma_0}\R).
	\end{align*}
\end{proof}



\end{document}
