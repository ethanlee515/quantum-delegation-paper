\documentclass{article}
\usepackage{authblk}
\usepackage{amsmath, amssymb, amsthm}

\newtheorem{theorem}{Theorem}[section]
\theoremstyle{definition}
\newtheorem{definition}{Definition}[section]

\title{Placeholder Title... Something About Delegating Quantum Computations}

\author[1]{Kai-Min Chung}
\author[1]{Yi Lee}
\author[2]{Han-Hsuan Lin}
\author[3]{Xiaodi Wu}
\affil[1]{Institute of Information Science, Academia Sinica, Taipei, Taiwan}
\affil[2]{Department of Computer Science, University of Texas at Austin}
\affil[3]{
	Department of Computer Science, Institute for Advanced Computer Studies,
	and Joint Center for Quantum Information and Computer Science,
	University of Maryland, USA
}

\begin{document}

\maketitle

\begin{abstract}

This is a placeholder. Lorem ipsum. Lorem ipsum. Lorem ipsum...

\end{abstract}

\section{Introduction}

Below is some intro of it copied from my SoP.
This is unpolished and mostly a placeholder at the moment though.

It was proven that BQP=BQIP. That is, if a quantum computer can efficiently solve a given decision problem, then it can also efficiently convince a classical machine of its solution. I'm generalizing this to arbitrary efficient quantum computations. The proof for decision problems involves the classical verifier reducing the problem to a local Hamiltonian instance; the quantum prover would then commit its certificate and act as the verifier’s trusted measurement device as put forth in "Classical Verification of Quantum Computations" by Mahadev. It isn't as trivial as it may seem. Repeating the scheme for each qubit loses the information carried by entanglements and throws off the joint distribution between qubits. Simply measuring the entire output register instead is difficult to analyze. For decision problems, it’s not hard to argue that a malicious prover cannot do better than sending identical copies of some pure state unentangled with each others. That same reasoning doesn't apply here a priori. I've been trying to get a grasp on the particular structure of the local Hamiltonian reduction in order to better analyze it.

\section{Preliminaries}

\subsection{Notations}

Let $\mathcal{B}$ denote the Hilbert space corresponding to a qubit.
Let $H:\mathcal{B}^{\otimes N}\rightarrow\mathcal{B}^{\otimes N}$ be Hermitian.

\begin{definition}
	Let $H\geq0$ denote $H$ is positive semidefinite.
\end{definition}

\begin{definition}
	Let $\lambda(H)$ as the least eigenvalue of $H$.
\end{definition}

\begin{definition}
	Let the \emph{ground state} of $H$ be the eigenvector corresponding to $\lambda(H)$.
\end{definition}

\begin{definition}
	Denote the \emph{controlled phase gate}: $C_\phi|xy\rangle=(-1)^{xy}|xy\rangle$.
\end{definition}

\begin{definition}
	Let $H\big|_S=\prod_SH\prod_S$, where $\prod_S$ is the projection onto the subspace $S$.
\end{definition}

\begin{definition}
	Let $|\widehat{t}\rangle$ denote $t$ represented in unary. That is, a state of the form $|11\ldots1\rangle |00\ldots0\rangle$; that is, $t$ $1$s followed by all $0$s.
\end{definition}

\subsection{Local Hamiltonian}

The local Hamiltonian problem is defined as follows:

\begin{definition}
	Let $H_1, \ldots, H_n:\mathcal{B}^{\otimes N}\rightarrow\mathcal{B}^{\otimes N}$ be Hermitian. Then $$H=\sum_{j=1}^nH_j$$ is called a \emph{Hamiltonian}.
	If each $H_j$ acts on at most $k$ qubits, then $H$ is called a \emph{$k$-local Hamiltonian}.
\end{definition}

The local Hamiltonian problem is given by let $a<b$ with some inverse polynomial gap, let $H=\sum H_j$, determine whether $\lambda(H)<a$ or $\lambda(H)>b$..

\begin{theorem}
	The Local Hamiltonian problem is QMA-complete.
\end{theorem}

\subsection{Projection Lemma}

Here is a lemma that we will use, taken from \cite{quant-ph/0406180}.

Let $H_1, H_2$ be Hamiltonians where $H_2\geq0$, and $S=\ker H_2$.
Then $\exists J\in\mathbb{R}$ s.t.
$\lambda(H_1\big|_S)-\frac{1}{8}\leq
	\lambda(H_1+JH_2)\leq\lambda(H_1\big|_S)$.

\section{The Scheme}

Here we describe the scheme between the classical client and an honest quantum server.

\subsection{Reduction to Real Quantum Computation}

The transform is found in \cite{1109.0795}. It turns the states and the unitary matrices into real numbers.

TODO: maybe describe the transform here as needed. Need to show that this doesn't change X/Z measurements. Also need to show that the result of this can still be rewritten into controlled phase gates and one-qubit gates.

\subsection{Reduction to 2-local Hamiltonians}

Then we reduce the circuit to an instance of 2-local Hamiltonian. This reduction is mostly based on \cite{quant-ph/0406180}.

\subsection{preprocess the circuit}

First, we rewrite the circuit to consist of only controlled phase gates and 1-qubit gates. TODO: prove that these 1-qubit gates are all real. We then double (TODO maybe multiply by $\gamma$ instead of doubling?) the size of the circuit by padding identity matrices at the end.

\subsection{2-local Hamiltonian construction}

We then attempt to construct a 2-local Hamiltonian with ground state $$\phi=\sum_{t=0}^TU_t...U_1|x\rangle\otimes|\hat{t}\rangle$$.

We then follow the rest of \cite{quant-ph/0406180} and reduce the circuit into a 2-Local Hamiltonian instance. We can then turn all the Hamiltonians involved into some combinations of $X$ and $Z$ matrices following \cite{0704.1287}.
It will drop the $H_{out}$ term and simply set $H=J_{in}H_{in}+J_2H_{prop2}+J_1H_{prop1}+J_{clock}H_{clock}$.

\subsection{2-local ZX Hamiltonians}

We then follow \cite{0704.1287} in writing each terms of $H$ into only $Z$ and $X$ measurements...

This probably involves some kind of basis arguments on the Hamiltonians and maybe some stuff that's a bit more ugly. This should not change the state. I'll have to write this out later...

\subsection{Checking computation}

We then follow \cite{1804.01082}. That is, the server would commit copies of the ground state of the Hamiltonian for the classical client to measure. The client would check whether the computation is done correctly using a modified Hamiltonian from \cite{quant-ph/0406180}. 


\subsection{Getting the Output}



\begin{thebibliography}{9}
	\bibitem{1804.01082}
	Urmila Mahadev.
	\newblock Classical Verification of Quantum Computations, 2018;
	\newblock arXiv:1804.01082.

	\bibitem{quant-ph/0406180}
	Julia Kempe, Alexei Kitaev and Oded Regev.
	\newblock The Complexity of the Local Hamiltonian Problem, 2004,
	\newblock SIAM Journal of Computing, Vol. 35(5), p. 1070-1097 (2006),
		conference version in Proc. 24th FSTTCS, p. 372-383 (2004);
	\newblock arXiv:quant-ph/0406180.

	\bibitem{1109.0795}
	Matthew McKague.
	\newblock On the power quantum computation over real Hilbert spaces, 2011,
	\newblock Int. J. Quantum Inform., 11, 1350001 (2013);
	\newblock arXiv:1109.0795.
	\newblock DOI: 10.1142/S0219749913500019.
	
	\bibitem{0704.1287}
	Jacob D. Biamonte and Peter J. Love.
	\newblock Realizable Hamiltonians for Universal Adiabatic Quantum Computers, 2007,
	\newblock Phys. Rev. A 78, 012352 (2008).;
	\newblock arXiv:0704.1287.
	\newblock DOI: 10.1103/PhysRevA.78.012352.

\end{thebibliography}

\end{document}
