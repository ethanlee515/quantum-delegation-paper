\documentclass{article}
\usepackage{authblk}
\usepackage{amsmath, amssymb, amsthm}
\usepackage{braket}
\usepackage{algorithm}
\usepackage{algpseudocode}
\newcommand{\sgn}{\operatorname{sgn}}
\DeclareMathOperator*{\E}{\mathbb{E}}
\DeclareMathOperator*{\spn}{\operatorname{span}}
\DeclareMathOperator*{\poly}{\operatorname{poly}}
\newcommand{\norm}[1]{\left\lVert#1\right\rVert}
\usepackage{etoolbox}
\usepackage{hyperref}
\apptocmd{\thebibliography}{\raggedright}{}{}
\usepackage{cite,color,float}


\newcommand{\KM}[1]{{\footnotesize\color{cyan}[KM: #1]}}


\newtheorem{theorem}{Theorem}[section]
\theoremstyle{definition}
\newtheorem{definition}{Definition}[section]
\newtheorem{lemma}[theorem]{Lemma}

\title{Placeholder Title... Something About Delegating Quantum Computations}

\author[1]{Kai-Min Chung}
\author[1]{Yi Lee}
\author[2]{Han-Hsuan Lin}
\author[3]{Xiaodi Wu}
\affil[1]{Institute of Information Science, Academia Sinica, Taipei, Taiwan}
\affil[2]{Department of Computer Science, University of Texas at Austin}
\affil[3]{
	Department of Computer Science, Institute for Advanced Computer Studies,
	and Joint Center for Quantum Information and Computer Science,
	University of Maryland, USA
}
\usepackage{graphicx,amsmath, amssymb,color,url,booktabs,comment}  %cite
\urlstyle{sf}
%\usepackage[margin=1in]{geometry}
% \usepackage{fancyhdr}
%\usepackage[colorlinks]{hyperref} %pagebackref

\newcommand{\nc}{\newcommand}
\nc{\rnc}{\renewcommand}

\def\View{\mathsf{View}}

\def\GS{\mathsf{Ham}}
\nc{\cVGS}{\ensuremath{\cV_\GS}}

\def\Samp{\mathsf{Samp}}
\nc{\PiSamp}{\ensuremath{\Pi_\Samp}}
\nc{\VSamp}{\ensuremath{V_\Samp}}
\nc{\PSamp}{\ensuremath{P_\Samp}}
\nc{\PSampstar}{\ensuremath{P_\Samp^*}}
\nc{\cVSamp}[1]{\ensuremath{\cV_{\Samp,#1}}}
\nc{\cPSamp}[1]{\ensuremath{\cP_{\Samp,#1}}}

\def\HE{\mathsf{HE}}
\def\HGen{\mathsf{HE.Keygen}}
\def\HEnc{\mathsf{HE.Enc}}
\def\HEval{\mathsf{HE.Eval}}
\def\HDec{\mathsf{HE.Dec}}
\def\Rej{\mathsf{Rej}}

\def\blind{\mathsf{blind}}
\nc{\Piblind}{\ensuremath{\Pi_\blind}}
\nc{\Vblind}{\ensuremath{V_\blind}}
\nc{\Pblind}{\ensuremath{P_\blind}}
\nc{\Pblindstar}{\ensuremath{P_\blind^*}}
\nc{\cVblind}[1]{\ensuremath{\cV_{\blind,#1}}}
\nc{\cPblind}[1]{\ensuremath{\cP_{\blind,#1}}}
\def\Pstar{P^*}
\nc{\cPstar}[1]{\ensuremath{\cP^*_{#1}}}
\nc{\ctx}[3]{\ensuremath{{{\widehat{#1}}_{#2}^{(#3)}}}}

%
%\newcommand{\bra}[1]{\langle #1|}
%\newcommand{\ket}[1]{|#1\rangle}
\newcommand{\proj}[1]{|#1\rangle\langle #1|}
% \newcommand{\braket}[2]{\langle #1|#2\rangle}
% \newcommand{\Bra}[1]{\left\langle #1\right|}
% \newcommand{\Ket}[1]{\left|#1\right\rangle}
\newcommand{\Proj}[1]{\left|#1\right\rangle\left\langle #1\right|}
% \newcommand{\Braket}[2]{\left\langle #1\middle|#2\right\rangle}
\nc{\vev}[1]{\langle#1\rangle}
\nc{\grad}{{\vec{\nabla}}}
\nc{\abs}[1]{\lvert#1\rvert}
%\DeclareMathOperator{\abs}{abs}
\DeclareMathOperator{\Bin}{Bin}
\DeclareMathOperator{\conv}{conv}
\DeclareMathOperator{\eig}{eig}
\DeclareMathOperator{\Hist}{Hist}
\DeclareMathOperator{\Hyb}{Hyb}
\DeclareMathOperator{\id}{id}
\DeclareMathOperator{\Img}{Im}
\DeclareMathOperator{\Par}{Par}
% \DeclareMathOperator{\poly}{poly}
\DeclareMathOperator{\negl}{negl}
\DeclareMathOperator{\polylog}{polylog}
\DeclareMathOperator{\tr}{tr}
\DeclareMathOperator{\rank}{rank}
% \DeclareMathOperator{\sgn}{sgn}
\DeclareMathOperator{\Sep}{Sep}
\DeclareMathOperator{\SepSym}{SepSym}
\DeclareMathOperator{\Span}{span}
\DeclareMathOperator{\supp}{supp}
\DeclareMathOperator{\swap}{SWAP}
\DeclareMathOperator{\Sym}{Sym}
\DeclareMathOperator{\ProdSym}{ProdSym}
\DeclareMathOperator{\SEP}{SEP}
\DeclareMathOperator{\PPT}{PPT}
\DeclareMathOperator{\Wg}{Wg}
\DeclareMathOperator{\WMEM}{WMEM}
\DeclareMathOperator{\WOPT}{WOPT}

\DeclareMathOperator{\BPP}{\mathsf{BPP}}
\DeclareMathOperator{\QPIP}{\mathsf{QPIP}}
\DeclareMathOperator{\SampBQP}{\mathsf{SampBQP}}
\DeclareMathOperator{\BQP}{\mathsf{BQP}}
\DeclareMathOperator{\FBQP}{\mathsf{FBQP}}
\DeclareMathOperator{\cnot}{\normalfont\textsc{cnot}}
\DeclareMathOperator{\DTIME}{\mathsf{DTIME}}
\DeclareMathOperator{\NTIME}{\mathsf{NTIME}}
\DeclareMathOperator{\MA}{\mathsf{MA}}
\DeclareMathOperator{\NP}{\mathsf{NP}}
\DeclareMathOperator{\NEXP}{\mathsf{NEXP}}
\DeclareMathOperator{\Ptime}{\mathsf{P}}
\DeclareMathOperator{\QMA}{\mathsf{QMA}}
\DeclareMathOperator{\QCMA}{\mathsf{QCMA}}
\DeclareMathOperator{\BellQMA}{\mathsf{BellQMA}}

\newcommand{\be}{\begin{equation}}
\newcommand{\ee}{\end{equation}}
\newcommand{\bea}{\begin{eqnarray}}
\newcommand{\eea}{\end{eqnarray}}
\newcommand{\nn}{\nonumber}
\newcommand{\bi}{\begin{itemize}}
\newcommand{\ei}{\end{itemize}}
\newcommand{\bn}{\begin{enumerate}}
\newcommand{\en}{\end{enumerate}}
\def\beas#1\eeas{\begin{eqnarray*}#1\end{eqnarray*}}
\def\ba#1\ea{\begin{align}#1\end{align}}
\nc{\bas}{\[\begin{aligned}}
\nc{\eas}{\end{aligned}\]}
\nc{\bpm}{\begin{pmatrix}}
\nc{\epm}{\end{pmatrix}}
\def\non{\nonumber}
\def\nn{\nonumber}
\def\eq#1{(\ref{eq:#1})}
\def\eqs#1#2{(\ref{eq:#1}) and (\ref{eq:#2})}
%\def\eq#1{Eq.~(\ref{eq:#1})}
%\def\eqs#1#2{Eqs.~(\ref{eq:#1}) and (\ref{eq:#2})}
\def\L{\left} 
\def\R{\right}
\def\ra{\rightarrow}
\def\ot{\otimes}
\nc{\given}{\ensuremath{\;\middle|\;}}

\newtheorem{thm}{Theorem}[section]
\newtheorem{theorem}{Theorem}[section]
%\newtheorem*{thm*}{Theorem}
%\newtheorem{claim}[thm]{Claim}
\newtheorem{cor}{Corollary}[thm]
\newtheorem{lem}{Lemma}[section]
\newtheorem{lemma}{Lemma}[section]
\newtheorem{rmk}{Remark}[thm]
%\newtheorem{prop}[thm]{Proposition}
\newtheorem{dfn}{Definition}[section]
\newtheorem{definition}{Definition}[section]
%\newtheorem{con}[thm]{Conjecture}

\newenvironment{prf}{\begin{proof}}{\end{proof}}

\def\eps{\epsilon}
\def\va{{\vec{a}}}
\def\vb{{\vec{b}}}
\def\vn{{\vec{n}}}
\def\cvs{{\cdot\vec{\sigma}}}
\def\vx{{\vec{x}}}
\def\Usch{U_{\text{Sch}}}

\def\cA{\mathcal{A}}
\def\cB{\mathcal{B}}
\def\cD{\mathcal{D}}
\def\cE{\mathcal{E}}
\def\cF{\mathcal{F}}
\def\cH{\mathcal{H}}
\def\cI{{\cal I}}
\def\cL{{\cal L}}
\def\cM{{\cal M}}
\def\cN{\mathcal{N}}
\def\cO{{\cal O}}
\def\cP{\mathcal{P}}
\def\cQ{\mathcal{Q}}
\def\cS{\mathcal{S}}
\def\cT{{\cal T}}
\def\cU{\mathcal{U}}
\def\cV{\mathcal{V}}
\def\cW{{\cal W}}
\def\cX{{\cal X}}
\def\cY{{\cal Y}}

\def\bp{\mathbf{p}}
\def\bq{\mathbf{q}}
\def\bP{{\bf P}}
\def\bQ{{\bf Q}}
\def\gl{\mathfrak{gl}}

\def\bbC{\mathbb{C}}
% \DeclareMathOperator*{\E}{\mathbb{E}}
\DeclareMathOperator*{\bbE}{\mathbb{E}}
%\DeclareMathOperator*{\Pr}{Pr}
\nc{\Prob}[1]{\ensuremath{\Pr\left[#1\right]}}
\def\bbM{\mathbb{M}}
\def\bbN{\mathbb{N}}
\def\bbR{\mathbb{R}}
\def\bbZ{\mathbb{Z}}
\def\bbP{\mathbb{P}}
\def\bbV{\mathbb{V}}
\newcommand{\Real}{\textrm{Re}}

\def\benum{\begin{enumerate}}
\def\eenum{\end{enumerate}}
% \def\bit{\begin{itemize}}
% \def\eit{\end{itemize}}
\def\bdesc{\begin{description}}
\def\edesc{\end{description}}
\newcommand{\fig}[1]{Fig.~\ref{fig:#1}}
\newcommand{\tab}[1]{Table~\ref{tab:#1}}
\newcommand{\secref}[1]{Section~\ref{sec:#1}}
\newcommand{\appref}[1]{Appendix~\ref{sec:#1}}
\newcommand{\lemref}[1]{Lemma~\ref{lem:#1}}
\newcommand{\thmref}[1]{Theorem~\ref{thm:#1}}
\newcommand{\propref}[1]{Proposition~\ref{prop:#1}}
\newcommand{\protoref}[1]{Protocol~\ref{proto:#1}}
\nc{\myprotoref}[1]{\hyperref[#1]{Protocol~\ref*{#1}}}
\newcommand{\defref}[1]{Definition~\ref{def:#1}}
\newcommand{\corref}[1]{Corollary~\ref{cor:#1}}
\newcommand{\conref}[1]{Conjecture~\ref{con:#1}}

\newcommand{\FIXME}[1]{{\color{red}FIXME: #1}}
\nc{\todo}[1]{\textcolor{red}{todo: #1}}



\newcommand{\boxdfn}[2]{
\begin{figure}[h]
\begin{center}
\noindent \framebox{
\begin{minipage}{0.8\textwidth}
\begin{dfn}[{\bf #1}]
\ \\ \\
#2
\end{dfn}
\end{minipage}
}
\end{center}
\end{figure}
}

\newcommand{\boxproto}[2]{
\begin{figure}[h]
\begin{center}
\noindent \framebox{
\begin{minipage}{0.8\textwidth}
\begin{proto}[{\bf #1}]
\ \\ \\
#2
\end{proto}
\end{minipage}
}
\end{center}
\end{figure}
}

\def\begsub#1#2\endsub{\begin{subequations}\label{eq:#1}\begin{align}#2\end{align}\end{subequations}}
\nc\qand{\qquad\text{and}\qquad}
\nc\mnb[1]{\medskip\noindent{\bf #1}}
\nc\mn{\medskip\noindent}

\renewcommand{\arraystretch}{1.5}
%\nc{\problem}[1]{\item\noindent {\bf #1}}

\setlength{\tabcolsep}{10pt}

%%%%%% Han-Hsuan's commands %%%%%%%%
\nc{\nl}{\nn \\ &=}  %new line
\nc{\nnl}{\nn \\ &}  %new new line
\nc{\fot}{\frac{1}{2}} %frac one two
\nc{\oo}[1]{\frac{1}{#1}} % one over
\newcommand{\ben}{\begin{enumerate}}
\newcommand{\een}{\end{enumerate}}
\nc{\mc}{\mathcal}
\nc{\beq}{\begin{equation}}
\nc{\eeq}{\end{equation}}
% \nc{\norm}[1]{\L\| #1 \R\|}

\nc{\onenorm}[1]{\L\| #1 \R\|_1} %one norm
%\nc{\span}{\ensuremath{\mathrm{span}}}

\DeclareMathOperator*{\argmax}{arg\,max}

%\nc{1}

\newcommand{\hannote}[1]{\textcolor{blue}{\small {\textbf{(Han:} #1\textbf{) }}}}

\newcommand{\Knote}[1]{\textcolor{red}{\small {\textbf{(KM:} #1\textbf{) }}}}

\nc{\Ra}{\Rightarrow}
\nc{\zo}{\{0,1\}}	

%%%%import..
% \newcommand{\secpar}{n}


% %%%Efficient Verifier%
% \newcommand{\setupeff}{\setup_{\mathsf{eff}}}
% \newcommand{\vereff}{V_{\mathsf{eff}}}
% \newcommand{\vereffone}{V_{\mathsf{eff},1}}
% \newcommand{\vereffthree}{V_{\mathsf{eff},3}}
% \newcommand{\vereffout}{V_{\mathsf{eff},\mathsf{out}}}
% \newcommand{\proeff}{P_{\mathsf{eff}}}
% \newcommand{\proefftwo}{P_{\mathsf{eff},2}}
% \newcommand{\proefffour}{P_{\mathsf{eff},4}}
% \newcommand{\advPH}{{P^*}^{H}}
% \newcommand{\setup}{\mathsf{Setup}}
% \newcommand{\re}{\mathsf{re}}
% \newcommand{\crh}{\mathsf{CRH}}
% \newcommand{\transcript}{\mathsf{trans}}

% \newcommand{\setupefffs}{\setup_{\mathsf{eff}\text{-}\mathsf{fs}}}
% \newcommand{\proefffs}{P_{\mathsf{eff}\text{-}\mathsf{fs}}}
% \newcommand{\proefffstwo}{P_{\mathsf{eff}\text{-}\mathsf{fs},2}}
% \newcommand{\verefffs}{V_{\mathsf{eff}\text{-}\mathsf{fs}}}
% \newcommand{\verefffsone}{V_{\mathsf{eff}\text{-}\mathsf{fs},1}}
% \newcommand{\verefffsout}{V_{\mathsf{eff}\text{-}\mathsf{fs},\out}}
% %Games%
% \newcommand{\game}{\mathsf{Game}}

% %\newcommand*{\bra}[1]{\langle#1|}
% %\newcommand*{\ket}[1]{|#1\rangle}
% \newcommand*{\opro}[2]{|#1\rangle\langle#2|}
% \newcommand*{\ipro}[2]{\langle #1|#2\rangle}
% \newcommand{\TD}{\mathsf{TD}}

% %%%%% Registers %%%%%%%
% \newcommand*{\regK}{\mathbf{K}}
% \newcommand*{\regI}{\mathbf{I}}
% \newcommand*{\regR}{\mathbf{R}}
% \newcommand*{\regX}{\mathbf{X}}
% \newcommand*{\regY}{\mathbf{Y}}
% \newcommand*{\regZ}{\mathbf{Z}}
% \newcommand{\regW}{\mathbf{W}}
% \newcommand*{\regC}{\mathbf{C}}
% \newcommand*{\regO}{\mathbf{O}}
% \newcommand*{\regF}{\mathbf{F}}

\begin{document}

\maketitle

\begin{abstract}

This is a placeholder. Lorem ipsum. Lorem ipsum. Lorem ipsum...

\end{abstract}

\section{Introduction}
% \XW{
% \begin{itemize}
%     \item add a lot of references.
%     \item comparison with the most relevant results:
%       \begin{itemize}
%           \item Sampling, the only paper; how about classical sampling?
%           \item the following for BQP  
%           \item blind and verifiable ~\cite{GV19}; we  constant round; technique-wise very different.
%           \item there is a table in~\cite{Grilo19}. Safe to say ~\cite{GV19} only existing blind protocol in the computational setting?
%           \item all previous either quantum clients, or at least 2 provers.
%           \item the following for blindness
%           \item Mahadev in her thesis~\cite{mahadev_2018} discussed a bit about the relation between verifiability and blindness. She hoped to get verifiability out of blindness by designing some non-malleable QFHE but failed.   
%           \item what's the high-level message we can say here?  Use QFHE in a different way? It is correct that not much work in the classical setting either.  Maybe existing classical work employs the principle but with different implementation.
%           \item old approach, first get blindness  (measurement-based, self-testing), and then try to make it verifiable; our approach, first have a verifiable protocol, and upgrade by a QFHE.
%           \item directly QFHE (blindness) won't give verifiability. some thoughts from Mahadev.
%           \item
%       \end{itemize}
% \end{itemize}
% }
Can quantum computation, with potential computational advantages that are intractable for classical computers,
be efficiently verified by classical human beings?
This seeming paradox has been one of the central problems in quantum complexity theory and delegation of quantum computation~\cite{web:Aaronson}.
From a philosophical point of view, this question is also known to have a fascinating connection to the \emph{falsifiability} of quantum mechanics in the potential high complexity regime~\cite{survey:AV12}.

A complexity theoretic formulation of this problem by Gottesman in 2004~\cite{web:Aaronson} asks the possibility for an efficient classical verifier/client (a $\BPP$ machine) to verify the output of an
efficient quantum prover (a $\BQP$ machine).
In the absence of techniques for directly tackling this question, earlier feasibility results on this problem have been focusing on two weaker formulations.
The first type of feasibility results (e.g.,~\cite{BFK09,arXiv:ABOEM17,FK17,mf16}) considers the case where the $\BPP$ verifier is equipped with limited quantum power.
The second type of feasibility results (e.g,~\cite{Nat:RUV13, CGJV19, Gheorghiu_2015, HPF15})
considers a $\BPP$ verifier interacting with at least two entangled, non-communicating quantum provers.
In a recent breakthrough, Mahadev~\cite{FOCS:Mahadev18a} proposed the first protocol of classical verification of quantum computation (CVQC) whose soundness is based on a widely recognized computational assumption that the learning with error (LWE) problem~\cite{JACM:Regev09} is hard for $\BQP$ machines.
The technique invented therein has inspired many  subsequent developments of CVQC protocols with improved parameters and functionality (e.g.~\cite{FOCS:GheVid19,arXiv:AlaChiHun19,arXiv:ChiaChungYam19}).
We refer curious readers to the survey~\cite{survey:GKK19} for details.

With the newly developed techniques, we revisit the classical verification of quantum computation problems from both a philosophical and a practical point of view.
We first observe that the \emph{sampling} version of $\BQP$ (e.g., the class $\SampBQP$ formulated by Aaronson~\cite{aaronson_2013}) might be a more appropriate notion to serve the purpose of the original problem.
Philosophically, the outcomes of quantum mechanical experiments are usually samples or statistical information, which is well demonstrated in the famous double-slit experiment.
Moreover, a lot of quantum algorithms from Shor's~\cite{Shor} and Grover's~\cite{Grover} algorithms to some recent developments in machine learning and optimization (e.g.~\cite{brando_et_al:LIPIcs:2019:10603, AGGW17,pmlr-v97-li19b}) contain a significant quantum sampling component.
The fact that almost all quantum supremacy tasks (e.g.,~\cite{Boson, IQP, nature-google}) are sampling ones strengthens the importance of delegation for quantum sampling problems.
 %Even though the relation between $\BQP$ and $\SampBQP$ is relatively understood in the plain model,
However, it is far from clear whether the subtle difference between $\BQP$ and its sampling version could lead to
technical barrier in the context of CVQC, or
 %what are the potential technical challenges raised by their subtle differences in the context of CVQC and how to resolve them is however far from clear.
whether one can develop a CVQC protocol for $\SampBQP$ based on Mahadev's technique~\cite{FOCS:Mahadev18a}.

Another desirable property of CVQC protocols is the \emph{blindness} where the prover cannot distinguish the particular computation in the protocol from another one of the same size, and hence is blind about the client's input.
Historically, blindness has been achieved in the weaker formulations of CVQC based on various techniques: e.g., the measurement-based quantum computation exploited in~\cite{BFK09}, the quantum authentication scheme exploited in~\cite{arXiv:ABOEM17}, and the self-testing technique exploited in~\cite{Nat:RUV13}.
Moreover, the blindness property is known to be helpful to establish the verifiability of CVQC protocols. However, this is never an easy task.
See for example the significant amount of efforts to add verifiability to blind CVQC protocols in~\cite{FK17}.
Achieving both blindness and verifiability on top of Mahadev's technique~\cite{FOCS:Mahadev18a} is a conceivably much more challenging task.
The only successful attempt~\cite{FOCS:GheVid19} so far applies Mahadev's technique to the measurement-based quantum computation,
whereas the analysis is still very specific to the construction.
Could there be a \emph{generic} way to achieve blindness and verifiability for CVQC protocols at the same time?


\vspace{2mm} \noindent \textbf{Contribution.} We provide \emph{affirmative} solutions to both of our questions.
In particular, we demonstrate the feasibility of the classical verification of quantum sampling by
constructing a constant-round CVQC protocol for $\SampBQP$, the sampling version of $\BQP$ formulated by Aaronson~\cite{aaronson_2013}. Formally, $\SampBQP$ consists of sampling problems $(D_x)_{x\in\zo^*}$ that can be approximately sampled by a $\BQP$ machine with an inverse polynomial accurate. %Namely, $A(x,1^{1/\eps})$ outputs a sample that is $\eps$-close to the distribution $D_x$ in statistical distance.
%, where given an input $x \in \zo^n$, the goal  
Our protocol leverages the Hamiltonian model and the computational X-Z measurement from~\cite{FOCS:Mahadev18a}.
However, a significant amount of new techniques have been developed to deal with the difference between $\SampBQP$ and $\BQP$, which will be highlighted in the technical contribution section. Precisely,
\begin{theorem}[informal]
Assuming the QLWE assumption, there exists a four-message CVQC protocol for all sampling problems in $\SampBQP$ with negligible completeness error and computational soundness.
\end{theorem}
%\XW{Insert the theorem statement for the first result here! and a pointer!}

Somewhat surprisingly, our second contribution is a simple yet powerful generic compiler that transforms any CVQC protocol to a blind one while preserving completeness and soundness errors.
Our construction builds upon another important primitive called the Quantum Fully Homomorphic Encryption (QFHE)~\cite{BJ15, DSS16, LC18, NS18, OTF18, mahadev_qfhe}.
Intuitively, QFHE allows fully homomorphic operations on encrypted quantum data and thus could be an ideal technical candidate for achieving blindness.
Indeed, in another paper~\cite{mahadev_qfhe}, Mahadev constructed the first leveled QFHE based on similar techniques and computational assumptions from~\cite{FOCS:Mahadev18a}.
The constructed QFHE automatically implies a blind CVQC protocol, however, without verifiability.
Extending this protocol with verifiability seems challenging as hinted by failed attempts in Section 2.2.2 of~\cite{mahadev_2018}.
In fact, most existing blind and verifiable CVQC protocols require a notable amount of effort in achieving each property respectively.

We observe that QFHE, especially the one from~\cite{mahadev_qfhe}, can be used to transform any CVQC protocol to a blind one with the same number of round communication, while preserving completeness and soundness error.
As a result, one can \emph{upgrade} every verifiable CVQC protocol with blindness almost for free with the help of QFHE.
Conceptually, we take a very different approach from previous results (e.g.,~\cite{FK17}) which use the blindness as the start point and then work to extend it with verifiability.
At a high level, our strategy is to simulate the original CVQC protocol under QFHE per each message.
To that end, we do require a special property of QFHE that the classical part of the ciphertext can be operated on separately from the quantum part, which is satisfied by the construction from~\cite{mahadev_qfhe}.
Our construction makes a modular use of QFHE and only requires a minor technicality in the analysis, which will be explained below. As a result, we obtain
\begin{theorem}[informal]
Assuming the QLWE assumption, there exists a protocol compiler that transforms any CVQC protocol $\Pi$ to a CVQC protocol $\Piblind$ that achieves blindness while preserves its round complexity, completeness, and soundness.
\end{theorem}

%\XW{theorem statement for the second contribution and pointer}



As a simple corollary of combining both results above, we achieve a constant-round blind CVQC protocol for $\SampBQP$. %with negligible completeness error and statistical soundness.  
\begin{theorem}[informal]
        Assuming the QLWE assumption, there exists a blind, four-message CVQC protocol for all sampling problems in $\SampBQP$ with negligible completeness error and computational soundness.
\end{theorem}

We can also the first blind and constant-round CVQC protocol for $\BQP$ by applying our compiler to the parallel repetition of Mahadev's protocol for $\BQP$ from \cite{arXiv:ChiaChungYam19, arXiv:AlaChiHun19}.


\begin{theorem}[informal]
    Assuming the QLWE assumption, there exists a blind, four-message CVQC protocol for all languages in $\BQP$ with negligible completeness and soundness errors.
\end{theorem}



%\XW{here for both $\BQP$ and $\SampBQP$}
%\XW{check the para/terminology here; consider adding a theorem statement,or a pointer to the later section}

To the authors' best knowledge, we are the first to study CVQC protocols for $\SampBQP$ and establish a generic compiler to upgrade CVQC protocols with blindness.
Our result also entails a \emph{constant-round} blind and verifiable CVQC protocol for $\BQP$.
The closest result to ours is by Gheorghiu and Vidick~\cite{FOCS:GheVid19} which shows such a CVQC protocol for $\BQP$, however, with a polynomial number of rounds.
Their protocol was obtained by first constructing a remote state preparation primitive and then combining it with an existing blind and verifiable protocol~\cite{FK17} where the verifier has some limited quantum power.
Our technical approach is quite different and seems incomparable.

% \XW{any more to say about parameters, techniques?}
% \XW{anything we want to say about composability?}
% \Ethan{Last time we checked, Vidick might have better composability since he's got some kind of ideal box and simulator-based proof}

% Related work
% \begin{itemize}
%     \item Comparison with related works here?
% \end{itemize}
%       \begin{itemize}
%           \item Sampling, the only paper; how about classical sampling?
%           \item the following for BQP  
%           \item blind and verifiable ~\cite{GV19}; we  constant round; technique-wise very different.
%           \item there is a table in~\cite{Grilo19}. Safe to say ~\cite{GV19} only existing blind protocol in the computational setting?
%           \item all previous either quantum clients, or at least 2 provers.
%           \item the following for blindness
%           \item Mahadev in her thesis~\cite{mahadev_2018} discussed a bit about the relation between verifiability and blindness. She hoped to get verifiability out of blindness by designing some non-malleable QFHE but failed.   
%           \item what's the high-level message we can say here?  Use QFHE in a different way? It is correct that not much work in the classical setting either.  Maybe existing classical work employs the principle but with different implementation.
%           \item old approach, first get blindness  (measurement-based, self-testing), and then try to make it verifiable; our approach, first have a verifiable protocol, and upgrade by a QFHE.
%           \item directly QFHE (blindness) won't give verifiability. some thoughts from Mahadev.
%
%           technical comparison with the past parallel % repetition.
%           \item

\vspace{2mm} \noindent \textbf{Techniques.} Following~\cite{FOCS:Mahadev18a}, we formally define $\QPIP_{\tau}$ as classes of CVQC protocols where $\tau$ refers to the size of quantum register in the possession of the classical verifier, or equivalently, the limited quantum computation power of the verifier.
It is known that $\BQP$ can be efficiently verified by a classical verifier that can perform a single qubit X or Z measurement~\cite{PhysRevA.93.022326, mf16}.
Namely, there is a $\QPIP_1$ protocol for $\BQP$.
The main contribution of Mahadev~\cite{FOCS:Mahadev18a} can hence be deemed as a way to compile this $\QPIP_1$ protocol into a $\QPIP_0$ protocol (i.e., with a fully classical verifier).

\vspace{2mm} \noindent \textbf{(1) Construction of a $\QPIP_0$ protocol for $\SampBQP$}.
We will follow the same road map above (i.e., from $\QPIP_1$ to $\QPIP_0$) for $\SampBQP$. However, since there is no existing $\QPIP_1$ protocol for $\SampBQP$, we make original contributions to both steps as follows:

\vspace{2mm} \noindent \emph{$\diamond \, \QPIP_1$ protocol for $\SampBQP$}: We will employ the local Hamiltonian technique~\cite{kitaev2002classical} and its ground state (known as the history state) as a key technical ingredient to certify the $\SampBQP$ circuits.
However, there are important differences between the cases for $\BQP$ and $\SampBQP$.
Recall that the original construction of local Hamiltonian $H$ for $\BQP$ (or $\QMA$) contains two parts $H=H_{\mathrm{circuit}}+ H_{\mathrm{out}}$.
Roughly speaking, $H_{\mathrm{circuit}}$ helps guarantee its ground space only contains \emph{valid} history states with correct input and circuit evolution, while $H_{\mathrm{out}}$'s energy encodes the 0/1 output for $\BQP$ circuits.
Thus, its outcome can be encoded by the \emph{ground energy} of $H$.
For $\SampBQP$, one still uses $H_{\mathrm{circuit}}$ to certify the validity of the history state.
However, in this case, one needs to measure on the entire final state of the circuit, rather than a single output qubit,
which can no longer be encoded solely by the ground energy.
Our approach is to have the valid history state lie in the ground space of a different local Hamiltonian $H'_{\mathrm{circuit}}$  that has a large \emph{spectral} gap between its ground energy and excited ones.
It is hence guaranteed that any state with close-to-ground energy must also be close to the history state.
In other words, a certification of the energy $H'_{\mathrm{circuit}}$ could lead to a certification of the history state.
We construct such $H'_{\mathrm{circuit}}$ from $H_{\mathrm{circuit}}$ by using the \emph{perturbation} technique (e.g.,~\cite{kempe_kitaev_regev_2006}) with further restriction to X/Z terms. (\Cref{sec:LHXZ}.)


Another high-level difficulty in constructing a $\QPIP_1$ protocol for $\SampBQP$ is due to the distinction between the test part and the output part.
Specifically, we will certify the energy of $H'_{\mathrm{circuit}}$ to guarantee the underlying state is close to the valid history state.
However, this procedure could be vastly different from outputting a sample by measuring the final state of $\SampBQP$ circuits.
We design a \emph{cut-and-choose} protocol on multiple copies of the history state for both testing and outputting.
We also employ a variant of quantum \emph{de Finetti} theorem~\cite{Brandão2017}
to prevent potential cheating strategies by entangling different copies of history states.
(\Cref{sec:qpip1}.)

\vspace{2mm} \noindent  \emph{$\diamond$ Compile $\QPIP_1$ into $\QPIP_0$}:
A naive attempt is to directly apply Mahadev's protocol on the aforementioned $\QPIP_1$ protocol.
Unfortunately, the plain version of Mahadev's protocol does not yield favorable parameters by itself.
In fact, there are some recent results~\cite{arXiv:AlaChiHun19, arXiv:ChiaChungYam19} that provide a parallel repetition of Mahadev's original protocol for $\BQP$ with
much more favorable parameters.

However, we cannot directly make use of these parallel repetition results due to the subtle difference between protocols for $\BQP$ and $\SampBQP$.
One of the major difficulties here is still to deal with both the test part and the output part in $\SampBQP$ protocols.
However, because we are now in the computational setting, there is no longer any available quantum de Finetti theorem that is usually derived in the information-theoretic setting.
We end up developing a weaker version of  parallel repetition of Mahadev's protocol inspired by the technique from~\cite{arXiv:ChiaChungYam19}.
Due to the nature of parallel repetition in the computational setting, our analysis is much less modular and significantly involved for this part.  
More intuitions and detailed analysis are given in \Cref{sec:qpip0_all}.

\vspace{2mm} \noindent \textbf{(2) A generic compiler to upgrade $\QPIP_0$ protocols with blindness}. At a high-level, the idea is simple: we run the original protocol under a QFHE with the verifier's key. Intuitively, this allows the prover to compute his next message under encryption without learning the underlying verifier's message, and hence achieves blindness while preserving the properties of the original protocol.
One subtlety with this approach is due to the fact that the verifier is classical while the QFHE cipher text could depend on both quantum and classical data.
In order to make the classical verifier work in this construction, the ciphertext and the encryption/decryption algorithm needs to be classical when the underlying message is classical, which is fortunately satisfied by~\cite{mahadev_qfhe}.

A more subtle issue is to preserve the soundness.
In particular, compiled protocols with only one-time use of QFHE might (1) leak information about the circuit being evaluated during the homomorphic evaluation of QFHE ciphertexts (i.e., no \emph{circuit privacy});
or (2) fail to simulate original protocols upon receiving invalid ciphertexts.
We address these issues by letting the verifier switch to a fresh new key for each round of the protocol.
Details are given in \Cref{sec:BlindBQP2}.

\vspace{2mm} \noindent \textbf{Open Questions.} Our main focus is on the feasibility of the desired functionality and properties, which nevertheless leaves a big room for the improvement of efficiency.
Some of our parameter dependence inherits from previous works (e.g.~\cite{FOCS:Mahadev18a}), whereas some is due to our own construction. 
It will be extremely interesting to improve the parameter dependence with potentially new techniques. 

\Ethan{TODO add organization of paper}

% \begin{itemize}
%     \item Open questions, and also explain for some associated high-cost. specifically
%     \item T dependence. In general, improve the efficiency.  
%     \item negligible soundness error.  compare with classical? what's the state-of-the art.
% \end{itemize}

% \Ethan{This section is currently all rough draft. We'll probably rewrite almost all of it.}

% \Ethan{application: verifiable private constant round delegation}

% \Ethan{Below is my attempt to talk about it in my SoP}

% It was proven that BQP=BQIP\hannote{who  when and cite}. That is, if a quantum computer can efficiently solve a given decision problem, then it can also efficiently convince a classical machine of its solution. I'm generalizing this to arbitrary efficient quantum computations. The proof for decision problems involves the classical verifier reducing the problem to a local Hamiltonian instance; the quantum prover would then commit its certificate and act as the verifier’s trusted measurement device as put forth in ``Classical Verification of Quantum Computations" by Mahadev. It isn't as trivial as it may seem. Repeating the scheme for each qubit loses the information carried by entanglements and throws off the joint distribution between qubits. Simply measuring the entire output register instead is difficult to analyze. For decision problems, it’s not hard to argue that a malicious prover cannot do better than sending identical copies of some pure state unentangled with each other. That same reasoning doesn't apply here a priori. I've been trying to get a grasp on the particular structure of the local Hamiltonian reduction in order to better analyze it.

% \Ethan{Below is my attempt to talk about it in my research proposal}

% We are interested in delegating quantum computations from a classical client to an untrusted quantum server. Under this setting, the client would send the server a quantum circuit and an initial state. Then, through interaction with the honest server, the client obtains a measurement result as if he measured the true output of the circuit. If the server attempts to deceive the client, the client should reject it. The case where the circuit encodes a decision problem has been well-studied, and we're now trying to generalize those results to circuits with possibly many bits of output.

% If the circuit encodes a decision problem, then by considering adiabatic quantum computation, there exists a reduction to local Hamiltonian. Local Hamiltonian is QMA-complete, so there's a certificate for every yes-instance, and no valid certificates for any no-instances. An introduction can be found in Kitaev, Shen, and Vyalyi's "Classical and Quantum Computation". Furthermore, Biamonte and Love's "Realizable Hamiltonians for Universal Adiabatic Quantum Computers" states that these local Hamiltonians have very simple forms, which in turn implies that in order to check such certificates one only requires abilities to receive qubits and perform X/Z measurements. Based on this observation, Mahadev constructed a protocol in "Classical Verification of Quantum Computations" which, under the LWE assumption (a widely believed conjecture in quantum cryptography), allows the prover to commit qubits and act as the verfier's trusted X/Z measurement device. This solves the delegation of decision problem from a fully classical client to a quantum server.

% To generalize delegation of quantum computations to allow long output, simply repeating the known protocol for every output qubit doesn't work. The joint probability distribution between qubits would be incorrect due to entanglements. In fact, generally the output qubits encode sampling problems rather than decision problems. Furthermore, for decision problems one can argue that the prover's optimal strategy is to send identical copies of a certificate, so Chernoff bound can be applied, but said argument doesn't generalize to sampling problems either. To overcome these challenges, we start by modifying the local Hamiltonian construction so it is compatible with long output. We then analyze our protocol's soundness more carefully, before using Mahadev's result as a black box to solve the long output case too for fully classical clients.

% A possible application for our long output protocol is to make the computation not only verifiable, but also private in the sense of homomorphic encryptions. That is, the input is encrypted before being sent to the server. The server computes on the encrypted input, obtaining an encrypted output. The client then receives and decrypts the output. Here we can combine results from Mahadev's "Classical Homomorphic Encryption for Quantum Circuits" with our long output scheme. The client can simply send the homomorphic evaluation circuit to the server with the encrypted input.




\section{Preliminaries}

\subsection{Notations}

Let $\mathcal{B}$ be the Hilbert space corresponding to one qubit. Let $H:\mathcal{B}^{\otimes n}\rightarrow\mathcal{B}^{\otimes n}$ be Hermitian matrices. We use $H\geq0$ to denote $H$ being positive semidefinite. Let $\lambda(H)$ be the smallest eigenvalue of $H$. The ground states of $H$ are the eigenvectors corresponding to $\lambda(H)$. For matrix $H$ and subspace $S$, let $H\big|_S=\Pi_S H \Pi_S$, where $\Pi_S$ is the projector onto the subspace $S$. For a $T$-qubit Hilbert space, let the state $\ket{\widehat{t}}=\ket{1}^{\otimes t}\otimes \ket{0}^{{\otimes (T-t)}}$.
We write $F(\rho_1, \rho_2)=\left(\tr\sqrt{\sqrt{\rho_1}\rho_2\sqrt{\rho_1}}\right)^2$ for the fidelity between $\rho_1$ and $\rho_2$.
We write $\frac{1}{2}\norm{\rho_1-\rho_2}_1$ for the trace distance between $\rho_1$ and $\rho_2$. For all $n$-qubit states $\rho_1, \rho_2\in\cB^{\otimes n}$ we have $\frac{1}{2}\norm{\rho_1-\rho_2}_1\leq\sqrt{1-F(\rho_1, \rho_2)}$.

\begin{definition} [quantum-classical channels]
	\label{def:QCChannel}
	A quantum measurement is given by a set of matrices $\set{M_k}$ such that $M_k\geq0$ and $\sum_k M_k=\id$.
	We associate to any measurement a map $\Lambda(\rho)=\sum_k \tr(M_k\rho)\ket{k}\bra{k}$
	with $\set{\ket{k}}$ an orthonormal basis.
	This map is also called a \emph{quantum-classical channel}.
\end{definition}

The phase gate and Pauli matrices are denoted as follows.

\begin{definition}
	$P(i)=\begin{pmatrix}1&0\\0&i\end{pmatrix}$, $X=\begin{pmatrix}0&1\\1&0\end{pmatrix}$,
	$Y=\begin{pmatrix}0&-i\\i&0\end{pmatrix}$,
	$Z=\begin{pmatrix}1&0\\0&-1\end{pmatrix}$
\end{definition}

\subsection{Relevant complexity classes}

We define a few relevant complexity classes.

\begin{definition} [$\BQP$]
	Definition from Kitaev:
	A \emph{quantum algorithm} for the computation of a function $F:\zo^*\rightarrow\zo^*$ is a classical algorithm (i.e., a Turing machine) that computes a function of the form $x\mapstochar\rightarrow Z(x)$, where $Z(x)$ is a description of a quantum circuit which computes $F(x)$ on empty input. The function $F$ is said to belong to class $\BQP$ if there is a quantum algorithm that computes $F$ in time $\poly(n)$.

	Definition from Complexity Zoo:
	$\BQP$ is the class of languages $L$ for which for all $n\in\bbN$ there exists a quantum circuit constructible in time $\poly(n)$ that, given any $x\in\set{0, 1}^n$ as input, correctly decides whether $x\in L$ at least $\frac{2}{3}$ of the time.
	\Ethan{Just copy this from somewhere... Does anyone even define this?}
\end{definition}

\begin{definition} [$\FBQP$]
	A function $f:\set{0,1}^*\rightarrow\set{0,1}^*$ is in $\FBQP$ if there is a $\BQP$ machine that, $\forall x$, outputs $f(x)$ with overwhelming probability.
	\Ethan{Need to be more formal. Also should be efficient verifiable}
\end{definition}

We define search and sampling versions of $\BQP$ based on \cite{aaronson_2013}.

\begin{definition} [search problem]
	A search problem $R$ is a collection of nonempty sets $(A_x)_{x\in\set{0, 1}^*}$, one for each input string $x\in\set{0, 1}^*$, where $A_x$... \Ethan{Great, interface doesn't line up correctly}
\end{definition}

\begin{definition} [sampling problem]
	A sampling problem $S$ is a collection of probability distributions $(D_x)_{x\in\set{0, 1}^*}$, one for each input string $x\in\set{0,1}^n$, where $D_x$ is a distribution over $\set{0,1}^{p(n)}$ for some fixed polynomial $p$.
\end{definition}

\begin{definition} [$\SampBQP$]
	$\SampBQP$ is the class of sampling problems $S=\left(D_x\right)_{x\in\set{0, 1}^*}$ for which there exists a polynomial-time quantum algorithm $B$ that, given $(x, 0^{1/\varepsilon})$ as input, samples from a probability distribution $C_x$ such that $\norm{C_x-D_x}\leq\varepsilon$.
\end{definition}

\subsection{Quantum Prover Interactive Protocol (QPIP)}
We classify the interaction between a (almost classical) client and a quantum server for sampling problems, extending the classification by \cite{FOCS:Mahadev18a}.

\begin{definition}
	We say $\Pi=(P, V)(x)$ is a protocol for the sampling problem $(D_x)_{x\in\zo^*}$ with completeness error $c$ and soundness error $s$ \Ethan{Might need these to be functions of $\abs{x}$} if
	\Ethan{Look up ``interactive protocols for BQP". Right now it's missing quantifier for all x. Also need to mention d is decision bit; maybe do that in next definition and swap locations}
	\begin{itemize}
		\item Let $(d, z)\leftarrow(P, V)(x)$. Then $d=rej$ with probability at most $c$.
		\item For all cheating prover $P^*$, let $(d, z)\leftarrow(P^*, V)(x)$. Let \Ethan{Use display math to make it obvious I'm defining this} $z_{ideal}\leftarrow D_x$ if $d=acc$, else $z_{ideal}=\bot$. Then $\norm{(d, z) - (d, z_{ideal})} \leq s$ \Ethan{make stat. distance notation consistent}.
	\end{itemize}
\end{definition}

\Ethan{Acc, rej, P, V fonts}

\Ethan{Might need to write our own def. here}

\Ethan{Look at thesis for this}

\begin{definition}
	A sampling problem $S=(D_x)_{x\in\set{0, 1}^*}$ is said to be \Ethan{Try to make this more general; remove mentions of sampling/decision problems} in $\QPIP_\tau$ with completeness $c$ and soundness $s$ \Ethan{Don't tie this with completeness and soundness yet} if there exists a protocol $(\bbP, \bbV)(x)$ for $S$ with the following properties:
	\begin{itemize}
		\item $\bbP$ is run by the prover, a $\BQP$ machine, which also has access to a quantum channel that can transmit $\tau$ qubits to the verifier per use.
		\item $\bbV$ is run by the verifier, which is a hybrid machine of a classical part and a limited quantum part. The classical part is a $\BPP$ machine. The quantum part is a register of $\tau$ qubits, on which the verifier can perform arbitrary quantum operations and which has access to a quantum channel which can transmit $\tau$ qubits. At any given time, the verifier is not allowed to possess more than $\tau$ qubits. The interaction between the quantum and classical parts of the verifier is the usual one: the classical part controls which operations are to be performed on the quantum register, and outcomes of measurements of the quantum register can be used as input to the classical part.
		\item There is also a classical communication channel between the prover and the verifier, which can transmit $\poly(\abs{x})$ many bits to either direction. 
	\end{itemize}
\end{definition}

\Ethan{Two separate definitions for comp and soundness}

\Ethan{Soundness?}

\subsection{Semantic security for interactive protocols}
\Ethan{Just call this blindness and put this under interactive protocols}

\Ethan{See Thomas' paper if he defined this}

We present the security definition for interactive protocols:

\begin{definition}
	Let $\lambda$ be a security parameter.
	Let $(\bbP, \bbV)$ be an interactive protocol with security parameter $\lambda$.
	Then it is IND-CPA secure if $\forall x\in\set{0,1}^n$ no polynomial time adversary $\cA$ can win \protoref{indcpa} with probability better than $\frac{1}{2}+\negl(\lambda)$
\end{definition}

\begin{protocol}{Attack against semantic security}
	\label{proto:indcpa}
	\begin{enumerate}
		\item The challenge picks $b\in\set{0,1}$ at random
		\item If $b=0$, the challenger runs the protocol with the adversary, acting as the verifier with input $0^n$
		\item Otherwise, the challenger runs the protocol with the adversary, acting as the verifier with input $x$
		\item $\cA$ attempts to guess $b$
	\end{enumerate}
\end{protocol}

\subsection{Chernoff bound}

Taken from \href{http://math.mit.edu/~goemans/18310S15/chernoff-notes.pdf}{here}.

\begin{thm}
\label{thm:Chernoff}
Let $X=\sum_{i=1}^n X_i$ where $X_i$ are i.i.d. Bernoulli trials, and $\mu=\E[X]$.
Then for all $0<\delta<1$,
$$P[\abs{X-\mu}\geq\delta\mu]\leq2e^{-\frac{\mu\delta^2}{3}}$$
\end{thm}

\subsection{Projection Lemma}

We use the projection lemma from \cite{kempe_kitaev_regev_2006}, which describes the conditions under which we can estimate the ground state energy of $H_1 + H_2$ with that of $H_1\big|_{\ker H_2}$.

\begin{thm}
	Let $H=H_1+H_2$ be the sum of two Hamiltonians operating on some Hilbert space $\cH=\cS+\cS^\bot$.
	The Hamiltonian $H_2$ is such that $\cS$ is a zero eigenspace and the eigenvectors in $\cS^\bot$ have eigenvalues at least $J>2\norm{H_1}$. Then,
	$$\lambda\left(H_1\big|_\cS\right)-\frac{\norm{H_1}^2}{J-2\norm{H_1}^2}\leq\lambda(H)\leq\lambda\left(H_1\big|_\cS\right)$$
\end{thm}

We will instead use the following formulation, which can be obtained by relabeling variables from above.

\begin{thm}
	\label{thm:projection}
	Let $H_1, H_2$ be local Hamiltonians where $H_2\geq0$. Let $K=\ker H_2$ and
	$$J=\frac{10\norm{H_1}^2}{\lambda\left(H_2\big|_{K^\bot}\right)}$$
	then we have
	$$\lambda(H_1+JH_2)\geq\lambda\left(H_1\big|_K\right)-\frac{1}{8}$$
\end{thm}

\subsection{Quantum de Finetti Theorem under Local Measurements}

De Finetti theorem provides a way to obtain close to independent samples by taking random subsystems of a quantum system.
There are many formulations; we use the one from \cite{Brandão2017} because we need to avoid exponential dependence on number of qubits in each subsystem.
\begin{thm}
	\label{deFinetti}
	Let $\rho^{A_1\ldots A_k}$ be a permutation-invariant state on registers $A_1,\ldots,A_k$ where each register is $s$ qubits,
	then for every $0\leq l\leq k$ there exists states $\set{\rho_i}$ and $\set{p_i}\subset\bbR$ such that
	$$\max_{\Lambda_1,\ldots,\Lambda_l}
	\norm{(\Lambda_1\otimes\ldots\otimes\Lambda_l)\left(\rho^{A_1\ldots A_l}-\sum_ip_i\rho_i^{A_1}\otimes\ldots\otimes\rho_i^{A_l}\right)}_1
	\leq\sqrt{\frac{2l^2s}{k-l}}$$
	where $\Lambda_i$ are quantum-classical channels.
\end{thm}

\subsection{Quantum Homomorphic Encryption Schemes}

\def\QHE{\mathsf{QHE}}
\def\QGen{\mathsf{QHE.Keygen}}
\def\QEnc{\mathsf{QHE.Enc}}
\def\QEval{\mathsf{QHE.Eval}}
\def\QDec{\mathsf{QHE.Dec}}

We use the quantum fully homomorphic encryption scheme given in \cite{mahadev_qfhe} which is compatible with our use of a classical client. We start by presenting the interface of a homomorphic encryption scheme:
\begin{definition}
	A leveled homomophic encryption scheme is tuple of algorithms \linebreak $\mathsf{HE}=(\mathsf{HE.Keygen}, \mathsf{HE.Enc}, \mathsf{HE.Dec}, \mathsf{HE.Eval})$ with the following descriptions:
	\begin{itemize}
		\item $\mathsf{HE.Keygen}(1^\lambda, 1^L)\rightarrow(pk, sk)$
		\item $\mathsf{HE.Enc}_{pk}(\mu)\rightarrow c$
		\item $\mathsf{HE.Dec}_{sk}(c)\rightarrow \mu^*$
		\item $\mathsf{HE.Eval}_{pk}(f, c_1, \ldots, c_l)\rightarrow c_f$
	\end{itemize}
\end{definition}

$\mathsf{HE}$ also satisfies, with overwhelming probability in $\lambda$, that
$$\mathsf{HE.Dec}_{sk}(\mathsf{HE.Eval}_{pk}(f, c_1, \ldots, c_l)=f(\mathsf{HE.Dec}_{sk}(c_0),\ldots,\mathsf{HE.Dec}_{sk}(c_l))$$
where $f$ is specified by a circuit of depth at most $L$.

\Ethan{To be pedantic, the above doesn't imply Dec undoes Enc even if we sub in $f=\id$.}

We also recall the security definition for a FHE scheme.

\begin{definition}
	A FHE scheme $\mathsf{HE}$ is IND-CPA secure if, for any polynomial time adversary $\cA$, there exists a negligible function $\mu(\cdot)$ such that
	$$\abs{Pr[\cA(pk, \mathsf{HE.Enc}_{pk}(0))=1]-Pr[\cA(pk, \mathsf{HE.Enc}_{pk}(1))=1]}=\mu(\lambda)$$
	where $(pk, sk)\leftarrow\mathsf{QHE.Keygen}(1^\lambda)$
\end{definition}

The quantum homomorphic encryption scheme $\mathsf{QHE}$ from \cite{mahadev_qfhe} has additional properties that facilitates the use of classical clients:
\begin{itemize}
	\item $\QGen$ can be done classically.
	\item In the case where the plaintext is classical, $\QEnc$ can be done classically.
	\item Its ciphertext takes the form $(X^xZ^z\rho Z^zX^x, c_{x, z})$, where $\rho$ is the plaintext and $c_{x, z}$ is a ciphertext that decodes to $(x, z)$ under a certain classical homomorphic encryption scheme.
\end{itemize}


\section{Quantum Delegation Schemes}

Here we describe schemes that allows a classical client to interact with a quantum server, and various properties that such schemes can have.

\begin{definition}
	A \emph{quantum delegation protocol} is an polynomial-round interactive protocol between a BPP verifier and a BQP prover. The BPP verifier has an arbitrary quantum circuit with classical input that it needs to evaluate and measure the result of by interacting with the prover.
\end{definition}

\begin{definition}
	A quantum delegation protocol has \emph{long output} if the output register contains multiple qubits.
\end{definition}

\begin{definition}
	A quantum delegation protocol is \emph{CPA-secure} if both the input and output of the circuit are CPA-secure as ciphertexts. In this case, the scheme should be seen as a homomorphic encryption scheme.
\end{definition}

\begin{definition}
	A quantum delegation protocol is \emph{$\delta(\epsilon)$-verifiable} if the verifier's output distribution is within $\epsilon$ of the true distribution whenever it interacts with a prover whose probability to be accepted is greater than $\delta$.
\end{definition}

In \cite{mahadev_delegation}, a verifiable scheme is proposed. In \cite{mahadev_qfhe}, a secure scheme is proposed. (TODO just make this CPA or whatever it actually is...) 

TODO check where does \cite{1904.06320} fit in.

\subsection{Our contributions}

We propose a verifiable scheme with long output which can also be made secure. We start by constructing a verifiable long output scheme. Then, we use \cite{mahadev_delegation} to achieve security. That is, we the techniques in \cite{mahadev_delegation} to encrypt the circuit and the input. We also tack on the evaluation key as extra input, so the entire computation can still be encoded as a single circuit.


\section{Our Long Output Scheme}

Here we lay our theoretical fundation by describing a reduction from $\SampBQP$ that we'll use throughout this paper.

First, for ease of notation, write

\begin{definition}
	Let $\mathcal{G}_{XZ}=\set{U_0\otimes U_1\otimes\ldots\otimes U_n: U_i\in\set{I,X,Z}}$.
\end{definition}

We define and construct an instance of Local Hamiltonian that encodes the history of the computation.

\begin{definition}
	Let $C(x)$ be a quantum circuit. Then let $H_{C(x)}$ be with the following properties: TODO
	\begin{itemize}
		\item $H_{C(x)}$ can be written as $\sum_{S\in\mathcal{G}_{XZ}} d_S S$, where each $S$ has at most constantly many non-identity components.
		\item 0 eigenvalue ground state and has certain form TODO type it out
		\item at least whatever eigenvalue in all other dimensions
		\item has ground state of a particular form we like
	\end{itemize}
\end{definition}

TODO Theorem statement for XZ LH and whatnot

\subsection{Preprocessing circuit}

We pick our universal gate set to be Hadamard gate and controlled phase gate following \cite{quant-ph/0301040}:
$$H=\frac{1}{\sqrt{2}}\begin{pmatrix}1&1\\1&-1\end{pmatrix}$$
	$$\Lambda(P(i))=\begin{pmatrix}1&0&0&0\\0&1&0&0\\0&0&1&0\\0&0&0&i\end{pmatrix}$$
		Which is proven to be universal in \cite{kitaev_1997}.

In order to estimate the energy of a Hamiltonian using only $X$ and $Z$ measurements, the gates in the circle must be combinations of $X$ and $Z$ matrices. To begin, we apply a transformation to the circuit to use only real states and gates. This transform is taken from \cite{quant-ph/0301040}.

\begin{theorem}
	Let $C$ be a circuit that:
	\begin{itemize}
		\item consists of $T$ gates, each either $H$ or $\Lambda(P(i))$.
		\item uses $n$ qubits
	\end{itemize}
	Then a classical machine given $C$ can compute a circuit $C'$ that:
	\begin{itemize}
		\item consists of at most $4T$ gates, each either $H$ or $\Lambda^2(X)$.
		\item uses $n+1$ qubits
		\item Let $x\in\set{0,1}^n$. Let $x||0\in\set{0,1}^{n+1}$ be $x$ concatenated with $0$. The $X$ and $Z$ measurement results of $C(x)$ and the first $n$ qubits of $C'(x||0)$ are identically distributed.
	\end{itemize}
\end{theorem}

\begin{proof}

	Consider the transform on quantum states
	$$\mathcal{F}(\ket{\phi})=(\Re\ket{\phi})\otimes\ket{0}+(\Im\ket{\phi})\otimes\ket{1}$$
	where $\Re$ and $\Im$ denote real and imaginary parts respectively.

	This transform commutes with Hadamard gates on the respective qubit. On the other hand, exchanging $\mathcal{F}$ with a controlled phase gate turn it into a combination of Hadamard and Toffoli gates. Mathematically,
	$$\mathcal{F}\circ H_s=H_s\circ\mathcal{F}$$
	$$\mathcal{F}\circ\Lambda_f(P(i)_s)=\Lambda^2_{f,s}(X_0Z_0)\circ\mathcal{F}=\Lambda^2_{f,s}(X_0)H_0\Lambda^2_{f,s}(X_0)H_0\circ\mathcal{F}$$

	We construct $C'$ so that $\mathcal{F}\circ C=C'\circ\mathcal{F}$ by following the computation above. It satisfies the required properties by construction.
	\begin{itemize}
		\item $C'$ uses only Hadamard and Toffoli gates.
		\item Exchanging $\mathcal{F}$ with $H$ doesn't change the circuit size. Exchanging $\mathcal{F}$ with controlled phase gate turns it into $4$ gates. So the final result is at most $4T$ gates.
		\item When $x$ is classical, $\mathcal{F}(x)=x||0$. So $F\circ C(x)=C'(x||0)$. It is simple to verify that $F$ preserves $X$ and $Z$ measurement results on the first $n$ qubits.
	\end{itemize}
\end{proof}

We verify that our new gate set can be written as combinations of $X$ and $Z$ matrices.

\begin{theorem}
	$H,\Lambda^2(X)\in\spn\mathcal{G}_{XZ}$ with $O(1)$ nonzero components.
\end{theorem}

\begin{proof}
	$$H=\frac{1}{\sqrt{2}}(X+Z)$$
	$$\Lambda^2_{1,2}(X_3)=\ket{11}\bra{11}\otimes X+(I-\ket{11}\bra{11})\otimes I$$
	$$\ket{11}\bra{11}=\frac{1}{4}(I\otimes I+Z\otimes Z-I\otimes Z-Z\otimes I)$$
\end{proof}

\subsection{Constructing a local Hamiltonian instance}

Here we reduce a circuit into a local Hamiltonian instance with a few additional properties.

\begin{theorem}
	Let $C=U_T\ldots U_1$ where $U_i$ are either $H$ or $\Lambda^2(X)$, and let $x\in\set{0, 1}^*$. Then a classical machine can construct in polynomial time a local Hamiltonian instance $H_{circuit}$ such that:
	\begin{itemize}
		\item its ground state is $$\ket{\psi_{circuit}}=\sum_{t=0}^TU_t...U_1\ket{x}\otimes\ket{\hat{t}}$$
		\item $\braket{\psi_{circuit}|H_{circuit}|\psi_{circuit}}=0$
		\item $\forall\ket\phi,\braket{\phi|\psi_{circuit}}=0\Rightarrow\braket{\phi|H_{circuit}|\phi}\geq\frac{3}{4}$
		\item $H_{circuit}\in\spn\mathcal{G}_{XZ}$, with $O(T)$ nonzero components whose coefficients are each at most $O(T^3)$.
	\end{itemize}
\end{theorem}

\begin{proof}
	We ensure states perpendicular to $\ket{\psi_{circuit}}$ have high eigenvalues. The base construction comes from \cite{kitaev2002classical}. The simplification to $\spn\mathcal{G}_{XZ}$ is taken from \cite{PhysRevA.78.012352}.

	Let $x_i$ denote the $i$-th bit of $x$, and let $n$ be the number of qubits in $C$.

	First, we ensure that the invalid clock states have high eigenvalues by applying the following Hamiltonian to the time register.
	$$H_{clock}=\sum_{t=1}^{T-1}\ket{01}\bra{01}_{t,t+1}$$
	As a sum of projections, clearly $H_{clock}\geq 0$. We shall also show that $H_{clock}\in\spn\mathcal{G}_{XZ}$.
	$$H_{clock}=\frac{1}{4}(Z_1 - Z_T) + \frac{1}{4}\sum_{t=1}^{T-1}(I-Z_tZ_{t+1}) $$
	This can be checked by fixing the first and last qubit, then doing induction on the number of switches.

	Then, we ensure that the initial condition is set up correctly.
	$$H_{in}=\sum_{i=1}^n(I-\ket{x_i}\bra{x_i})\otimes\ket{0}\bra{0}_1$$
	The kernel of this is precisely where everything is set up consistently with $\phi$ in time step $t=0$. Furthermore, $H_{in}\in\spn\mathcal{G}_{XZ}$
	$$H_{in}=\frac{1}{4}\sum_{i=1}^n(I-(-1)^{x_i}Z_i)\otimes(I+Z_1)$$

	Then, we ensure that the gates are applied correctly.
	$$H_{prop}=\sum_{t\in T_1}H_{prop,t}$$
	$$H_{prop,t}=I\otimes\ket{\widehat{t}}\bra{\widehat{t}}
	+I\otimes\ket{\widehat{t-1}}\bra{\widehat{t-1}}
	-U_t\otimes\ket{\widehat{t}}\bra{\widehat{t-1}}
	-U_t^\dagger\otimes\ket{\widehat{t-1}}\bra{\widehat{t}}$$

	Next, we check that $H_{prop}\geq0$ by the following transform.

	$$W=\sum_{j=0}^L U_j\ldots U_1\otimes\ket{j}\bra{j}$$
	$$W^\dagger\phi=\sum_{t=0}^T\ket{x}\otimes\ket{\hat{t}}$$
	$$W^\dagger H_{prop} W=
	\begin{pmatrix}
		\frac{1}{2} & -\frac{1}{2} & & & &  \\
		-\frac{1}{2} & 1 & -\frac{1}{2} & & & \\
		& -\frac{1}{2} & 1 & \ddots & & \\
		& & \ddots & \ddots & -\frac{1}{2} & \\
		& & & -\frac{1}{2} & 1 & -\frac{1}{2} \\
		& & & & -\frac{1}{2} & \frac{1}{2}
	\end{pmatrix}$$

	According to \cite{2002quant.ph.10077A}, $H_{prop}\geq 0$ due to the above form being relevant to random walks. Furthermore, the last nonzero eigenvalue of $H_{prop}$ is at least $\frac{1}{2(T+1)^2}$.

	We can also write $H=\spn\mathcal{G}_{XZ}$. Note that $U^\dagger=U$, since the gates are either Hadamard or Toffoli. Additionally, $\frac{1}{2}(I-Z_{t-1})$ annihilates time steps before $t-1$. $\frac{1}{2}(I+Z_{t+1})$ similarly annihilates steps $t+1$ and after.
	$$H_{prop,t}=\frac{I}{4}\otimes(I-Z_{t-1})(I+Z_{t+1})-\frac{U}{4}\otimes(I-Z_{t-1})X_t(I+Z_{t+1})$$
	Extra care must be taken for boundary cases.
	$$H_{prop,1}=\frac{1}{2}(I+Z_2)-U_1\otimes\frac{1}{2}(X_1+X_1Z_2)$$
	$$H_{prop,T}=\frac{1}{2}(I-Z_{t-1})-U_T\otimes\frac{1}{2}(X_T-Z_{T-1}X_T)$$

	To properly combine the three Hamiltonians defined, we consider the kernels of the Hamiltonians we defined.
	$$K_{clock}=\ker H_{clock}$$
	$$K_{in}=\ker H_{in}$$
	$$K_{prop}=\ker H_{prop}$$
	It is clear that $K_{clock}$ consists of the states with time registers of the form $\ket{\hat{t}}$.

	\begin{lemma}
		$$K_{clock}\cap K_{prop}=\set{\sum_{t=0}^TU_t\ldots U_1\ket{y}\otimes\ket{\hat{t}}: \ket{y}\in\mathcal{B}^{\otimes n}}$$
	\end{lemma}

	\begin{proof}
		Let $\ket\phi\in K_{clock}\cap K_{prop}$. As $\ket\phi\in K_{clock}$, the time register components are always valid unary representations. Now consider some $\widetilde{t}$ such that $\braket{\widetilde{t}|\phi}\ne0$. As $\ket\phi\in K_{prop}$, we can then say that $\bra{\widetilde{t+1}}\ket\phi=\bra{\widetilde{t}}U_{t+1}\ket\phi$ for $t\ne T$, and $\bra{\widetilde{t}}\ket\phi=U_t\bra{\widetilde{t-1}}\ket\phi$ for $t\ne0$. So we can obtain the required statement by induction.
	\end{proof}

	\begin{corollary}
		$K_{clock}\cap K_{in}\cap K_{prop}=\spn\set{\ket{\psi_{circuit}}}$
	\end{corollary}

We now consider the space outside the desired state.
$$S=(\spn\set{\ket{\psi_{circuit}}})^\bot$$
$$H_{clock}\big|_S,H_{in}\big|_S,H_{prop}\big|_S$$
To combine the Hamiltonians, we apply the projection lemma twice.
$$\exists J_{clock}
=\frac{\poly\left(\norm{H_{in}\big|_S}\right)}{\lambda\left(H_{clock}\big|_{S\cap K^\bot_{clock}}\right)}
=O(n)=O(T)$$
$$\lambda(H_{in}\big|_S+J_{clock}H_{clock}\big|_S)\geq
\lambda(H_{in}\big|_{S\cap K_{clock}})-\frac{1}{8}$$
$$\exists J_{prop}=\frac{\poly\left(\norm{H_{in}\big|_S+J_{clock}H_{clock}\big|_S}\right)}{\lambda\left(H_{prop}\big|_{S\cap K^\bot_{prop}}\right)}
=\frac{O(n+T)}{\Omega(T^{-2})}=O(T^3)$$
$$\lambda(H_{in}\big|_S+J_{clock}H_{clock}\big|_S+J_{prop}H_{prop}\big|_S)\geq
\lambda(H_{in}\big|_{S\cap K_{clock}\cap K_{prop}})-\frac{1}{4}$$
\begin{align*}
	S\cap K_{clock}\cap K_{prop}&=S\cap\set{\sum_{t=0}^TU_t\ldots U_1\ket{y}\otimes\ket{\hat{t}}|\ket{y}\in\mathcal{B}^{\otimes n}}\\
	&=\set{\sum_{t=0}^TU_t\ldots U_1\ket{y}\otimes\ket{\hat{t}}:\braket{\psi_{circuit}|y}=0}
\end{align*}
$$\Rightarrow\lambda((H_{in}+J_{clock}H_{clock}+J_{prop}H_{prop})\big|_S)\geq\frac{3}{4}$$
So we set $H_{circuit}=H_{in}+J_{clock}H_{clock}+J_{prop}H_{prop}$, which satisfies the required properties by construction.
\end{proof}

\subsection{Measuring ground state}

TODO Combine and rewrite the following subsections

Now we put the above results in term of measurement outputs.

\begin{algorithm}
	\caption{Check for ground state}
	\label{AlgGroundStateCheck}
	\begin{algorithmic}[1]
		\Require $H=\sum_{S\in\mathcal{G}_{XZ}} d_S S$
		\Procedure{GroundStateCheck}{$\ket\phi$}
		\State $D\gets\sum_{S\in\mathcal{G}_{XZ}}|d_S|$
		\State $p_S\gets\frac{|d_S|}{D}$
		\State $\widetilde{S}\gets$ Pick weighted using $p_S$
		\State $\lambda_{\widetilde{S}}\gets$ Measure $\ket\phi$ in the $\widetilde{S}$ basis
		\If{$\sgn(d_{\widetilde{S}})\lambda_{\widetilde{S}}=-1$}
		\State Accept
		\Else
		\State Reject
		\EndIf
		\EndProcedure
	\end{algorithmic}
\end{algorithm}

\begin{theorem}\label{ThmXZCheck}
	Let $H$ have $O(\poly(T))$ nonzero components whose coefficients are each at most $O(\poly(T))$.
	Then \autoref{AlgGroundStateCheck} accepts $\ket\phi$ with probability $\frac{1}{2}-\Omega(\frac{1}{\poly(T)})\braket{\phi|H|\phi}$.
\end{theorem}
\begin{proof}

	Here we follow \cite{PhysRevA.93.022326}.
	\begin{align*}
		\frac{1}{D}\braket{\phi|H|\phi}&=\sum_{S\in\mathcal{G}_{XZ}} p_S\sgn(d_S)\braket{\phi|S|\phi}\\
		&=\sum_{S\in\mathcal{G}_{XZ}} p_S\sgn(d_S)\E[\lambda_S]\\
		&=\E_{\widetilde{S}}[\sgn(d_{\widetilde{S}})\E[\lambda_{\widetilde{S}}]]\\
		&=\E_{\widetilde{S}}[\sgn(d_{\widetilde{S}})\lambda_{\widetilde{S}}]
	\end{align*}

	Note that $\sgn(d_{\widetilde{S}})\lambda_{\widetilde{S}}=\pm1$. Let $p$ be the probability that $\sgn(d_{\widetilde{S}})\lambda_{\widetilde{S}}=-1$.
	$$\Rightarrow \frac{1}{D}\braket{\phi|H|\phi}=\E_{\widetilde{S}}[\sgn(d_{\widetilde{S}})\lambda_{\widetilde{S}}]=-p+(1-p)$$
	\begin{align*}
		\Rightarrow p&=\frac{1}{2}-\frac{1}{2D}\braket{\phi|H|\phi}\\
		&=\frac{1}{2}-\Omega\left(\frac{1}{\poly(T)}\right)\braket{\phi|H|\phi}
	\end{align*}

\end{proof}

\subsection{Analysis and amplification}

In the case for decision problems, amplifying the gap between acceptance and rejection probabilities is trivial. As seen in \cite{kitaev2002classical}, a prover, honest or not, cannot do better than sending $n$ identical copies of some state. As a result, a simple Chernoff bound would suffice. The same reasoning does not generalize to our case, so instead we give the following proof. We start by considering simpler inputs before moving on to the general case.

Suppose $\mathcal{F}$ accepts $\ket{\psi}$ with probability $\frac{1}{2}$, and any states perpendicular to it with probability $\frac{1}{2}-\Delta$.

Let $\varepsilon>0$, $p>0$. Pick $n\in\mathbb{N}$ so that by Chernoff bound:
\begin{itemize}
	\item $\ket{\psi^{\otimes n}}$ is rejected with probability at most $p$.
	\item If $\mathcal{F}(\ket{\phi})$ has less than $\frac{1}{2}-\varepsilon$ accept probability, then $\ket{\phi^{\otimes n}}$ is accepted with probability at most $p$.
\end{itemize}

\begin{algorithm}
	\caption{Amplification with simple input}
	\label{AlgAmp1}
	\begin{algorithmic}[1]
		\Require $\phi=\sum_i w_i\ket{\rho_i^{\otimes n+m}}\bra{\rho_i^{\otimes n+m}}$
		\Procedure{Amplification}{$\phi$}
		\State Apply $\mathcal{F}$ to first $n$ registers of $\phi$
		\If{number of accepted copies $\geq (\frac{1}{2}-\frac{\varepsilon}{2})n$}
			\State Output the last $m$ registers
		\Else
			\State Reject
		\EndIf
		\EndProcedure
	\end{algorithmic}
\end{algorithm}

\begin{observation}
	\autoref{AlgAmp1} applies a Chernoff bound to the first $n$ registers
\end{observation}

\begin{theorem}
	When \autoref{AlgAmp1} accepts an input with probability greater than $\delta$, conditioned on this acceptance, $\mathcal{F}$ has probability at least $1-\frac{p}{\delta}$ to accept each register of the output.
\end{theorem}
\begin{proof}
	Note that the input is a classical probability distribution over inputs $\ket{\rho_i^{\otimes n+m}}$ with $w_i$ as weights.

	Let $q_j$ be the accept probability of $\ket{\rho_j^{\otimes n+m}}$ under \autoref{AlgAmp1}. By Bayes' theorem, conditioned on acceptance, the probability of the input being $\ket{\rho_j^{\otimes n+m}}$ is then $\frac{w_j q_j}{\sum_i w_i q_i}$.

	Let $J=\set{j:q_j<p}$.

	$\Rightarrow\sum_{j\in J} w_j q_j<p$, since $\sum_i w_i=1$.
	
	$\Rightarrow\frac{\sum_{j\in J} w_j q_j}{\sum_i w_i q_i}<\frac{p}{\delta}$.
\end{proof}

Now we consider the general case. We use \autoref{deFinetti}. Let $n=l$ and $k=\poly(n)$, so $\sqrt{\frac{2l^2\ln\abs{A}}{k-l}}$ can be arbitrarily small. That is, with some arbitrarily small error in measurement results, we can assume without the loss of generality that the client receives an input of the simple form above. Hence concludes the proof.

\subsection{Getting the Output}

Thanks to having padded the circuit with identity matrices at the end, if we measured $t>\frac{T}{2}$ on the time register, we would know that the other registers include the output of the required quantum computation. So simply take $m$ to be high enough that it happens at least once with overwhelming probability. When it does, measure the entire register. All the errors up to this point are $O(\varepsilon)$, so will be the distance between this output distribution and the true one.

\Ethan{The above is incredibly handwavy.}

\Ethan{Now use \cite{mahadev_delegation} to turn this into a server-client thing}


\section{Conclusion}

This is a placeholder. Lorem ipsum...


\bibliographystyle{plain}
\bibliography{refs}

\end{document}
