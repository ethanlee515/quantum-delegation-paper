\documentclass{article}
\usepackage{authblk}
\usepackage{amsmath, amssymb, amsthm}
\usepackage{braket}

\newtheorem{theorem}{Theorem}[section]
\theoremstyle{definition}
\newtheorem{definition}{Definition}[section]

\title{Placeholder Title... Something About Delegating Quantum Computations}

\author[1]{Kai-Min Chung}
\author[1]{Yi Lee}
\author[2]{Han-Hsuan Lin}
\author[3]{Xiaodi Wu}
\affil[1]{Institute of Information Science, Academia Sinica, Taipei, Taiwan}
\affil[2]{Department of Computer Science, University of Texas at Austin}
\affil[3]{
	Department of Computer Science, Institute for Advanced Computer Studies,
	and Joint Center for Quantum Information and Computer Science,
	University of Maryland, USA
}

\begin{document}

\maketitle

\begin{abstract}

This is a placeholder. Lorem ipsum. Lorem ipsum. Lorem ipsum...

\end{abstract}

\section{Introduction}

Below is some intro of it copied from my SoP.
This is unpolished and mostly a placeholder at the moment though.

It was proven that BQP=BQIP. That is, if a quantum computer can efficiently solve a given decision problem, then it can also efficiently convince a classical machine of its solution. I'm generalizing this to arbitrary efficient quantum computations. The proof for decision problems involves the classical verifier reducing the problem to a local Hamiltonian instance; the quantum prover would then commit its certificate and act as the verifier’s trusted measurement device as put forth in "Classical Verification of Quantum Computations" by Mahadev. It isn't as trivial as it may seem. Repeating the scheme for each qubit loses the information carried by entanglements and throws off the joint distribution between qubits. Simply measuring the entire output register instead is difficult to analyze. For decision problems, it’s not hard to argue that a malicious prover cannot do better than sending identical copies of some pure state unentangled with each others. That same reasoning doesn't apply here a priori. I've been trying to get a grasp on the particular structure of the local Hamiltonian reduction in order to better analyze it.

\section{Preliminaries}

\subsection{Notations}

Let $\mathcal{B}$ denote the Hilbert space corresponding to a qubit.
Let $H:\mathcal{B}^{\otimes N}\rightarrow\mathcal{B}^{\otimes N}$ be Hermitian.

\begin{definition}
	Let $H\geq0$ denote $H$ is positive semidefinite.
\end{definition}

\begin{definition}
	Let $\lambda(H)$ as the least eigenvalue of $H$.
\end{definition}

\begin{definition}
	Let the \emph{ground state} of $H$ be the eigenvector corresponding to $\lambda(H)$.
\end{definition}

\begin{definition}
	Let $H\big|_S=\prod_SH\prod_S$, where $\prod_S$ is the projection onto the subspace $S$.
\end{definition}

\begin{definition}
	Let $|\widehat{t}\rangle=|11\ldots1\rangle |00\ldots0\rangle$; $t$ $1$s followed by all $0$s. $t$ represented in unary.
\end{definition}

\begin{definition}
	Let $\Lambda_c(U)$ denote the gate $U$ controlled on qubit $c$. $\Lambda_{f, s}(U)$ would be the gate $U$ controlled by both $f$ and $s$. I.e. $\Lambda{1, 2}(X_3)$ would be a Toffoli ((CCNOT) gate.
\end{definition}

\begin{definition}
	Let $P(i)=\begin{pmatrix}1&0\\0&i\end{pmatrix}$
\end{definition}

\subsection{Local Hamiltonian}

The local Hamiltonian problem is defined as follows:

\begin{definition}
	Let $H_1, \ldots, H_n:\mathcal{B}^{\otimes N}\rightarrow\mathcal{B}^{\otimes N}$ be Hermitian. Then $$H=\sum_{j=1}^nH_j$$ is called a \emph{Hamiltonian}.
	If each $H_j$ acts on at most $k$ qubits, then $H$ is called a \emph{$k$-local Hamiltonian}.
\end{definition}

The local Hamiltonian problem is given by let $a<b$ with some inverse polynomial gap, let $H=\sum H_j$, determine whether $\lambda(H)<a$ or $\lambda(H)>b$..

\begin{theorem}
	The Local Hamiltonian problem is QMA-complete.
\end{theorem}

\subsection{Projection Lemma}

Here is a lemma that we will use, taken from \cite{quant-ph/0406180}.

Let $H_1, H_2$ be Hamiltonians where $H_2\geq0$, and $S=\ker H_2$.
Then $\exists J\in\mathbb{R}$ s.t.
$\lambda(H_1\big|_S)-\frac{1}{8}\leq
	\lambda(H_1+JH_2)\leq\lambda(H_1\big|_S)$.

\section{Quantum Delegation Schemes}

Here we describe schemes that allows a classical client to interact with a quantum server, and various properties that such schemes can have.

\begin{definition}
	A \emph{quantum delegation protocol} is an polynomial-round interactive protocol between a BPP verifier and a BQP prover. The BPP verifier has an arbitrary quantum circuit with classical input that it needs to evaluate with the help of the prover.
\end{definition}

TODO (maybe don't assume universality?)

\begin{definition}
	A quantum delegation protocol has \emph{long output with error $\epsilon$} if the output can have multiple bits. In this case, the verifier should end with a measurement result of the output register. Said result's probability distribution would be within $\epsilon$ of the true distribution.
\end{definition}

\begin{definition}
	A quantum delegation protocol is \emph{blind} if a malicious prover cannot learn the input or the output of the computation. In particular, security notions such as \emph{CPA} can be applied here.
\end{definition}

\begin{definition}
	A quantum delegation protocol is \emph{verifiable} if the verifier can check if the output is correct. TODO need to be more rigorous here.
\end{definition}

In \cite{1804.01082}, a verifiable scheme is proposed. In \cite{1708.02130}, a blind scheme is proposed. (TODO just make this CPA or whatever it actually is...) We propose a verifiable scheme with long output which can also be made blind.

TODO check where does \cite{1904.06320} fit in.

\section{Long Output scheme}

Here we describe the scheme between the classical client and an honest quantum server. We mostly follow the logic in \cite{1804.01082}. We first follow \cite{quant-ph/0406180} to reduce the problem to an instance of local Hamiltonian.

We pick our universal gate set to be Hadamard gate and controlled phase gate following \cite{quant-ph/0301040}:
$$H=\frac{1}{\sqrt{2}}\begin{pmatrix}1&1\\1&-1\end{pmatrix}$$
$$\Lambda(P(i))=\begin{pmatrix}1&0&0&0\\0&1&0&0\\0&0&1&0\\0&0&0&i\end{pmatrix}$$

\subsection{Reduction to Real Quantum Computation}

Here we first reduce the circuit to real gates and states only. The proof is taken from \cite{quant-ph/0301040}.

We consider the transform $T\ket{\psi}=\ket{0}\otimes\Re|\psi\rangle+\ket{1}\otimes\Im\ket{\psi}$, where $\Re$ and $\Im$ denote real and imaginary parts respectively. To simplify some notations, let $\ket{\psi}$ be 1-indexed, and let the added qubit be the 0th. Note that after this transformation, the state is real. This transformation preserves measurement results in the standard basis. Under the transform, Hadamard gates are unchanged, but controlled phase gates become a combination of Hadamard and Toffoli gates. Mathematically,
$$T\circ H_f=H_f\circ T$$
$$T\circ\Lambda_f(P(i)_s)=\Lambda_{f,s}(X_0)\Lambda_{f,s}(Z_0)\circ T=\Lambda_{f,s}(X_0)H_0\Lambda_{f,s}(X_0)H_0\circ T$$
We then double (TODO: maybe multiply by $\gamma>1$ instead of doubling?) the size of the circuit by padding identity matrices at the end.
After these transformations, let $x$ be the new circuit input, and $U_T...U_1$ be the circuit to evaluate.

\subsection{preprocess the circuit}

First, we rewrite the circuit to consist of only controlled phase gates and 1-qubit gates. TODO: prove that these 1-qubit gates are all real. We ensure that each $C_\phi$s are preceded and followed by two $Z$ gates. That is, if the $C_\phi$ acts on the qubits $f_t$ and $s_t$, then the circuit would look like $Z_{f_t}Z_{s_t}C_\phi Z_{s_t}Z_{f_t}$ around each $C_\phi$. We then pad the circuit so that the $C_\phi$s are applied at $t\in T_2=\{L, 2L, 3L, \ldots\}$ for some $L\in\mathbb{N}$. Let $T_1=[T]\setminus T_2$ be time steps corresponding to 1-qubit gates.

\subsection{Constructing the 2-local Hamiltonian instance}

We then attempt to construct a 2-local Hamiltonian with ground state $$\phi=\sum_{t=0}^TU_t...U_1|x\rangle\otimes|\hat{t}\rangle$$. We do so by causing states perpendicular to this to have high eigenvalues.

First, we ensure that the invalid clock states have high eigenvalues. That is, we take $$H_{clock}=\sum_{1\leq i<j\leq T} I\otimes |01\rangle\langle01|_{ij}$$. We also set $S_{legal}=\ker H_{clock}$.
We note that when stricted to $S_{legal}$, 
\begin{itemize}
	\item $|10\rangle\langle10|_{t,t+1}
		=|\widehat{t}\rangle\langle\widehat{t}|$ 
	\item $|1\rangle\langle0|_{t+1}
		=|\widehat{t+1}\rangle\langle\widehat{t}|$
	\item $|11\rangle\langle00|_{t,t+1}
		=|\widehat{t+2}\rangle\langle\widehat{t}|$
\end{itemize}

Then, we ensure that 1-qubit gates are propagated properly. To do so, we set
	$$H_{prop1}=\sum_{t\in T_1}H_{prop,t}$$
	$$H_{prop,t}=I\otimes|\widehat{t}\rangle\langle\widehat{t}|
		+I\otimes|\widehat{t-1}\rangle\langle\widehat{t-1}|
		-U_t\otimes|\widehat{t}\rangle\langle\widehat{t-1}|
		-U_t^\dagger\otimes|\widehat{t-1}\rangle\langle\widehat{t}|$$

I probably need to explain why this works but I'll leave it for another day.

Let $S_{prop1}=\ker H_{prop1}$.
Then, $S_{prop1}$ has the following as basis:
\begin{itemize}
	\item $|00\rangle_{f_t,s_t}|\xi_{00}\rangle\otimes
		(|\widehat{t}\rangle+|\widehat{t+1}\rangle+|\widehat{t+2}\rangle)$
	\item $|01\rangle_{f_t,s_t}|\xi_{01}\rangle\otimes
		(|\widehat{t}\rangle+|\widehat{t+1}\rangle-|\widehat{t+2}\rangle)$
	\item $|10\rangle_{f_t,s_t}|\xi_{10}\rangle\otimes
		(|\widehat{t}\rangle-|\widehat{t+1}\rangle-|\widehat{t+2}\rangle)$
	\item $|11\rangle_{f_t,s_t}|\xi_{11}\rangle\otimes
		(|\widehat{t}\rangle-|\widehat{t+1}\rangle+|\widehat{t+2}\rangle)$
\end{itemize}

Also, if $t\in T_2$, that is, $t$ corresponds to a time step where $C_\phi$ is applied, then the following are true due to $C_\phi$ being surrounded by $Z$ gates:
\begin{itemize}
	\item $|0\rangle\langle0|_{f_t}\otimes
		|\widehat{t}\rangle\langle\widehat{t}|
		=(|\widehat{t}\rangle+|\widehat{t+1}\rangle)
		(\langle\widehat{t}|+\langle\widehat{t+1}|)$
	\item $|1\rangle\langle1|_{f_t}\otimes
		|\widehat{t}\rangle\langle\widehat{t}|
		=(|\widehat{t}\rangle-|\widehat{t+1}\rangle)
		(\langle\widehat{t}|-\langle\widehat{t+1}|)$
	\item $|0\rangle\langle0|_{s_t}\otimes
		|\widehat{t}\rangle\langle\widehat{t}|
		=(|\widehat{t+1}\rangle+|\widehat{t+2}\rangle)
		(\langle\widehat{t+1}|+\langle\widehat{t+2}|)$
	\item $|1\rangle\langle1|_{s_t}\otimes
		|\widehat{t}\rangle\langle\widehat{t}|
		=(|\widehat{t+1}\rangle-|\widehat{t+2}\rangle)
		(\langle\widehat{t+1}|-\langle\widehat{t+2}|)$
	\item $(|00\rangle\langle00|_{f_t,s_t}
		+|11\rangle\langle11|_{f_t,s_t})\otimes
		|\widehat{t}\rangle\langle\widehat{t}|
		=(|\widehat{t}\rangle+|\widehat{t+2}\rangle)
		(\langle\widehat{t}|+\langle\widehat{t+2}|)$
	\item $(|01\rangle\langle01|_{f_t,s_t}
		+|10\rangle\langle10|_{f_t,s_t})\otimes
		|\widehat{t}\rangle\langle\widehat{t}|
		=(|\widehat{t}\rangle-|\widehat{t+2}\rangle)
		(\langle\widehat{t}|-\langle\widehat{t+2}|)$
\end{itemize}

Similarly,
\begin{itemize}
	\item $|0\rangle\langle0|_{f_t}\otimes
		|\widehat{t-1}\rangle\langle\widehat{t-1}|
		=(|\widehat{t-1}\rangle+|\widehat{t-2}\rangle)
		(\langle\widehat{t-1}|+\langle\widehat{t-2}|)$
	\item $|1\rangle\langle1|_{f_t}\otimes
		|\widehat{t-1}\rangle\langle\widehat{t-1}|
		=(|\widehat{t-1}\rangle-|\widehat{t-2}\rangle)
		(\langle\widehat{t-1}|-\langle\widehat{t-2}|)$
	\item $|0\rangle\langle0|_{s_t}\otimes
		|\widehat{t-1}\rangle\langle\widehat{t-1}|
		=(|\widehat{t-2}\rangle+|\widehat{t-3}\rangle)
		(\langle\widehat{t-2}|+\langle\widehat{t-3}|)$
	\item $|1\rangle\langle1|_{s_t}\otimes
		|\widehat{t-1}\rangle\langle\widehat{t-1}|
		=(|\widehat{t-2}\rangle-|\widehat{t-3}\rangle)
		(\langle\widehat{t-2}|-\langle\widehat{t-3}|)$
	\item $(|00\rangle\langle00|_{f_t,s_t}
		+|11\rangle\langle11|_{f_t,s_t})\otimes
		|\widehat{t-1}\rangle\langle\widehat{t-1}|
		=(|\widehat{t-1}\rangle+|\widehat{t-3}\rangle)
		(\langle\widehat{t-1}|+\langle\widehat{t-3}|)$
	\item $(|01\rangle\langle01|_{f_t,s_t}
		+|10\rangle\langle10|_{f_t,s_t})\otimes
		|\widehat{t-1}\rangle\langle\widehat{t-1}|
		=(|\widehat{t-1}\rangle-|\widehat{t-3}\rangle)
		(\langle\widehat{t-1}|-\langle\widehat{t-3}|)$
\end{itemize}

Now we ensure the 1-qubit gates are applied properly. The naive way would be to set
	$$H_t=I\otimes|\widehat t\rangle\langle\widehat t|
		+I\otimes|\widehat{t-1}\rangle\langle\widehat{t-1}|
		-C_\phi\otimes|\widehat t\rangle\langle\widehat{t-1}|
		-C_\phi^\dagger\otimes|\widehat{t-1}\rangle\langle\widehat{t}|$$
	$$H'=\sum_{t\in T_2}H_t$$
	Equivalently, $H_t=$

	$|00\rangle\langle00|_{f_t,s_t}\otimes
		(|\widehat{t}\rangle-|\widehat{t-1}\rangle)
		(\langle\widehat{t}|-\langle\widehat{t-1}|)+$

	$|01\rangle\langle01|_{f_t,s_t}\otimes
		(|\widehat{t}\rangle-|\widehat{t-1}\rangle)
		(\langle\widehat{t}|-\langle\widehat{t-1}|)+$

	$|10\rangle\langle10|_{f_t,s_t}\otimes
		(|\widehat{t}\rangle-|\widehat{t-1}\rangle)
		(\langle\widehat{t}|-\langle\widehat{t-1}|)+$

	$|11\rangle\langle11|_{f_t,s_t}\otimes
		(|\widehat{t}\rangle+|\widehat{t-1}\rangle)
		(\langle\widehat{t}|+\langle\widehat{t-1}|)$

However, this cannot be turned into 2-local, so we change the coefficients and set 
	$$H_{prop2}=\sum_{t\in T_2}H_{prop2,t}$$

	$H_{prop2,t}=$

	$|00\rangle\langle00|_{f_t,s_t}\otimes
		4(|\widehat{t}\rangle-|\widehat{t-1}\rangle)
		(\langle\widehat{t}|-\langle\widehat{t-1}|)+$

	$|01\rangle\langle01|_{f_t,s_t}\otimes
		2(|\widehat{t}\rangle-|\widehat{t-1}\rangle)
		(\langle\widehat{t}|-\langle\widehat{t-1}|)+$

	$|10\rangle\langle10|_{f_t,s_t}\otimes
		2(|\widehat{t}\rangle-|\widehat{t-1}\rangle)
		(\langle\widehat{t}|-\langle\widehat{t-1}|)+$

	$|11\rangle\langle11|_{f_t,s_t}\otimes
		(|\widehat{t}\rangle+|\widehat{t-1}\rangle)
		(\langle\widehat{t}|+\langle\widehat{t-1}|)$
instead.

Now, if restricted to $S_{prop}$,

	$H_{prop2,t}=H_{qubit,t}+H_{time,t}=$
	$$(-2|0\rangle\langle0|_{f_t}-2|0\rangle\langle0|_{s_t}
		+|1\rangle\langle1|_{f_t}+|1\rangle\langle1|_{s_t})\otimes$$
	$$(|\widehat{t}\rangle\langle\widehat{t-1}|+
		|\widehat{t-1}\rangle\langle\widehat{t}|)+$$
	$$(2|0\rangle\langle0|_{f_t}+2|0\rangle\langle0|_{s_t}
		+|1\rangle\langle1|_{f_t}+|1\rangle\langle1|_{s_t}
		-2|01\rangle\langle01|_{f_t,s_t}
		-2|10\rangle\langle10|_{f_t,s_t})\otimes$$
	$$(|\widehat{t-1}\rangle\langle\widehat{t-1}|+
		|\widehat{t}\rangle\langle\widehat{t}|)$$


Lastly, we check if the input is right...
	$$H_{in}=\sum_{i=m+1}^N|1\rangle\langle1|_i\otimes|0\rangle\langle0|$$

Now we use the projection lemma here. This step is different now that $H_{out}$ is gone. I think $J_{in}$ gets dropped or something like that?

\subsection{2-local ZX Hamiltonians}

We then follow \cite{0704.1287} in writing each terms of $H$ into only $Z$ and $X$ measurements...

This probably involves some kind of basis arguments on the Hamiltonians and maybe some stuff that's a bit more ugly. This should not change the state. I'll have to write this out later...

\subsection{Checking computation}

We then follow \cite{1804.01082}. That is, the server would commit copies of the ground state of the Hamiltonian for the classical client to measure. The client would check whether the computation is done correctly using a modified Hamiltonian from \cite{quant-ph/0406180}. 

And then, this in fact involves some probabilistic analysis shenanigans that we haven't fully figured out yet.

\subsection{Getting the Output}

Thanks to having padded the circuit with identity matrices at the end, if we measured $T>\frac{T}{2}$ on the time register, we would know that the other registers include the output of the required quantum computation. Now I would analyze the chances of getting this right, maybe in terms of fidelity or in terms of distance between this measured distribution and the ideal distribution...

\section{Conclusion}

This is a placeholder. Lorem ipsum...

\begin{thebibliography}{9}
	\bibitem{1804.01082}
	Urmila Mahadev.
	\newblock Classical Verification of Quantum Computations, 2018;
	\newblock arXiv:1804.01082.

	\bibitem{quant-ph/0406180}
	Julia Kempe, Alexei Kitaev and Oded Regev.
	\newblock The Complexity of the Local Hamiltonian Problem, 2004,
	\newblock SIAM Journal of Computing, Vol. 35(5), p. 1070-1097 (2006),
		conference version in Proc. 24th FSTTCS, p. 372-383 (2004);
	\newblock arXiv:quant-ph/0406180.

	\bibitem{1109.0795}
	Matthew McKague.
	\newblock On the power quantum computation over real Hilbert spaces, 2011,
	\newblock Int. J. Quantum Inform., 11, 1350001 (2013);
	\newblock arXiv:1109.0795.
	\newblock DOI: 10.1142/S0219749913500019.
	
	\bibitem{0704.1287}
	Jacob D. Biamonte and Peter J. Love.
	\newblock Realizable Hamiltonians for Universal Adiabatic Quantum Computers, 2007,
	\newblock Phys. Rev. A 78, 012352 (2008).;
	\newblock arXiv:0704.1287.
	\newblock DOI: 10.1103/PhysRevA.78.012352.

	\bibitem{1904.06320}
	Alexandru Gheorghiu and Thomas Vidick.
	\newblock Computationally-secure and composable remote state preparation, 2019;
	\newblock arXiv:1904.06320.

	\bibitem{1708.02130}
	Urmila Mahadev.
	\newblock Classical Homomorphic Encryption for Quantum Circuits, 2017;
	\newblock arXiv:1708.02130.

	\bibitem{quant-ph/0301040}
	Dorit Aharonov.
	\newblock A Simple Proof that Toffoli and Hadamard are Quantum Universal, 2003;
	\newblock arXiv:quant-ph/0301040.

\end{thebibliography}

\end{document}
