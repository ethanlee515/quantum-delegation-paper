\section{Proofs for \Cref{sec:qpip0_all}}
\label{sec:qpip0_proof}

% \begin{theorem}[binding property of $\PiNaive$]
% 	\label{lem:naive-qpip0-binding}
% 	Let $\PNaiveStar$ be a cheating $\BQP$ prover for $\PiNaive$ and $\lambda$ be the security parameter.
% 	Suppose that $\Prob{d=\Acc\mid y\ne\bot, c=0}$ is overwhelming, 
% 	under the QLWE assumption, then the verifier's output in the Hadamard round is $O(\eps)$-computationally indistinguishable from $(d, z_{ideal})$.
% \end{theorem}
\begin{proof}[\Cref{lem:naive-qpip0-binding}]
	We first introduce the \emph{dummy strategy} for ~\Cref{proto:urmila4}, where the prover chooses $\rho$ as the maximally mixed state and executes the rest of the protocol honestly.
	It is straightforward to verify that this prover would be accepted in the testing round with probability $1-\negl(\lambda)$,
	but has negligible probability passing the verification  after the Hadamard round.

  
   %for $\PiMeasure$ $\Pstarsub$ that is almost perfect as follows. 
	Now we construct a cheating $\BQP$ prover for \Cref{proto:qpip0_naive}, $\Pstar$, that does the same thing as $\PNaiveStar$ except at Step~\ref{step:urmila-in-naive}, where the prover and verifier runs \Cref{proto:urmila4}. $\Pstar$ does the following in Step~\ref{step:urmila-in-naive}:
	for the second message, run $(y, \sigma)\leftarrow\cPNaiveStar{2}(pk, \rho)$.
	If $y\ne\bot$, then reply $y$;
	else, run the corresponding step of the dummy strategy and reply with its results.
	For the fourth message, if $y\ne\bot$, run and reply with $a\leftarrow\cPNaiveStar{4}(pk, c, \sigma)$;
	else, continue the dummy strategy.


%	so we can apply \Cref{lem:urmila-binding} to the $\PiMeasure$ call to use its binding property (\Cref{lem:urmila-binding}).
%That is, there exists some $\rho$ such that $v=M_{XZ}(\rho, h)$.\hannote{only comp}


	 In the following we fix an $x$. Let the distribution on $h$ specified in Step~\ref{step:naive1} of the protocol be $p_x(h)$. Define $\Pstarsub(x)$ as $\Pstar$'s responds in Step~\ref{step:urmila-in-naive}. Note that we can view $\Pstarsub(x)$ as a prover strategy for \Cref{proto:urmila4}. By construction $\Pstarsub(x)$ passes testing round with overwhelming probability over $p_x(h)$, i.e. $\sum_h p_x(h) p_{h,T} =\negl(\lambda)$, where $p_{h,T}$ is $\Pstar$'s probability of getting accepted by the prover on the testing round on basis choice $h$. By \Cref{lem:urmila-binding} and Cauchy's inequality, there exists some $\rho$ such that  $\sum_h p_x(h) \norm{v_h -M_{XZ}(\rho, h)}_c = \negl(\lambda)$, where we use $\norm{A-B}_c=\alpha$ to denote that $A$ is $\alpha$-computational indistinguishable to $B$. Therefore $v= \sum_h p_x(h) v_h$ is computationally indistinguishable to $\sum_h p_x(h) M_{XZ}(\rho, h)$. Combining it with $\PiSamp$'s soundness (\Cref{QPIP1thm}), 
	we see that $(d', z')\leftarrow(\Pstar, \VNaive)(1^\lambda, 1^{1/\epsilon}, x)$  is $\eps$-computationally indistinguishable to $(d', z_{ideal}')$.

	Now we relate $(d', z')$ back to $(d, z)$.
	First, conditioned on that $\PNaiveStar$ aborts, since dummy strategy will be rejected with overwhelming probability in Hadamard round,
	we have $(d', z')$ is computationally indistinguishable to $(\Rej, \bot)=(d, z)$.
	On the other hand, conditioned on $\PNaiveStar$ not aborting, clearly $(d, z)=(d', z')$.
	So $(d, z)$ is computationally indistinguishable to $(d', z')$,
	which in turn is $O(\eps)$-computationally indistinguishable to $(d', z_{ideal}')$.
	Since $\norm{d-d'}_{tr}= O(\eps)$,
	 $(d, z_{ideal})$ is $O(\eps)$-computationally indistinguishable to $(d', z_{ideal}')$.
	Combining everything, we conclude that $(d, z)$ is $O(\eps)$-computationally indistinguishable to $(d, z_{ideal})$.
\end{proof}


\begin{proof}[\Cref{thm:zi-zgoodi}]
	We take expectation of \Cref{eq:partition-string} over $\gamma$
	$$\ket{\psi}=\E_{\gamma}\left[
		\sum_{j=0}^{i-1} \ket{\psi_{1^j0,\gamma}} +\ket{\psi_{1^i,\gamma}} +\sum_{j=1}^{i}\ket{\psi_{err,j,\gamma}}
	\right],$$
	and expand $z_i$ from \Cref{eq:zi-def} as
	\begin{align}
		z_i &= z_{good,i}+ \E_{pk, y, \gamma} \sum_z \L[\sum_{k=0}^{i-1} \bra{\psi_{1^k0,\gamma}}U^\dag  P_{acc,i,z}U   \sum_{j=0}^{i-1} \ket{\psi_{1^j0,\gamma}} \R. \nn \\
		&+
		\sum_{k=0}^{i-1} \bra{\psi_{1^k0,\gamma}}U^\dag  P_{acc,i,z}U \ket{\psi_{1^i,\gamma}} +\sum_{k=0}^{i-1} \bra{\psi_{1^k0,\gamma}}U^\dag  P_{acc,i,z}U\sum_{j=1}^{i}\ket{\psi_{err,j,\gamma}} \nn \\
		&+\bra{\psi_{1^i,\gamma}} U^\dag  P_{acc,i,z}U \sum_{j=0}^{i-1} \ket{\psi_{1^j0,\gamma}} +\bra{\psi_{1^i,\gamma}} U^\dag  P_{acc,i,z}U \sum_{j=1}^{i}\ket{\psi_{err,j,\gamma}}
		\nn \\
		&+ \sum_{k=1}^{i}\bra{\psi_{err,k,\gamma}} U^\dag  P_{acc,i,z}U  \sum_{j=0}^{i-1} \ket{\psi_{1^j0,\gamma}} + \sum_{k=1}^{i}\bra{\psi_{err,k,\gamma}} U^\dag  P_{acc,i,z}U \ket{\psi_{1^i,\gamma}}
		\nn \\
		&\L.    +\sum_{k=1}^{i}\bra{\psi_{err,k,\gamma}} U^\dag  P_{acc,i,z}U \sum_{j=1}^{i}\ket{\psi_{err,j,\gamma}} \R] \proj{z} , \nn     
		%=& z_{good,i} +(\text{terms with } \psi_{1^j0},\, j\neq i ) + (\text{terms with } \psi_{1^{i-1}0}) +(\text{terms with }err )
	\end{align}
	where we omitted writing out $e_i$.
	Therefore we have
	\begin{align*}
		\tr|z_i-z_{good,i}|\leq \E_{pk, y, \gamma} \sum_z &\L[ \sum_{k=0}^{i-1} \sum_{j=0}^{i-1} \L| \bra{\psi_{1^k0,\gamma}}U^\dag  P_{acc,i,z}U \ket{\psi_{1^j0,\gamma}} \R|\R.\\
		&+
		2 \sum_{k=0}^{i-1} \L|\bra{\psi_{1^k0,\gamma}}U^\dag  P_{acc,i,z}U \ket{\psi_{1^i,\gamma}} \R| \\
		&+ 2 \sum_{k=0}^{i-1}\sum_{j=1}^{i}\L| \bra{\psi_{1^k0,\gamma}}U^\dag  P_{acc,i,z}U\ket{\psi_{err,j,\gamma}}\R| \\   
		&+2 \sum_{j=1}^{i}\L|\bra{\psi_{1^i,\gamma}} U^\dag  P_{acc,i,z}U \ket{\psi_{err,j,\gamma}}\R| \\
		&+\L. \sum_{k=1}^{i}\sum_{j=1}^{i}\L| \bra{\psi_{err,k,\gamma}} U^\dag  P_{acc,i,z}U \ket{\psi_{err,j,\gamma}}\R| \R] \\ %%%%%%%%%
	\end{align*}
	by the triangle inequality.
	The last three error terms sum to $O\L(\frac{m^2}{\sqrt{T}}\R)$ by \Cref{lem:samp-tech} and property~\ref{property:partition-err} of \Cref{lem:partition2}.
	As for the first two terms, by \Cref{lem:samp-tech} and \Cref{lem:partition-testing}, we see that
	\begin{align*}
		\sum_z \sum_{k=0}^{i-1}\sum_{j=0}^{i-1}
		&\abs{\bra{\psi_{1^k0,\gamma}}U^\dag  P_{acc,i,z}U \ket{\psi_{1^j0,\gamma}}} \\
		&\leq\sum_z \abs{\bra{\psi_{1^{i-1}0,\gamma}}U^\dag  P_{acc,i,z}U \ket{\psi_{1^{i-1}0,\gamma}}} + O\L(m^2(m-1)\gamma_0\R) \\
		&\leq\norm{\ket{\psi_{1^{i-1}0,\gamma}}}^2 + O\L(m^2(m-1)\gamma_0\R)
	\end{align*}
	and similarly
	\begin{align*}
		\sum_z\sum_{k=0}^{i-1}
		&\abs{\bra{\psi_{1^k0,\gamma}}U^\dag  P_{acc,i,z}U \ket{\psi_{1^i,\gamma}}}\\
		&\leq\sum_z\abs{\bra{\psi_{1^{i-1}0,\gamma}}U^\dag  P_{acc,i,z}U \ket{\psi_{1^i,\gamma}}}+O\L(m\sqrt{(m-1)\gamma_0}\R)\\
		&\leq\norm{\ket{\psi_{1^i,\gamma}}}+O\L(m\sqrt{(m-1)\gamma_0}\R).
	\end{align*}
\end{proof}
