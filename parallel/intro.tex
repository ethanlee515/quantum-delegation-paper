\section{Introduction}
Can we verify quantum computations by a classical computer? This problem has been a major open problem in the field until Mahadev~\cite{FOCS:Mahadev18a} finally gave an affirmative solution.
Specifically, she constructed an interactive protocol between an efficient classical verifier (a BPP machine) and an efficient quantum prover (a BQP machine) where the verifier can verify the result of the BQP computation.
(In the following, we call such a protocol a CVQC protocol.\footnote{``CVQC" stands for ``Classical Verification of Quantum Computations"}) 
Soundness of her protocol relies on a computational assumption that the learning with error (LWE) problem \cite{JACM:Regev09} is hard for an efficient quantum algorithm, which has been widely used in the field of cryptography. We refer to the extensive survey by Peikert \cite{FTTCS:Peikert16} for details about LWE and its cryptographic applications.

Though her result is a significant breakthrough, there are still several drawbacks. First, her protocol has soundness error $3/4$, which means that a cheating prover may convince the verifier even if it does not correctly computes the BQP computation with probability at most $3/4$. Though we can exponentially reduce the soundness error by sequential repetition, we need super-constant rounds to reduce the soundness error to be negligible.
If parallel repetition works to reduce the soundness error, then we need not increase the number of round.
However, parallel repetition may not reduce soundness error for computationally sound protocol in general \cite{FOCS:BelImpNao97,TCC:PieWik07}.
Thus, it is still open to construct constant round protocol with negligible soundness error.

Another issue is about verifier's efficiency. In her protocol, for verifying a computation that is done by a quantum computer in time $T$, the verifier's running time is as large as $\poly(T)$.
Considering a situation where a device with weak classical computational power outsources computations to untrusted quantum server, we may want to make the verifier's running time as small as possible.
Such a problem has been studied well in the setting where the prover is classical and we know solutions where verifier's running time only logarithmically depends on $T$~\cite{STOC:Kilian92,SIAM:Micali00,STOC:KalRazRot13,STOC:KalRazRot14,JACM:GolKalRot15,STOC:ReiRotRot16,STOC:BraHolKal17,STOC:BKKSW18,FOCS:HolRot18,STOC:CCHLRRW19,STOC:KalPanYan19}.
%\takashi{We may need not cite such a lot. I just cited all works I know.}
Hopefully, we want to obtain a CVQC protocol where the verifier runs in logarithmic time.  
%However, it is non-trivial to combine the Mahadev's protocol of verification of quantum computations and a verifiable delegation protocol in the classical setting.
%Thus, 

\subsection{Our Results}
In this paper, we solve the above drawbacks of the Mahadev's protocol. 
Our contribution is divided into three parts:
\begin{itemize}
    \item We show that parallel repetition version of Mahadev's protocol has negligible soundness error. This gives the first constant round CVQC protocol with negligible soundness error.
    \item We construct a two-round CVQC protocol in the quantum random oracle model (QROM) \cite{AC:BDFLSZ11} where a cryptographic hash function is idealized to be a random function that is only accessible as a quantum oracle.
    This is obtained by applying the Fiat-Shamir transform \cite{C:FiaSha86,C:LiuZha19,C:DFMS19} to the parallel repetition version of the Mahadev's protocol.
    \item We construct a two-round CVQC protocol with logarithmic-time verifier in the CRS+QRO model where both prover and verifier can access to a (classical) common reference string generated by a trusted third party in addition to quantum access to QRO.
    For proving soundness, we assume that a standard model instantiation of our two-round protocol with a concrete hash function (say, SHA-3) is sound and the existence of post-quantum indistinguishability obfuscation \cite{JACM:BGIRSVY12,SIAM:GGH0SW16} and (post-quantum) fully homomorphic encryption (FHE) \cite{STOC:Gentry09} in addition to the quantum hardness of the LWE problem. 
\end{itemize}

\subsection{Related Works}
\paragraph{Verification of Quantum Computation.}
There are long line of researches on verification of quantum computation.
Except for solutions relying on computational assumptions, there are two type of settings where verification of quantum computation is known to be possible.
In the first setting, instead of considering purely classical verifier, we assume that a verifier can perform a certain kind of weak quantum computations \cite{FOCS:BroFitKas09,PR:FitKas17,arXiv:ABOEM17,PR:MorFit18}.
In the second setting, we assume that a prover is splitted into two remote servers that share entanglement but do not communicate \cite{Nat:RUV13}.
Though these works do not give a CVQC protocol in our sense, the advantage is that we need not assume any computational assumption for the proof of soundness, and thus they are incomparable to Mahadev's result and ours.

Subsequent to Mahadev's breakthrough result, Gheorghiu and Vidick \cite{FOCS:GheVid19} gave a CVQC protocol that also satisfies blindness, which ensures that a prover cannot learn what computation is delegated.
We note that their protocol requires polynomial number of rounds.

\paragraph{Concurrent Work.}
In a concurrent and independent work, Alagic et al. \cite{arXiv:AlaChiHun19} also shows similar results to our first and second results, parallel repetition theorem for the Madadev's protocol and a two-round CVQC protocol by the Fiat-Shamir transform.
We note that our third result, a two-round CVQC protocol with efficient verification, is unique in this paper.   








