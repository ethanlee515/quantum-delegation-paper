\section{Two-Round Protocol via Fiat-Shamir Transform}\label{sec:tworound}
In this section, we show that if we apply the Fiat-Shamir transform to $m$-parallel version of the Mahadev's protocol, then we obtain two-round protocol in the QROM. 
That is, we prove the following theorem.
\begin{theorem}\label{thm:MahFS}
Assuming LWE assumption, there exists a two-round CVQC protocol with overwhelming completeness and negligible soundness error in the QROM.
\end{theorem}

\begin{proof}%(of Theorem~\ref{thm:MahFS})
Let $m>\secpar$ be a sufficiently large integer so that $m$-parallel version of the Mahadev's protocol has negligible soundness.
For notational simplicity, we abuse the notation to simply use $\ver_{i}$, $\pro_{i}$, and $\ver_{\out}$ to mean the $m$-parallel repetitions of them. 
Let $H:\calY\ra \bit^{m}$ be a hash function idealized as a quantum random oracle where $\calX$ is the space of the second message $y$ and $\calY=\bit^{m}$.
Our two-round protocol is described below:
\begin{description}
\item[First Message:] The verifier runs $\ver_1$ to generate $(k,\td)$. Then it sends $k$ to the prover and keeps $\td$ as its state.
\item[Second Message:] The prover runs $\pro_2$ on input $k$ to generate $y$ along with the prover's state $\ket{\st_\pro}$. Then set $c\defeq H(y)$, and runs $\pro_4$ on input $\ket{\st_\pro}$ and $y$ to generate $a$. Finally, it returns $(y,a)$ to the verifier.
\item[Verification:] The verifier computes $c=H(y)$, runs $\ver_{\out}(k,\td,y,c,a)$, and outputs as $\ver_{\out}$ outputs.
\end{description}

It is clear that the completeness is preserved given that $H$ is a random oracle.
We reduce the soundness of this protocol to the soundness of $m$-parallel version of the Mahadev's protocol.
For proving this, we borrow the following lemma shown in \cite{C:DFMS19}.

%\begin{lemma}[\cite{C:DFMS19}]\label{lem:FS}
%Let $\calY$ be finite non-empty sets. There exists a black-box polynomial-time two-stage quantum algorithm $\calS$ with the following property. Let $\A$ be an arbitrary oracle quantum algorithm that makes $q$ queries to a uniformly random $H:\calY\ra \bit^{m}$ and that outputs some $y\in \calY$ and output $a$. 
%Then, the two-stage algorithm $\calS^{\A}$ outputs $y\in\calY$ in the first stage
%and, upon a random $c^*\in \bit^{m}$ as input to the second stage, output $a$ so that for any $x_\circ\in \calX$ and any predicate $V$:
%\begin{align*}
%    \Pr_{c^*}\left[y=y_\circ \land V(y,c^*,a):(y,a)\sample \langle\calS^{\A},c^* \rangle \right]\lapprox \frac{1}{O(q^2)}\Pr_{H}\left[y=y_{\circ}\land V(y,H(y),a):(y,a)\sample \A^H\right],
%\end{align*}
%where 
%$(y,a)\sample \langle\calS^{\A},c^* \rangle$ means that $\calS^{\A}$ outputs $y$ and $a$ in the first and second stages respectively on the second stage input $c^*$, and  $\lapprox$ hides a term that is bounded by $\frac{1}{2^{m+1}q}$ when summed over all $y_{\circ}\in \calY$.
%\end{lemma}

\begin{lemma}[{\cite[Theorem 2]{C:DFMS19}}]\label{lem:FS}
Let $\calY$ be finite non-empty sets. There exists a black-box polynomial-time two-stage quantum algorithm $\calS$ with the following property. Let $\A$ be an arbitrary oracle quantum algorithm that makes $q$ queries to a uniformly random $H:\calY\ra \bit^{m}$ and that outputs some $y\in \calY$ and output $a$. 
Then, the two-stage algorithm $\calS^{\A}$ outputs $y\in\calY$ in the first stage
and, upon a random $c\in \bit^{m}$ as input to the second stage, output $a$ so that for any $x_\circ\in \calX$ and any predicate $V$:
\begin{align*}
    \Pr_{c}\left[V(y,c,a):(y,a)\sample \langle\calS^{\A},c \rangle \right]\leq \frac{1}{O(q^2)}\Pr_{H}\left[V(y,H(y),a):(y,a)\sample \A^H\right]-\frac{1}{2^{m+1}q},
\end{align*}
where 
$(y,a)\sample \langle\calS^{\A},c \rangle$ means that $\calS^{\A}$ outputs $y$ and $a$ in the first and second stages respectively on the second stage input $c$.
\end{lemma}

We assume that there exists an efficient adversary $\A$ that breaks the soundness of the above two-round protocol.
We fix $x\notin \lang$ on which $\A$ succeeds in cheating.
We fix $(k,\td)$ that is in the support of the verifier's first message.
We apply Lemma~\ref{lem:FS} for $\A=\A(k)$ and $V=\ver_\out(k,\td,\cdot,\cdot,\cdot)$, to obtain an algorithm $\calS^{\A(k)}$ that satisfies
\begin{align*}
    &\Pr_{c}\left[V_{\out}(k,\td,y,c,a):(y,a)\sample \langle\calS^{\A(k)},c \rangle \right]\\
    \leq &\frac{1}{O(q^2)}\Pr_{H}\left[V_{\out}(k,\td,y,H(y),a):(y,a)\sample \A^H(k)\right]-\frac{1}{2^{m+1}q}.
\end{align*}
Averaging over all possible $(k,\td)$, we have
\begin{align*}
    &\Pr_{k,\td,c}\left[V_{\out}(k,\td,y,c,a):(y,a)\sample \langle\calS^{\A(k)},c \rangle \right]\\
    \leq &\frac{1}{O(q^2)}\Pr_{k,\td,H}\left[V_{\out}(k,\td,y,H(y),a):(y,a)\sample \A^H(k)\right]-\frac{1}{2^{m+1}q}.
\end{align*}
Since we assume that $\A$ breaks the soundness of the above two-round protocol,
\[
\Pr_{k,\td,H}\left[V_{\out}(k,\td,y,H(y),a):(y,a)\sample \A^H(k)\right]
\]
is non-negligible in $\secpar$.
Therefore, as long as $q=\poly(\secpar)$, 
\[
\Pr_{k,\td,c^*}\left[V_{\out}(k,\td,y,c^*,a):(y,a)\sample \langle\calS^{\A(k)},c^* \rangle \right]\]
is also non-negligible in $\secpar$.
Then, we construct an adversary $\B$ that breaks the soundness of parallel version of Mahadev's protocol as follows:
\begin{description}
\item[Second Message:] Given the first message $k$, $\B$ runs the first stage of $\calS^{\A(k)}$ to obtain $y$. It sends $y$ to the verifier.
\item[Forth Message:] Given the third message $c$, $\B$ gives $c$ to $\calS^{\A(k)}$ as the second stage input, and let $a$ be the output of it.
Then $\B$ sends $a$ to the verifier.
\end{description}
Clearly, the probability that $\B$ succeeds in cheating is 
\[
\Pr_{k,\td,c^*}\left[V_{\out}(k,\td,y,c^*,a):(y,a)\sample \langle\calS^{\A(k)},c^* \rangle \right],\]
which is non-negligible in $\secpar$.
This contradicts the soundness of $m$-parallel version of Mahadev's protocol (Theorem~\ref{thm:rep_soundness}).
Therefore we conclude that there does not exists an adversary that succeeds in the two-round protocol with non-negligible probability assuming LWE in the QROM.
\end{proof}