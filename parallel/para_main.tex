\documentclass[11pt]{article}
\usepackage[utf8]{inputenc}
\usepackage{amsmath}
\usepackage{amsfonts,amsthm}
\usepackage{braket}
\usepackage{authblk}
%\usepackage[colorinlistoftodos]{todonotes}
\usepackage[colorlinks=true,allcolors=blue]{hyperref}
\usepackage[margin=1in]{geometry}
\usepackage{xcolor}
\usepackage{algorithm}
\usepackage{mdframed} 
\mdfdefinestyle{figstyle}{ %
  linecolor=black!7, %
  backgroundcolor=black!7, %
  innertopmargin=10pt, %
  innerleftmargin=\textwidth, %
  innerrightmargin=\textwidth, %
  innerbottommargin=5pt %
}

\usepackage{comment}


\newtheorem{theorem}{Theorem}[section]
\newtheorem{conjecture}[theorem]{Conjecture}
\newtheorem{lemma}[theorem]{Lemma}
\newtheorem{claim}[theorem]{Claim}
\newtheorem{proposition}[theorem]{Proposition}
\newtheorem{corollary}[theorem]{Corollary}
\newtheorem{definition}[theorem]{Definition}
\newtheorem{fact}{Fact}
\newtheorem{observation}{Observation}
\newtheorem{remark}{Remark}
\newtheorem{problem}{Problem}

\usepackage{graphicx,amsmath, amssymb,color,url,booktabs,comment}  %cite
\urlstyle{sf}
%\usepackage[margin=1in]{geometry}
% \usepackage{fancyhdr}
%\usepackage[colorlinks]{hyperref} %pagebackref

\newcommand{\nc}{\newcommand}
\nc{\rnc}{\renewcommand}

\def\View{\mathsf{View}}

\def\GS{\mathsf{Ham}}
\nc{\cVGS}{\ensuremath{\cV_\GS}}

\def\Samp{\mathsf{Samp}}
\nc{\PiSamp}{\ensuremath{\Pi_\Samp}}
\nc{\VSamp}{\ensuremath{V_\Samp}}
\nc{\PSamp}{\ensuremath{P_\Samp}}
\nc{\PSampstar}{\ensuremath{P_\Samp^*}}
\nc{\cVSamp}[1]{\ensuremath{\cV_{\Samp,#1}}}
\nc{\cPSamp}[1]{\ensuremath{\cP_{\Samp,#1}}}

\def\HE{\mathsf{HE}}
\def\HGen{\mathsf{HE.Keygen}}
\def\HEnc{\mathsf{HE.Enc}}
\def\HEval{\mathsf{HE.Eval}}
\def\HDec{\mathsf{HE.Dec}}
\def\Rej{\mathsf{Rej}}

\def\blind{\mathsf{blind}}
\nc{\Piblind}{\ensuremath{\Pi_\blind}}
\nc{\Vblind}{\ensuremath{V_\blind}}
\nc{\Pblind}{\ensuremath{P_\blind}}
\nc{\Pblindstar}{\ensuremath{P_\blind^*}}
\nc{\cVblind}[1]{\ensuremath{\cV_{\blind,#1}}}
\nc{\cPblind}[1]{\ensuremath{\cP_{\blind,#1}}}
\def\Pstar{P^*}
\nc{\cPstar}[1]{\ensuremath{\cP^*_{#1}}}
\nc{\ctx}[3]{\ensuremath{{{\widehat{#1}}_{#2}^{(#3)}}}}

%
%\newcommand{\bra}[1]{\langle #1|}
%\newcommand{\ket}[1]{|#1\rangle}
\newcommand{\proj}[1]{|#1\rangle\langle #1|}
% \newcommand{\braket}[2]{\langle #1|#2\rangle}
% \newcommand{\Bra}[1]{\left\langle #1\right|}
% \newcommand{\Ket}[1]{\left|#1\right\rangle}
\newcommand{\Proj}[1]{\left|#1\right\rangle\left\langle #1\right|}
% \newcommand{\Braket}[2]{\left\langle #1\middle|#2\right\rangle}
\nc{\vev}[1]{\langle#1\rangle}
\nc{\grad}{{\vec{\nabla}}}
\nc{\abs}[1]{\lvert#1\rvert}
%\DeclareMathOperator{\abs}{abs}
\DeclareMathOperator{\Bin}{Bin}
\DeclareMathOperator{\conv}{conv}
\DeclareMathOperator{\eig}{eig}
\DeclareMathOperator{\Hist}{Hist}
\DeclareMathOperator{\Hyb}{Hyb}
\DeclareMathOperator{\id}{id}
\DeclareMathOperator{\Img}{Im}
\DeclareMathOperator{\Par}{Par}
% \DeclareMathOperator{\poly}{poly}
\DeclareMathOperator{\negl}{negl}
\DeclareMathOperator{\polylog}{polylog}
\DeclareMathOperator{\tr}{tr}
\DeclareMathOperator{\rank}{rank}
% \DeclareMathOperator{\sgn}{sgn}
\DeclareMathOperator{\Sep}{Sep}
\DeclareMathOperator{\SepSym}{SepSym}
\DeclareMathOperator{\Span}{span}
\DeclareMathOperator{\supp}{supp}
\DeclareMathOperator{\swap}{SWAP}
\DeclareMathOperator{\Sym}{Sym}
\DeclareMathOperator{\ProdSym}{ProdSym}
\DeclareMathOperator{\SEP}{SEP}
\DeclareMathOperator{\PPT}{PPT}
\DeclareMathOperator{\Wg}{Wg}
\DeclareMathOperator{\WMEM}{WMEM}
\DeclareMathOperator{\WOPT}{WOPT}

\DeclareMathOperator{\BPP}{\mathsf{BPP}}
\DeclareMathOperator{\QPIP}{\mathsf{QPIP}}
\DeclareMathOperator{\SampBQP}{\mathsf{SampBQP}}
\DeclareMathOperator{\BQP}{\mathsf{BQP}}
\DeclareMathOperator{\FBQP}{\mathsf{FBQP}}
\DeclareMathOperator{\cnot}{\normalfont\textsc{cnot}}
\DeclareMathOperator{\DTIME}{\mathsf{DTIME}}
\DeclareMathOperator{\NTIME}{\mathsf{NTIME}}
\DeclareMathOperator{\MA}{\mathsf{MA}}
\DeclareMathOperator{\NP}{\mathsf{NP}}
\DeclareMathOperator{\NEXP}{\mathsf{NEXP}}
\DeclareMathOperator{\Ptime}{\mathsf{P}}
\DeclareMathOperator{\QMA}{\mathsf{QMA}}
\DeclareMathOperator{\QCMA}{\mathsf{QCMA}}
\DeclareMathOperator{\BellQMA}{\mathsf{BellQMA}}

\newcommand{\be}{\begin{equation}}
\newcommand{\ee}{\end{equation}}
\newcommand{\bea}{\begin{eqnarray}}
\newcommand{\eea}{\end{eqnarray}}
\newcommand{\nn}{\nonumber}
\newcommand{\bi}{\begin{itemize}}
\newcommand{\ei}{\end{itemize}}
\newcommand{\bn}{\begin{enumerate}}
\newcommand{\en}{\end{enumerate}}
\def\beas#1\eeas{\begin{eqnarray*}#1\end{eqnarray*}}
\def\ba#1\ea{\begin{align}#1\end{align}}
\nc{\bas}{\[\begin{aligned}}
\nc{\eas}{\end{aligned}\]}
\nc{\bpm}{\begin{pmatrix}}
\nc{\epm}{\end{pmatrix}}
\def\non{\nonumber}
\def\nn{\nonumber}
\def\eq#1{(\ref{eq:#1})}
\def\eqs#1#2{(\ref{eq:#1}) and (\ref{eq:#2})}
%\def\eq#1{Eq.~(\ref{eq:#1})}
%\def\eqs#1#2{Eqs.~(\ref{eq:#1}) and (\ref{eq:#2})}
\def\L{\left} 
\def\R{\right}
\def\ra{\rightarrow}
\def\ot{\otimes}
\nc{\given}{\ensuremath{\;\middle|\;}}

\newtheorem{thm}{Theorem}[section]
\newtheorem{theorem}{Theorem}[section]
%\newtheorem*{thm*}{Theorem}
%\newtheorem{claim}[thm]{Claim}
\newtheorem{cor}{Corollary}[thm]
\newtheorem{lem}{Lemma}[section]
\newtheorem{lemma}{Lemma}[section]
\newtheorem{rmk}{Remark}[thm]
%\newtheorem{prop}[thm]{Proposition}
\newtheorem{dfn}{Definition}[section]
\newtheorem{definition}{Definition}[section]
%\newtheorem{con}[thm]{Conjecture}

\newenvironment{prf}{\begin{proof}}{\end{proof}}

\def\eps{\epsilon}
\def\va{{\vec{a}}}
\def\vb{{\vec{b}}}
\def\vn{{\vec{n}}}
\def\cvs{{\cdot\vec{\sigma}}}
\def\vx{{\vec{x}}}
\def\Usch{U_{\text{Sch}}}

\def\cA{\mathcal{A}}
\def\cB{\mathcal{B}}
\def\cD{\mathcal{D}}
\def\cE{\mathcal{E}}
\def\cF{\mathcal{F}}
\def\cH{\mathcal{H}}
\def\cI{{\cal I}}
\def\cL{{\cal L}}
\def\cM{{\cal M}}
\def\cN{\mathcal{N}}
\def\cO{{\cal O}}
\def\cP{\mathcal{P}}
\def\cQ{\mathcal{Q}}
\def\cS{\mathcal{S}}
\def\cT{{\cal T}}
\def\cU{\mathcal{U}}
\def\cV{\mathcal{V}}
\def\cW{{\cal W}}
\def\cX{{\cal X}}
\def\cY{{\cal Y}}

\def\bp{\mathbf{p}}
\def\bq{\mathbf{q}}
\def\bP{{\bf P}}
\def\bQ{{\bf Q}}
\def\gl{\mathfrak{gl}}

\def\bbC{\mathbb{C}}
% \DeclareMathOperator*{\E}{\mathbb{E}}
\DeclareMathOperator*{\bbE}{\mathbb{E}}
%\DeclareMathOperator*{\Pr}{Pr}
\nc{\Prob}[1]{\ensuremath{\Pr\left[#1\right]}}
\def\bbM{\mathbb{M}}
\def\bbN{\mathbb{N}}
\def\bbR{\mathbb{R}}
\def\bbZ{\mathbb{Z}}
\def\bbP{\mathbb{P}}
\def\bbV{\mathbb{V}}
\newcommand{\Real}{\textrm{Re}}

\def\benum{\begin{enumerate}}
\def\eenum{\end{enumerate}}
% \def\bit{\begin{itemize}}
% \def\eit{\end{itemize}}
\def\bdesc{\begin{description}}
\def\edesc{\end{description}}
\newcommand{\fig}[1]{Fig.~\ref{fig:#1}}
\newcommand{\tab}[1]{Table~\ref{tab:#1}}
\newcommand{\secref}[1]{Section~\ref{sec:#1}}
\newcommand{\appref}[1]{Appendix~\ref{sec:#1}}
\newcommand{\lemref}[1]{Lemma~\ref{lem:#1}}
\newcommand{\thmref}[1]{Theorem~\ref{thm:#1}}
\newcommand{\propref}[1]{Proposition~\ref{prop:#1}}
\newcommand{\protoref}[1]{Protocol~\ref{proto:#1}}
\nc{\myprotoref}[1]{\hyperref[#1]{Protocol~\ref*{#1}}}
\newcommand{\defref}[1]{Definition~\ref{def:#1}}
\newcommand{\corref}[1]{Corollary~\ref{cor:#1}}
\newcommand{\conref}[1]{Conjecture~\ref{con:#1}}

\newcommand{\FIXME}[1]{{\color{red}FIXME: #1}}
\nc{\todo}[1]{\textcolor{red}{todo: #1}}



\newcommand{\boxdfn}[2]{
\begin{figure}[h]
\begin{center}
\noindent \framebox{
\begin{minipage}{0.8\textwidth}
\begin{dfn}[{\bf #1}]
\ \\ \\
#2
\end{dfn}
\end{minipage}
}
\end{center}
\end{figure}
}

\newcommand{\boxproto}[2]{
\begin{figure}[h]
\begin{center}
\noindent \framebox{
\begin{minipage}{0.8\textwidth}
\begin{proto}[{\bf #1}]
\ \\ \\
#2
\end{proto}
\end{minipage}
}
\end{center}
\end{figure}
}

\def\begsub#1#2\endsub{\begin{subequations}\label{eq:#1}\begin{align}#2\end{align}\end{subequations}}
\nc\qand{\qquad\text{and}\qquad}
\nc\mnb[1]{\medskip\noindent{\bf #1}}
\nc\mn{\medskip\noindent}

\renewcommand{\arraystretch}{1.5}
%\nc{\problem}[1]{\item\noindent {\bf #1}}

\setlength{\tabcolsep}{10pt}

%%%%%% Han-Hsuan's commands %%%%%%%%
\nc{\nl}{\nn \\ &=}  %new line
\nc{\nnl}{\nn \\ &}  %new new line
\nc{\fot}{\frac{1}{2}} %frac one two
\nc{\oo}[1]{\frac{1}{#1}} % one over
\newcommand{\ben}{\begin{enumerate}}
\newcommand{\een}{\end{enumerate}}
\nc{\mc}{\mathcal}
\nc{\beq}{\begin{equation}}
\nc{\eeq}{\end{equation}}
% \nc{\norm}[1]{\L\| #1 \R\|}

\nc{\onenorm}[1]{\L\| #1 \R\|_1} %one norm
%\nc{\span}{\ensuremath{\mathrm{span}}}

\DeclareMathOperator*{\argmax}{arg\,max}

%\nc{1}

\newcommand{\hannote}[1]{\textcolor{blue}{\small {\textbf{(Han:} #1\textbf{) }}}}

\newcommand{\Knote}[1]{\textcolor{red}{\small {\textbf{(KM:} #1\textbf{) }}}}

\nc{\Ra}{\Rightarrow}
\nc{\zo}{\{0,1\}}	

%%%%import..
% \newcommand{\secpar}{n}


% %%%Efficient Verifier%
% \newcommand{\setupeff}{\setup_{\mathsf{eff}}}
% \newcommand{\vereff}{V_{\mathsf{eff}}}
% \newcommand{\vereffone}{V_{\mathsf{eff},1}}
% \newcommand{\vereffthree}{V_{\mathsf{eff},3}}
% \newcommand{\vereffout}{V_{\mathsf{eff},\mathsf{out}}}
% \newcommand{\proeff}{P_{\mathsf{eff}}}
% \newcommand{\proefftwo}{P_{\mathsf{eff},2}}
% \newcommand{\proefffour}{P_{\mathsf{eff},4}}
% \newcommand{\advPH}{{P^*}^{H}}
% \newcommand{\setup}{\mathsf{Setup}}
% \newcommand{\re}{\mathsf{re}}
% \newcommand{\crh}{\mathsf{CRH}}
% \newcommand{\transcript}{\mathsf{trans}}

% \newcommand{\setupefffs}{\setup_{\mathsf{eff}\text{-}\mathsf{fs}}}
% \newcommand{\proefffs}{P_{\mathsf{eff}\text{-}\mathsf{fs}}}
% \newcommand{\proefffstwo}{P_{\mathsf{eff}\text{-}\mathsf{fs},2}}
% \newcommand{\verefffs}{V_{\mathsf{eff}\text{-}\mathsf{fs}}}
% \newcommand{\verefffsone}{V_{\mathsf{eff}\text{-}\mathsf{fs},1}}
% \newcommand{\verefffsout}{V_{\mathsf{eff}\text{-}\mathsf{fs},\out}}
% %Games%
% \newcommand{\game}{\mathsf{Game}}

% %\newcommand*{\bra}[1]{\langle#1|}
% %\newcommand*{\ket}[1]{|#1\rangle}
% \newcommand*{\opro}[2]{|#1\rangle\langle#2|}
% \newcommand*{\ipro}[2]{\langle #1|#2\rangle}
% \newcommand{\TD}{\mathsf{TD}}

% %%%%% Registers %%%%%%%
% \newcommand*{\regK}{\mathbf{K}}
% \newcommand*{\regI}{\mathbf{I}}
% \newcommand*{\regR}{\mathbf{R}}
% \newcommand*{\regX}{\mathbf{X}}
% \newcommand*{\regY}{\mathbf{Y}}
% \newcommand*{\regZ}{\mathbf{Z}}
% \newcommand{\regW}{\mathbf{W}}
% \newcommand*{\regC}{\mathbf{C}}
% \newcommand*{\regO}{\mathbf{O}}
% \newcommand*{\regF}{\mathbf{F}}


\title{Classical Verification of Quantum Computations with \\ Efficient Verifier}
\begin{document}

\author[1]{Nai-Hui Chia}
\author[2]{Kai-Min Chung}
\author[3]{Takashi Yamakawa\thanks{This work was done in part while the author was visiting Academia Sinica.}}
\affil[1]{Department of Computer Science, University of Texas at Austin}
\affil[2]{Institute of Information Science, Academia Sinica}
\affil[3]{NTT Secure Platform Laboratories}
%\author{Nai-Hui Chia \and
%Kai-Min Chung \and
%Takashi Yamakawa}
%\date{November 2019}



\maketitle
\begin{abstract}
In this paper, we extend the protocol of classical verification of quantum computations (CVQC) recently proposed by Mahadev to make the verification efficient.
Our result is obtained in the following three steps:
\begin{itemize}
    \item We show that parallel repetition of Mahadev's protocol has negligible soundness error. This gives the first constant round CVQC protocol with negligible soundness error. In this part, we only assume the quantum hardness of the learning with error (LWE) problem similarly to the Mahadev's work.
    \item We construct a two-round CVQC protocol in the quantum random oracle model (QROM) where a cryptographic hash function is idealized to be a random function.
    This is obtained by applying the Fiat-Shamir transform to the parallel repetition version of the Mahadev's protocol.
    \item We construct a two-round CVQC protocol with efficient verifier in the CRS+QRO model where both prover and verifier can access to a (classical) common reference string generated by a trusted third party in addition to quantum access to QRO.
    Specifically, the verifier can verify a $\QTIME(T)$ computation in time $\poly(\secpar,\log T)$ where $\secpar$ is the security parameter.
    For proving soundness, we assume that a standard model instantiation of our two-round protocol with a concrete hash function (say, SHA-3) is sound and the existence of post-quantum indistinguishability obfuscation and post-quantum fully homomorphic encryption in addition to the quantum hardness of the LWE problem. 
\end{itemize}
\end{abstract}
\section{Introduction}
Can we verify quantum computations by a classical computer? This problem has been a major open problem in the field until Mahadev~\cite{FOCS:Mahadev18a} finally gave an affirmative solution.
Specifically, she constructed an interactive protocol between an efficient classical verifier (a BPP machine) and an efficient quantum prover (a BQP machine) where the verifier can verify the result of the BQP computation.
(In the following, we call such a protocol a CVQC protocol.\footnote{``CVQC" stands for ``Classical Verification of Quantum Computations"}) 
Soundness of her protocol relies on a computational assumption that the learning with error (LWE) problem \cite{JACM:Regev09} is hard for an efficient quantum algorithm, which has been widely used in the field of cryptography. We refer to the extensive survey by Peikert \cite{FTTCS:Peikert16} for details about LWE and its cryptographic applications.

Though her result is a significant breakthrough, there are still several drawbacks. First, her protocol has soundness error $3/4$, which means that a cheating prover may convince the verifier even if it does not correctly computes the BQP computation with probability at most $3/4$. Though we can exponentially reduce the soundness error by sequential repetition, we need super-constant rounds to reduce the soundness error to be negligible.
If parallel repetition works to reduce the soundness error, then we need not increase the number of round.
However, parallel repetition may not reduce soundness error for computationally sound protocol in general \cite{FOCS:BelImpNao97,TCC:PieWik07}.
Thus, it is still open to construct constant round protocol with negligible soundness error.

Another issue is about verifier's efficiency. In her protocol, for verifying a computation that is done by a quantum computer in time $T$, the verifier's running time is as large as $\poly(T)$.
Considering a situation where a device with weak classical computational power outsources computations to untrusted quantum server, we may want to make the verifier's running time as small as possible.
Such a problem has been studied well in the setting where the prover is classical and we know solutions where verifier's running time only logarithmically depends on $T$~\cite{STOC:Kilian92,SIAM:Micali00,STOC:KalRazRot13,STOC:KalRazRot14,JACM:GolKalRot15,STOC:ReiRotRot16,STOC:BraHolKal17,STOC:BKKSW18,FOCS:HolRot18,STOC:CCHLRRW19,STOC:KalPanYan19}.
%\takashi{We may need not cite such a lot. I just cited all works I know.}
Hopefully, we want to obtain a CVQC protocol where the verifier runs in logarithmic time.  
%However, it is non-trivial to combine the Mahadev's protocol of verification of quantum computations and a verifiable delegation protocol in the classical setting.
%Thus, 

\subsection{Our Results}
In this paper, we solve the above drawbacks of the Mahadev's protocol. 
Our contribution is divided into three parts:
\begin{itemize}
    \item We show that parallel repetition version of Mahadev's protocol has negligible soundness error. This gives the first constant round CVQC protocol with negligible soundness error.
    \item We construct a two-round CVQC protocol in the quantum random oracle model (QROM) \cite{AC:BDFLSZ11} where a cryptographic hash function is idealized to be a random function that is only accessible as a quantum oracle.
    This is obtained by applying the Fiat-Shamir transform \cite{C:FiaSha86,C:LiuZha19,C:DFMS19} to the parallel repetition version of the Mahadev's protocol.
    \item We construct a two-round CVQC protocol with logarithmic-time verifier in the CRS+QRO model where both prover and verifier can access to a (classical) common reference string generated by a trusted third party in addition to quantum access to QRO.
    For proving soundness, we assume that a standard model instantiation of our two-round protocol with a concrete hash function (say, SHA-3) is sound and the existence of post-quantum indistinguishability obfuscation \cite{JACM:BGIRSVY12,SIAM:GGH0SW16} and (post-quantum) fully homomorphic encryption (FHE) \cite{STOC:Gentry09} in addition to the quantum hardness of the LWE problem. 
\end{itemize}

\subsection{Related Works}
\paragraph{Verification of Quantum Computation.}
There are long line of researches on verification of quantum computation.
Except for solutions relying on computational assumptions, there are two type of settings where verification of quantum computation is known to be possible.
In the first setting, instead of considering purely classical verifier, we assume that a verifier can perform a certain kind of weak quantum computations \cite{FOCS:BroFitKas09,PR:FitKas17,arXiv:ABOEM17,PR:MorFit18}.
In the second setting, we assume that a prover is splitted into two remote servers that share entanglement but do not communicate \cite{Nat:RUV13}.
Though these works do not give a CVQC protocol in our sense, the advantage is that we need not assume any computational assumption for the proof of soundness, and thus they are incomparable to Mahadev's result and ours.

Subsequent to Mahadev's breakthrough result, Gheorghiu and Vidick \cite{FOCS:GheVid19} gave a CVQC protocol that also satisfies blindness, which ensures that a prover cannot learn what computation is delegated.
We note that their protocol requires polynomial number of rounds.

\paragraph{Concurrent Work.}
In a concurrent and independent work, Alagic et al. \cite{arXiv:AlaChiHun19} also shows similar results to our first and second results, parallel repetition theorem for the Madadev's protocol and a two-round CVQC protocol by the Fiat-Shamir transform.
We note that our third result, a two-round CVQC protocol with efficient verification, is unique in this paper.   









\section{Preliminaries}

\subsection{Notations}

Let $\mathcal{B}$ be the Hilbert space corresponding to one qubit. Let $H:\mathcal{B}^{\otimes n}\rightarrow\mathcal{B}^{\otimes n}$ be Hermitian matrices. We use $H\geq0$ to denote $H$ being positive semidefinite. Let $\lambda(H)$ be the smallest eigenvalue of $H$. The ground states of $H$ are the eigenvectors corresponding to $\lambda(H)$. For matrix $H$ and subspace $S$, let $H\big|_S=\Pi_S H \Pi_S$, where $\Pi_S$ is the projector onto the subspace $S$. For a $T$-qubit Hilbert space, let the state $\ket{\widehat{t}}=\ket{1}^{\otimes t}\otimes \ket{0}^{{\otimes (T-t)}}$.
We write $F(\rho_1, \rho_2)=\left(\tr\sqrt{\sqrt{\rho_1}\rho_2\sqrt{\rho_1}}\right)^2$ for the fidelity between $\rho_1$ and $\rho_2$.
We write $\frac{1}{2}\norm{\rho_1-\rho_2}_1$ for the trace distance between $\rho_1$ and $\rho_2$. For all $n$-qubit states $\rho_1, \rho_2\in\cB^{\otimes n}$ we have $\frac{1}{2}\norm{\rho_1-\rho_2}_1\leq\sqrt{1-F(\rho_1, \rho_2)}$.

\begin{definition} [quantum-classical channels]
	\label{def:QCChannel}
	A quantum measurement is given by a set of matrices $\set{M_k}$ such that $M_k\geq0$ and $\sum_k M_k=\id$.
	We associate to any measurement a map $\Lambda(\rho)=\sum_k \tr(M_k\rho)\ket{k}\bra{k}$
	with $\set{\ket{k}}$ an orthonormal basis.
	This map is also called a \emph{quantum-classical channel}.
\end{definition}

The phase gate and Pauli matrices are denoted as follows.

\begin{definition}
	$P(i)=\begin{pmatrix}1&0\\0&i\end{pmatrix}$, $X=\begin{pmatrix}0&1\\1&0\end{pmatrix}$,
	$Y=\begin{pmatrix}0&-i\\i&0\end{pmatrix}$,
	$Z=\begin{pmatrix}1&0\\0&-1\end{pmatrix}$
\end{definition}

\subsection{Relevant complexity classes}

We define a few relevant complexity classes.

\begin{definition} [$\BQP$]
	Definition from Kitaev:
	A \emph{quantum algorithm} for the computation of a function $F:\zo^*\rightarrow\zo^*$ is a classical algorithm (i.e., a Turing machine) that computes a function of the form $x\mapstochar\rightarrow Z(x)$, where $Z(x)$ is a description of a quantum circuit which computes $F(x)$ on empty input. The function $F$ is said to belong to class $\BQP$ if there is a quantum algorithm that computes $F$ in time $\poly(n)$.

	Definition from Complexity Zoo:
	$\BQP$ is the class of languages $L$ for which for all $n\in\bbN$ there exists a quantum circuit constructible in time $\poly(n)$ that, given any $x\in\set{0, 1}^n$ as input, correctly decides whether $x\in L$ at least $\frac{2}{3}$ of the time.
	\Ethan{Just copy this from somewhere... Does anyone even define this?}
\end{definition}

\begin{definition} [$\FBQP$]
	A function $f:\set{0,1}^*\rightarrow\set{0,1}^*$ is in $\FBQP$ if there is a $\BQP$ machine that, $\forall x$, outputs $f(x)$ with overwhelming probability.
	\Ethan{Need to be more formal. Also should be efficient verifiable}
\end{definition}

We define search and sampling versions of $\BQP$ based on \cite{aaronson_2013}.

\begin{definition} [search problem]
	A search problem $R$ is a collection of nonempty sets $(A_x)_{x\in\set{0, 1}^*}$, one for each input string $x\in\set{0, 1}^*$, where $A_x$... \Ethan{Great, interface doesn't line up correctly}
\end{definition}

\begin{definition} [sampling problem]
	A sampling problem $S$ is a collection of probability distributions $(D_x)_{x\in\set{0, 1}^*}$, one for each input string $x\in\set{0,1}^n$, where $D_x$ is a distribution over $\set{0,1}^{p(n)}$ for some fixed polynomial $p$.
\end{definition}

\begin{definition} [$\SampBQP$]
	$\SampBQP$ is the class of sampling problems $S=\left(D_x\right)_{x\in\set{0, 1}^*}$ for which there exists a polynomial-time quantum algorithm $B$ that, given $(x, 0^{1/\varepsilon})$ as input, samples from a probability distribution $C_x$ such that $\norm{C_x-D_x}\leq\varepsilon$.
\end{definition}

\subsection{Quantum Prover Interactive Protocol (QPIP)}
We classify the interaction between a (almost classical) client and a quantum server for sampling problems, extending the classification by \cite{FOCS:Mahadev18a}.

\begin{definition}
	We say $\Pi=(P, V)(x)$ is a protocol for the sampling problem $(D_x)_{x\in\zo^*}$ with completeness error $c$ and soundness error $s$ \Ethan{Might need these to be functions of $\abs{x}$} if
	\Ethan{Look up ``interactive protocols for BQP". Right now it's missing quantifier for all x. Also need to mention d is decision bit; maybe do that in next definition and swap locations}
	\begin{itemize}
		\item Let $(d, z)\leftarrow(P, V)(x)$. Then $d=rej$ with probability at most $c$.
		\item For all cheating prover $P^*$, let $(d, z)\leftarrow(P^*, V)(x)$. Let \Ethan{Use display math to make it obvious I'm defining this} $z_{ideal}\leftarrow D_x$ if $d=acc$, else $z_{ideal}=\bot$. Then $\norm{(d, z) - (d, z_{ideal})} \leq s$ \Ethan{make stat. distance notation consistent}.
	\end{itemize}
\end{definition}

\Ethan{Acc, rej, P, V fonts}

\Ethan{Might need to write our own def. here}

\Ethan{Look at thesis for this}

\begin{definition}
	A sampling problem $S=(D_x)_{x\in\set{0, 1}^*}$ is said to be \Ethan{Try to make this more general; remove mentions of sampling/decision problems} in $\QPIP_\tau$ with completeness $c$ and soundness $s$ \Ethan{Don't tie this with completeness and soundness yet} if there exists a protocol $(\bbP, \bbV)(x)$ for $S$ with the following properties:
	\begin{itemize}
		\item $\bbP$ is run by the prover, a $\BQP$ machine, which also has access to a quantum channel that can transmit $\tau$ qubits to the verifier per use.
		\item $\bbV$ is run by the verifier, which is a hybrid machine of a classical part and a limited quantum part. The classical part is a $\BPP$ machine. The quantum part is a register of $\tau$ qubits, on which the verifier can perform arbitrary quantum operations and which has access to a quantum channel which can transmit $\tau$ qubits. At any given time, the verifier is not allowed to possess more than $\tau$ qubits. The interaction between the quantum and classical parts of the verifier is the usual one: the classical part controls which operations are to be performed on the quantum register, and outcomes of measurements of the quantum register can be used as input to the classical part.
		\item There is also a classical communication channel between the prover and the verifier, which can transmit $\poly(\abs{x})$ many bits to either direction. 
	\end{itemize}
\end{definition}

\Ethan{Two separate definitions for comp and soundness}

\Ethan{Soundness?}

\subsection{Semantic security for interactive protocols}
\Ethan{Just call this blindness and put this under interactive protocols}

\Ethan{See Thomas' paper if he defined this}

We present the security definition for interactive protocols:

\begin{definition}
	Let $\lambda$ be a security parameter.
	Let $(\bbP, \bbV)$ be an interactive protocol with security parameter $\lambda$.
	Then it is IND-CPA secure if $\forall x\in\set{0,1}^n$ no polynomial time adversary $\cA$ can win \protoref{indcpa} with probability better than $\frac{1}{2}+\negl(\lambda)$
\end{definition}

\begin{protocol}{Attack against semantic security}
	\label{proto:indcpa}
	\begin{enumerate}
		\item The challenge picks $b\in\set{0,1}$ at random
		\item If $b=0$, the challenger runs the protocol with the adversary, acting as the verifier with input $0^n$
		\item Otherwise, the challenger runs the protocol with the adversary, acting as the verifier with input $x$
		\item $\cA$ attempts to guess $b$
	\end{enumerate}
\end{protocol}

\subsection{Chernoff bound}

Taken from \href{http://math.mit.edu/~goemans/18310S15/chernoff-notes.pdf}{here}.

\begin{thm}
\label{thm:Chernoff}
Let $X=\sum_{i=1}^n X_i$ where $X_i$ are i.i.d. Bernoulli trials, and $\mu=\E[X]$.
Then for all $0<\delta<1$,
$$P[\abs{X-\mu}\geq\delta\mu]\leq2e^{-\frac{\mu\delta^2}{3}}$$
\end{thm}

\subsection{Projection Lemma}

We use the projection lemma from \cite{kempe_kitaev_regev_2006}, which describes the conditions under which we can estimate the ground state energy of $H_1 + H_2$ with that of $H_1\big|_{\ker H_2}$.

\begin{thm}
	Let $H=H_1+H_2$ be the sum of two Hamiltonians operating on some Hilbert space $\cH=\cS+\cS^\bot$.
	The Hamiltonian $H_2$ is such that $\cS$ is a zero eigenspace and the eigenvectors in $\cS^\bot$ have eigenvalues at least $J>2\norm{H_1}$. Then,
	$$\lambda\left(H_1\big|_\cS\right)-\frac{\norm{H_1}^2}{J-2\norm{H_1}^2}\leq\lambda(H)\leq\lambda\left(H_1\big|_\cS\right)$$
\end{thm}

We will instead use the following formulation, which can be obtained by relabeling variables from above.

\begin{thm}
	\label{thm:projection}
	Let $H_1, H_2$ be local Hamiltonians where $H_2\geq0$. Let $K=\ker H_2$ and
	$$J=\frac{10\norm{H_1}^2}{\lambda\left(H_2\big|_{K^\bot}\right)}$$
	then we have
	$$\lambda(H_1+JH_2)\geq\lambda\left(H_1\big|_K\right)-\frac{1}{8}$$
\end{thm}

\subsection{Quantum de Finetti Theorem under Local Measurements}

De Finetti theorem provides a way to obtain close to independent samples by taking random subsystems of a quantum system.
There are many formulations; we use the one from \cite{Brandão2017} because we need to avoid exponential dependence on number of qubits in each subsystem.
\begin{thm}
	\label{deFinetti}
	Let $\rho^{A_1\ldots A_k}$ be a permutation-invariant state on registers $A_1,\ldots,A_k$ where each register is $s$ qubits,
	then for every $0\leq l\leq k$ there exists states $\set{\rho_i}$ and $\set{p_i}\subset\bbR$ such that
	$$\max_{\Lambda_1,\ldots,\Lambda_l}
	\norm{(\Lambda_1\otimes\ldots\otimes\Lambda_l)\left(\rho^{A_1\ldots A_l}-\sum_ip_i\rho_i^{A_1}\otimes\ldots\otimes\rho_i^{A_l}\right)}_1
	\leq\sqrt{\frac{2l^2s}{k-l}}$$
	where $\Lambda_i$ are quantum-classical channels.
\end{thm}

\subsection{Quantum Homomorphic Encryption Schemes}

\def\QHE{\mathsf{QHE}}
\def\QGen{\mathsf{QHE.Keygen}}
\def\QEnc{\mathsf{QHE.Enc}}
\def\QEval{\mathsf{QHE.Eval}}
\def\QDec{\mathsf{QHE.Dec}}

We use the quantum fully homomorphic encryption scheme given in \cite{mahadev_qfhe} which is compatible with our use of a classical client. We start by presenting the interface of a homomorphic encryption scheme:
\begin{definition}
	A leveled homomophic encryption scheme is tuple of algorithms \linebreak $\mathsf{HE}=(\mathsf{HE.Keygen}, \mathsf{HE.Enc}, \mathsf{HE.Dec}, \mathsf{HE.Eval})$ with the following descriptions:
	\begin{itemize}
		\item $\mathsf{HE.Keygen}(1^\lambda, 1^L)\rightarrow(pk, sk)$
		\item $\mathsf{HE.Enc}_{pk}(\mu)\rightarrow c$
		\item $\mathsf{HE.Dec}_{sk}(c)\rightarrow \mu^*$
		\item $\mathsf{HE.Eval}_{pk}(f, c_1, \ldots, c_l)\rightarrow c_f$
	\end{itemize}
\end{definition}

$\mathsf{HE}$ also satisfies, with overwhelming probability in $\lambda$, that
$$\mathsf{HE.Dec}_{sk}(\mathsf{HE.Eval}_{pk}(f, c_1, \ldots, c_l)=f(\mathsf{HE.Dec}_{sk}(c_0),\ldots,\mathsf{HE.Dec}_{sk}(c_l))$$
where $f$ is specified by a circuit of depth at most $L$.

\Ethan{To be pedantic, the above doesn't imply Dec undoes Enc even if we sub in $f=\id$.}

We also recall the security definition for a FHE scheme.

\begin{definition}
	A FHE scheme $\mathsf{HE}$ is IND-CPA secure if, for any polynomial time adversary $\cA$, there exists a negligible function $\mu(\cdot)$ such that
	$$\abs{Pr[\cA(pk, \mathsf{HE.Enc}_{pk}(0))=1]-Pr[\cA(pk, \mathsf{HE.Enc}_{pk}(1))=1]}=\mu(\lambda)$$
	where $(pk, sk)\leftarrow\mathsf{QHE.Keygen}(1^\lambda)$
\end{definition}

The quantum homomorphic encryption scheme $\mathsf{QHE}$ from \cite{mahadev_qfhe} has additional properties that facilitates the use of classical clients:
\begin{itemize}
	\item $\QGen$ can be done classically.
	\item In the case where the plaintext is classical, $\QEnc$ can be done classically.
	\item Its ciphertext takes the form $(X^xZ^z\rho Z^zX^x, c_{x, z})$, where $\rho$ is the plaintext and $c_{x, z}$ is a ciphertext that decodes to $(x, z)$ under a certain classical homomorphic encryption scheme.
\end{itemize}

%\section{Parallel Repetition of Mahadev's Protocol}

\subsection{Overview of Mahadev's Protocol}
Here, we recall the Mahadev's protocol \cite{FOCS:Mahadev18a}. We only give a high-level description of the protocol and properties of it and omit the details since they are not needed to show our result. 

The protocol is run between a quantum prover $\pro$ and a classical verifier $\ver$ on a common input $x$. The aim of the protocol is to enable a verifier to classically verify $x\in \lang$ for a BQP language $\lang$ with the help of interactions with a quantum prover.
The protocol is a 4-round protocol where the first message is sent from $\ver$ to $\pro$. 
We denote the $i$-th message generation algorithm by $\ver_i$ for $i\in\{1,3\}$ or $\pro_i$ for $i\in \{2,4\}$ and denote the verifier's final decision algorithm by $\ver_\out$.
Then a high-level description of the protocol is given below.
\begin{description}
\item[$\ver_1$:] On input the security parameter $1^\secpar$ and $x$, it generates a pair $(\key,\td)$ of a``key" and ``trapdoor", sends $\key$ to $\pro$, and keeps $\td$ as its internal state.
\item[$\pro_2$:] On input $x$ and $\key$, it generates a classical ``commitment" $\comy$ along with a quantum state $\ket{\st_\pro}$, sends $\comy$ to $\pro$, and keeps $\ket{\st_\pro}$ as its internal state.
\item[$\ver_3$:] It randomly picks $c\sample \bit$ and sends $c$ to $\pro$.\footnote{The third message is just a public-coin, and does not depend on the transcript so far or $x$.}
For a knowledgeable reader, the case of $c=0$ corresponds to the ``test round" and the case of $c=1$ corresponds to the ``Hadamard round" in the terminology in \cite{FOCS:Mahadev18a}.
\item[$\pro_4$:] On input $\ket{\st_\pro}$ and $c$, it generates a classical string $\ans$ and sends $\ans$ to $\pro$.
\item[$\ver_\out$:] On input $\key$, $\td$, $y$, $c$, and $\ans$, it returns $\top$ indicating acceptance or $\bot$ indicating rejection.
In case $c=0$, the verification can be done publicly, that is, $\ver_\out$ need not take $\td$ as input.
\end{description}

For the protocol, we have the following properties:\\
\noindent\textbf{Completeness:}
For all $x\in \lang$, we have $\Pr[\langle \pro,\ver \rangle(x)]=\bot]= \negl(\secpar)$.\\
\noindent\textbf{Soundness:}
If the LWE problem is hard for quantum polynomial-time algorithms, then for any $x\notin \lang$ and a quantum polynomial-time cheating prover $\pro^*$, we have  $\Pr[\langle \pro^*,\ver \rangle(x)]=\bot]\leq 3/4$.

We need a slightly different form of soundness implicitly shown in \cite{FOCS:Mahadev18a}, which roughly says that if a cheating prover can pass the ``test round" (i.e., the case of $c=0$) with overwhelming probability, then it can pass the ``Hadamard round" (i.e., the case of $c=1$) only with a negligible probability. 
\begin{lemma}[implicit in \cite{FOCS:Mahadev18a}]\label{lem:Mah_soundness}
If the LWE problem is hard for quantum polynomial-time algorithms, then for any $x\notin \lang$ and a quantum polynomial-time cheating prover $\pro^*$ such that  $\Pr[\langle \pro^*,\ver \rangle(x)]=\bot\mid c=0]=\negl(\secpar)$, we have $\Pr[\langle \pro^*,\ver \rangle(x)]=\top\mid c=1]=\negl(\secpar)$.
\end{lemma}

\subsection{Parallel Repetition}
Here, we prove that the parallel repetition of the Mahadev's protocol decrease the soundness bound to be negligible.
Let $\pro^m$ and $\ver^m$ be $m$-parallel repetitions of the honest prover $\pro$ and verifier $\ver$ in the Mahadev's protocol. Then we have the following:
\begin{theorem}[Completeness]\label{thm:rep_completeness}
For all $x\in \lang$ and $m= \poly(\secpar)$, we have $\Pr[\langle \pro^m,\ver^m \rangle(x)]=\bot]= \negl(\secpar)$.\\
\end{theorem}
\begin{theorem}[Soundness]\label{thm:rep_soundness}
If the LWE problem is hard for quantum polynomial-time algorithms, then for any $x\notin \lang$ and a quantum polynomial-time cheating prover $\pro^*$, we have  $\Pr[\langle \pro^*,\ver^m \rangle(x)]=\top]\leq \negl(\secpar)$ for $m=\poly(\secpar)$.
\end{theorem}

The completeness (Theorem~\ref{thm:rep_completeness}) easily follows from the completeness of the Mahadev's protocol.
In the next subsection, we prove the soundness (Theorem~\ref{thm:rep_soundness}).

\subsection{Proof of Soundness}
\noindent\textbf{Characterization of cheating prover.}
Any cheating prover can be characterized by a tuple $(U_0,\{U_c\}_{c\in \bit^n})$ of unitaries over Hilbert space $\hil_{\regX}\otimes \hil_{\regZ} \otimes \hil_{\regY}  \otimes \hil_{\regK} $. 
A prover characterized by $(U_0,\{U_{\bfc}\}_{\bfc\in \bit^n})$ works as follows.\footnote{Here, we hardwire into the cheating prover the instance $x\notin \lang$ on which it will cheat instead of giving it as an input.}
\begin{description}
\item[Second Message:] Upon receiving $k=(\key_1,...,\key_m)$, it applies $U_0$ to the state $\ket{0}\otimes\ket{0}\otimes\ket{0}\otimes  \ket{k}$, and then measures the $Y$ register to obtain $y=(\comy_1,...,\comy_m)$. Then it sends $\bfy$ to $\ver$ and keeps the resulting state $\ket{\psi_{k,y}}$ over  $\hil_{\regX,\regZ}$.
\item [Forth Message:] Upon receiving $c\in \bit^{m}$, it applies $U_c$ to $\ket{\psi}_{\pro^*}$ and then measures the $\regX$ register in computational basis to obtain $a=(a_1,...,a_m)$. We denote the designated register for $a_i$ by $\regX_i$ 
\end{description}

%Let $\ket{\psi_{k,y}}$ be the prover's state over $\hil_{\regX,\regZ}$ after sending $y=(y_1,...,y_m)$ conditioned on a fixed value of $k$ and $y$.
For $i\in[m]$ and $c_i\in \bit$, let $\Pi_{i,c_i}^{out}$ be the projection to the space spanned by states that 
contains a accepting message in $\regX_i$ w.r.t. the challenge $c_i$.
Then we have
\begin{align*}
\Pr[\langle P^*,V \rangle=accept]=\mathbb{E}_{k,y,c}[\|\Pi_{m,c_m}^{out}...\Pi_{1,c_1}^{out}U_{c}\ket{\psi_{k,y}}\|^{2}]
\end{align*}
We prove this is negligible in the following.

First, we observe that it suffices to show that for any polynomial $p$, there exists $m=O(\log(\secpar))$ such that we have 
\begin{align*}
\mathbb{E}_{k,y,c}[\|\Pi_{m,c_m}^{out}...\Pi_{1,c_1}^{out}U_{c}\ket{\psi_{k,y}}\|^{2}]\leq 1/p(\secpar).
\end{align*}
This is because if we set $m=O(\secpar)$, then we can consider the first $m'=O(\log \secpar)$ coordinates and do the same analysis ignoring the rest of coordinates.
\takashi{Is this correct?}

In the following, we fix a polynomial $p$, and set $m$ and $\epsilon$ so that 
$\epsilon=1/\poly(\secpar)$ and  $(m+1)m2^{m-1}2\epsilon+2^{-m}\leq 1/2p(\secpar)$.



%\begin{lemma}
%Let $U$ be an efficient unitary acting on registers $\regX$, $\regY$, and $\regZ$.
%Let $f:\calX \ra \bit$ be an efficiently computable function and $\Pi_f$ be a projector that projects $\regZ$ register onto the space spanned by $\{\ket{x}\}_{X\in S_f}$
%where $S_f\defeq {x\in \calX \text{~s.t.~}f(x)=1}$ (i.e., we let $\Pi\defeq \sum_{x\in S_f}I\ot I\ot\ket{x}\bra{x})$.
%Then for any $\epsilon=1/\poly(\secpar)$, there exists a decomposition of $\hil_\regX$ into two subspaces $\Sgood$ and $\Sbad$ and an efficient quantum algorithm $\ext$ such that the following is satisfied:
%\begin{enumerate}
%    \item $\ext$ outputs $x\in S_f$ with probability $1$ on any input $\ket{\psigood}\in \Sgood$.    
%    \item For any $\ket{\psibad}\in \Sbad$, we have $\|\Pi_f U\ket{\psibad}\ket{0}\ket{0}\|^2\leq \epsilon$
    %\item If there is an efficient quantum algorithm with input $z$ to generate $\ket{\psi}$ such that $\|\Pi_{\Sgood}\ket{\psi}\|$ is non-negligible where $\Pi_{\Sgood}$ denotes the projection onto $\Sgood$, then there exists an efficient quantum algorithm with input $z$ to generate $\Pi_{\Sgood}\ket{\psi}$ with overwhelming probability. 
%\end{enumerate}
%\end{lemma}

For proving the soundness, we prepare several lemmas.
\begin{lemma}\label{lem:decomp}
Let $\A_i$ be a quantum algorithm that takes a quantum state $\ket{\psi}\in \hil$, picks random $c\sample \bit^m$ such that $c_i=0$, applies a unitary $U_c$ to $\ket{\psi}$ and finally measures $\regX$ to output a classical string $a_i\in \calX$.
%Let $f_i:\calX \ra \bit$ be a function such that $f(x_i)=1$ if and only if $\ver_\out(k_i,y_i,0,x_i)=\top$. 
%Then for any $\epsilon=1/\poly(\secpar)$, 
Then there exist a decomposition of $\hil_{\regX,\regZ}$ into two subspaces $T_{i,0}$ and $T_{i,1}$ and an efficient quantum algorithm $\ext_i$ such that for any fixed $k,y$ and any state $\ket{\psi}$ the followings are satisfied: Let $\Pi_{i,b}^{in}$ be the projection to $T_{i,b}$.
\begin{enumerate}
    \item
    We have $\Pr[\ver_\out(k_i,y_i,0,a_i)=\top: a_i\sample \ext_i(\Pi_{i,0}^{in}\ket{\psi})]=1-\negl(\secpar)$.
    \item We have $\Pr[\ver_\out(k_i,y_i,0,a_i)=\top:a_i\sample \A_i(\Pi_{i,1}^{in}\ket{\psi})]\leq \epsilon$.
    \item Measurement w.r.t. $\{\Pi_{i,0}^{in}, \Pi_{i,1}^{in}\}$ can be done efficiently.
    \takashi{approximately?}
\end{enumerate}
\end{lemma}


\begin{lemma}\label{lem:decomp2}
%For any adversary's second stage strategy $\{U_c\}_{v\in\bit^{m}}$,
%we can define partition $\hil_{\regX,\regZ}$ into $T_{i,0}$ and $T_{i,1}$ for each $i\in[m]$, such that the following properties are satisfied.
For the partitions $(T_{i,0},T_{i,1})_{i\in[m]}$ as defined in Lemma~\ref{lem:decomp}, the following is satisfied:
Assuming LWE, for all $i\in[m]$ and $c\in \bit^{m}$, we have 
    \begin{align*}
     \mathbb{E}_{k,y}\left[\left\|\Pi_{i,c_i}^{out}U_c\Pi_{i,1-c_i}^{in}\Pi_{i-1,c_{i-1}}^{in}...\Pi_{1,c_1}^{in}\ket{\psi_{k,y}}\right\|^2\right]\leq  2^{m-1}\epsilon +\negl(\secpar).
    \end{align*}
%  \item We have 
%    \begin{align*}
%     \mathbb{E}_{k,y}\left[\left\|\Pi_{i,0}^{out}U_c\Pi_{i,1}^{in}\Pi_{i-1,c_{i-1}}^{in}...\Pi_{1,c_1}^{in}\ket{\psi_{k,y}}\right\|\right]\leq  (2^m\epsilon)^{1/2}
%    \end{align*}
    
%  \item Assuming LWE, we have 
%    \begin{align*}
%     \mathbb{E}_{k,y}\left[\left\|\Pi_{i,1}^{out}U_c\Pi_{i,0}^{in}\Pi_{i-1,c_{i-1}}^{in}...\Pi_{1,c_1}^{in}\ket{\psi_{k,y}}\right\|\right]\leq  \negl(\secpar)
%    \end{align*}
\end{lemma}

\begin{proof}
%For each $i\in [m]$, we apply Lemma~\ref{lem:decomp} w.r.t $\A_i$ 
%that applies the second stage $\A$ with a random $c$ such that $c_i=0$ and then measures $\regX_i$ and $f$ that performs the verification on the $i$-th coordinate in the test round. 
%More precisely, given an input $\ket{\psi}$, it picks random $c_{-i}$, set $c_i\defeq 0$ to define $c$, applies $U_c$ to $\ket{\psi}$, measures $\regX_i$, and outputs the measurement outcome. 
%We define $f(x)=1$ if and only if $\ver_{\out}(k,y,0,x)=accept$.
%By Lemma~\ref{lem:decomp}, there exist a partition ($T_{i,0},T_{i,1}$) of $\hil_{\regX,\regZ}$ and an efficient quantum algorithm $\ext_i$ such that for any state $\ket{\psi}$, 
%\begin{itemize} 
%  \item
%    We have $\Pr[f(x)=1: x\sample \ext_i(\Pi_{i,0}^{in}\ket{\psi})]=1-\negl(\secpar)$.
%    \item We have $\Pr[f(x)=1:x\sample \A_i(\Pi_{i,1}^{in}\ket{\psi})]\leq \epsilon$.
%    \item Measurement w.r.t. $\{\Pi_{i,0}^{in}, \Pi_{i,1}^{in}\}$ can be done efficiently.
%\end{itemize}
First, we show the inequality for the case of $c_i=0$.
For any fixed $k$, $y$, the second claim of Lemma~\ref{lem:decomp} implies 
\begin{align*}
    &~~~\mathbb{E}_{c_{-i}}\left[\left\|\Pi_{i,0}^{out}U_c\Pi_{i,1}^{in}\Pi_{i-1,c_{i-1}}^{in}...\Pi_{1,c_1}^{in}\ket{\psi_{k,y}}\right\|^2\right] \leq \epsilon,
\end{align*}
which implies 
\begin{align*}
    \left\|\Pi_{i,0}^{out}U_c\Pi_{i,1}^{in}\Pi_{i-1,c_{i-1}}^{in}...\Pi_{1,c_1}^{in}\ket{\psi_{k,y}}\right\|^2\leq 2^{m-1}\epsilon
\end{align*}
for all $c_{-i}\in \bit^{m-1}$.
Averaging over all $(k,y)$, we obtain the inequality with the case $c_i=0$.

We move on to the proof of the case of $c_i=1$.
To prove this, we assume that 
\begin{align*}
     \mathbb{E}_{k,y}\left[\left\|\Pi_{i,1}^{out}U_c\Pi_{i,0}^{in}\Pi_{i-1,c_{i-1}}^{in}...\Pi_{1,c_1}^{in}\ket{\psi_{k,y}}\right\|^2\right]\geq 1/\poly(\secpar). 
\end{align*}
for some polynomial $\poly$, and construct an adversary $\B$ that breaks Lemma~\ref{lem:Mah_soundness}.

$\B$ is given $k_i$ as the first round message, and repeats the following procedure $\Theta(\poly(\secpar))$ times:
\begin{enumerate}
\item Pick $k_{-i}$ according to the protocol to complete $k=(k_1,...,k_m)$
\item Pick $c\sample \bit^{m}$ such that $c_i=1$.
\item Run the first stage of $\A$ to obtain $y$ along with its internal state $\ket{\psi_{k,y}}$.
\item Apply a projector $\Pi_{i,0}^{in}\Pi_{i-1,c_{i-1}}^{in}...\Pi_{1,c_1}^{in}$ to $\ket{\psi_{k,y}}$ by applying iterative projective measurements and unitary $U_c$. We note that these projective measurements can be implemented efficiently by the third claim of  Lemma~\ref{lem:decomp}. 
\takashi{Here, we assume perfect projection. In the actual proof, we have to think about the approximate version.}
If it fails in any of the projection, then it immediately aborts this trial and goes back to the 1st step of the loop.
Otherwise, it goes out the loop to continue the attack.
\end{enumerate}
Since we assume \begin{align*}
    \mathbb{E}_{k,y}\left[\left\|\Pi_{i,1}^{out}U_c\Pi_{i,0}^{in}\Pi_{i-1,c_{i-1}}^{in}...\Pi_{1,c_1}^{in}\ket{\psi_{k,y}}\right\|^2\right]\geq 1/\poly(\secpar),
\end{align*}
we especially have  
\begin{align*}
\mathbb{E}_{k,y}\left[\left\|\Pi_{i,0}^{in}\Pi_{i-1,c_{i-1}}^{in}...\Pi_{1,c_1}^{in}\ket{\psi_{k,y}}\right\|^2\right]\geq 1/\poly(\secpar).
\end{align*}
Therefore, in each trial in the loop, the probability that $\B$ does not abort is at least $1/\poly(\secpar)$. 
Thus, $\B$ goes out the loop within $\Theta(\poly(\secpar))$ times repetitions with overwhelming probability.
Assuming this happens, $\B$ now has $k,y$ along with the state $\ket{\psi_{\B}}\defeq\Pi_{i,0}^{in}\Pi_{i-1,c_{i-1}}^{in}...\Pi_{1,c_1}^{in}\ket{\psi_{k,y}}/\left\|\Pi_{i,0}^{in}\Pi_{i-1,c_{i-1}}^{in}...\Pi_{1,c_1}^{in}\ket{\psi_{k,y}}\right\|$.
Then it sends $y_i$ to the verifier.
Then verifier returns $c'_i$.
\begin{itemize}
\item If $c'_i=0$, $\B$ runs $\ext_i(\ket{\psi_{\B}})$ to obtain $a_i$. This is an accepting answer with probability $1-\negl(\secpar)$ by the first claim of  Lemma~\ref{lem:decomp}.
\item If $c'_i=1$, $\B$ applies $U_c$ to $\ket{\psi_{\B}}$ and then measures $\regX_i$ to obtain $a_i$. 
Let $\fail$ be the event that $\B$ fails to generate $\ket{\psi_{\B}}$ within $\Theta(\poly(\secpar))$ times of trials.
Then we have 
%The probability that $\B$ fails to generate $\ket{\psi_{\B}}$ within $\Theta(\poly(\secpar))$ times of trials and $a_i$ is accepted is 
\begin{align*}
% &~~~\mathbb{E}_{k,y}\left[\left\|\Pi_{i,1}^{out}U_c\left(\Pi_{i,0}^{in}\Pi_{i-1,c_{i-1}}^{in}...\Pi_{1,c_1}^{in}\ket{\psi_{k,y}}/\left\|\Pi_{i,0}^{in}\Pi_{i-1,c_{i-1}}^{in}...\Pi_{1,c_1}^{in}\ket{\psi_{k,y}}\right\|\right)\right\|^2\right]\\
%&\leq 
 &~~~\Pr\left[\ver_{\out}(k_i,\td,y_i,1,a_i) \mid \overline{\fail}\right]\\
 &\leq \Pr\left[\ver_{\out}(k_i,\td,y_i,1,a_i) \wedge \overline{\fail}\right]-\Pr[\fail]\\
 &\leq \mathbb{E}_{k,y}\left[\left\|\Pi_{i,1}^{out}U_c\Pi_{i,0}^{in}\Pi_{i-1,c_{i-1}}^{in}...\Pi_{1,c_1}^{in}\ket{\psi_{k,y}}\right\|^2\right]-\negl(\secpar)\\
 &\geq 1/\poly(\secpar)-\negl(\secpar).
\end{align*}
\end{itemize}
\takashi{More careful analysis needed.}

Overall, $\B$ succeeds in answering a correct answer with overwhelming probability for the case of $c_i=0$ and with non-negligible probability for the case of $c_i=1$, which contradicts Lemma~\ref{lem:Mah_soundness}.
Therefore, under the LWE assumption, we have
\begin{align*}
   \mathbb{E}_{k,y}\left[\left\|\Pi_{i,1}^{out}U_c\Pi_{i,0}^{in}\Pi_{i-1,c_{i-1}}^{in}...\Pi_{1,c_1}^{in}\ket{\psi_{k,y}}\right\|^2\right]=\negl(\secpar), 
\end{align*}
which concludes the proof of the proof of Lemma~\ref{lem:decomp2}.
\end{proof}

\begin{lemma}\label{lem:decomp3}
For any $\ket{\psi}\in \hil_{\regX,\regZ}$ %and decompositions of $\hil_{\regX,\regZ}$ into $T_{i,0}$ and $T_{i,1}$ for $i\in[m]$ 
and $c\in[m]$, we define
\begin{align*}
 \ket{\psi_i^{c}}\defeq \Pi_{i,1-c_i}^{in}\Pi_{i-1,c_{i-1}}^{in}...\Pi_{1,c_1}^{in}\ket{\psi} 
\end{align*}
and 
\begin{align*}
 \ket{\psi_{m+1}^{c}}\defeq \ket{\psi}-\sum_{i=1}^{m}\ket{\psi_{i}^{c}}.
\end{align*}
Then we have
\begin{align*}
 \mathbb{E}_{c}\left[\left\|\ket{\psi_{m+1}^{c}}\right\|^{2}\right]=2^{-m}.
\end{align*}
\end{lemma}
\begin{proof}
\takashi{easy induction?}
\end{proof}

\begin{lemma}\label{lem:parallel_soundness}
\begin{align*}
\mathbb{E}_{k,y,c}\left[\left\|\Pi_{m,c_m}...\Pi_{1,c_1}U_{c}\ket{\psi_{k,y}}\right\|^{2}\right]\leq 2^{-m} + (m+1)m 2^{m-1}\epsilon +\negl(\secpar).
\end{align*}
\end{lemma}
\begin{proof}
Fix $c$.
We decompose $\ket{\psi_{k,y}}$ as in Lemma~\ref{lem:decomp3} to write 
\begin{align*}
\ket{\psi_{k,y}}=\sum_{i=1}^{m+1}\ket{\psi_{k,y,i}^{c}}.
\end{align*}
Then we have 
\begin{align*}
&~~~\mathbb{E}_{k,y}\left[\left\|\Pi_{m,c_m}^{out}...\Pi_{1,c_1}^{out}U_{c}\ket{\psi_{k,y}}\right\|^2\right]\\
&\leq \mathbb{E}_{k,y}\left[\left(\sum_{i=1}^{m+1}\left\|\Pi_{m,c_m}^{out}...\Pi_{1,c_1}^{out}U_{c}\ket{\psi_{k,y,i}}\right\|\right)^2\right]\\
&\leq \mathbb{E}_{k,y}\left[(m+1)\sum_{i=1}^{m+1}\left\|\Pi_{m,c_m}^{out}...\Pi_{1,c_1}^{out}U_{c}\ket{\psi_{k,y,i}}\right\|^2\right]\\
&\leq (m+1)\left(\sum_{i=1}^{m}\mathbb{E}_{k,y}\left[\left\|\Pi_{i,c_i}^{out}U_{c}\ket{\psi_{k,y,i}}\right\|^2\right]+\mathbb{E}_{k,y}\left[\left\|\ket{\psi_{m+1}^{c}}\right\|^2\right]\right)\\
&\leq (m+1)m 2^{m-1}\epsilon+\negl(\secpar)+ \mathbb{E}_{k,y}\left[\left\|\ket{\psi_{m+1}^{c}}\right\|^2\right]
\end{align*}
by Lemma~\ref{lem:decomp2}.
By taking average over all $c$, we have 
\begin{align*}
&~~~\mathbb{E}_{k,y,c}\left[\left\|\Pi_{m,c_m}...\Pi_{1,c_1}U_{c}\ket{\psi_{k,y}}\right\|^2\right]\\
&\leq (m+1)m 2^{m-1}\epsilon+\negl(\secpar) +\mathbb{E}_{k,y,c}\left[\left\|\ket{\psi_{m+1}^{c}}\right\|^2\right]\\
&\leq (m+1)m 2^{m-1}\epsilon+\negl(\secpar) +2^{-m}
\end{align*}
by Lemma~\ref{lem:decomp3}.
\end{proof}

By Lemma~\ref{lem:parallel_soundness} and $(m+1)m2^{m-1}2\epsilon+2^{-m}\leq 1/2p(\secpar)$, we have 
\begin{align*}
    \mathbb{E}_{k,y,c}\left[\left\|\Pi_{m,c_m}...\Pi_{1,c_1}U_{c}\ket{\psi_{k,y}}\right\|^{2}\right]\leq 1/p(\secpar).
\end{align*}
As discussed in the beginning of this subsection, this suffices to show the soundness of the parallel repetition of the Mahadev's protocol (Theorem~\ref{thm:rep_soundness}).

\subsection{Jordan's Lemma and the Subspace Supporting Test Round}


%In this section, for each trial $i\in [m]$, we will use Jordan's lemma to partition the the Prover's space $\hil$ into $T_i$ and $T^{\bot}_i$, which satisfies the following properties. 
%\begin{itemize}
%    \item $T^{\bot}_i$ is the orthogonal complement of $T_i$ in $\hil$.  
%    \item For $\ket{s} \in T_i$, there exists an efficient quantum algorithm $U_T$, such that $U_T\ket{s}$ can pass the test round with probability $1-\frac{1}{\poly(n)}$; but there is no efficient quantum algorithm $U_H$, such that $U_H\ket{s}$ can pass the Hadamard round. 
%\end{itemize}


Let $m$ be the number of repetitions. We consider the prover's state after sending $y_1,\dots,y_m$. We let the prover's state space be $\hil_{\regX,\regZ}$, where $\regX$ is the register corresponding to the last message and $\regZ$ is the prover's working qubits. We can decompose $\regX$ into $\regX_1,\dots,\regX_m$ without loss of generality, where $\regX_i$ is corresponding to the output message of the $i$th trial.     


\begin{lemma}\label{lem:test_subspace}
For $i\in [m]$, we can partition $\hil_{\regX,\regZ}$ into $T_i$, $T^{\bot}_i$, and $T^{err}_i$, such that the following properties are satisfied. 
\begin{itemize}
    \item $T^{\bot}_i$ is the orthogonal complement of $T_i$ in $\hil_{\regX,\regZ}$.
    \item For $\ket{\psi}\in T_i$, for any efficient quantum algorithm $A$ used in Hadamard round, 
    \begin{align*}
        &\Pr[V\left(A(\ket{\psi})\right)=accept]\leq \delta. 
    \end{align*} 
    \takashi{This holds only if $\ket{\psi}$ is generated efficiently.}
    \item For $\ket{\psi^{\bot}}\in T^{\bot}_i$, 
    \begin{align*}
        &\Pr[f^{-1}(y_i)=M_{\regX_i}\circ\ket{\psi^{\bot}}]\leq \epsilon,  
    \end{align*}
    where $M_{\regX_i}$ is the standard-basis measurement in the register $\regX_i$, 
\end{itemize}
\end{lemma}
\nai{Need to specify the relation between $\epsilon$ and $\delta$.}
%\nai{I want to say that $T_i$ is the subspace where the prover can pass the text round but must fail in the Hadamard round. Needs to be more formal.}

%\nai{For $T_i^{\bot}$, do we need to say it cannot pass the test round with high probability? Or, we only need to care $T_i$.}

\begin{lemma}[Jordan's lemma]~\label{lemma:Jordan}
Given any two projectors $\Pi_1$ and $\Pi_2$. There exists a decomposition of the Hilbert space into one-dimensional and two-dimensional subspaces, which satisfy the following properties: 
\begin{itemize}
    \item All subspaces are orthogonal to each other.
    \item For any two-dimensional subspace $S$, for all $\ket{s}\in S$, $\Pi_1 \ket{s} \in S$ and $\Pi_2 \ket{s} \in S$.
    \item For any two-dimensional subspace $S$, $\Pi_1$ and $\Pi_2$ are rank-one projectors, i.e., there exist two vectors $\ket{v_1}$ and $\ket{v_2}$ in $S$ such that for all $ \ket{s}\in S$,  $\Pi_1 \ket{s} = \ipro{v_1}{s}\ket{v_1}$ and $\Pi_2 |s\rangle =\ipro{v_2}{s}\ket{v_2}$. 
\end{itemize}
\end{lemma}
\begin{proof}
We defer the proof to the Appendix. 
\end{proof}

\begin{proof}[Proof of Lemma~\ref{lem:test_subspace}]

Let $\regC$ be the register for the messages $c_1,\dots,c_m$ from the verifier. We define two projectors
\begin{align*}
    &\Pi_{in} := \opro{0}{0}_{\regC}\otimes I_{\regX,\regZ}\quad and \\
    &\Pi_{out} := H_{\regC_{-i}}U_T^{\dag}\left(I_{\regC}\otimes(\opro{f^{-1}(y_i)}{f^{-1}(y_i)})_{\regX_i}\otimes I_{\regX_{-i},\regZ}\right)U_T H_{\regC_{-i}}.  
\end{align*}
\nai{Need to clarify why we set $\regC_{-i}$ be uniform superposition.}

By Jordan's lemma, we can decompose the space $\hil_{\regC,\regX,\regZ}$ into one-dimensional and two-dimensional subspaces. Let $S_1,\dots,S_{\ell}$ be the two-dimensional subspaces from decomposition. Then, $\Pi_{in}$ on $S_1,\dots,S_{\ell}$ can be represented as $\ket{v_1},\dots,\ket{v_{\ell}}$. We can represent $\ket{v_i}$ as $\ket{0}_{\regC}\ket{v'_j}_{\regX,\regZ}$ for all $j\in [\ell]$ since $\Pi_{in}$ has register $\regC$ be all-zero state. Similarly, we represent $\Pi_{out}$ on $S_1,\dots,S_{\ell}$ as $\ket{w}_1,\dots,\ket{w}_{\ell}$. 

Now, we define a basis 
\begin{align*}
    B_i := \{\ket{v'_j}_{\regX,\regZ}:\; |\ipro{v_j}{w_j}|^2\geq \epsilon\}. 
\end{align*}
We let $T_i := span(B_i)$ and $T_i^{\bot}$ be the orthogonal complement of $T_i$ in $\hil_{\regX,\regZ}$. It is obvious that $\ket{\psi^{\bot}}\in T_i^{\bot}$ accepted with probability at most $\epsilon$. 


Finally, for $\ket{\psi}\in T_i$, we can use amplitude amplification to amplify the accepting probability in the test round to $1$. This implies that the prover only has negligible probability to be rejected in the test round, the probability the prover passes the test round is negligible according to Lemma~\ref{lem:Mah_soundness}. 
\nai{Make it clear how to do amplitude amplification here. We need to ``call the verifier'' to implement the projector $\Pi_{out}$ or the rotation $I-2\Pi_{out}$ since only the verifier knows $f^{-1}(y_i)$.}
\takashi{Lemma~\ref{lem:Mah_soundness} can be applied only if $\ket{\psi}$ is generated by an efficient algorithm. On the other hand, we cannot implement the projection onto $T_i$ efficiently. So here is somehow tricky to argue.}

\end{proof}

\subsection{Approximate and efficient subspace}

For $i\in [k]$, we define $T_i$, $T_i^{err}$, and $T_i^{\bot}$ on space $\hil_{X,Z,C_{-i}}$. Here, $T_i$ is the subspace where  


The following procedure define a projector $G_{i\gamma}$. Let $\epsilon = 1/poly$, $T = poly$, and $T << 1/\epsilon$. 
\begin{itemize}
    \item Let $U = V_i P H_{C_{-i}}$. 
    \item $\Pi_{in} := \opro{0}{0}\otimes I_A$. 
    \item $\Pi_{out}$, $R_1$, $R_2$. 
    \item $Q = R_2R_1$. 
    \item Do phase estimation $E$ with precision $\epsilon/2$. 
    \item i.e. $Q\ket{u} = e^{i\hat{\theta}}\ket{u}$. 
    \begin{align*}
        E\ket{u}\ket{0} \rightarrow \sum_{\theta} \alpha_{\theta} \ket{u}\ket{\theta}. 
    \end{align*}
    such that 
    \begin{align*}
        \sum_{\theta\notin \hat{\theta}+\epsilon/2}|\alpha_{\theta}|^2\leq 2^{-n}. 
    \end{align*}
    \item Sample $\gamma$ from $j\epsilon$ for $j\in [T]$
    \begin{align*}
        \ket{u}\ket{\theta}\ket{0} \rightarrow \ket{u}\ket{\theta}\ket{b}
    \end{align*}
    $b=1$ iff $2\pi -(\gamma-\epsilon/2)>\theta>\gamma-\frac{\epsilon}{2}$. 
    \item Uncompute $E$. 
    \item Measure the last qubit (the index of the threshold). 
\end{itemize}

\begin{align*}
    S_{in}:= \opro{0}{0}\otimes I_A \mbox{ wrt }Q_i
\end{align*}
decompose to $\{\ket{u_j}, \theta_j\}$. Here $u_j$ is given by Jordan's Lemma. 

for $\ket{\psi}\in S_{in}$
let 
\begin{align*}
    \ket{\psi} = \sum_{j} \alpha_j \ket{u_j}. 
\end{align*}
Fix $\gamma$ define 
\begin{align*}
    &\ket{\psi_0} = \sum_{j: \theta_j <\gamma-\epsilon} \alpha_j\ket{u_j} = \Pi_{\geq \gamma} \ket{\psi}\\
    &\ket{\psi_1} = \sum_{j: \theta_j >\gamma }\alpha_j\ket{u_j} = \Pi_{\leq \gamma-\epsilon} \ket{\psi} \\
    &\ket{\psi_{err}} = \sum_{j: others} \alpha_j\ket{u_j} 
\end{align*}

\begin{align*}
    G_{i,\gamma} (\ket{\psi_0} + \ket{\psi_1}) 
\end{align*}
project to $\ket{\psi_0}$ or $\ket{\psi_1}$ with small error from phase estimation. 

\begin{align*}
    &G_{i,\gamma} \ket{\psi} \rightarrow \ket{\tilde{\psi}_0}\mbox{ or }\ket{\tilde{\psi}_1}.\\ 
    &\ket{\tilde{\psi}_b} = \ket{\psi_b}+G_{i,\gamma}\ket{\psi_{err}}.
\end{align*}

Need to prove the following inequality. 
\begin{align*}
    \|\ket{\tilde{\psi}_b} - \ket{\psi_b}\|<\|\ket{\psi_{err}}\|.
\end{align*}

\begin{align*}
    E[\|\ket{\psi_{err}}\|] \leq \sqrt{1/T} \mbox{ or } 1/T. 
\end{align*}

Note: 
\begin{align*}
    \ket{\tilde{\psi}_1}\mbox{ reject by Hadamard round w.h.p}. 
\end{align*}
It seems that we can just analyze $\ket{\tilde{\psi}_b}$ instead of $\ket{\psi_b}$. 



\subsection{older proof}
\begin{lemma}
For $i\in [m]$, we can partition $\hil_{\regX,\regZ}$ into $T_{i}$ and $T_{i}^{\bot}$, such that the following properties are satisfied. 
For all $i\in[m]$, $c_{<i}\in \bit^{i-1}$, and efficiently implementable unitary $U$,
\begin{itemize}
    \item $T_{i}^{\bot}$ is the orthogonal complement of $T_{i}$ in $\hil_{\regX,\regZ}$.

    \item We have 
    \begin{align*}
     \mathbb{E}_{k,y}\left[\left\|\Pi_{i,0}\Pi_{i-1,c_{i-1}}...\Pi_{1,c_1}U\ket{\psi_{k,y}}\right\|^2\right]\leq   \mathbb{E}_{k,y}\left[\left\|\Pi_{T_{i}}U^{\dagger}\Pi_{i-1,c_{i-1}}...\Pi_{1,c_1}U\ket{\psi_{k,y}}\right\|^{2}\right]+\left(2^{m+4}\epsilon\right)^{1/2}
    \end{align*}
    
     \item Assuming LWE, we have 
    \begin{align*}
     \mathbb{E}_{k,y}\left[\left\|\Pi_{i,1}\Pi_{i-1,c_{i-1}}...\Pi_{1,c_1}U\ket{\psi_{k,y}}\right\|^{2}\right]\leq   \mathbb{E}_{k,y}\left[\left\|\Pi_{T_{i}^{\bot}}U^{\dagger}\Pi_{i-1,c_{i-1}}...\Pi_{1,c_1}U\ket{\psi_{k,y}}\right\|^{2}\right]+\negl(\secpar)
    \end{align*}
\end{itemize}
\end{lemma}
\begin{proof}
In the following, we denote $\ket{\psi_{k,y,i}}\defeq U^{\dagger}\Pi_{i-1,c_{i-1}}...\Pi_{1,c_1}U\ket{\psi_{k,y}}$.
For each $i\in [m]$, we apply Lemma~\ref{lem:decomp} w.r.t $\A_i$ 
%that applies the second stage $\A$ with a random $c$ and then measures $\regX_i$. 
%More precisely, 
that, given an input $\ket{\psi}$, applies $U$ to $\ket{\psi}$, measures $\regX_i$, and outputs the measurement outcome. 
By Lemma~\ref{lem:decomp}, there exist a partition ($T_i,T_i^{\bot}$) of $\hil_{\regX,\regZ}$ and an efficient quantum algorithm $\ext_i$ such that for any state $\ket{\psi}$, 
\begin{itemize} 
  \item
    We have $\Pr[f(x)=1: x\sample \ext_i(\Pi_{T_i}\ket{\psi})]=1$.
    \item We have $\Pr[f(x)=1:x\sample \A_i(\Pi_{T_i^{\bot}}\ket{\psi})]\leq \epsilon$.
    \item Measurement w.r.t. $\{\Pi_{T_i}, \Pi_{T_i^{\bot}}\}$ can be done efficiently.
\end{itemize}
For any fixed $k$, $y$, and $c$, we have
\begin{align*}
    &~~~\left\|\Pi_{i,1}\Pi_{i-1,c_{i-1}}...\Pi_{1,c_1}U\ket{\psi_{k,y}}\right\|\\
    &=\left\|\Pi_{i,1}U\ket{\psi_{k,y,i}}\right\|\\
    &\leq \left\|\Pi_{i,1}U\Pi_{T_i}\ket{\psi_{k,y,i}}\right\|+\left\|\Pi_{i,1}U\Pi_{T_i^\bot}\ket{\psi_{k,y,i}}\right\|\\
    &\leq  \left\|\Pi_{T_i}\ket{\psi_{k,y,i}}\right\|+\left\|\Pi_{i,1}U\Pi_{T_i^\bot}\ket{\psi_{k,y,i}}\right\|.
\end{align*}
By squaring both sides of the inequality, we obtain
\begin{align*}
    &~~~\left\|\Pi_{i,1}\Pi_{i-1,c_{i-1}}...\Pi_{1,c_1}U\ket{\psi_{k,y}}\right\|^2\\
    &\leq  \left\|\Pi_{T_i}\ket{\psi_{k,y,i}}\right\|^2+\left\|\Pi_{i,1}U\Pi_{T_i^\bot}\ket{\psi_{k,y,i}}\right\|^2+2\left\|\Pi_{T_i}\ket{\psi_{k,y,i}}\right\|\left\|\Pi_{i,1}U\Pi_{T_i^\bot}\ket{\psi_{k,y,i}}\right\|.
\end{align*}

By the second property above, we have
\begin{align*}
    \mathbb{E}_{c}\left[\left\|\Pi_{i,1}U\Pi_{T_i^\bot}\ket{\psi_{k,y,i}}\right\|^{2}\right]\leq \epsilon
\end{align*}
which implies 
\begin{align*}
    \left\|\Pi_{i,1}U\Pi_{T_i^\bot}\ket{\psi_{k,y,i}}\right\|^{2}\leq 2^{m}\epsilon
\end{align*}
for all $c\in \bit^{m}$.
In addition, it is trivial that we have
\begin{align*}
    \left\|\Pi_{T_i}\ket{\psi_{k,y,i}}\right\|\leq 1.
\end{align*}
Then we obtain
\begin{align*}
    &~~~\left\|\Pi_{i,1}\Pi_{i-1,c_{i-1}}...\Pi_{1,c_1}U\ket{\psi_{k,y}}\right\|^2\\
    &\leq  \left\|\Pi_{T_i}\ket{\psi_{k,y,i}}\right\|^2+2^{m}\epsilon+2\left(2^{m}\epsilon\right)^{1/2}\\
    &\leq \left\|\Pi_{T_i}\ket{\psi_{k,y,i}}\right\|^2+\left(2^{m+4}\epsilon\right)^{1/2}
\end{align*}

Summing up over all $(k,y)$, we obtain the first inequality.

We move on to the proof of the second inequality.
%First, if $\mathbb{E}_{k,y}[\|\ket{\psi_{k,y,i}}\|]=\negl(\secpar)$, then we have 
%\begin{align*}
%\mathbb{E}_{k,y}[\|\Pi_{i,1}\Pi_{i-1,c_{i-1}}...\Pi_{1,c_1}U_{c}\ket{\psi_{k,y}}\|]\leq \mathbb{E}_{k,y}[\|\ket{\psi_{k,y,i}}\|]=\negl(\secpar)   
%\end{align*}
%and thus the inequality trivially holds.
%In the following, we assume that $\mathbb{E}_{k,y}[\|\ket{\psi_{k,y,i}}\|]$ is non-negligible.
Similarly to the above, we have 
\begin{align*}
    &~~~~\mathbb{E}_{k,y}\left[\left\|\Pi_{i,0}\Pi_{i-1,c_{i-1}}...\Pi_{1,c_1}U\ket{\psi_{k,y}}\right\|\right]\\
    &=\mathbb{E}_{k,y}\left[\left\|\Pi_{i,0}U\ket{\psi_{k,y,i}}\right\|\right]\\
    &\leq \mathbb{E}_{k,y}\left[\left\|\Pi_{i,0}U\Pi_{T_i}\ket{\psi_{k,y,i}}\right\|\right]+\mathbb{E}_{k,y}\left[\left\|\Pi_{i,0}U\Pi_{T_i^\bot}\ket{\psi_{k,y,i}}\right\|\right]\\
    &\leq \mathbb{E}_{k,y}\left[\left\|\Pi_{i,0}U\Pi_{T_i}\ket{\psi_{k,y,i}}\right\|\right]+\mathbb{E}_{k,y}\left[\left\|\Pi_{T_i^\bot}\ket{\psi_{k,y,i}}\right\|\right]
\end{align*}
Thus it suffices to show 
\begin{align*}
   \mathbb{E}_{k,y}\left[\left\|\Pi_{i,0}U\Pi_{T_i}\ket{\psi_{k,y,i}}\right\|\right]=\negl(\secpar). 
\end{align*}
To prove this, we assume that 
\begin{align*}
   \mathbb{E}_{k,y}\left[\left\|\Pi_{i,0}U\Pi_{T_i}\ket{\psi_{k,y,i}}\right\|\right]\geq 1/\poly(\secpar). 
\end{align*}
for some polynomial $\poly$, and construct an adversary $\B$ that breaks Lemma~\ref{lem:Mah_soundness}.

$\B$ is given $k_i$ as the first round message, and repeats the following procedure $\Theta(\poly(\secpar))$ times:
\begin{enumerate}
\item Pick $k_{-i}$ according to the protocol to complete $k=(k_1,...,k_m)$
\item Pick $c_{<i}\sample \bit^{i-1}$.
\item Run the first stage of $\A$ to obtain $y$ along with its internal state $\ket{\psi_{k,y}}$.
\item Apply $U$ to $\ket{\psi_{k,y}}$ and measures $\regX_1,..,\regX_{i-1}$ to obtain $a_{1}$,...,$a_{i-1}$.
\item If at least one of $a_1$,...,$a_{i-1}$ is not an accepting answer w.r.t. the transcript of the corresponding coordinate, then immediately abort this trial and go back to the 1st step of the loop. 
\item Apply $U^{\dagger}$ to the post-measurement state to obtain $\ket{\psi_{k,y,i}}$.
\item Apply a measurement $\{\Pi_{T_i},\Pi_{T_i^{\bot}}\}$ to $\ket{\psi_{k,y,i}}$. 
If the measurement gives  $\Pi_{T_i^{\bot}}\ket{\psi_{k,y,i}}$, then it immediately aborts this trial and goes back to the 1st step of the loop. Otherwise, it goes out the loop to continue the attack.
\end{enumerate}
Since we assume \begin{align*}
   \mathbb{E}_{k,y}\left[\left\|\Pi_{i,0}U\Pi_{T_i}\ket{\psi_{k,y,i}}\right\|\right]\geq 1/\poly(\secpar),
\end{align*}
we especially have 
\begin{align*}
   \mathbb{E}_{k,y}\left[\left\|\Pi_{T_i}\ket{\psi_{k,y,i}}\right\|\right]\geq 1/\poly(\secpar).    
\end{align*}
Therefore in each trial in the loop, the probability that $\B$ does not abort is at least $1/\poly(\secpar)$. 
Thus, $\B$ goes out the loop within $\Theta(\poly(\secpar))$ times repetitions with overwhelming probability.
Assuming this happens, $\B$ now has $k,y$ along with the corresponding state $\ket{\psi_{\B}}\defeq \Pi_{T_i}\ket{\psi_{k,y,i}}/\left\|\Pi_{T_i}\ket{\psi_{k,y,i}}\right\|$.
Then it sends $y_i$ to the verifier.
Then verifier returns $c_i$.
\begin{itemize}
\item If $c_i=0$, $\B$ runs $\ext_i(\ket{\psi_{\B}})$ to obtain $a_i$. This is an accepting answer with probability 1 by the first property above.
\item If $c_i=1$, $\B$ applies $U$ to  $\ket{\psi_{\B}}$ and then measures $\regX_i$ to obtain $a_i$. The probability that $a_i$ is accepted is 
\begin{align*}
   &~~~~\mathbb{E}_{k,y}\left[\left\|\Pi_{i,0}U\left(\Pi_{T_i}\ket{\psi_{k,y,i}}/\|\Pi_{T_i}\ket{\psi_{k,y,i}}\|\right)\right\|\right]\\
    &\leq \mathbb{E}_{k,y}\left[\left\|\Pi_{i,0}U\Pi_{T_i}\ket{\psi_{k,y,i}}\right\|\right],
\end{align*}
which is assumed to be non-negligible.

\takashi{More careful analysis may be needed here since non-aborting probability in the loop depends on $(k,y)$.}
\end{itemize}

Overall, $\B$ succeeds in answering a correct answer with overwhelming probability for the case of $c_i=0$ and with non-negligible probability for the case of $c_i=1$, which contradicts Lemma~\ref{lem:Mah_soundness}.
Therefore, under the LWE assumption, we have
\begin{align*}
   \mathbb{E}_{k,y}[\|\Pi_{i,0}U\Pi_{T_i}\ket{\psi_{k,y,i}}\|]=\negl(\secpar), 
\end{align*}
which concludes the proof of the proof of Lemma~\ref{lem:decomp2}.





\end{proof}

\begin{theorem}
\begin{align*}
\Pr[\langle P^*,V \rangle=accept]=\mathbb{E}_{k,y,c}\left[\left\|\Pi_{m,c_m}...\Pi_{1,c_1}U_{c}\ket{\psi_{k,y}}\right\|^{2}\right]\leq 2^{-m}+(2^{m+4}\epsilon)^{1/2} +\negl(\secpar).
\end{align*}
\end{theorem}
\begin{proof}
Let $N_{k,y,i,c}\defeq \left\|\Pi_{i,c_i}...\Pi_{1,c_1}U_{c}\ket{\psi_{k,y}}\right\|^{2}$ for notational simplicity.
Summing up the two inequalities in Lemma~\ref{lem:decomp2} and dividing it by $2$, under the LWE assumption, for any $i$, $c_{<i}$, and $U$, we have 
\begin{align*}
     \mathbb{E}_{k,y,c_i}\left[\left\|\Pi_{i,c_i}\Pi_{i-1,c_{i-1}}...\Pi_{1,c_1}U\ket{\psi_{k,y}}\right\|^{2}\right]\leq \frac{1}{2}\left(\mathbb{E}_{k,y}\left[\left\|\Pi_{i-1,c_{i-1}}...\Pi_{1,c_1}U\ket{\psi_{k,y}}\right\|^{2}\right]+\left(2^{m+4}\epsilon\right)^{1/2}+\negl(\secpar)\right).  
\end{align*}
By setting $U\defeq U_c$, for any fixed choice of $c_{-i}$, we have
\begin{align*}
\mathbb{E}_{k,y,c_{i}}\left[N_{k,y,i,c}\right]\leq \frac{1}{2}\left(\mathbb{E}_{k,y,c_i}\left[N_{k,y,i-1,c}\right]+\left(2^{m+4}\epsilon\right)^{1/2}+\negl(\secpar)\right).
\end{align*}
\takashi{Should check if $U$ can depend on $c$. I guess the current argument is wrong.}

Then we have
\begin{align*}
 &~~~~\mathbb{E}_{k,y,c}\left[N_{k,y,i,c}\right]\\
 &=  \mathbb{E}_{c_{-i}}\left[\mathbb{E}_{k,y,c_i}\left[N_{k,y,m,c}\right]\right]\\
 &\leq \mathbb{E}_{c_{-i}}\left[\frac{1}{2}\left(\mathbb{E}_{k,y,c_i}\left[N_{k,y,m-1,c}\right]+\left(2^{m+4}\epsilon\right)^{1/2}+\negl(\secpar)\right)\right]\\
 &=\frac{1}{2}\mathbb{E}_{k,y,c}\left[N_{k,y,i-1,c}\right]+\frac{1}{2}\left(\left(2^{m+4}\epsilon\right)^{1/2}+\negl(\secpar)\right)
\end{align*}

Applying this ineuqality for each $i=m,...,1$, we obtain
\begin{align*}
\mathbb{E}_{k,y,c}\left[N_{k,y,m,c}\right]\leq 2^{-m}+\left(2^{m+4}\epsilon\right)^{1/2}+\negl(\secpar).
\end{align*}
\end{proof}



%\section{Parallel Repetition of Mahadev's Protocol}

\subsection{Overview of Mahadev's Protocol}\label{sec:mahadev_overview}
Here, we recall the Mahadev's protocol \cite{FOCS:Mahadev18a}. We only give a high-level description of the protocol and properties of it and omit the details since they are not needed to show our result. 

The protocol is run between a quantum prover $\pro$ and a classical verifier $\ver$ on a common input $x$. The aim of the protocol is to enable a verifier to classically verify $x\in \lang$ for a BQP language $\lang$ with the help of interactions with a quantum prover.
The protocol is a 4-round protocol where the first message is sent from $\ver$ to $\pro$. 
We denote the $i$-th message generation algorithm by $\ver_i$ for $i\in\{1,3\}$ or $\pro_i$ for $i\in \{2,4\}$ and denote the verifier's final decision algorithm by $\ver_\out$.
Then a high-level description of the protocol is given below.
\begin{description}
\item[$\ver_1$:] On input the security parameter $1^\secpar$ and $x$, it generates a pair $(\key,\td)$ of a``key" and ``trapdoor", sends $\key$ to $\pro$, and keeps $\td$ as its internal state.
\item[$\pro_2$:] On input $x$ and $\key$, it generates a classical ``commitment" $\comy$ along with a quantum state $\ket{\st_\pro}$, sends $\comy$ to $\pro$, and keeps $\ket{\st_\pro}$ as its internal state.
\item[$\ver_3$:] It randomly picks $c\sample \bit$ and sends $c$ to $\pro$.\footnote{The third message is just a public-coin, and does not depend on the transcript so far or $x$.}
For a knowledgeable reader, the case of $c=0$ corresponds to the ``test round" and the case of $c=1$ corresponds to the ``Hadamard round" in the terminology in \cite{FOCS:Mahadev18a}.
\item[$\pro_4$:] On input $\ket{\st_\pro}$ and $c$, it generates a classical string $\ans$ and sends $\ans$ to $\pro$.
\item[$\ver_\out$:] On input $\key$, $\td$, $y$, $c$, and $\ans$, it returns $\top$ indicating acceptance or $\bot$ indicating rejection.
In case $c=0$, the verification can be done publicly, that is, $\ver_\out$ need not take $\td$ as input.
\end{description}

For the protocol, we have the following properties:\\
\noindent\textbf{Completeness:}
For all $x\in \lang$, we have $\Pr[\langle \pro,\ver \rangle(x)]=\bot]= \negl(\secpar)$.\\
\noindent\textbf{Soundness:}
If the LWE problem is hard for quantum polynomial-time algorithms, then for any $x\notin \lang$ and a quantum polynomial-time cheating prover $\pro^*$, we have  $\Pr[\langle \pro^*,\ver \rangle(x)]=\bot]\leq 3/4$.

We need a slightly different form of soundness implicitly shown in \cite{FOCS:Mahadev18a}, which roughly says that if a cheating prover can pass the ``test round" (i.e., the case of $c=0$) with overwhelming probability, then it can pass the ``Hadamard round" (i.e., the case of $c=1$) only with a negligible probability. 
\begin{lemma}[implicit in \cite{FOCS:Mahadev18a}]\label{lem:Mah_soundness}
If the LWE problem is hard for quantum polynomial-time algorithms, then for any $x\notin \lang$ and a quantum polynomial-time cheating prover $\pro^*$ such that  $\Pr[\langle \pro^*,\ver \rangle(x)]=\bot\mid c=0]=\negl(\secpar)$, we have $\Pr[\langle \pro^*,\ver \rangle(x)]=\top\mid c=1]=\negl(\secpar)$.
\end{lemma}

We will also use the following simple fact:
\begin{fact}\label{fact:perfectly_pass_test}
There exists an efficient prover that passes the test round with probability $1$ (but passes the Hadamard round with probability $0$) even if $x\notin \lang$. 
\end{fact}

\subsection{Parallel Repetition}
Here, we prove that the parallel repetition of the Mahadev's protocol decrease the soundness bound to be negligible.
Let $\pro^m$ and $\ver^m$ be $m$-parallel repetitions of the honest prover $\pro$ and verifier $\ver$ in the Mahadev's protocol. Then we have the following:
\begin{theorem}[Completeness]\label{thm:rep_completeness}
For all $m= \Omega(\log^2(\secpar))$, for all $x\in \lang$, we have $\Pr[\langle \pro^m,\ver^m \rangle(x)]=\bot]= \negl(\secpar)$.\\
\end{theorem}
\begin{theorem}[Soundness]\label{thm:rep_soundness}
For all $m= \Omega(\log^2(\secpar))$, if the LWE problem is hard for quantum polynomial-time algorithms, then for any $x\notin \lang$ and a quantum polynomial-time cheating prover $\pro^*$, we have  $\Pr[\langle \pro^*,\ver^m \rangle(x)]=\top]\leq \negl(\secpar)$.
\end{theorem}
The completeness (Theorem~\ref{thm:rep_completeness}) easily follows from the completeness of the Mahadev's protocol.
In the next subsection, we prove the soundness (Theorem~\ref{thm:rep_soundness}).

\subsection{Proof of Soundness}

We prove the soundness by showing that for all noticeable error $\epsilon$, there exists a number $m$ such that by parallelly repeating the protocol $m$ times, the error can be reduced to less than $\epsilon$. 

\noindent\textbf{Characterization of cheating prover.}
Any cheating prover can be characterized by a tuple $(U_0,\{U_c\}_{c\in \bit^m})$ of unitaries over Hilbert space $\hil_{\regX}\otimes \hil_{\regZ} \otimes \hil_{\regY}  \otimes \hil_{\regK}\otimes \hil_{\regC} $. 
A prover characterized by $(U_0,\{U_{\bfc}\}_{\bfc\in \bit^m})$ works as follows.\footnote{Here, we hardwire into the cheating prover the instance $x\notin \lang$ on which it will cheat instead of giving it as an input.}
\begin{description}
\item[Second Message:] Upon receiving $k=(\key_1,...,\key_m)$, it applies $U_0$ to the state $\ket{0}_{\regX}\otimes\ket{0}_{\regZ}\otimes\ket{0}_{\regY}\otimes  \ket{k}_{\regK}$, and then measures the $Y$ register to obtain $y=(\comy_1,...,\comy_m)$. Then it sends $\bfy$ to $\ver$ and keeps the resulting state $\ket{\psi(k,y)}$ over  $\hil_{\regX,\regZ}$.
\item [Forth Message:] Upon receiving $c\in \bit^{m}$, it applies $U_c$ to $\ket{\psi(k,y)}_{\regX,\regZ}\ket{0}_{\regC}$ and then measures the $\regX$ register in computational basis to obtain $a=(a_1,...,a_m)$. We denote the designated register for $a_i$ by $\regX_i$. Then, we can view the verifier's verification procedure on $i$th trial as a unitary $V_i$. 
\end{description}

%\begin{lemma}[{\cite[Theorem 3]{NWZ09}}]
%Let $\Pi_0$ and $\Pi_1$ be projectors on a Hilbert space $\hil$ and let $R_0:= 2\Pi_0-I$, $R_1:= 2\Pi_1-I$, and $Q:=R_1R_0$. $\hil$ can be decomposed into two-dimensional subspaces $S_1,...,S_\ell$ and one-dimensional subspaces $T_1,...,T_{\ell'}$ invariant under $\Pi_0$ and $\Pi_1$.
%For each $i\in[\ell]$, we can choose a basis $\{\ket{v_i},\ket{v_i}^\bot\}$ of $S_i$ such that $Q$ is a rotation with eigenvalues $e^{\pm i2\theta_i}$ in $S_i$ where $\theta_i=\arccos\sqrt{\bra{v_i}\Pi_1\ket{v_i}}$.
%For $j\in[\ell']$, $Q$ has eigenvalues $\pm 1$ in $T_j$. 
%corresponding to eigenvectors
%\begin{align*}
%    \ket{\phi_i^+}=\frac{1}{\sqrt{2}}(\ket{v_i}+i\ket{v_i^\bot}),~~~\ket{\phi_i^-}=\frac{1}{\sqrt{2}}(\ket{v_i}-i\ket{v_i^\bot})
%\end{align*}
%\end{lemma}

In the following, we first introduced the Jordan's lemma, which we will use to prove Lemma~\ref{lem:partition}. 
\begin{lemma}[Jordan's lemma]~\label{lemma:Jordan}
Given any two projectors $\Pi_1$ and $\Pi_2$. There exists a decomposition of the Hilbert space into one-dimensional and two-dimensional subspaces, which satisfy the following properties: 
\begin{itemize}
    \item All subspaces are orthogonal to each other.
    \item For any two-dimensional subspace $S$, for all $\ket{s}\in S$, $\Pi_1 \ket{s} \in S$ and $\Pi_2 \ket{s} \in S$.
    \item For any two-dimensional subspace $S$, $\Pi_1$ and $\Pi_2$ are rank-one projectors, i.e., there exist two vectors $\ket{v_1}$ and $\ket{v_2}$ in $S$ such that for all $ \ket{s}\in S$,  $\Pi_1 \ket{s} = \ipro{v_1}{s}\ket{v_1}$ and $\Pi_2 |s\rangle =\ipro{v_2}{s}\ket{v_2}$. 
\end{itemize}
\end{lemma}


Fix $k,y$ and the function $f$. Let $i\in [m]$, we consider two projectors 
\begin{align*}
    &\Pi_{in}:= \opro{0}{0}_{\regC}\otimes I_{\regX,\regZ}\\
    %&\Pi_{i,out} := (UH_{\regC_{-i}})^{\dag}(\sum_{b,x: f_{k}(b,x)=y}\opro{b,x}{b,x}_{\regX_i})\otimes I_{\regC,\regX_{-i},\regZ} (UH_{\regC_{-i}}),
   & \Pi_{i,out} := (UH_{\regC_{-i}})^{\dag}(\sum_{a_i\in \Acc_{k_i,y_i}}\opro{a_i}{a_i}_{\regX_i}\otimes I_{\regC,\regX_{-i},\regZ}) (UH_{\regC_{-i}}),
\end{align*}
where $U$ can be any prover's strategy and $\Acc_{k_i,y_i}$ denotes the set of $a_i$ such that the verifier accepts $a_i$ in the test round on the $i$-th coordinate when the first and second messages are $k_i$ and $y_i$, respectively. 
Note that one can efficiently check if $a_i\in \Acc_{k_i,y_i}$ without knowing the trapdoor behind $k_i$ since verification in the test round can be done publicly as explained in Sec. \ref{sec:mahadev_overview}.

$\regX_{-i}:= \regX_1,\dots,\regX_{i-1}, \regX_{i+1},\dots, \regX_{m}$, and $H_{\regC_{-i}}$ means applying Hadamard operators to registers $\regC_1,\dots,\regC_{i-1}, \regC_{i+1},\dots, \regC_{m}$. By using Jordan's lemma, we can decompose the space $\hil_{\regC,\regX,\regZ}$ in the two-dimensional subspaces $S_1,\dots,S_{\ell}$ and one-dimensional subspaces $S_{\ell+1},\dots,S_{p'}$ which are vectors on either $\Pi_{in}$ or $\Pi_{i,out}$. Furthermore, $\Pi_{in}$ and $\Pi_{i,out}$ on $S_1,\dots,S_{p'}$ are rank-one projectors $\opro{\alpha_1}{\alpha_1},\dots,\opro{\alpha_{p'}}{\alpha_{p'}}$ and $\opro{\beta_1}{\beta_1},\dots,\opro{\beta_{p'}}{\beta_{p'}}$. 
For $j\in [p']$, we let the angles between $\ket{\alpha_j}$ and $\ket{\beta_j}$ as $\theta_j$.  Then, we define projectors 
\begin{align*}
    &\Pi_{in, \geq\gamma} := I_{\regC}\otimes(\sum_{j: \theta_j\geq \gamma}\opro{\alpha_j}{\alpha_j}_{\regX,\regZ})\\
    &\Pi_{in, \leq\gamma} := I_{\regC}\otimes(\sum_{j: \theta_j\leq \gamma}\opro{\alpha_j}{\alpha_j}_{\regX,\regZ}).
\end{align*}


\begin{lemma}\label{lem:partition}
Let $(U_0,U)$ be any prover's strategy. Let $i\in[m]$. Let $\gamma_0,\delta\in [0,1]$ where $2^m\gamma_0<<1$. Let $T\in \mathbb{N}$, where $\gamma_0/T>> \delta$ and $\gamma_0/T = 1/\poly(n)$. Let $\gamma$ be sampled uniformly randomly from $[\frac{\gamma_0}{T},\frac{2\gamma_0}{T},\dots,\frac{T\gamma_0}{T}]$ and $2^{m-1}<< \frac{\gamma_0}{T}$. Then, there exists an efficient quantum algorithm $G_{i,\gamma,\delta}$ such that for any efficiently generated \takashi{"efficiently generated" may not be needed here.} quantum state $\ket{\psi}_{\regX,\regZ}$, 
\begin{align*}
    G_{i,\gamma,\delta} \ket{0}_{\regC}\ket{\psi}_{\regX,\regZ}\ket{0}_{ph}\ket{0}_{th}\ket{0}_{in} = \ket{0}_{\regC}\ket{\psi_{0}}_{\regX,\regZ}\ket{001}+ \ket{0}_{\regC}\ket{\psi_{1}}_{\regX,\regZ}\ket{011} + \ket{\psi'_{err}}.
\end{align*}
Furthermore, the following properties are satisfied. 
\begin{enumerate}
    \item If we define $\ket{\psi_{err}}_{\regX,\regZ}\defeq \ket{\psi}_{\regX,\regZ} - \ket{\psi_{0}}_{\regX,\regZ}- \ket{\psi_{1}}_{\regX,\regZ}$, then we have  $E_{\gamma}[\|\ket{\psi_{err}}\|^2]\leq \frac{1}{T}+\negl(n)$.
    \item $\Pr_{\gamma}[M_{ph,th,in}\circ (G_{i,\gamma,\delta} \ket{0}_{\regC}\ket{\psi}_{\regX,\regZ}\ket{0}_{ph}\ket{0}_{th}\ket{0}_{in})\notin \{0^t01,0^t11\}] \leq \frac{1}{T}+\negl(n)$, where $M_{ph,th,in}$ is the standard-basis measurement in the register $(ph,th,in)$, and $t$ is the number of qubits in $ph$.   
    \item $E_{c\in \{0,1\}} [\|\ket{\psi_c}\|^2]\leq 1/2$. 
    \item $\|\Pi_{in,\geq\gamma-2\delta} \ket{\psi_1}\|^2 \geq (1-\negl(n))\|\ket{\psi_1}\|^2$. This implies that there exists an polynomial-time cheating prover with $\ket{\psi_1}$ that can be accepted in the test round with $1-\negl(n)$ probability. 
    \takashi{How about saying as follows: There exists an efficient unitary $U'$ such that $\Pr[M_{\regX_i}\circ U'\ket{\psi_1}\in \Acc_{k_i,y_i}]=1-\negl(\secpar)$. Especially, we need not talk about $\Pi_{in,\geq\gamma-2\delta}$ in the statement.}
    \item $\|\Pi_{in,\leq\gamma} \ket{\psi_0}\|^2 \geq (1-\negl(n))\|\ket{\psi_0}\|^2$. %This implies that $\ket{\psi_0}$ will be accepted in the test round with probability at most $2^{m}\gamma$. 
    \takashi{Similarly, we can say as follow: $\Pr[M_{\regX_i}\circ U'\ket{\psi_0}\in \Acc_{k_i,y_i}]\leq 2^{m}\gamma+\negl(\secpar)$}
\end{enumerate}
\end{lemma}

\begin{remark}
To prove Theorem~\ref{thm:rep_soundness}, we only need $m$ to be at most $\log(n)$. Hence, $\gamma_0$ and $T$ can be $1/\poly(n)$. 
\end{remark}



\begin{proof}[Proof of Lemma~\ref{lem:partition}]
We can consider $U_c$ as a unitary $U$ operating on registers $\regC$, $\regX$, and $\regZ$. Procedure~\ref{fig:process_G} defines an efficient process $G_{i,\gamma,\delta}$, which decomposes $\ket{0}_{\regC}\ket{\psi}_{\regX,\regZ}$ into states described in Lemma~\ref{lem:partition}.    

\floatname{algorithm}{Procedure}
\begin{algorithm}[h]
    \begin{mdframed}[style=figstyle,innerleftmargin=10pt,innerrightmargin=10pt]
    We define $R_1:= I-2\Pi_{i,out} $, $R_2:= 2\Pi_{in}-I$, and $Q:= R_2R_1$. 
    \begin{enumerate}
    \item Do quantum phase estimation $U_{est}$ on $Q$ with input state $\ket{\psi}$ and $t$-bit precision for parameter $t$ which will be specified later, i.e.,  
    \begin{align*}
        U_{est}\ket{u}\ket{0} \rightarrow \sum_{\theta} \alpha_{\theta} \ket{u}\ket{\theta},
    \end{align*}
    such that $\sum_{\theta\notin \hat{\theta}\pm \epsilon/2}|\alpha_{\theta}|^2\leq 2^{-n}$.
    \item Sample $\gamma$ from $\{\gamma_0/T,\dots,\gamma_0\}$ and then apply $U_{th}:\ket{u}\ket{\theta}\ket{0}_{th} \xrightarrow{U_{th}} \ket{u}\ket{\theta}\ket{b}_{th} $, 
    where $b=1$ if $\theta\in [\gamma-\delta/2, 2\pi - \gamma + \delta/2]$. 
    \item Apply $U^{\dag}_{est}$. 
    \item Apply $U_{in}: \ket{c}_{\regC}\ket{0}_{in} \xrightarrow{U_{in}}  \ket{c}_{\regC}\ket{b'}_{in}$,
    where $b'=1$ if $c\neq 0$. 
    %\item Measure the single-qubit registers $th$ and $err$. 
\end{enumerate}
    \caption{$G_{i,\gamma,\delta}$}
    \label{fig:process_G}
    \end{mdframed}
\end{algorithm}

Here, $G_{i,\gamma,\delta} := U_{in}U^{\dag}_{est}U_{th}U_{est}$ operates on register $\regC$, $\regX$, $\regZ$, and additional registers $ph$, $th$, and $in$.

We let $\ket{u^+_{j}} := \frac{1}{\sqrt{2}}(\ket{\beta_j}+ i\ket{\beta^{\bot}_j})$ and $\ket{u^-_{j}} := \frac{1}{\sqrt{2}}(\ket{\beta_j}- i\ket{\beta^{\bot}_j})$ for $j\in [\ell]$, which are eigenvectors of $Q$. For each one-dimensional subspace, it is either a vector in $\Pi_{in}$ or $\Pi_{i,out}$. We only consider vectors in $\Pi_{in}$, and denote them as $\ket{\alpha_{\ell+1}},\dots,\ket{\alpha_{p}}$. Obviously, they are also eigenvectors of $Q$ (with eigenvalues equal to zero). The eigenvalues corresponding to $\ket{u^+_j}$ are $e^{i\theta_j}$ and $\ket{u^-_j}$ are $e^{i(2\pi-\theta_j)}$.  

Now, we can decompose any input state as 
\begin{align*}
    \ket{0}_{\regC}\ket{\psi}_{\regX,\regZ}\ket{0}_{ph}\ket{0}_{th}\ket{0}_{in}:= \sum_{j=1}^p d_j \ket{0}_{\regC}\ket{\alpha_j}_{\regX,\regZ}\ket{0}_{ph}\ket{0}_{th}\ket{0}_{in}. 
\end{align*}

We suppose that $\ket{\psi}$ are on the two-dimensional subspaces without loss of generality. Then, since each $\ket{\alpha_j}$ for $j\in [\ell]$ can be represented as $a^+_j\ket{u^+_j} + a^-_{j}\ket{u^-_{j}}$, we rewrite the state in the basis of eigenvectors as 
\begin{align}
    \ket{0}_{\regC}\ket{\psi}_{\regX,\regZ}\ket{0}_{ph}\ket{0}_{th}\ket{0}_{in}:= \sum_{j=1}^{\ell} (e^+_j \ket{u^+_j}_{\regC,\regX,\regZ} +e^-_j \ket{u^-_j}_{\regC,\regX,\regZ} )\ket{0}_{ph}\ket{0}_{th}\ket{0}_{in},  \label{eq:input_state}
\end{align}
where $e^+_j = a^+_jd_j$ and $e^-_j = a^-_jd_j$. 

We define a function 
\begin{align*}
    \Delta(a;b) = \left\{\begin{matrix} 1 & \mbox{if } a\in [b,2\pi-b]\\
    0 & \mbox{otherwise}\end{matrix}\right.
\end{align*}

In the following, we apply $U_{est}$ and $U_{th}$ to the state in Eq.~\ref{eq:input_state} to estimate the eigenvalues of each $\ket{u_j}$. 
\begin{align}
    &U_{th}U_{est} \ket{0}_{\regC}\ket{\psi}_{\regX,\regZ}\ket{0}_{ph}\ket{0}_{th}\ket{0}_{in} \\
    &= U_{th}\sum_{j=1}^{\ell} \left(e^+_j \ket{u^+_j}(\sum_{\tilde{\theta}^+_j} w_{\tilde{\theta}^+_j}\ket{\tilde{\theta}^+_j}_{ph}) + e^-_j \ket{u^-_j}(\sum_{\tilde{\theta}^-_j} w_{\tilde{\theta}^-_j}\ket{\tilde{\theta}^-_j}_{ph})\right) \ket{0}_{th}\ket{0}_{in}\\
    &= \sum_{j: \theta_j\in [\gamma,2\pi-\gamma]} e^+_j \ket{u^+_j}\left(\sum_{\tilde{\theta}^+_j}\ket{\tilde{\theta}^+_j} \ket{\Delta(\tilde{\theta}^+_j,\gamma-\delta)}\ket{0}\right) + e^-_j \ket{u^-_j} \left(\sum_{\tilde{\theta}^-_j}\ket{\tilde{\theta}^-_j} \ket{\Delta(\tilde{\theta}^-_j,\gamma-\delta)}\ket{0}\right)
    \label{eq:correct_1}\\
    &+ \sum_{j: \theta_j\notin [\gamma-2\delta,2\pi-\gamma+2\delta]} e^+_j \ket{u^+_j}\left(\sum_{\tilde{\theta}^+_j}\ket{\tilde{\theta}^+_j} \ket{\Delta(\tilde{\theta}^+_j,\gamma-\delta)}\ket{0}\right) + e^-_j \ket{u^-_j} \left(\sum_{\tilde{\theta}^-_j}\ket{\tilde{\theta}^-_j} \ket{\Delta(\tilde{\theta}^-_j,\gamma-\delta)}\ket{0}\right)\label{eq:correct_2}\\
    &+ \sum_{j: \theta_j \in [\gamma-2\delta,\gamma]\vee[2\pi - \gamma,2\pi - \gamma+\delta]} \left(e^+_j\ket{u^+_j}\ket{\phi_j^+}_{ph,th} + e^-_j\ket{u^-_j}\ket{\phi_j^-}_{ph,th}\right) \ket{0}_{in}. 
    \label{eq:grey_part}
\end{align}
Here, $w_{\tilde{\theta_j^+}} = \frac{1}{2^t}\left( \frac{1-e^{2\pi i(2^t\theta_j^{+}-(b_j^+ + \tilde{\theta}_j^+))}}{1-e^{2\pi i(\theta_j^{+}-(b_j^+ + \tilde{\theta}_j^+)/2^t)}}\right)$ and $w_{\tilde{\theta_j^-}} = \frac{1}{2^t}\left( \frac{1-e^{2\pi i(2^t\theta_j^{-}-(b_j^- + \tilde{\theta}_j^-))}}{1-e^{2\pi i(\theta_j^{-}-(b_j^- + \tilde{\theta}_j^-)/2^t)}}\right)$ for $b_j^{\pm}$ the best $t$ bit approximation to $\theta_j^{\pm}$ which is less than $\theta_j^{\pm}$ for $j\in [\ell]$.


To successfully obtain $\theta_j$ with accuracy $\delta$ with probability $1-\epsilon$ for $\epsilon$ be negligible, we can choose $t=O(\log n)$. Note that simply applying phase estimation with $O(\log n)$-bit precision can not guarantee $\epsilon$ to be negligible. However, by parallelly applying phase estimation polynomially times and taking the most commonly occurring outcome, one can reduce $\epsilon$ to be negligible as shown by Watrous in~\cite{Watrous06}.  


By applying $U_{est}^{\dag}$, the state above will be 
\begin{align}
    &\sum_{j: \theta_j\in [\gamma,2\pi-\gamma]} \left[e^+_j \ket{u^+_j} \left( z^+_j\ket{0}_{ph}\ket{1}_{th} + \sqrt{1-|z^+_j|^2} (\sum_{i\neq 0^t1} h^+_{i}\ket{i}_{ph,th}) \right)\ket{0}\right. \\
    &\left. +e^-_j \ket{u^-_j} \left( z^-_j\ket{0}_{ph}\ket{1}_{th} + \sqrt{1-|z^-_j|^2} (\sum_{i\neq 0^t1} h^-_{i}\ket{i}_{ph,th}) \right)\ket{0}\right]\\
    &+\sum_{j: \theta_j\notin [\gamma-2\delta,2\pi-\gamma+2\delta]} \left[e^+_j \ket{u^+_j} \left( z^+_j\ket{0}_{ph}\ket{1}_{th} + \sqrt{1-|z^+_j|^2} (\sum_{i\neq 0^t0} h^+_{i}\ket{i}_{ph,th}) \right)\ket{0}\right.\\
    &\left. +e^-_j \ket{u^-_j} \left( z^-_j\ket{0}_{ph}\ket{1}_{th} + \sqrt{1-|z^-_j|^2} (\sum_{i\neq 0^t0} h^-_{i}\ket{i}_{ph,th}) \right)\ket{0}\right]\\
    &+  \sum_{j: \theta_j \in [\gamma-2\delta,\gamma]\vee[2\pi - \gamma,2\pi - \gamma+\delta]} \left(e^+_j\ket{u^+_j}\ket{\phi_j^+}_{ph,th} + e^-_j\ket{u^-_j}\ket{\phi_j^-}_{ph,th}\right) \ket{0}_{in}.
\end{align}
Here, $z_j^+ = z_j^-$ for all $j$. Let $z_j = z_j^+ = z_j^-$, then
$|z_j|^2 \geq 1-\epsilon$ if $\theta_j\in [\gamma,2\pi-\gamma]$ or $\theta_j\notin [\gamma-2\delta,2\pi-\gamma+2\delta]$. We can rewrite the state above in the basis of $\ket{\alpha_1},\dots,\ket{\alpha_{p}}$. 
\begin{align}
    &\sum_{j: \theta_j\in [\gamma,2\pi-\gamma]} z_jd_j \ket{0}_{\regC}\ket{\alpha_j}_{\regX,\regZ} \ket{0}_{ph}\ket{1}_{th}\ket{0}_{in}\label{eq:correct1}\\
    &+ \sum_{j: \theta_j\notin [\gamma-2\delta,2\pi-\gamma+2\delta]} z_jd_j \ket{0}_{\regC}\ket{\alpha_j}_{\regX,\regZ} \ket{0}_{ph}\ket{0}_{th}\ket{0}_{in}\label{eq:correct2}\\
    &+ \sum_{j: \theta_j\in [\gamma,2\pi-\gamma]} \sqrt{1-|z_j|^2} \left(e^+_j  \ket{u^+_j}(\sum_{i\neq 01} h^+_{i}\ket{i}_{ph,th})\ket{0}+e^-_j \ket{u^-_j}(\sum_{i\neq 01} h^-_{i}\ket{i}_{ph,th})\right)\ket{0}\label{eq:error_1}\\
    &+ \sum_{j: \theta_j\notin [\gamma-2\delta,2\pi-\gamma+2\delta]}\sqrt{1-|z_j|^2} \left(e^+_j  \ket{u^+_j}(\sum_{i\neq 00} h^+_{i}\ket{i}_{ph,th})\ket{0}+e^-_j \ket{u^-_j}(\sum_{i\neq 00} h^-_{i}\ket{i}_{ph,th})\right)\ket{0}\label{eq:error_2}\\
    &+  \sum_{j: \theta_j \in [\gamma-2\delta,\gamma]\vee[2\pi - \gamma,2\pi - \gamma+2\delta]} \left(e^+_j\ket{u^+_j}\ket{\phi_j^+}_{ph,th} + e^-_j\ket{u^-_j}\ket{\phi_j^-}_{ph,th}\right) \ket{0}_{in}\label{eq:grey}.
\end{align}
Notably, the state Eq.~\ref{eq:grey} can only have expected norm $1/T$ over the choice of $\gamma\in \{\frac{\gamma_0}{T},\frac{2\gamma_0}{T},\dots,\gamma_0\}$. Moreover, the norms of states in Eq.~\ref{eq:error_1} and Eq.~\ref{eq:error_2} can only be negligible since the phase estimation has success probability at least $1-\negl(n)$.  


By applying $U_{in}$, we can write the output state of $G_{i,\gamma,\delta}$ as
\begin{align}
    &\left(\sum_{j: \theta_j\in [\gamma-2\delta,2\pi-\gamma+2\delta]} z_jd_j \ket{0}_{\regC}\ket{\alpha_j}_{\regX,\regZ}+\ket{\eta}\right) \ket{0}_{ph}\ket{1}_{th}\ket{1}_{in} \label{eq:output_1}\\
    &+ \left(\sum_{j: \theta_j\notin [\gamma,2\pi-\gamma]} z_jd_j \ket{0}_{\regC}\ket{\alpha_j}_{\regX,\regZ}+\ket{\eta'}\right) \ket{0}_{ph}\ket{0}_{th}\ket{1}_{in}\label{eq:output_2}\\
    &+\sum_{s\in \{0,1\}^{t+2}\setminus \{0^t11,0^{t+1}1\}}\ket{\psi'_{s}}_{\regC,\regX,\regZ}\ket{s}_{ph,th,in}\label{eq:output_err}.
\end{align}
Here, $\ket{\eta}$ and $\ket{\eta'}$ are errors from the states in Eq.~\ref{eq:error_2} and Eq.~\ref{eq:error_1} after applying $U_{in}$. Furthermore, the range of $\theta_j$ has been changed from $[\gamma,2\pi-\gamma]$ to $[\gamma-2\delta, 2\pi-\gamma+2\delta]$ in Eq.~\ref{eq:correct1} and from $[\gamma-2\delta,2\pi-\gamma+2\delta]$ to $[\gamma,2\pi-\gamma]$ in Eq.~\ref{eq:output_1} and Eq.~\ref{eq:output_2}. This follows from the fact that there can be additional state from the state in Eq.~\ref{eq:grey} after applying $U_{in}$. Therefore, $\|\ket{\eta}\|$ and $\|\ket{\eta'}\|$ can only be negligible, and the probability that measuring the register $ph,th,in$ of the state gives neither $0^t11$ nor $0^{t+1}1$ is at most $1/T+\negl(n)$.  



We then define
\begin{align*}
&\ket{\psi_1}:=\sum_{j: \theta_j\in [\gamma-2\delta,2\pi-\gamma+2\delta]} z_jd_j \ket{\alpha_j}_{\regX,\regZ}+\ket{\eta}\\
&\ket{\psi_0}:=\sum_{j: \theta_j\notin [\gamma,2\pi-\gamma]} z_jd_j \ket{\alpha_j}_{\regX,\regZ}+\ket{\eta'}.\\
&\ket{\psi_{err}}:= \ket{\psi}-\ket{\psi_0}-\ket{\psi_1}. 
\end{align*}
It is worth noting that 
\begin{align*}
    \ket{0}_{\regC}\ket{\psi}_{\regX,\regZ} &=  \ket{0}_{\regC}\ket{\psi_0}_{\regX,\regZ} + \ket{0}_{\regC}\ket{\psi_1}_{\regX,\regZ} + 
    \sum_{s\in \{0,1\}^{t+2}\setminus \{0^t11,0^{t+1}1\}}\ket{\psi'_{s}}_{\regC,\regX,\regZ}\\
    &= \ket{0}_{\regC}(\ket{\psi_0}+\ket{\psi_1}+\ket{\psi_{err}}).
\end{align*}


Now, we are going to prove that the five properties of the lemma are correct. First, it is obvious that $\ket{\psi} = \ket{\psi_0}+\ket{\psi_1}+\ket{\psi_{err}}$. Then, as we have explained in the previous paragraph, the probability that measuring the register $ph,th,in$ of the state gives neither $0^t11$ nor $0^{t+1}1$ is at most $1/T+\negl(n)$. Then, we prove that $E_{\gamma}[\|\ket{\psi_{err}}\|^2]$ is at most $1/T+\negl(n)$. The errors come from the state in Eq.~\ref{eq:grey_part} are on the eigenvectors with eigenvalues in $[\gamma-2\delta,\gamma]\vee[2\pi - \gamma,2\pi - \gamma+2\delta]$, therefore, the error can be at most $\sum_{j:\theta_j \in [\gamma-2\delta,\gamma]\vee[2\pi - \gamma,2\pi - \gamma+2\delta]}| |e_j^+|^2+|e_j^-|^2$, which expected value is at most $1/T$ over the choice of $\gamma$. The errors from the states in Eq.~\ref{eq:error_2} and Eq.~\ref{eq:error_1} are at most negligible. This proves the second property.  Then, $\ket{\psi_0}$ and $\ket{\psi_1}$ may not be orthogonal. However, $\|\ket{\psi_0}\|^2+ \|\ket{\psi_1}\|^2\leq 1$ since $\|\ket{0}_{\regC}\ket{\psi_{0}}_{\regX,\regZ}\ket{001}+ \ket{0}_{\regC}\ket{\psi_{1}}_{\regX,\regZ}\ket{011}\|^2 \leq1$ and  $\ket{0}_{\regC}\ket{\psi_{0}}_{\regX,\regZ}\ket{001}$ and $ \ket{0}_{\regC}\ket{\psi_{1}}_{\regX,\regZ}\ket{011}$ are orthogonal. This also implies that 
\begin{align*}
    E_{c\in \{0,1\}}[ \|\ket{\psi_{c}}\|^2 ] \leq 1/2. 
\end{align*}
Finally, fix $k_i,y_i$, and let $b_i,x_i$ satisfy that $f_{k_i}(b_i,x_i) = y_i$. 
\takashi{Here, we also need to replace $f_{k_i}(b_i,x_i) = y_i$ with $a_i\in \Acc_{k_i,y_i}$.}
Then, given any $c_{-i}$, there exists a unitary $U_c$ such that
\begin{align*}
    & \Pr[M_{\regX}\circ(U_{c} \ket{\psi_0}) = b_i,x_i] \leq 2^m \gamma\\ 
    &\Pr[M_{\regX}\circ(U_{c} \ket{\psi_1}) = b_i,x_i] \geq 1-\negl(n).
\end{align*}
This follow from the fact that on average over $c_{-i}$,  $\Pr[M_{\regX}\circ(UH_{-i} \ket{0}_{\regC}\ket{\psi_0}) = b_i,x_i] \leq \gamma$ and $\Pr[M_{\regX}\circ(UH_{-i} \ket{0}_{\regC}\ket{\psi_1}) = b_i,x_i] \geq 1-\negl(n)$. We can show that $\Pr[M_{\regX}\circ(UH_{-i} \ket{0}_{\regC}\ket{\psi_0}) = b_i,x_i] \leq \gamma$ and $\Pr[M_{\regX}\circ(UH_{-i} \ket{0}_{\regC}\ket{\psi_1}) = b_i,x_i] \geq 1-\negl(n)$ by using the error reduction procedure by Marriott and Watrous in~\cite{MW05}. This completes the proof. 






%Consider a linear map $N_{est}: \ket{\phi}_{\regF} \rightarrow \ket{0}_{\regF}$ for all $\ket{\phi}$. Obviously, $N_{est}U_{est} = N_{est}U_{est}^{\dag}= N_{est}$ and $N_{est}$ commutes with $U_{in}$ and $U_{th}$. Similarly, we can define $N_{th}$ and $N_{err}$ that map qubits $th$ and $err$ to $\ket{0}$. Similarly, $N_{th}$ commutes with $U_{in}$ and $U_{est}$ as well as $N_{err}$ commutes with $U_{th}$ and $U_{est}$. Then, 
%\begin{align*}
%    N_eN_{th} N_e N_{err} G_{i,\gamma,\delta}\ket{\psi}\ket{0} =  \ket{\psi}\ket{0}.
%\end{align*}
%Therefore, 
%\begin{align*}
%    \ket{0}_{\regC}\ket{\psi}_{\regX,\regZ}=\ket{0}_{\regC}\ket{\psi_{00}}_{\regX,\regZ}+\ket{0}_{\regC}\ket{\psi_{01}}_{\regX,\regZ}+\ket{\psi_{10}}_{\regC,\regX,\regZ}+ \ket{\psi_{11}}_{\regC,\regX,\regZ}.
%\end{align*}

\end{proof}

%\begin{remark}\label{remark:one_half}
%Note that $\ket{\psi_0}$ and $\ket{\psi_1}$ may not be orthogonal. However, $\|\ket{\psi_0}\|^2+ \|\ket{\psi_1}\|^2\leq 1$ since $\|\ket{0}_{\regC}\ket{\psi_{0}}_{\regX,\regZ}\ket{001}+ \ket{0}_{\regC}\ket{\psi_{1}}_{\regX,\regZ}\ket{011} + \ket{\psi_{err}}\|^2 =1$ and  $\ket{0}_{\regC}\ket{\psi_{0}}_{\regX,\regZ}\ket{001}$ and $ \ket{0}_{\regC}\ket{\psi_{1}}_{\regX,\regZ}\ket{011}$ are orthogonal. This implies that 
%\begin{align*}
%    E_{c_i}[ \|\ket{\psi_{c_i}}\|^2 ] \leq 1/2
%\end{align*}
%\end{remark}

In Lemma~\ref{lem:partition}, we show that by fixing any $i\in [m]$, we can partition any prover's state into $\ket{\psi_0}$, $\ket{\psi_1}$, and $\ket{\psi_{err}}$ such that $\ket{\psi_0}$ and $\ket{\psi_1}$ will be rejected and accepted in the test round with high probability.

In the following, we show another procedure that further decompose the prover's state according to any given $c\in \{0,1\}^n$. 
\floatname{algorithm}{Procedure}
\begin{algorithm}[h]
    \begin{mdframed}[style=figstyle,innerleftmargin=10pt,innerrightmargin=10pt]
    Let $(ph_1,th_1,in_1),\dots,(ph_m,th_m,in_m)$ be additional registers that $H_c$ will use. 
    \begin{enumerate}
        \item Sample $\gamma_1$ from $\{\frac{\gamma_0}{T},\frac{2\gamma_0}{T},\dots,\gamma_0\}$. Apply $G_{1,\gamma_1,\delta}$ on $\ket{\psi}\ket{0}_{ph_1,th_1,in_1}$ to obtain
    \begin{align*}
        \ket{\psi_0}\ket{0^t01}+ \ket{\psi_1}\ket{0^t11} + \sum_{s_1\in \{0,1\}^{t+2}\setminus \{0^t01,0^t11\}}\ket{\psi'_{s_1}}\ket{s_1}. 
    \end{align*}
    \item For $i=2,\dots,m$, 
    \begin{enumerate}
        \item Sample $\gamma_i$ from $\{\frac{\gamma_0}{T},\frac{2\gamma_0}{T},\dots,\gamma_0\}$. 
        \item Apply $G_{i,\gamma_i,\delta}$ on $\ket{\psi_{\bar{c}_1,\dots,\bar{c}_{i-1}}}\ket{0}_{ph_i,th_i,in_i}$ to decompose the state into
        \begin{align*}
            \ket{\psi_{\bar{c}_1,\dots,\bar{c}_{i-1},0}}\ket{0^t01}+ \ket{\psi_{\bar{c}_1,\dots,\bar{c}_{i-1},1}}\ket{0^t11} + \sum_{s_i\in \{0,1\}^{t+2}\setminus \{0^t01,0^t11\}}\ket{\psi'_{s_i}}\ket{s_i}.  
        \end{align*}
    \end{enumerate}
    \end{enumerate}

    \caption{$H_c$}
    \label{fig:process_H}
    \end{mdframed}
\end{algorithm}

\begin{lemma}\label{lem:partition_further}
%Fix $c\in \{0,1\}^m$. 
For any $c\in \bit^m$, the state $\ket{\psi}$ can be partitioned as follows by using Procedure~\ref{fig:process_H}. 
\takashi{More precisely, shouldn't we explicitly say that there is an efficient procedure to generate $\ket{\psi_{\bar{c}_1,\dots,\bar{c}_{i-1},c_i}}$ with probability $\|\ket{\psi_{\bar{c}_1,\dots,\bar{c}_{i-1},c_i}}\|^2$?}
\begin{align*}
    & \ket{\psi} = \ket{\psi_{c_1}} + \ket{\psi_{\bar{c}_1,c_2}} + \cdots +\ket{\psi_{\bar{c}_1,\dots,\bar{c}_{m-1},c_m}} + \ket{\psi_{\bar{c}_1,\dots,\bar{c}_m}}+ \ket{\psi_{err}}.
\end{align*}
Further, the following properties are satisfied. 
For any $c\in\bit^{m}$, we have 
\begin{enumerate}
    \item For any $i\in [m]$,
    $\ket{\psi_{\bar{c}_1,\dots,\bar{c}_{i-1},0}}$ is rejected in the $i$th test round with probability at most $2^m\gamma$.
    \takashi{$\Pr[M_{\regX_i}\circ U_c \ket{\psi_{\bar{c}_1,\dots,\bar{c}_{i-1},0}}\in \Acc_{k_i,y_i}]\leq 2^{m}\gamma+ \negl(\secpar)$.}
    
    \takashi{minor comment: $\ket{\psi_{\bar{c}_1,\dots,\bar{c}_{i-1},0}}$ is not defined when $c_i=1$. We may need to slightly change the way of the statement.}
    \item There exists an polynomial-time cheating prover with $\ket{\psi_{\bar{c}_1,\dots,\bar{c}_{i-1},1}}$ that can be accepted in the test round with $1-\negl(n)$ probability. 
    \takashi{There exists an efficient unitary $U'$ such that $\Pr[M_{\regX_i}\circ U' \ket{\psi_{\bar{c}_1,\dots,\bar{c}_{i-1},1}}\in \Acc_{k_i,y_i}]= 1-\negl(\secpar)$.}
\item $E_{\gamma_1,...,\gamma_m}[\|\ket{\psi_{err}}\|^2]\leq \frac{m^2}{T}+\negl(\secpar)$.
%    \item Measuring the register $(ph_1,th_1,in_1),\dots,(ph_i,th_i,in_i)$ gives error (i.e., $\exists i$, such that $(ph_i,th_i,in_i)\neq (0^t,0,1)$ or $(0^t,1,1)$) with probability at most $\frac{m}{T}$. 
 %   \item For any fixed $\gamma$, $E_c[\|\ket{\psi_{\bar{c}_1,\dots,\bar{c}_m}}\|^2] \leq 2^{-m}$.
\end{enumerate}
 Moreover, for any fixed $\gamma$, we have $E_c[\|\ket{\psi_{\bar{c}_1,\dots,\bar{c}_m}}\|^2] \leq 2^{-m}$.
\end{lemma}
\begin{proof}
We prove this lemma by induction on $i\in [m]$. 
When $i=1$, 
\begin{align*}
    \ket{\psi} =  \ket{\psi_0} + \ket{\psi_1} + \ket{\psi_{err_1}}, 
\end{align*}
where $\ket{\psi_0}$, $\ket{\psi_1}$, and $\ket{\psi_{err_1}}$ satisfy all the properties of Lemma~\ref{lem:partition_further} according to Lemma~\ref{lem:partition}. 

We suppose the hypothesis is true when $i=k$. Then, when $i=k+1$, we can decompose the state $\ket{\psi}$ as follows
\begin{align*}
    \ket{\psi} =  \ket{\psi_{c_1}}+ \ket{\psi_{\bar{c}_1,c_2}} + \cdots + \ket{\psi_{\bar{c}_1,\dots, \bar{c}_k,c_{k+1}}} + \ket{\psi_{\bar{c}_1,\dots, \bar{c}_k,\bar{c}_{k+1}}}+ \sum_{j=1}^{k+1}\ket{\psi_{err_j}}.
\end{align*}
$\ket{\psi_{\bar{c}_1,\dots, \bar{c}_k,c_k}}$ also satisfies the first two properties according to Lemma~\ref{lem:partition}, and the third property follows from the Cauchy-Schwarz inequality. Finally, as $E[\ket{\psi_{\bar{c}_1,\dots, \bar{c}_k}}]\leq 2^{-k}$, it is obvious that $E[\ket{\psi_{\bar{c}_1,\dots, \bar{c}_k,\bar{c}_{k=1}}}]\leq 2^{-(k+1)}$ according to Lemma~\ref{lem:partition}. 
\end{proof}

Given Lemma~\ref{lem:partition_further}, we can start proving Theorem~\ref{thm:rep_soundness}. 

\begin{proof}[Proof of Theorem~\ref{thm:rep_soundness}]

According to Lemma~\ref{lem:Mah_soundness}, we know that
\begin{align*}
    \Pr_{k,y}[U_{0}\ket{\psi(k,y)}\mbox{ wins test round}]\geq 1-\negl(n) \Rightarrow \Pr_{k,y}[U_{0}\ket{\psi(k,y)}\mbox{ wins Hadamard round}]\leq \negl(n),   
\end{align*}
Here,  
\begin{align*}
    &U_0\ket{0}_{\regY}\ket{0}_{\regX,\regZ}\ket{k}_{\regK} \xrightarrow{\mbox{measure }\regY} \ket{\psi(k,y)}_{\regX,\regZ}\ket{k}_{\regK}.
\end{align*}
In the following, we just write $\ket{\psi}$ to mean $\ket{\psi(k,y)}$.

For each $i\in [m]$, let $V_{i,0}$ and $V_{i,1}$ be unitaries that run the verification procedure in the test and Hadamard round on the $i$-th coordinate and write the verification result in a designated register and $M_i$ be the measurement on the designated register on the $i$-th coordinate.
For any state $\ket{\phi}$, we denote $M\circ \ket{\phi}=\top$ to mean $M_i\circ \ket{\phi}=\top$ for all $i\in[m]$ for notational simplicity. 
With this notation, a cheating prover's success probability can be written as 
\begin{align*}
    \Pr[M\circ\left((V_{1,c_1}\cdots V_{m,c_m})U_c\ket{\psi}\right) = \top].
\end{align*}

According to Lemma~\ref{lem:partition_further}, for any fixed $c\in \bit^{m}$, we can decompose $\ket{\psi}$ as 
\begin{align*}
    \ket{\psi} =  \ket{\psi_{c_1}}+ \ket{\psi_{\bar{c}_1,c_2}} + \cdots + \ket{\psi_{\bar{c}_1,\dots, \bar{c}_{m-1},c_{m}}} + \ket{\psi_{\bar{c}_1,\dots, \bar{c}_{m-1},\bar{c}_{m}}}+ \ket{\psi_{err}}.
\end{align*}

To prove the theorem, we first show the following two inequalities holds for any $i\in[m]$ and fixed $c\in \bit^{m}$:
\begin{align}
    \Pr[M_i\circ(V_{i,0}U_c\ket{\psi_{\bar{c}_1,\dots,\bar{c}_{i-1},0}}) = \top] \leq 2^{m}\gamma_{0}. \label{eq:Test}
\end{align}
\begin{align}
    \|\ket{\psi_{\bar{c}_1,\dots,\bar{c}_{i-1},1}}\|^2\Pr[M_i\circ(V_{i,1}U_c\ket{\psi_{\bar{c}_1,\dots,\bar{c}_{i-1},1}}) = \top] = \negl(n). \label{eq:Hada}
\end{align}

Eq.~\ref{eq:Test} easily follows from the first claim of Lemma~\ref{lem:partition_further} and $\gamma\leq \gamma_0$.

For proving Eq.~\ref{eq:Hada}, we consider a modified cheating adversary described below:

\begin{enumerate}
    \item Given $k$, it runs the first stage of the adversary to obtain $y$ along with the corresponding state $\ket{\psi}=\ket{\psi(k,y)}$.
    \item Apply $G_{1,\gamma,\delta}$,....,$G_{i-1,\gamma,\delta}$ sequentially to obtain the state $\ket{\psi_{\bar{c}_1,\dots,\bar{c}_{i-1},1}}$, which succeeds in non-negligible probability since we assume the LHS of Eq.~\ref{eq:Hada} is non-negligible. 
    We denote by $\Succ$ the event that it succeeds in generating the state.
    If it fails to generate the state, then it overrides $y$ by picking it in a way such that it can pass the test round with probability $1$, which can be done according to Fact~\ref{fact:perfectly_pass_test}.
    Then it sends $y$ to the verifier.
    \item Given a challenge $c_i$, it works as follows:
    \begin{itemize}
     \item When $c_i=0$ (i.e., Test round), if $\Succ$ occurred, then it runs the cheating prover that is assumed to exist in the second claim of Lemma~\ref{lem:partition_further} to generate an forth message accepted with overwhelming probability. 
     If $\Succ$ did not occur, then it returns a forth message accepted with probability $1$, which is possible by Fact~\ref{fact:perfectly_pass_test}.
    \item When $c_i=0$ (i.e., Hadamard round), if $\Succ$ occurred, then it runs the second stage of the adversary with the internal state  $\ket{\psi_{\bar{c}_1,\dots,\bar{c}_{i-1},1}}$ to generate the forth message $a$. If $\Succ$ did not occur, it just aborts.
    \end{itemize}
\end{enumerate}
Then we can see that this cheating adversary passes the test round with overwhelming probability and passes the Hadamard round with the probability equal to the LHS of Eq.~\ref{eq:Hada}.
Therefore, Eq.~\ref{eq:Hada} follows from Lemma~\ref{lem:Mah_soundness}.

Now, we are ready to prove the theorem. 
First, we remark that it suffices to show that for any $\mu=1/\poly(n)$, there exists $m=O(\log(n))$ such that the success probability of the cheating prover is at most $\mu$.
This is because we are considering $\omega(\log(n))$-parallel repetition, in which case the number of trials is larger than   any $m=O(\log(n))$ for sufficiently large $n$, and thus we can do the same analyses focusing on  the first $m$ trials and ignoring the rest of the trials.  
Specifically, we set $m = \log \frac{1}{\mu^2}$, $\gamma_0 = 2^{-2m}$, and $T=2^{m}$. Then we have
\begin{align*}
    &\Pr[M\circ\left((V_{1,c_1}\cdots V_{m,c_m})U_c\ket{\psi}\right) = \top] \\
    &\leq (m+2)(\sum_{i=1}^{m} \|\ket{\psi_{\bar{c}_1,\dots,\bar{c}_{i-1},c_i}}\|^2\Pr[M\circ\left((V_{1,c_1}\cdots V_{m,c_m})U_c\ket{\psi_{\bar{c}_1,\dots,\bar{c}_{i-1},c_i}}\right)=\top]\\
    &+\|\ket{\psi_{\bar{c}_1,\dots,\bar{c}_m}}\|^2\Pr[M\circ\left((V_{1,c_1}\cdots V_{m,c_m})U_c\ket{\psi_{\bar{c}_1,\dots,\bar{c}_m}}\right)=\top]\\
    &+ \|\ket{\psi_{err}}\|^2\Pr[M\circ\left((V_{1,c_1}\cdots V_{m,c_m})U_c\ket{\psi_{err}}\right)=\top]))\\
    &\leq (m+2)(\sum_{i=1}^{m} \|\ket{\psi_{\bar{c}_1,\dots,\bar{c}_{i-1},c_i}}\|^2\Pr[M_i\circ\left(V_{i,c_i}U_c\ket{\psi_{\bar{c}_1,\dots,\bar{c}_{i-1},c_i}}\right)=\top]\\
    &+\|\ket{\psi_{\bar{c}_1,\dots,\bar{c}_m}}\|^2+\|\ket{\psi_{err}}\|^2)\\
    &\leq (m+2)(m(2^m\gamma_0 +\negl(n))+ 2^{-m} + \frac{m^2}{T}+\negl(\secpar)) \leq \mu. 
\end{align*}
The first inequality follows from the Cauchy-Schwarz inequality. The second inequality follows from the fact that $V_1,\dots,V_m$ commute, and thus we can choose $V_{i,c_i}$ to be the first operator operating on $\ket{\psi_{\bar{c}_1,\dots,\bar{c}_{i-1},c_{i}}}$.
The third inequality follows from Eq.~\ref{eq:Test} and \ref{eq:Hada}, which give an upper bound of the first term and Lemma~\ref{lem:partition_further}, which give upper bounds of the second and third terms.
The last inequality follows from our choices of $\gamma_0$, $T$, and $m$.
\end{proof}

%---------------------Older  Proof--------------------------------
\begin{comment}
According to Lemma~\ref{lem:partition_further}, we can decompose any state $\ket{\psi}$ above as 
\begin{align*}
    &\ket{\psi} = \ket{\psi_{c_1}} + \ket{\psi_{\bar{c}_1,c_2}} + \cdots +\ket{\psi_{\bar{c}_1,\dots,c_m}} + \ket{\psi_{\bar{c}_1,\dots,\bar{c}_m}}+ \ket{\psi_{err}}.
\end{align*}
To prove the theorem, we need to show that 
\begin{align*}
    \Pr[M\circ(V_{i,H}U_c\ket{\psi_{\bar{c}_1,\dots,\bar{c}_{i-1},1}}) = accept] = \negl(n), 
\end{align*}
where $V_{i,H}$ is the verification procedure in the $i$th round and $M$ is the measurement to check whether the prover wins the Hadamard round. 

When $i=0$, $\ket{\psi} = \ket{\psi_0}+ \ket{\psi_1}+\ket{\psi_{err_1}}$, and $\ket{\psi_1}$ wins the test round with high probability. We prove that the prover with internal state $\ket{\psi_1}$ wins the Hadamard round with only negligible probability by contradiction. Suppose $\ket{\psi_1}$ wins the Hadamard round with noticeable probability. Then, we can construct the following attack for $\ket{\psi}$ such that Lemma~\ref{lem:Mah_soundness} fails. Without loss of generality, we can assume $\|\ket{\psi_1}\|^2\geq 1/poly(n)$. The prover first applies the corresponding $G_{1,\gamma,\delta}$ and measure the register $ph,th,in$ to obtain $\ket{\psi_1}$ with noticeable probability, which implies that the prover can win the test round with noticeable probability by Lemma~\ref{lem:partition}. Then, consider the Hadamard round, if the prover does not obtain $\ket{\psi_1}$ from $G_{1,\gamma,\delta}$, it just randomly outputs $u,d$ to the verifier; otherwise, if the prover obtains $\ket{\psi_1}$, based on our hypothesis, it can win with noticeable probability. Overall, the prover can win the Hadamard round with noticeable probability, which violates Lemma~\ref{lem:Mah_soundness}. Therefore, the prover with internal state $\ket{\psi_1}$ wins the Hadamard round with only negligible probability.   

We can decompose $\ket{\psi}$ by using Procedure~\ref{fig:process_H}
\begin{align*}
    \ket{\psi} =  \ket{\psi_{c_1}}+ \ket{\psi_{\bar{c}_1,c_2}} + \cdots + \ket{\psi_{\bar{c}_1,\dots, \bar{c}_{m-1},c_{m}}} + \ket{\psi_{\bar{c}_1,\dots, \bar{c}_{m-1},\bar{c}_{m}}}+ \ket{\psi_{err}}.
\end{align*}
Similar to the case with only one trial, we can show that the prover with internal state $\ket{\psi_{\bar{c}_1,\dots, \bar{c}_{m-1},1}}$ wins the Hadamard round with negligible probability by contradiction.
Suppose the prover with $\ket{\psi_{\bar{c}_1,\dots, \bar{c}_{m-1},0}}$ can win the Hadamard round with noticeable probability. Without loss of generality, $\|\ket{\psi_{\bar{c}_1,\dots, \bar{c}_{m-1},1}}\|^2>1/\poly(n)$. This implies that the prover has noticeable probability to obtain $\ket{\psi_{\bar{c}_1,\dots, \bar{c}_{m-1},1}}$ and thus it can win the test round with high probability. Then, in the Hadamard round, the prover can again use $H_c$ to obtain $\ket{\psi_{\bar{c}_1,\dots, \bar{c}_{m-1},1}}$ and win the Hadamard round with noticeable probability, which fails Lemma~\ref{lem:Mah_soundness}. Hence, the prover with $\ket{\psi_{\bar{c}_1,\dots, \bar{c}_{m-1},1}}$ can only win the Hadamard wound with negligible probability. 


Now, we are ready to prove the theorem by using contradiction. Let $m' = \log^2n$. We suppose that there exists a prover can win with probability $\mu=1/\poly(n)$. We choose $m = \log \frac{1}{\mu^2}$, $\gamma_0 = 2^{-2m}$, and $T=2^{-m}$. Then, we choose the first $m$ trials to do parallel repetition and show that by our choices of parameters, the prover can only succeed with probability less than $\mu$. Note that the verifier has $c_1,\dots,c_m$ be chosen uniformly independently. Hence, $c_{m+1},\dots,c_{m'}$ can be viewed as some redundant information uncorrelated to $c_1,\dots,c_m$ given to the prover, which does not change our analysis in Lemma~\ref{lem:Mah_soundness}, Lemma~\ref{lem:partition}, and Lemma~\ref{lem:partition_further}. Let the verifier's verification be $V_{1,c_1},\dots,V_{m,c_m}$, where $V_{i,0}$ is doing the test round in the $i$th trial and $V_{i,1}$ is doing the Hadamard round. Then, 
\begin{align}
    &\Pr[M\circ\left((V_{1,c_1}\cdots V_{m,c_m})U_c\ket{\psi}\right) = accept] \\
    &\leq (m+2)(\Pr[M\circ\left((V_{1,c_1}\cdots V_{m,c_m})U_c\ket{\psi_{c_1}}\right)=accept] \\
    &+ \Pr[M\circ\left((V_{1,c_1}\cdots V_{m,c_m})U_c\ket{\psi_{\bar{c}_1,c_2}}\right)=accept]\\
    &+\cdots+\Pr[M\circ\left((V_{1,c_1}\cdots V_{m,c_m})U_c\ket{\psi_{\bar{c}_1,\dots,c_m}}\right)=accept]\label{eq:last_1}\\
    &+\Pr[M\circ\left((V_{1,c_1}\cdots V_{m,c_m})U_c\ket{\psi_{\bar{c}_1,\dots,\bar{c}_m}}\right)=accept] \label{eq:last_2}\\
    &+ \Pr[M\circ\left((V_{1,c_1}\cdots V_{m,c_m})U_c\ket{\psi_{err}}\right)=accept])\\
    &\leq (m2^m\gamma_0 + 2^{-m} + \frac{m}{T})(m+2) \leq \mu. 
\end{align}
The first inequality follows from the Cauchy-Schwarz inequality. The second inequality follows from the fact that $V_1,\dots,V_m$ commute, and thus we can choose $V_{i,c_i}$ to be the first operator operating on $\ket{\psi_{\bar{c}_1,\dots,\bar{c}_{i-1},c_{i}}}$; then, by our analysis, the probability that the prover can win is at most $2^m\gamma_0$. The states considered in Eq.~\ref{eq:last_1} and Eq.~\ref{eq:last_2} have norm at most $1/2^m$ and $m/T$ according to Lemma~\ref{lem:partition_further}. The last inequality follows from our choices of $\gamma_0$, $T$, and $m$.

For all noticeable $\mu$, we can find corresponding $m$, $\gamma_0$, and $T$ such that the prover can only win with probability less than $\mu$. Therefore, the probability the prover can win the test can only be negligible when $m=\poly(n)$.  
\end{comment}


\section{Parallel Repetition of Mahadev's Protocol}

\subsection{Overview of Mahadev's Protocol}\label{sec:mahadev_overview}
Here, we recall the Mahadev's protocol \cite{FOCS:Mahadev18a}. We only give a high-level description of the protocol and properties of it and omit the details since they are not needed to show our result. 

The protocol is run between a quantum prover $\pro$ and a classical verifier $\ver$ on a common input $x$. The aim of the protocol is to enable a verifier to classically verify $x\in \lang$ for a BQP language $\lang$ with the help of interactions with a quantum prover.
The protocol is a 4-round protocol where the first message is sent from $\ver$ to $\pro$. 
We denote the $i$-th message generation algorithm by $\ver_i$ for $i\in\{1,3\}$ or $\pro_i$ for $i\in \{2,4\}$ and denote the verifier's final decision algorithm by $\ver_\out$.
Then a high-level description of the protocol is given below.
\begin{description}
\item[$\ver_1$:] On input the security parameter $1^\secpar$ and $x$, it generates a pair $(\key,\td)$ of a``key" and ``trapdoor", sends $\key$ to $\pro$, and keeps $\td$ as its internal state.
\item[$\pro_2$:] On input $x$ and $\key$, it generates a classical ``commitment" $\comy$ along with a quantum state $\ket{\st_\pro}$, sends $\comy$ to $\pro$, and keeps $\ket{\st_\pro}$ as its internal state.
\item[$\ver_3$:] It randomly picks $c\sample \bit$ and sends $c$ to $\pro$.\footnote{The third message is just a public-coin, and does not depend on the transcript so far or $x$.}
Following the terminology in \cite{FOCS:Mahadev18a}, we call the case of $c=0$ the ``test round" and the case of $c=1$ the ``Hadamard round".
\item[$\pro_4$:] On input $\ket{\st_\pro}$ and $c$, it generates a classical string $\ans$ and sends $\ans$ to $\pro$.
\item[$\ver_\out$:] On input $\key$, $\td$, $y$, $c$, and $\ans$, it returns $\top$ indicating acceptance or $\bot$ indicating rejection.
In case $c=0$, the verification can be done publicly, that is, $\ver_\out$ need not take $\td$ as input.
\end{description}

For the protocol, we have the following properties:\\
\noindent\textbf{Completeness:}
For all $x\in \lang$, we have $\Pr[\langle \pro,\ver \rangle(x)]=\bot]= \negl(\secpar)$.\\
\noindent\textbf{Soundness:}
If the LWE problem is hard for quantum polynomial-time algorithms, then for any $x\notin \lang$ and a quantum polynomial-time cheating prover $\pro^*$, we have  $\Pr[\langle \pro^*,\ver \rangle(x)]=\bot]\leq 3/4$.

We need a slightly different form of soundness implicitly shown in \cite{FOCS:Mahadev18a}, which roughly says that if a cheating prover can pass the ``test round" (i.e., the case of $c=0$) with overwhelming probability, then it can pass the ``Hadamard round" (i.e., the case of $c=1$) only with a negligible probability. 
\begin{lemma}[implicit in \cite{FOCS:Mahadev18a}]\label{lem:Mah_soundness}
If the LWE problem is hard for quantum polynomial-time algorithms, then for any $x\notin \lang$ and a quantum polynomial-time cheating prover $\pro^*$ such that  $\Pr[\langle \pro^*,\ver \rangle(x)]=\bot\mid c=0]=\negl(\secpar)$, we have $\Pr[\langle \pro^*,\ver \rangle(x)]=\top\mid c=1]=\negl(\secpar)$.
\end{lemma}

We will also use the following simple fact:
\begin{fact}\label{fact:perfectly_pass_test}
There exists an efficient prover that passes the test round with probability $1$ (but passes the Hadamard round with probability $0$) even if $x\notin \lang$. 
\end{fact}

\subsection{Parallel Repetition}
Here, we prove that the parallel repetition of the Mahadev's protocol decrease the soundness bound to be negligible.
Let $\pro^m$ and $\ver^m$ be $m$-parallel repetitions of the honest prover $\pro$ and verifier $\ver$ in the Mahadev's protocol. Then we have the following:
\begin{theorem}[Completeness]\label{thm:rep_completeness}
For all $m= \Omega(\log^2(\secpar))$, for all $x\in \lang$, we have $\Pr[\langle \pro^m,\ver^m \rangle(x)]=\bot]= \negl(\secpar)$.\\
\end{theorem}
\begin{theorem}[Soundness]\label{thm:rep_soundness}
For all $m= \Omega(\log^2(\secpar))$, if the LWE problem is hard for quantum polynomial-time algorithms, then for any $x\notin \lang$ and a quantum polynomial-time cheating prover $\pro^*$, we have  $\Pr[\langle \pro^*,\ver^m \rangle(x)]=\top]\leq \negl(\secpar)$.
\end{theorem}
The completeness (Theorem~\ref{thm:rep_completeness}) easily follows from the completeness of the Mahadev's protocol.
In the next subsection, we prove the soundness (Theorem~\ref{thm:rep_soundness}).

\subsection{Proof of Soundness}\label{sec:proof_of_soundness}
First, we remark that it suffices to show that for any $\mu=1/\poly(n)$, there exists $m=O(\log(n))$ such that the success probability of the cheating prover is at most $\mu$.
This is because we are considering $\omega(\log(n))$-parallel repetition, in which case the number of repetitions is larger than   any $m=O(\log(n))$ for sufficiently large $n$, and thus we can just focus on  the first $m$ coordinates ignoring the rest of the coordinates.  
Thus, we prove the above claim in this section.
%We prove the soundness by showing that for any noticeable $\mu=1/\poly(\secpar)$, there exists a number $m=O(\log \secpar)$ such that by parallelly repeating the protocol $m$ times, the soundness error can be reduced to less than $\mu$. 


\noindent\textbf{Characterization of cheating prover.}
Any cheating prover can be characterized by a tuple $(U_0,U)$ of unitaries over Hilbert space $\hil_{\regC}\otimes \hil_{\regX}\otimes \hil_{\regZ} \otimes \hil_{\regY}  \otimes \hil_{\regK}$. 
A prover characterized by $(U_0,U)$ works as follows.\footnote{Here, we hardwire into the cheating prover the instance $x\notin \lang$ on which it will cheat instead of giving it as an input.}
\begin{description}
\item[Second Message:] Upon receiving $k=(\key_1,...,\key_m)$, it applies $U_0$ to the state $\ket{0}_{\regX}\otimes\ket{0}_{\regZ}\otimes\ket{0}_{\regY}\otimes  \ket{k}_{\regK}$, and then measures the $Y$ register to obtain $y=(\comy_1,...,\comy_m)$. Then it sends $\bfy$ to $\ver$ and keeps the resulting state $\ket{\psi(k,y)}_{\regX,\regZ}$ over  $\hil_{\regX,\regZ}$.
\item [Forth Message:] Upon receiving $c\in \bit^{m}$, it applies $U$ to $\ket{c}_{\regC}\ket{\psi(k,y)}_{\regX,\regZ}$ and then measures the $\regX$ register in computational basis to obtain $a=(a_1,...,a_m)$. We denote the designated register for $a_i$ by $\regX_i$. %Then, we can view the verifier's verification procedure on $i$th coordinate as a unitary $V_i$. 
\end{description}




Here, we first introduce a general lemma about two projectors that was shown by Nagaj, Wocjan, and Zhang \cite{NWZ09} by using the Jordan's lemma.
\begin{lemma}[{\cite[Appendix A]{NWZ09}}]\label{lem:decomposition}
Let $\Pi_0$ and $\Pi_1$ be projectors on an $N$-dimensional Hilbert space $\hil$ and let $R_0:= 2\Pi_0-I$, $R_1:= 2\Pi_1-I$, and $Q:=R_1R_0$. $\hil$ can be decomposed into two-dimensional subspaces $S_j$ for $j\in[\ell]$ and $N-2\ell$ one-dimensional subspaces $T_j^{(bc)}$ for $b,c\in \bit$ that satisfies the following properties:
\begin{enumerate}
\item For each two-dimensional subspace $S_j$, there exist two orthonormal bases $(\ket{\alpha_j},\ket{\alpha_j^{\bot}})$ and $(\ket{\beta_j},\ket{\beta_j^{\bot}})$ of $S_j$ such that $\ipro{\alpha_j}{\beta_j}$ is a positive real and for all $ \ket{s}\in S_j$,  $\Pi_0 \ket{s} = \ipro{\alpha_j}{s}\ket{\alpha_j}$ and $\Pi_1 |s\rangle =\ipro{\beta_j}{s}\ket{\beta_j}$.  Moreover, $Q$ is a rotation with eigenvalues $e^{\pm i\theta_j}$ in $S_j$ corresponding to eigenvectors $\ket{\phi_j^+}=\frac{1}{\sqrt{2}}(\ket{\alpha_j}+i\ket{\alpha_j^\bot})$ and $\ket{\phi_j^-}=\frac{1}{\sqrt{2}}(\ket{\alpha_j}-i\ket{\alpha_j^\bot})$
where $\theta_j=2\arccos \ipro{\alpha_j}{\beta_j}=2\arccos\sqrt{\bra{\alpha_j}\Pi_1\ket{\alpha_j}}$.
\item  Each one-dimensional subspace $T_j^{(bc)}$ is spanned by a vector $\ket{\alpha_j^{(bc)}}$ such that $\Pi_0 \ket{\alpha_j^{(bc)}}=b \ket{\alpha_j^{(bc)}}$ and $\Pi_1 \ket{\alpha_j^{(bc)}}=c \ket{\alpha_j^{(bc)}}$.
\end{enumerate}
\end{lemma}


Fix $k$ and $y$ for now (until we finish the proof of Lemma~\ref{lem:partition_further}). For $i\in [m]$, we consider two projectors 
\begin{align*}
    &\Pi_{in}:= \opro{0^m}{0^m}_{\regC}\otimes I_{\regX,\regZ}\\
    %&\Pi_{i,out} := (UH_{\regC_{-i}})^{\dag}(\sum_{b,x: f_{k}(b,x)=y}\opro{b,x}{b,x}_{\regX_i})\otimes I_{\regC,\regX_{-i},\regZ} (UH_{\regC_{-i}}),
   & \Pi_{i,out} := (UH_{\regC_{-i}})^{\dag}(\sum_{a_i\in \Acc_{k_i,y_i}}\opro{a_i}{a_i}_{\regX_i}\otimes I_{\regC,\regX_{-i},\regZ}) (UH_{\regC_{-i}}),
\end{align*}
where 
$\regX_{-i}:= \regX_1,\dots,\regX_{i-1}, \regX_{i+1},\dots, \regX_{m}$,  $H_{\regC_{-i}}$ means applying Hadamard operators to registers $\regC_1,\dots,\regC_{i-1}, \regC_{i+1},\dots, \regC_{m}$, and $\Acc_{k_i,y_i}$ denotes the set of $a_i$ such that the verifier accepts $a_i$ in the test round on the $i$-th coordinate when the first and second messages are $k_i$ and $y_i$, respectively.
Note that one can efficiently check if $a_i\in \Acc_{k_i,y_i}$ without knowing the trapdoor behind $k_i$ since verification in the test round can be done publicly as explained in Sec. \ref{sec:mahadev_overview}. 
We apply Lemma~\ref{lem:decomposition} for $\Pi_0= \Pi_{in}$ and $\Pi_1=\Pi_{i,out}$ to decompose the space  $\hil_{\regC,\regX,\regZ}$ into the two-dimensional subspaces $\{S_j\}_{j}$ and one-dimensional subspaces $\{T_{j}^{bc}\}_{j,b,c}$.
In the following, we use notations defined in Lemma~\ref{lem:decomposition} for this particular application.
We can write $\ket{\alpha_j}_{\regC,\regX,\regZ}=\ket{0}\ot \ket{\hat{\alpha}_j}_{\regX,\regZ}$  since $\Pi_{in}\ket{\alpha_j}=\ipro{\alpha_j}{\alpha_j}\ket{\alpha_j}=\ket{\alpha_j}$.
Similarly, we can write  $\ket{\alpha_j^{(10)}}_{\regC,\regX,\regZ}=\ket{0}\ot \ket{\hat{\alpha}_j^{(10)}}_{\regX,\regZ}$ 
and  $\ket{\alpha_j^{(11)}}_{\regC,\regX,\regZ}=\ket{0}\ot \ket{\hat{\alpha}_j^{(11)}}_{\regX,\regZ}$. 
Then  $\{\ket{\hat{\alpha}_j}\}_{j}$ and $\{\ket{\hat{\alpha}_{j}^{(1c)}}\}_{j,c}$ span $\hil_{\regX,\regZ}$. 

First, we prove a lemma which gives a decomposition of a cheating prover's state.

\begin{lemma}\label{lem:partition}
Let $(U_0,U)$ be any prover's strategy. Let $m=O(\log \secpar)$, $i\in[m]$, 
$\gamma_0 \in[0,1]$, and $T\in \mathbb{N}$ such that $\frac{\gamma_0}{T}=1/\poly(\secpar)$. Let $\gamma$ be sampled uniformly randomly from $[\frac{\gamma_0}{T},\frac{2\gamma_0}{T},\dots,\frac{T\gamma_0}{T}]$. Then, there exists an efficient quantum procedure $G_{i,\gamma}$ such that for any (possibly sub-normalized) quantum state $\ket{\psi}_{\regX,\regZ}$,  
\begin{align*}
    G_{i,\gamma} \ket{0^m}_{\regC}\ket{\psi}_{\regX,\regZ}\ket{0^t}_{ph}\ket{0}_{th}\ket{0}_{in} = z_0\ket{0^m}_{\regC}\ket{\psi_{0}}_{\regX,\regZ}\ket{0^t01}_{ph,th,in}+ z_1\ket{0^m}_{\regC}\ket{\psi_{1}}_{\regX,\regZ}\ket{0^t11}_{ph,th,in} + \ket{\psi'_{err}}
\end{align*}
where $t$ is the number of qubits in the register $ph$, $z_0,z_1\in \mathbb{C}$ such that $|z_0|=|z_1|=1$, and 
$z_0$, $z_1$, $\ket{\psi_0}_{\regX,\regZ}$, $\ket{\psi_1}_{\regX,\regZ}$, and $\ket{\psi_{err}}_{\regX,\regZ}$ may depend on $\gamma$.

Furthermore, the following properties are satisfied.
%
\begin{enumerate}
%\item There exist (sub-normalized) states $\ket{\psi_0}_{\regX,\regZ}$ and $\ket{\psi_1}_{\regX,\regZ}$ such that when $G_{i,\gamma}$ runs on input $\ket{\psi}_{\regX,\regZ}$, the output is  $(0,\ket{\psi_0}_{\regX,\regZ}/\|\ket{\psi_0}_{\regX,\regZ}\|)$ with probability $\|\ket{\psi_0}_{\regX,\regZ}\|^2$, $(1,\ket{\psi_1}_{\regX,\regZ}/\|\ket{\psi_1}_{\regX,\regZ}\|)$ with probability $\|\ket{\psi_1}_{\regX,\regZ}\|^2$, and $\bot$ with probability  $1-\|\ket{\psi_0}_{\regX,\regZ}\|^2- \|\ket{\psi_1}_{\regX,\regZ}\|^2$,
    \item If we define $\ket{\psi_{err}}_{\regX,\regZ}\defeq \ket{\psi}_{\regX,\regZ} - \ket{\psi_{0}}_{\regX,\regZ}- \ket{\psi_{1}}_{\regX,\regZ}$, then we have  $E_{\gamma}[\|\ket{\psi_{err}}_{\regX,\regZ}\|^2]\leq \frac{6}{T}+\negl(n)$.
\item For any fixed $\gamma$, $\Pr[M_{ph,th,in}\circ \ket{\psi'_{err}} \in \{0^t01,0^t11\}] =0$. %where $M_{ph,th,in}$ is the computational-basis measurement in the register $(ph,th,in)$. %and $t$ is the number of qubits in $ph$.   
This implies that if we apply the measurement $M_{ph,th,in}$ on $\frac{G_{i,\gamma} \ket{0^m}_{\regC}\ket{\psi}_{\regX,\regZ}\ket{0^t}_{ph}\ket{0}_{th}\ket{0}_{in}}{\|\ket{\psi}_{\regX,\regZ}\|}$, then the outcome is $0^tb1$ with probability $\|\ket{\psi_b}_{\regX,\regZ}\|^2$ and the resulting state in the register $(\regX,\regZ)$  is $\frac{\ket{\psi_b}_{\regX,\regZ}}{\|\ket{\psi_b}_{\regX,\regZ}\|}$ ignoring a global phase factor.
    \item For any fixed $\gamma$, $E_{b\in \{0,1\}} [\|\ket{\psi_b}_{\regX,\regZ}\|^2]\leq \frac{1}{2}\|\ket{\psi}_{\regX,\regZ}\|^2$. 
        \item 
For any fixed $\gamma$ and $c\in \bit^m$ such that $c_i=0$, we have 
\begin{align*}
\Pr\left[M_{\regX_i}\circ U\frac{\ket{c}_{\regC}\ket{\psi_0}_{\regX,\regZ}}{\|\ket{\psi_0}_{\regX,\regZ}\|}\in \Acc_{k_i,y_i}\right]\leq 2^{m-1}\gamma+\negl(\secpar).
\end{align*}
    \item 
For any fixed $\gamma$, there exists an efficient quantum algorithm $\ext_i$ such that 
\begin{align*}  
  \Pr\left[\ext_i\left(\frac{\ket{0^m}_{\regC}\ket{\psi_1}_{\regX,\regZ}}{\|\ket{\psi_1}_{\regX,\regZ}\|}\right)\in \Acc_{k_i,y_i}\right]=1-\negl(\secpar).
  \end{align*}   
\end{enumerate}
\end{lemma}

%\begin{remark}
%To prove Theorem~\ref{thm:rep_soundness}, we only need $m$ to be at most $\log(n)$. Hence, $\gamma_0$ and $T$ can be $1/\poly(n)$. 
%\end{remark}

\begin{proof}[Proof of Lemma~\ref{lem:partition}]
Procedure~\ref{fig:process_G} defines an efficient process $G_{i,\gamma}$, which decomposes $\ket{\psi}_{\regX,\regZ}$ into $\ket{\psi_0}_{\regX,\regZ}$ , $\ket{\psi_1}_{\regX,\regZ}$ , and $\ket{\psi_{err}}_{\regX,\regZ}$  described in Lemma~\ref{lem:partition}.    
Here, $G_{i,\gamma} := U_{in}U^{\dag}_{est}U_{th}U_{est}$ operates on register $\regC$, $\regX$, $\regZ$, and additional registers $ph$, $th$, and $in$, and we let $\delta:=\frac{\gamma_0}{3T}$.


\floatname{algorithm}{Procedure}
\begin{algorithm}[h]
    \begin{mdframed}[style=figstyle,innerleftmargin=10pt,innerrightmargin=10pt]
    \begin{enumerate}
   % \item Uniformly choose $\gamma$ from $[\frac{\gamma_0}{T},\frac{2\gamma_0}{T},\dots,\frac{T\gamma_0}{T}]$ 
    \item Do quantum phase estimation $U_{est}$ on $Q=(2\Pi_{in}-I)(2\Pi_{i,out}-I)$ with input state $\ket{0^m}_{\regC}\ket{\psi}_{\regX,\regZ}$ and $\tau$-bit precision and failure probability $2^{-n}$ where the parameter $\tau$ will be specified later, i.e.,  
    \begin{align*}
        U_{est}\ket{u}_{\regC,\regX,\regZ}\ket{0^t}_{ph} \rightarrow \sum_{\theta\in(-\pi,\pi]} \alpha_{\theta} \ket{u}_{\regC,\regX,\regZ}\ket{\theta}_{ph}.
    \end{align*}
 such that $\sum_{\theta\notin \bar{\theta}\pm 2^{-\tau}}|\alpha_{\theta}|^2\leq 2^{-\secpar}$ for any eigenvector $\ket{u}_{\regC,\regX,\regZ}$ of $Q$ with eigenvalue $e^{i\bar{\theta}}$.
 %\item Apply $U_{cos}:\ket{u}_{\regC,\regX,\regZ}\ket{\theta'}_{ph}\ket{0}_{pr} \xrightarrow{U_{cos}} \ket{u}_{\regC,\regX,\regZ}\ket{\theta'}_{ph}\ket{\cos^2(\theta'/2)}_{pr}$. 
    \item Apply $U_{th}:\ket{u}_{\regC,\regX,\regZ}\ket{\theta}_{ph}\ket{0}_{th} \xrightarrow{U_{th}} \ket{u}_{\regC,\regX,\regZ}\ket{\theta}_{ph}\ket{b}_{th} $, 
    where $b=1$ if $\cos^2 (\theta/2)\geq \gamma-\delta$.
    %$\cos^2 \theta\in [\gamma-\delta/2, \gamma + \delta/2]$. 
    \item Apply $U_{est}^\dag$. 
   % \item Measure $ph$ and $th$ registers and let $m$ be the measurement outcome. If $m=0^{\tau}b$ for $b\in \bit$, then it returns $b$ and the quantum state in the registers $(\regX,\regZ)$, and $\bot$ otherwise where $\tau$ is the number of qubits in the $ph$ register.
   \item Apply $U_{in}: \ket{c}_{\regC}\ket{0}_{in} \xrightarrow{U_{in}}  \ket{c}_{\regC}\ket{b'}_{in}$,
    where $b'=1$ if $c= 0^m$. 
    %\item Measure the single-qubit registers $th$ and $err$. 
\end{enumerate}
    \caption{$G_{i,\gamma}$}
    \label{fig:process_G}
    \end{mdframed}
\end{algorithm}

%We can consider $U_c$ as a unitary $U$ operating on registers $\regC$, $\regX$, and $\regZ$. 



%First, we remark that $\ket{\alpha_1},...,\ket{\alpha_\ell}$ and $\{\ket{\alpha_{j}^{(1c)}}\}$ spans $\ket{0^m}_{\regC}\ot \hil_{\regX,\regZ}$. 

In the procedure, we choose $\tau$ so that for any $\theta$ and $\theta'$ such that $|\theta'-\theta|\leq 2^{-\tau}$, we have $|\cos^2(\theta'/2)-\cos^2(\theta/2)|\leq \delta/2$.
Since we can upper and lower bound $\cos^2(\theta'/2)-\cos^2(\theta/2)$ by polynomials in $\theta'-\theta$ by considering the Taylor series, we can set $\tau=O(-\log(\delta))$ for satisfying this property.\takashi{More explanation may be needed?}
Since phase estimation with $\tau$-bit precision and failure probability $2^{-\secpar}$ can be done in time $\poly(\secpar,2^{\tau})$ \cite{NWZ09} and $\delta=\frac{\gamma_0}{3T}=1/\poly(\secpar)$ by the assumption, the procedure runs in time $\poly(\secpar)$.

For each $j\in [\ell]$, we define $p_j:= \cos^2(\theta_j/2)=\bra{\alpha_j} \Pi_{i,out} \ket{\alpha_j}$.
We define the following projections on $\hil_{\regX,\regZ}$:
\begin{align*}
    &\Pi_{in, \leq \gamma-2\delta} := \sum_{j: p_j\leq \gamma-2\delta}\opro{\hat{\alpha}_j}{\hat{\alpha}_j}_{\regX,\regZ}+ \sum_{j} \opro{\hat{\alpha}_j^{10}}{\hat{\alpha}_j^{10}}_{\regX,\regZ},\\
     &\Pi_{in, \geq \gamma} := \sum_{j: p_j\geq  \gamma}\opro{\hat{\alpha}_j}{\hat{\alpha}_j}_{\regX,\regZ}+ \sum_{j} \opro{\hat{\alpha}_j^{11}}{\hat{\alpha}_j^{11}}_{\regX,\regZ},\\
    &\Pi_{in, mid} := \sum_{j: p_j \in (\gamma-2\delta,\gamma)}\opro{\hat{\alpha}_j}{\hat{\alpha}_j}_{\regX,\regZ}.
\end{align*}
 
We let $\ket{\psi_{\leq \gamma-2\delta}}_{\regX,\regZ}:=\Pi_{in, \leq \gamma-2\delta}\ket{\psi}_{\regX,\regZ}$,  $\ket{\psi_{\geq \gamma}}_{\regX,\regZ}:=\Pi_{in, \geq \gamma}\ket{\psi}_{\regX,\regZ}$, and $\ket{\psi_{mid}}_{\regX,\regZ}:=\Pi_{in, mid}\ket{\psi}_{\regX,\regZ}$.
Then we have 
\begin{align}
\ket{\psi}_{\regX,\regZ}=\ket{\psi_{\leq \gamma-2\delta}}_{\regX,\regZ}+\ket{\psi_{\geq\gamma}}_{\regX,\regZ}+\ket{\psi_{mid}}_{\regX,\regZ}. \label{eq:psi}
\end{align}
Roughly speaking, $\ket{\psi_{\leq \gamma-2\delta}}_{\regX,\regZ}$, $\ket{\psi_{\geq\gamma}}_{\regX,\regZ}$, $\ket{\psi_{mid}}_{\regX,\regZ}$ will correspond to $\ket{\psi_0}$, $\ket{\psi_1}$, and $\ket{\psi_{err}}$ with some error terms as explained in the following.

It is easy to see that $E_{\gamma}[\|\ket{\psi_{mid}}\|^2]\leq \frac{1}{T}$ since $\Pi_{in, mid}$ with different choice of $\gamma$ are disjoint. \takashi{Explanation may be needed.}
In the following, we analyze how the first two terms of Eq. \ref{eq:psi} develops by $G_{i,\gamma}$. 

$\ket{\psi_{\leq\gamma-2\delta}}_{\regX,\regZ}$ is a superposition of states $\{\ket{\hat{\alpha}_j}_{\regX,\regZ}\}_{j:p_j\leq \gamma-2\delta}$ and $\{\ket{\hat{\alpha}_j^{11}}_{\regX,\regZ}\}_{j}$.
By Lemma~\ref{lem:decomposition}, $\ket{\alpha_j}_{\regC,\regX,\regZ}=\ket{0^m}_{\regC}\ot \ket{\hat{\alpha}_j}_{\regX,\regZ}$ can be written as $\ket{\alpha_j}_{\regC,\regX,\regZ}=\frac{1}{\sqrt{2}}(\ket{\phi_j^+}_{\regC,\regX,\regZ}+\ket{\phi_j^-}_{\regC,\regX,\regZ})$ where $\ket{\phi_j^\pm}_{\regC,\regX,\regZ}$ is an eigenvector of $Q$ with eigenvalue $e^{\pm i\theta_j}$ where $\theta_j=2\arccos(\sqrt{p_j})\geq 2\arccos (\sqrt{\gamma-2\delta})$. 
Moreover,  $\ket{\alpha_j^{(10)}}_{\regC,\regX,\regZ}=\ket{0^m}_{\regC}\ot \ket{\hat{\alpha}_j^{(10)}}_{\regX,\regZ}$ is an eigenvector of $Q$ with eigenvalue $-1=e^{i\pi}$. 
Here, we remark that $\pi\geq 2\arccos x$ for any $0\leq x \leq 1$.
Thus, after applying $U_{est}$ to  $\ket{\psi_{\leq\gamma-2\delta}}_{\regX,\regZ}$, $(1-2^{-\secpar})$-fraction of the state contains $\theta$ in the register $ph$ such that $|\theta|\geq 2\arccos(\sqrt{\gamma-2\delta}) -2^{-\tau}$. which implies $\cos^2(\theta/2)\leq \gamma-\frac{3}{2}\delta<\gamma-\delta$ by our choice of $\tau$.
For this fraction of the state, $U_{th}$ does nothing.
 Thus, we have 
 \begin{align*}
 \TD(U_{est}\ket{0^m}_{\regC}\ket{\psi_{\leq\gamma-2\delta}}_{\regX,\regZ}\ket{0^t00}_{ph,th,in},U_{th}U_{est}\ket{0^m}_{\regC}\ket{\psi_{\leq\gamma-2\delta}}_{\regX,\regZ}\ket{0^t00}_{ph,th,in})\leq 2^{-n}
 \end{align*}
  and thus
  \begin{align*}
  \TD(\ket{0^m}_{\regC}\ket{\psi_{\leq\gamma-2\delta}}_{\regX,\regZ}\ket{0^t00}_{ph,th,in},U_{est}^\dag U_{th}U_{est}\ket{0^m}_{\regC}\ket{\psi_{\leq\gamma-2\delta}}_{\regX,\regZ}\ket{0^t00}_{ph,th,in})\leq 2^{-n}
  \end{align*}
   where $\TD$ denotes the trace distance. 
 \takashi{The explanation here makes sense? If not, we may need to write some expressions as in the previous manuscript.}

Therefore we can write
\begin{align}
 G_{i,\gamma}\ket{0^m}_{\regC}\ket{\psi_{\leq\gamma-2\delta}}_{\regX,\regZ}\ket{0^t00}_{ph,th,in}=z_{\leq \gamma-2\delta}\ket{0^m}_{\regC}\ket{\psi_{\leq\gamma-2\delta}}_{\regX,\regZ}\ket{0^t01}_{ph,th,in}+\ket{\psi'_{err,\leq \gamma-2\delta}} \label{eq:leq}
\end{align}

by using $z_{\leq \gamma-2\delta}$ such that $|z_{\leq \gamma-2\delta}|^2\geq 1-2^{-\secpar}$ and a state  $\ket{\psi'_{err,\leq \gamma-2\delta}}$ that is orthogonal to $\ket{0^m}_{\regC}\ket{\psi_{\leq\gamma-2\delta}}_{\regX,\regZ}\ket{0^t01}_{ph,th,in}$ such that $\|\ket{\psi'_{err,\leq \gamma-2\delta}}\|^2 \leq 2^{-n}$.

\begin{comment}
$\ket{\psi_{\geq\gamma}}_{\regX,\regZ}$ is a superposition of $\ket{\hat{\alpha}_j}_{\regX,\regZ}$ such that $p_j\geq \gamma$ and $\ket{\hat{\alpha}_j^{11}}_{\regX,\regZ}$.
By Lemma~??? $\ket{\alpha_j}_{\regC,\regX,\regZ}=\ket{0^m}_{\regC}\ot \ket{\hat{\alpha}_j}_{\regX,\regZ}$ can be written as $\ket{\alpha_j}_{\regC,\regX,\regZ}=\frac{1}{\sqrt{2}}(\ket{\phi_j^+}_{\regC,\regX,\regZ}+\ket{\phi_j^-}_{\regC,\regX,\regZ})$ where $\ket{\phi_j^\pm}_{\regC,\regX,\regZ}$ is an eigenvector of $Q$ with eigenvalue $e^{-i\theta_j}$. 
Moreover,  $\ket{\alpha_j^{(11)}}_{\regC,\regX,\regZ}=\ket{0^m}_{\regC}\ot \ket{\hat{\alpha}_j^{(11)}}_{\regX,\regZ}$ is an eigenvector of $Q$ with eigenvalue $1=e^{0}$. 
Thus, after applying $U_{est}$ to  $\ket{\psi_{\geq\gamma}}_{\regX,\regZ}$, $(1-2^{-\secpar})$-fraction of the state contains $\theta$ in the register $ph$ such that $\theta\leq 2\arccos(\sqrt{\gamma}) +2^{-t}$. which implies $\cos^2(\theta/2)\geq \gamma-\delta$ by our choice of $t$.
%For this fraction of the state, $U_{th}$ does nothing, and then $U_{\est}^\dag$ returns the state to the original state.  
 Thus, we have $\TD(U_{est}\ket{0^m}_{\regC}\ket{\psi_{\geq\gamma}}_{\regX,\regZ}\ket{0^t00}_{ph,th,in},U_{th}U_{est}\ket{0^m}_{\regC}\ket{\psi_{\geq\gamma}}_{\regX,\regZ}\ket{0^t00}_{ph,th,in})\leq 2^{-n}$ and thus$\TD(\ket{0^m}_{\regC}\ket{\psi_{\geq\gamma}}_{\regX,\regZ}\ket{0^t00}_{ph,th,in},U_{est}^\dag U_{th}U_{est}\ket{0^m}_{\regC}\ket{\psi_{\geq\gamma}}_{\regX,\regZ}\ket{0^t00}_{ph,th,in})\leq 2^{-n}$ where $\TD$ denotes the trace distance. 
 \takashi{The explanation here makes sense? If not, we may need to write some expressions as in the previous manuscript.}

Therefore we can write
\end{comment}

By a similar analysis, we can write 
\begin{align}
 G_{i,\gamma}\ket{0^m}_{\regC}\ket{\psi_{\geq\gamma}}_{\regX,\regZ}\ket{0^t00}_{ph,th,in}=z_{\geq \gamma}\ket{0^m}_{\regC}\ket{\psi_{\geq\gamma}}_{\regX,\regZ}\ket{0^t11}_{ph,th,in}+\ket{\psi'_{err,\geq \gamma}} \label{eq:geq}
\end{align}

by using $z_{\geq \gamma}$ such that $|z_{\geq \gamma}|^2\geq 1-2^{-\secpar}$ and a state  $\ket{\psi'_{err,\geq \gamma}}$ that is orthogonal to $\ket{0^m}_{\regC}\ket{\psi_{\geq\gamma}}_{\regX,\regZ}\ket{0^t01}_{ph,th,in}$ such that $\|\ket{\psi'_{err,\geq \gamma}}\|^2 \leq 2^{-n}$.


Combining Eq. \ref{eq:psi}, \ref{eq:leq}, and \ref{eq:geq}, we have
%we have 
%\begin{align*}
 %   G_{i,\gamma} \ket{0^m}_{\regC}\ket{\psi}_{\regX,\regZ}\ket{0^t}_{ph}\ket{0}_{th}\ket{0}_{in} = \ket{0^m}_{\regC}\ket{\psi_{0}}_{\regX,\regZ}\ket{0^t01}_{ph,th,in}+ \ket{0^m}_{\regC}\ket{\psi_{1}}_{\regX,\regZ}\ket{0^t11}_{ph,th,in} + \ket{\psi'_{err}}.
%\end{align*}
%where 
%\begin{align*}
%\ket{\psi_{0}}_{\regX,\regZ}=z_{\geq \gamma}\ket{\psi_{\geq\gamma}}_{\regX,\regZ}+
%\end{align*}
\begin{align}
\begin{split}
& G_{i,\gamma}\ket{0^m}_{\regC}\ket{\psi}_{\regX,\regZ}\ket{0^t00}_{ph,th,in} \\
&=\ket{0^m}_{\regC}(z_{\geq \gamma}\ket{\psi_{\geq\gamma}}_{\regX,\regZ}+\ket{\eta_{mid,0}}_{\regX,\regZ}+\ket{\eta_{other,0}}_{\regX,\regZ})\ket{0^t01}_{ph,th,in} \\
&+\ket{0^m}_{\regC}(z_{\leq \gamma-2\delta}\ket{\psi_{\leq\gamma-2\delta}}_{\regX,\regZ}+\ket{\eta_{mid,1}}_{\regX,\regZ}+\ket{\eta_{other,1}}_{\regX,\regZ})\ket{0^t11}_{ph,th,in} \\
&+ \ket{\psi'_{err}}. 
\end{split} \label{eq:resulting_state}
\end{align}

where $\ket{\eta_{mid,0}}_{\regX,\regZ}$, $\ket{\eta_{other,0}}_{\regX,\regZ}$, $\ket{\eta_{mid,1}}_{\regX,\regZ}$, and $\ket{\eta_{other,1}}_{\regX,\regZ}$ are defined so that
\begin{align}
& I_{\regC,\regX,\regZ} \ot \opro{0^t01}{0^t01}_{ph,th,in}G_{i,\gamma}\ket{0^m}_{\regC}\ket{\psi_{mid}}_{\regX,\regZ}\ket{0^t00}_{ph,th,in}=\ket{0^m}_{\regC}{\ket{\eta_{mid,0}}_{\regX,\regZ}}\ket{0^t01}_{ph,th,in}, \label{eq:etamid} \\
&I_{\regC,\regX,\regZ} \ot \opro{0^t11}{0^t11}_{ph,th,in}G_{i,\gamma}\ket{0^m}_{\regC}\ket{\psi_{mid}}_{\regX,\regZ}\ket{0^t00}_{ph,th,in}=\ket{0^m}_{\regC}{\ket{\eta_{mid,1}}_{\regX,\regZ}}\ket{0^t11}_{ph,th,in}, \label{eq:etamidprime} \\
&I_{\regC,\regX,\regZ} \ot \opro{0^t01}{0^t01}_{ph,th,in}(\ket{\psi'_{err,\leq \gamma-2\delta}}+\ket{\psi'_{err,\geq \gamma}})=\ket{0^m}_{\regC}{\ket{\eta_{other,0}}_{\regX,\regZ}}\ket{0^t01}_{ph,th,in}, \label{eq:etaother} \\
&I_{\regC,\regX,\regZ} \ot \opro{0^t11}{0^t11}_{ph,th,in}(\ket{\psi'_{err,\leq \gamma-2\delta}}+\ket{\psi'_{err,\geq \gamma}})=\ket{0^m}_{\regC}{\ket{\eta_{other,1}}_{\regX,\regZ}}\ket{0^t11}_{ph,th,in}. \label{eq:etaotherprime} 
\end{align}
and $\ket{\psi'_{err}}$  is defined by 
\begin{align}
%\begin{split}
\ket{\psi'_{err}}&:=
\sum_{s\notin \{0^t01,0^t11\}} I_{\regC,\regX,\regZ} \ot \opro{s}{s}_{ph,th,in}(G_{i,\gamma}\ket{0^m}_{\regC}\ket{\psi_{mid}}_{\regX,\regZ}\ket{0^t00}_{ph,th,in}+\ket{\psi'_{err,\leq \gamma-2\delta}}+\ket{\psi'_{err,\geq \gamma}}) \label{eq:psierrprime}
%&-\ket{0^m}_{\regC}(\ket{\eta_{mid,0}}_{\regX,\regZ}+\ket{\eta_{other,0}}_{\regX,\regZ})\ket{0^t01}_{ph,th,in} \\
%&-\ket{0^m}_{\regC}(\ket{\eta_{mid,1}}_{\regX,\regZ}+\ket{\eta_{other,1}}_{\regX,\regZ})\ket{0^t11}_{ph,th,in}. 
%\end{split} 
\end{align}

We remark that $\ket{\eta_{mid,0}}_{\regX,\regZ}$, $\ket{\eta_{other,0}}_{\regX,\regZ}$, $\ket{\eta_{mid,1}}_{\regX,\regZ}$, and $\ket{\eta_{other,1}}_{\regX,\regZ}$ are well-defined since after applying $G_{i,\gamma}$, the value in the register $in$ is $1$ if and only if the value in the register $\regC$ is $0^m$.

We let $z_0:=\frac{z_{\leq \gamma-2\delta}}{|z_{\leq \gamma-2\delta}|}$, $z_1:=\frac{z_{\geq \gamma}}{|z_{\geq \gamma}|}$, and 
\begin{align}
&\ket{\psi_0}_{\regX,\regZ}:=|z_{\leq \gamma-2\delta}|\ket{\psi_{\leq\gamma-2\delta}}_{\regX,\regZ}+ \overline{z}_0(\ket{\eta_{mid,0}}_{\regX,\regZ}+\ket{\eta_{other,0}}_{\regX,\regZ}), \label{eq:psizero} \\
&\ket{\psi_1}_{\regX,\regZ}:=|z_{\geq \gamma}|\ket{\psi_{\geq\gamma}}_{\regX,\regZ}+ \overline{z}_1(\ket{\eta_{mid,1}}_{\regX,\regZ}+\ket{\eta_{other,1}}_{\regX,\regZ}), \label{eq:psione}
\end{align}
where $\overline{z}_0$ and $\overline{z}_1$ denotes complex conjugates of $z_0$ and $z_1$. By Eq~\ref{eq:resulting_state}, \ref{eq:psizero}, and \ref{eq:psione}, we have
\begin{align*}
    G_{i,\gamma} \ket{0^m}_{\regC}\ket{\psi}_{\regX,\regZ}\ket{0^t}_{ph}\ket{0}_{th}\ket{0}_{in} = z_0\ket{0^m}_{\regC}\ket{\psi_{0}}_{\regX,\regZ}\ket{0^t01}_{ph,th,in}+ z_1\ket{0^m}_{\regC}\ket{\psi_{1}}_{\regX,\regZ}\ket{0^t11}_{ph,th,in} + \ket{\psi'_{err}}.
\end{align*}

Now, we are ready to prove the five claims in Lemma~\ref{lem:partition}.

\paragraph{Proof of the first claim.}
By Eq.~\ref{eq:psi}, \ref{eq:psizero}, and \ref{eq:psione}, we have
\begin{align*}
\ket{\psi_{err}}_{\regX,\regZ}&=\ket{\psi}_{\regX,\regZ}-\ket{\psi_0}_{\regX,\regZ}-\ket{\psi_1}_{\regX,\regZ}\\
&=(1-|z_{\leq \gamma-2\delta}|)\ket{\psi_{\leq\gamma-2\delta}}_{\regX,\regZ}+(1-|z_{\geq \gamma}|)\ket{\psi_{\geq\gamma}}_{\regX,\regZ}+\ket{\psi_{mid}}_{\regX,\regZ}\\
&-\overline{z}_0(\ket{\eta_{mid,0}}_{\regX,\regZ}+\ket{\eta_{other,0}}_{\regX,\regZ})
-\overline{z}_1(\ket{\eta_{mid,1}}_{\regX,\regZ}+\ket{\eta_{other,1}}_{\regX,\regZ}),
\end{align*}

Since $|z_{\leq \gamma-2\delta}|$ and $|z_{\geq \gamma}|$ are $1-\negl(\secpar)$, the norms of the first two terms are negligible.
By Eq.~\ref{eq:etaother} and \ref{eq:etaotherprime}, we have $\|\ket{\eta_{other,0}}_{\regX,\regZ}\|^2 + \|\ket{\eta_{other,1}}_{\regX,\regZ}\|^2\leq \|\ket{\psi'_{err,\leq \gamma-2\delta}}+\ket{\psi'_{err,\geq \gamma}}\|^2 \leq \negl(\secpar)$.
Therefore we have
\begin{align*}
\|\ket{\psi_{err}}_{\regX,\regZ}\|^2 &\leq \|\ket{\psi_{mid}}_{\regX,\regZ}-\overline{z}_0\ket{\eta_{mid,0}}_{\regX,\regZ}-\overline{z}_1\ket{\eta_{mid,1}}_{\regX,\regZ}\|^2+\negl(\secpar)\\
&\leq 3(\|\ket{\psi_{mid}}_{\regX,\regZ}\|^2+\|\ket{\eta_{mid,0}}_{\regX,\regZ}\|^2 + \|\ket{\eta_{mid,1}}_{\regX,\regZ}\|^2)+\negl(\secpar)
\end{align*}
where the latter inequality follows from the Cauchy-Schwarz inequality.
As already noted, we have $E_{\gamma}[\|\ket{\psi_{mid}}_{\regX,\regZ}\|^2]\leq \frac{1}{T}$.
By Eq.~\ref{eq:etamid} and \ref{eq:etamidprime}, we have $E_{\gamma}[\|\ket{\eta_{mid,0}}_{\regX,\regZ}\|^2 + \|\ket{\eta_{mid,1}}_{\regX,\regZ}\|^2]\leq E_{\gamma}[\|\ket{\psi_{mid}}_{\regX,\regZ}\|^2]\leq \frac{1}{T}$.
Therefore, we have $E_{\gamma}[\|\ket{\psi_{err}}_{\regX,\regZ}\|^2]\leq \frac{6}{T}+\negl(\secpar)$ and the first claim is proven.
\takashi{The constant $6$ is somewhat ugly. Is there a better analysis?}

\paragraph{Proof of the second claim.}
By Eq~\ref{eq:psierrprime}, we can see that 
\begin{align*}
I_{\regC,\regX,\regZ} \ot (\opro{001}{001}_{ph,th,in}+ \opro{011}{011}_{ph,th,in})\ket{\psi'_{err}}=0.
\end{align*}
This immediately implies the second claim.
%
\paragraph{Proof of the third claim.}
By the second claim,  $\ket{0^m}_{\regC}\ket{\psi_{0}}_{\regX,\regZ}\ket{0^t01}_{ph,th,in}$,  $\ket{0^m}_{\regC}\ket{\psi_{1}}_{\regX,\regZ}\ket{0^t11}_{ph,th,in}$, and $\ket{\psi'_{err}}$ are orthogonal with one another.
Therefore we have 
\begin{align*}
&\|G_{i,\gamma}\ket{0^m}_{\regC}\ket{\psi}_{\regX,\regZ}\ket{0^t00}_{ph,th,in}\|^2\\
=&\|z_0\ket{0^m}_{\regC}\ket{\psi_{0}}_{\regX,\regZ}\ket{0^t01}_{ph,th,in}\|^2+\|z_1\ket{0^m}_{\regC}\ket{\psi_{1}}_{\regX,\regZ}\ket{0^t11}_{ph,th,in}\|^2+\|\ket{\psi'_{err}}\|^2.
\end{align*}
Since we have 
$\|G_{i,\gamma}\ket{0^m}_{\regC}\ket{\psi}_{\regX,\regZ}\ket{0^t00}_{ph,th,in}\|^2=\|\ket{\psi}_{\regX,\regZ}\|^2$ and $\|z_b\ket{0^m}_{\regC}\ket{\psi_{b}}_{\regX,\regZ}\ket{0^tb1}_{ph,th,in}\|^2=\|\ket{\psi_{b}}_{\regX,\regZ}\|^2$, the above implies $\|\ket{\psi_{0}}_{\regX,\regZ}\|^2+\|\ket{\psi_{1}}_{\regX,\regZ}\|^2\leq \|\ket{\psi}_{\regX,\regZ}\|^2$, which implies the third claim.
%
\paragraph{Proof of the forth claim.}
By the definition of $\ket{\psi_{mid}}_{\regX,\regZ}$, the state $\ket{0^m}_{\regC}\ket{\psi_{mid}}_{\regX,\regZ}$ is in the subspace $S_{mid}$, which is the subspace spanned by $\{S_j\}_{j:p_j\in (\gamma-2\delta,\gamma)}$.
We define $\ket{\psi''_{mid,s}}_{\regC,\regX,\regZ}$ so that 
\begin{align*}
G_{i,\gamma}\ket{0^m}_{\regC}\ket{\psi_{mid}}\ket{0^t00}_{ph,th,in}=\sum_{s\in \bit^{t+2}} \ket{\psi''_{mid,s}}_{\regC,\regX,\regZ}\ket{s}_{ph,th,in}.
\end{align*}
Since each subspace $S_j$ is invariant under the projections $\Pi_{in}$ and $\Pi_{i,out}$, each $\ket{\psi''_{mid,s}}_{\regC,\regX,\regZ}$ is also in the subspace $S_{mid}$. \takashi{More explanation needed?}
In particular, $\ket{0^m}_{\regC}\ket{\eta_{mid,0}}_{\regX,\regZ}=\ket{\psi''_{mid,0^t01}}_{\regC,\regX,\regZ}$ is in the subspace $S_{mid}$.
%Then by Eq~\ref{eq:etamid} and that $\Pi_{in}S_{mid}$ is spanned by $\{\hat{\alpha}_j\}_{j: p_j\in (\gamma-2\delta,\gamma)}$, we can express $\ket{\eta_{mid,0}}_{\regX,\regZ}$ as a superposition of states $\{\hat{\alpha}_j\}_{j: p_j\in (\gamma-2\delta,\gamma)}$,
%\begin{align*}
%\ket{\eta_{mid,0}}_{\regX,\regZ}=\sum_{j: p_j\in (\gamma-2\delta,\gamma)} d_j \ket{\hat{\alpha}_j}. 
%\end{align*}
Therefore, by Eq. \ref{eq:psizero}, we can write 
\begin{align}
\ket{0^m}_{\regC}\ket{\psi_0}_{\regX,\regZ}=\sum_{j:p_j<\gamma} d_j \ket{\alpha_j}_{\regC,\regX,\regZ}+\sum_{j} d_j^{(10)} \ket{\alpha_j^{(10)}}_{\regC,\regX,\regZ}+\ket{\psi''_{err,0}}_{\regC,\regX,\regZ}  \label{eq:psizero_decompose} 
\end{align}
where $\ket{\psi''_{err,0}}_{\regC,\regX,\regZ}:=\overline{z}_0 \ket{0^m}_{\regC}\ket{\eta_{other,0}}_{\regX,\regZ}$.
We remark that $\|\ket{\psi''_{err,0}}_{\regC,\regX,\regZ}\|=\negl(\secpar)$ since we have $\|\ket{\psi''_{err,0}}_{\regC,\regX,\regZ}\|=\|\ket{\eta_{other,0}}_{\regX,\regZ}\|\leq \|\ket{\psi_{err,\leq \gamma-2\delta}}+\ket{\psi_{err,\geq \gamma}})\|=\negl(\secpar)$ by Eq. \ref{eq:etamid}.

By the definition of $\Pi_{i,out}$, we have 
\begin{align}
 \Pr_{c_{-i}}\left[M_{\regX_i}\circ U\frac{\ket{c_1...c_{i-1}0c_{i+1}...c_m}_{\regC}\ket{\psi_0}_{\regX,\regZ}}{\|\ket{\psi_0}_{\regX,\regZ}\|}\in \Acc_{k_i,y_i}\right]=\frac{\bra{0^m}_{\regC}\bra{\psi_0}_{\regX,\regZ} \Pi_{i,out} \ket{0^m}_{\regC}\ket{\psi_0}_{\regX,\regZ}}{\|\ket{\psi_0}_{\regX,\regZ}\|^2} \label{eq:accept_probability}
\end{align}
where $c_{-i}$ denotes $c_1...c_{i-1}c_{i+1}...c_{m}$.

By Lemma~\ref{lem:decomposition}, we can see that $\bra{\alpha_j}\Pi_{i,out} \ket{\alpha_{j'}}=0$ for all $j\neq j'$ and $\Pi_{i,out} \ket{\alpha_{j}^{(10)}}=0$ for all $j$.
By substituting Eq. \ref{eq:psizero_decompose} for Eq. \ref{eq:accept_probability}, we have
\begin{align*}
&~~~\Pr_{c_{-i}}\left[M_{\regX_i}\circ U\frac{\ket{c_1...c_{i-1}0c_{i+1}...c_m}_{\regC}\ket{\psi_0}_{\regX,\regZ}}{\|\ket{\psi_0}_{\regX,\regZ}\|}\in \Acc_{k_i,y_i}\right]\\
%&=(\sum_{j:p_j<\gamma} \overline{a}_j \bra{\alpha_j}_{\regC,\regX,\regZ}+\bra{\psi''_{err,0}}_{\regC,\regX,\regZ}) \Pi_{i,out} (\sum_{j:p_j<\gamma} a_j \ket{\alpha_j}_{\regC,\regX,\regZ}+\ket{\psi''_{err,0}}_{\regC,\regX,\regZ})\\
&=\frac{1}{\|\ket{\psi_0}_{\regX,\regZ}\|^2}\left(\sum_{j:p_j<\gamma} |d_j|^2 \bra{\alpha_j} \Pi_{i,out} \ket{\alpha_j}+ \sum_{j:p_j<\gamma} (\overline{d}_j \bra{\alpha_j}\Pi_{i,out} \ket{\psi''_{err,0}} + d_j \bra{\psi''_{err,0}}\Pi_{i,out} \ket{\alpha_j})\right)\\
%&\leq \sum_{j:p_j<\gamma} |a_j|^2 \gamma + \negl(\secpar)\\
&\leq \gamma+\negl(\secpar)
\end{align*}
where the last inequality follows from $\sum_{j:p_j<\gamma}|d_j|^2\leq \|\ket{\psi_0}_{\regX,\regZ}\|^2$ and $\|\ket{\psi''_{err,0}}_{\regC,\regX,\regZ}\|=\negl(\secpar)$. 
This immediately implies the forth claim considering that the number of possible $c_{-i}$ is $2^{m-1}$ and $m=O(\log \secpar)$.
%
\paragraph{Proof of the fifth claim.}
By a similar argument to the one in the proof of the forth claim, we can write  
\begin{align}
\ket{0^m}_{\regC}\ket{\psi_1}_{\regX,\regZ}=\sum_{j:p_j>\gamma-2\delta} d_j \ket{\alpha_j}_{\regC,\regX,\regZ}+\sum_{j} `d_j^{(11)} \ket{\alpha_j^{(11)}}_{\regC,\regX,\regZ}+\ket{\psi''_{err,1}}_{\regC,\regX,\regZ}  \label{eq:psione_decompose} 
\end{align}
where $\ket{\psi''_{err,1}}_{\regC,\regX,\regZ}$ is a state such that $\|\ket{\psi''_{err,1}}_{\regC,\regX,\regZ}\|=\negl(\secpar)$.

The algorithm $\ext_{i}$ is described below:

\begin{description}
\item[$\ext_{i}\left(\frac{\ket{0^m}_{\regC}\ket{\psi_1}_{\regX,\regZ}}{\|\ket{\psi_1}_{\regX,\regZ}\|}\right)$:]
Given $\frac{\ket{0^m}_{\regC}\ket{\psi_1}_{\regX,\regZ}}{\|\ket{\psi_1}_{\regX,\regZ}\|}$ as input, $\ext_{i}$ works as follows:
\begin{itemize}
%\item Append $\ket{0^m}_{\regC}$ to prepare $\ket{0^m}_{\regC}\ket{\psi_1}_{\regX,\regZ}$.
\item Repeat the following procedure $N=\poly(\secpar)$ times where $N$ is specified later:
\begin{enumerate}
\item Perform a measurement $\{\Pi_{i,out},I_{\regC,\regX,\regZ}-\Pi_{i,out}\}$. If the outcome is $0$, i,e, $\Pi_{i,out}$ is applied, then measure the register $\regX_i$ in computational basis to obtain $a_i$, outputs $a_i$, and immediately halts.
\item  Perform a measurement $\{\Pi_{in},I_{\regC,\regX,\regZ}-\Pi_{in}\}$.
\end{enumerate}
\item If it does not halts within $N$ trials in the previous step, output $\bot$.
\end{itemize}
\end{description}

By the definition of $\Pi_{i,out}$, it is clear that $\ext_{i}$ succeeds, (i.e., returns $a_i\in \Acc_{k_i,y_i}$) if it does not output $\bot$.
Since the algorithm $\ext_{i}$ just alternately applies measurements $\{\Pi_{i,out},I_{\regC,\regX,\regZ}-\Pi_{i,out}\}$ and $\{\Pi_{in},I_{\regC,\regX,\regZ}-\Pi_{in}\}$ and each subspaces $S_j$ and $T_j^{(11)}$ are invariant under $\Pi_{in}$ and $\Pi_{i,out}$, we can analyze the success probability of the algorithm separately on each subspace.
Therefore, it suffices to show that $\ext_{i}$ succeeds with probability $1-\negl(\secpar)$ on any input $\ket{\alpha_j}_{\regX,\regZ}$ such that  $p_j> \gamma-2\delta$ or $\ket{\alpha_j^{(11)}}$ for any $j$.
First, it is easy to see that on input $\ket{\alpha_j^{(11)}}$, $\ext_{i}$ returns $a_i\in \Acc_{k_i,y_i}$ at the first trial with probability $1$ since we have $\bra{\alpha_{j}^{(11)}} \Pi_{i,out} \ket{\alpha_{j}^{11}}=1$.
What is left is to prove that $\ext_{i}$ succeeds with probability $1-\negl(\secpar)$ on any input $\ket{\alpha_j}_{\regX,\regZ}$ such that  $p_j> \gamma-2\delta$. 

By Lemma~\ref{lem:decomposition}, it is easy to see that we have
\begin{align*}
&\ket{\alpha_j}_{\regX,\regZ}=\sqrt{p_j}\ket{\beta_j}_{\regX,\regZ}+\sqrt{1-p_j}\ket{\beta_j^\bot}_{\regX,\regZ},\\
&\ket{\beta_j}_{\regX,\regZ}=\sqrt{p_j}\ket{\alpha_j}_{\regX,\regZ}+\sqrt{1-p_j}\ket{\alpha_j^\bot}_{\regX,\regZ}.
\end{align*}

Let  $P_k$ and $P_k^\bot$ be the probability that $\ext_i$ succeeds within $k$ trials starting from the initial state $\ket{\alpha_j}_{\regX,\regZ}$   and $\ket{\alpha_j^\bot}_{\regX,\regZ}$, respectively.
Then by the above equations, it is easy to see that we have $P_0=P_0^\bot=0$ and
\begin{align*}
&P_{k+1}=p_j+(1-p_j)^2 P_{k}+ (1-p_j)p_j P_{k}^\bot, \\
&P_{k+1}^\bot=(1-p_j)+ p_j(1-p_j) P_{k}+ p_j^2 P_{k}^\bot.
\end{align*}

Solving this, we have 
\begin{align*}
P_N=1-(1-2p_j+2p_j^2)^{N-1}(1-p_j).
\end{align*}
\takashi{Just a back-of-envelope calculation and not super confident. Should double check. Or is there any better way of analysis?}

Since we assume $p_j> \gamma-2\delta>\frac{\gamma_0}{3T}=1/\poly(\secpar)$, we have $1-2p_j+2p_j^2=1-1/\poly(\secpar)$.
Therefore if we take $N=\poly(\secpar)$ sufficiently large, then $P_N=1-\negl(\secpar)$.
This means that $\ext_i$ succeeds within $N$ steps with probability $1-\negl(\secpar)$ starting from the initial state $\ket{\alpha_j}_{\regX,\regZ}$.
This completes the proof of the fifth claim and the proof of Lemma~\ref{lem:partition}. 
%\end{align*}
\end{proof}

%\begin{remark}\label{remark:one_half}
%Note that $\ket{\psi_0}$ and $\ket{\psi_1}$ may not be orthogonal. However, $\|\ket{\psi_0}\|^2+ \|\ket{\psi_1}\|^2\leq 1$ since $\|\ket{0^m}_{\regC}\ket{\psi_{0}}_{\regX,\regZ}\ket{001}+ \ket{0^m}_{\regC}\ket{\psi_{1}}_{\regX,\regZ}\ket{011} + \ket{\psi_{err}}\|^2 =1$ and  $\ket{0^m}_{\regC}\ket{\psi_{0}}_{\regX,\regZ}\ket{001}$ and $ \ket{0^m}_{\regC}\ket{\psi_{1}}_{\regX,\regZ}\ket{011}$ are orthogonal. This implies that 
%\begin{align*}
%    E_{c_i}[ \|\ket{\psi_{c_i}}\|^2 ] \leq 1/2
%\end{align*}
%\end{remark}

In Lemma~\ref{lem:partition}, we showed that by fixing any $i\in [m]$, we can partition any prover's state $\ket{\psi}_{\regX,\regZ}$ into $\ket{\psi_0}_{\regX,\regZ}$, $\ket{\psi_1}_{\regX,\regZ}$, and $\ket{\psi_{err}}_{\regX,\regZ}$ with certain properties. %such that $\ket{\psi_0}$ and $\ket{\psi_1}$ will be rejected and accepted in the test round with high probability.
%In the following, we show another procedure that further decompose the prover's state according to any given $c\in \{0,1\}^n$. 
In the following, we sequentially apply Lemma~\ref{lem:partition} for each $i\in[m]$ to further decompose the prover's state.




\begin{lemma}\label{lem:partition_further}
%Fix $c\in \{0,1\}^m$. 
Let $m$, $\gamma_0$, $T$ be as in Lemma~\ref{lem:partition}, and let $\gamma_i\sample [\frac{\gamma_0}{T},\frac{2\gamma_0}{T},\dots,\frac{T\gamma_0}{T}]$ for each $i\in [m]$.
For any $c\in \bit^m$, a state $\ket{\psi}_{\regX,\regZ}$ can be partitioned as follows.% by using Procedure~\ref{fig:process_H}. 
\begin{align*}
    & \ket{\psi}_{\regX,\regZ} = \ket{\psi_{c_1}}_{\regX,\regZ} + \ket{\psi_{\bar{c}_1,c_2}}_{\regX,\regZ} + \cdots +\ket{\psi_{\bar{c}_1,\dots,\bar{c}_{m-1},c_m}}_{\regX,\regZ} + \ket{\psi_{\bar{c}_1,\dots,\bar{c}_m}}_{\regX,\regZ}+ \ket{\psi_{err}}_{\regX,\regZ}
\end{align*}
where the way of partition may depend on the choice of $\hat{\gamma}=\gamma_1...\gamma_m$.
Further, the following properties are satisfied. 
\begin{enumerate}
    \item For any fixed $\hat{\gamma}$ and any $c$, $i\in [m]$ such that $c_i=0$, we have 
    \begin{align*}
    \Pr\left[M_{\regX_i}\circ U \frac{\ket{0^m}_{\regC}\ket{\psi_{\bar{c}_1,\dots,\bar{c}_{i-1},0}}_{\regX,\regZ}}{|\ket{\psi_{\bar{c}_1,\dots,\bar{c}_{i-1},0}}_{\regX,\regZ}|}\in \Acc_{k_i,y_i}\right]\leq 2^{m-1}\gamma_0+ \negl(\secpar).
    \end{align*}
    
    \item For any fixed $\hat{\gamma}$ and any $c$, $i\in[m]$ such that $c_i=1$, there exists an efficient algorithm $\ext_i$ such that 
    \begin{align*}  
  \Pr\left[\ext_i\left(\frac{\ket{0^m}_{\regC}\ket{\psi_{\bar{c}_1,\dots,\bar{c}_{i-1},1}}_{\regX,\regZ}}{\|\ket{\psi_{\bar{c}_1,\dots,\bar{c}_{i-1},1}}_{\regX,\regZ}\|}\right)\in \Acc_{k_i,y_i}\right]=1-\negl(\secpar).
  \end{align*}   
  \item For any fixed $\hat{\gamma}$, we have $E_c[\|\ket{\psi_{\bar{c}_1,\dots,\bar{c}_m}}_{\regX,\regZ}\|^2] \leq 2^{-m}$.
\item For any fixed $c$, we have $E_{\hat{\gamma}}[\|\ket{\psi_{err}}_{\regX,\regZ}\|^2]\leq \frac{6m^2}{T}+\negl(\secpar)$.
    \item For any fixed $\hat{\gamma}$ and $c$ there exists an efficient quantum algorithm $H_{\hat{\gamma},c}$ that is given $\ket{\psi}_{\regX,\regZ}$ as input and produces  $\frac{\ket{\psi_{\bar{c}_1,\dots,\bar{c}_{i-1},c_i}}_{\regX,\regZ}}{\|\ket{\psi_{\bar{c}_1,\dots,\bar{c}_{i-1},c_i}}_{\regX,\regZ}\|}$ with probability $\|\ket{\psi_{\bar{c}_1,\dots,\bar{c}_{i-1},c_i}}_{\regX,\regZ}\|^2$ ignoring a global phase factor.
%    \item Measuring the register $(ph_1,th_1,in_1),\dots,(ph_i,th_i,in_i)$ gives error (i.e., $\exists i$, such that $(ph_i,th_i,in_i)\neq (0^t,0,1)$ or $(0^t,1,1)$) with probability at most $\frac{m}{T}$. 
 %   \item For any fixed $\gamma$, $E_c[\|\ket{\psi_{\bar{c}_1,\dots,\bar{c}_m}}\|^2] \leq 2^{-m}$.
\end{enumerate}
\end{lemma}
\begin{proof}
We inductively define $\ket{\psi_{c_1}}_{\regX,\regZ}$,...,$\ket{\psi_{\bar{c}_1,...,\bar{c}_m}}_{\regX,\regZ}$ as follows.

First, we apply Lemma \ref{lem:partition} for the state $\ket{\psi}_{\regX,\regZ}$ with $\gamma=\gamma_1$ to give a decomposition
\begin{align*}
\ket{\psi}_{\regX,\regZ}=\ket{\psi_0}_{\regX,\regZ}+\ket{\psi_1}_{\regX,\regZ} + \ket{\psi_{err,1}}_{\regX,\regZ} 
\end{align*}
where $\ket{\psi_{err,1}}_{\regX,\regZ}$ corresponds to $\ket{\psi_{err}}_{\regX,\regZ}$ in Lemma \ref{lem:partition}.

For each $i=2,...,m$, we apply  Lemma \ref{lem:partition} for the state $\ket{\psi_{\bar{c}_1,...,\bar{c}_{i-1}}}_{\regX,\regZ}$ with $\gamma=\gamma_i$ to give a decomposition 
\begin{align*}
\ket{\psi_{\bar{c}_1,...,\bar{c}_{i-1}}}_{\regX,\regZ}=\ket{\psi_{\bar{c}_1,...,\bar{c}_{i-1},0}}_{\regX,\regZ}+\ket{\psi_{\bar{c}_1,...,\bar{c}_{i-1},1}}_{\regX,\regZ} + \ket{\psi_{err,i}}_{\regX,\regZ} 
\end{align*}  
where 
$\ket{\psi_{\bar{c}_1,...,\bar{c}_{i-1},0}}_{\regX,\regZ}$, $\ket{\psi_{\bar{c}_1,...,\bar{c}_{i-1},1}}_{\regX,\regZ}$, and $\ket{\psi_{err,i}}_{\regX,\regZ}$ corresponds to $\ket{\psi_{0}}_{\regX,\regZ}$, $\ket{\psi_{1}}_{\regX,\regZ}$, and $\ket{\psi_{err}}_{\regX,\regZ}$ in Lemma \ref{lem:partition}, respectively.
  
 Then it is easy to see that we have
 \begin{align*}
    & \ket{\psi}_{\regX,\regZ} = \ket{\psi_{c_1}}_{\regX,\regZ} + \ket{\psi_{\bar{c}_1,c_2}}_{\regX,\regZ} + \cdots +\ket{\psi_{\bar{c}_1,\dots,\bar{c}_{m-1},c_m}}_{\regX,\regZ} + \ket{\psi_{\bar{c}_1,\dots,\bar{c}_m}}_{\regX,\regZ}+ \ket{\psi_{err}}_{\regX,\regZ}
\end{align*}
where we define $\ket{\psi_{err}}_{\regX,\regZ}\defeq \sum_{i=1}^{m}\ket{\psi_{err,i}}_{\regX,\regZ}$. 
  
The first and second claims immediately follow from the forth and fifth claims of Lemma~\ref{lem:partition} and $\gamma_i\leq \gamma_0$ for each $i\in[m]$.  

By the third claim of  Lemma~\ref{lem:partition}, we have $E_{c_1...c_{i}}[\|\ket{\psi_{\bar{c}_1,...,\bar{c}_{i}}}_{\regX,\regZ}\|]\leq \frac{1}{2}E_{c_1...c_{i-1}}[\|\ket{\psi_{\bar{c}_1,...,\bar{c}_{i-1}}}_{\regX,\regZ}\|]$.
Ths implies the third claim.

 
By the first claim of  Lemma~\ref{lem:partition}, we have $E_{\gamma_i}[\|\ket{\psi_{err,i}}_{\regX,\regZ}\|^2]\leq \frac{6}{T}+\negl(\secpar)$.
The forth claim follows from this and the Cauchy-Schwarz inequality. 

Finally, for proving the fifth claim, we define the procedure $H_{\hat{\gamma},c}$ as described in Procedure~\ref{fig:process_H}
We can easily see that $H_{\hat{\gamma},c}$ satisfies the desired property by the second claim of  Lemma~\ref{lem:partition}.

\floatname{algorithm}{Procedure}
\begin{algorithm}[h]
    \begin{mdframed}[style=figstyle,innerleftmargin=10pt,innerrightmargin=10pt]
   On input $\ket{\psi}_{\regX,\regZ}$, it works as follows:
   
   For each $i=1,...,m$, it applies 
    \begin{enumerate}
    \item Prepare registers $\regC$, $(ph_1,th_1,in_1)$,..., $(ph_m,th_m,in_m)$ all of which are initialized to be $\ket{0}$.
    \item For each $i=1,...,m$, do the following: 
      \begin{enumerate}
      \item Apply $G_{i,\gamma_i}$ on the quantum state in the registers $(\regC,\regX,\regZ,ph_i,th_i,in_i)$.
      \item Measure the registers $(ph_i,th_i,in_i)$ in the computational basis.
      \item If the outcome is $0^tc_{i}1$, then it halts and returns the state in the register $(\regX,\regZ)$. If the outcome is $0^t\bar{c}_{i}1$, continue to run. Otherwise, immediately halt and abort.
      \end{enumerate}   
    \end{enumerate}

    \caption{$H_{\hat{\gamma},c}$}
    \label{fig:process_H}
    \end{mdframed}
\end{algorithm}
\end{proof}

Given Lemma~\ref{lem:partition_further}, we can start proving Theorem~\ref{thm:rep_soundness}. 

\begin{proof}[Proof of Theorem~\ref{thm:rep_soundness}]

%According to Lemma~\ref{lem:Mah_soundness}, we know that
%\begin{align*}
 %   \Pr_{k,y}[U_{0}\ket{\psi(k,y)}\mbox{ wins test round}]\geq 1-\negl(n) \Rightarrow \Pr_{k,y}[U_{0}\ket{\psi(k,y)}\mbox{ wins Hadamard round}]\leq \negl(n),   
%\end{align*}
%Here,  
First, we recall how a cheating prover characterized by $(U_0,U)$ works.
When the first message $k$ is given, it first applies 
\begin{align*}
    &U_0\ket{0}_{\regX,\regZ}\ket{0}_{\regY}\ket{k}_{\regK} \xrightarrow{\mbox{measure }\regY} \ket{\psi(k,y)}_{\regX,\regZ}\ket{k}_{\regK}.
\end{align*}
to generate the second message $y$ and $\ket{\psi(k,y)}_{\regX,\regZ}$.
Then after receiving the third message $c$, it applies $U$ on $\ket{c}_{\regC}\ket{\psi(k,y)}_{\regX,\regZ}$ and measures the register $\regX$ in the computational basis to obtain the forth message $a$.
In the following, we just write $\ket{\psi}_{\regX,\regZ}$ to mean $\ket{\psi(k,y)}_{\regX,\regZ}$ for notational simplicity.
%let $V_{i,b}$ be a unitary over $\hil_{\regC,\regX,\regZ}$ that runs the verification procedure for $c_i=b$ on the $i$-th coordinate and write the verification result in a designated register (say, $i$-th qubit of $\regZ$), and $M_i$ be the measurement on the designated register that contains the verification result on the $i$-th coordinate.
%Note that $V_{i,1}$ cannot be applied publicly without knowing the trapdoor,  but this does not affect our analysis below.
Let $M_{i,k_i,\td_i,y_i,c_i}$ be the measurement that outputs the verification result of the value in the register $\regX_i$ w.r.t.  $k_i,\td_i,y_i,c_i$, and let $M_{k,\td,y,c}$ be the measurement that returns $\top$ if and only if $M_{i,k_i,\td_i,y_i,c_i}$ returns $\top$ for all $i\in[m]$ where $k=(k_1,...,k_m)$, $\td=(\td_1,...,\td_m)$, $y=(y_1,...,y_m)$ and $c=(c_1,...,c_m)$.
%For any state $\ket{\phi}_{\regC,\regX,\regZ}$, we denote $M\circ \ket{\phi}_{\regC,\regX,\regZ}=\top$ to mean $M_i\circ \ket{\phi}_{\regC,\regX,\regZ}=\top$ for all $i\in[m]$ for notational simplicity. 
With this notation, a cheating prover's success probability can be written as 
\begin{align*}
    \Pr_{k,\td,y,c}[M_{k,\td,y,c}U\ket{c}_{\regC}\ket{\psi}_{\regX,\regZ} = \top].
\end{align*}

Let $\gamma_0$, $\hat{\gamma}$, and $T$ be as in Lemma~\ref{lem:partition_further}.
According to Lemma~\ref{lem:partition_further}, for any fixed $\hat{\gamma}$ and $c\in \bit^{m}$, we can decompose $\ket{\psi}_{\regX,\regZ}$ as 
\begin{align*}
    \ket{\psi}_{\regX,\regZ} =  \ket{\psi_{c_1}}_{\regX,\regZ}+ \ket{\psi_{\bar{c}_1,c_2}}_{\regX,\regZ} + \cdots + \ket{\psi_{\bar{c}_1,\dots, \bar{c}_{m-1},c_{m}}}_{\regX,\regZ} + \ket{\psi_{\bar{c}_1,\dots, \bar{c}_{m-1},\bar{c}_{m}}}_{\regX,\regZ}+ \ket{\psi_{err}}_{\regX,\regZ}.
\end{align*}

To prove the theorem, we show the following two inequalities.
First,  for any  fixed $\hat{\gamma}$, $i\in[m]$, $c\in \bit^{m}$ such that $c_i=0$, $k_i$, $\td_i$, and $y_i$, we have
\begin{align}
 \Pr\left[M_{i,k_i,\td_i,y_i,0} \circ \frac{U\ket{c}_{\regC}\ket{\psi_{\bar{c}_1,\ldots,\bar{c}_{i-1},0}}_{\regX,\regZ}}{\|\ket{\psi_{\bar{c}_1,\ldots,\bar{c}_{i-1},0}}_{\regX,\regZ}\|}=\top\right]\leq 2^{m-1}\gamma_0+\negl(\secpar). \label{eq:Test}
\end{align}
This easily follows from the first claim of Lemma~\ref{lem:partition_further}

Second, for any  fixed $\hat{\gamma}$, $i\in[m]$, and $c\in \bit^{m}$ such that $c_i=1$,
\begin{align}
    \underset{k,\td,y}{E}\left[\|\ket{\psi_{\bar{c}_1,\dots,\bar{c}_{i-1},1}}_{\regX,\regZ}\|^2\Pr\left[M_{i,k_i,\td_i,y_i,1}\circ U\frac{\ket{c}_{\regC}\ket{\psi_{\bar{c}_1,\dots,\bar{c}_{i-1},1}}_{\regX,\regZ}}{\|\ket{\psi_{\bar{c}_1,\dots,\bar{c}_{i-1},1}}_{\regX,\regZ}\|} = \top\right]\right] = \negl(n) \label{eq:Hada}
\end{align}
assuming the quantum hardness of LWE problem.

For proving Eq.~\ref{eq:Hada}, we consider a cheating prover against the original Mahadev's protocol on the $i$-th corrdinate described below:

\begin{enumerate}
    \item Given $k_i$, it picks $k_{-i}=k_1...k_{i-1},k_{i+1},...,k_{m}$ as in the protocol and computes $U_0\ket{0}_{\regX,\regZ}\ket{0}_{\regY}\ket{k}_{\regK}$ and measure the register $\regY$ to obtain $y=(y_1,...,y_m)$ along with the corresponding state $\ket{\psi}_{\regX,\regZ}=\ket{\psi(k,y)}_{\regX,\regZ}$.
    \item Apply $H_{\hat{\gamma},c}$ to generate the state $\frac{\ket{\psi_{\bar{c}_1,\dots,\bar{c}_{i-1},1}}_{\regX,\regZ}}{\|\ket{\psi_{\bar{c}_1,\dots,\bar{c}_{i-1},1}}_{\regX,\regZ}\|}$, which succeeds with probability $\|\ket{\psi_{\bar{c}_1,\dots,\bar{c}_{i-1},1}}_{\regX,\regZ}\|^2$ (ignoring a global phase factor).
    We denote by $\Succ$ the event that it succeeds in generating the state.
    If it fails to generate the state, then it overrides $y_i$ by picking it in a way such that it can pass the test round with probability $1$, which can be done according to Fact~\ref{fact:perfectly_pass_test}.
    Then it sends $y_i$ to the verifier.
    \item Given a challenge $c'_i$, it works as follows:
    \begin{itemize}
     \item When $c'_i=0$ (i.e., Test round), if $\Succ$ occurred, then it runs $\ext_i$ in the second claim of Lemma~\ref{lem:partition_further} on input $\frac{\ket{0^m}_{\regC}\ket{\psi_{\bar{c}_1,\dots,\bar{c}_{i-1},1}}_{\regX,\regZ}}{\|\ket{\psi_{\bar{c}_1,\dots,\bar{c}_{i-1},1}}_{\regX,\regZ}\|}$ to generate a forth message accepted with probability $1-\negl(\secpar)$. 
     If $\Succ$ did not occur, then it returns a forth message accepted with probability $1$, which is possible by Fact~\ref{fact:perfectly_pass_test}.
    \item When $c'_i=1$ (i.e., Hadamard round), if $\Succ$ occurred, then it computes  $U\frac{\ket{c}_{\regC}\ket{\psi_{\bar{c}_1,\dots,\bar{c}_{i-1},1}}_{\regX,\regZ}}{\|\ket{\psi_{\bar{c}_1,\dots,\bar{c}_{i-1},1}}_{\regX,\regZ}\|}$ and measure the register $\regX_i$ to obtain the forth message $a_i$.
    If $\Succ$ did not occur, it just aborts.
    \end{itemize}
\end{enumerate}
Then we can see that this cheating adversary passes the test round with overwhelming probability and passes the Hadamard round with the probability equal to the LHS of Eq.~\ref{eq:Hada}.
Therefore, Eq.~\ref{eq:Hada} follows from Lemma~\ref{lem:Mah_soundness} assuming the quantum hardness of LWE problem.

Now, we are ready to prove the theorem. 
As remarked at the beginning of Sec. \ref{sec:proof_of_soundness}, it suffices to show that for any $\mu=1/\poly(n)$, there exists $m=O(\log(n))$ such that the success probability of the cheating prover is at most $\mu$.
Here we set $m = \log \frac{1}{\mu^2}$, $\gamma_0 = 2^{-2m}$, and $T=2^{m}$. 
Note that this parameter setting satisfies the requirement for Lemma~\ref{lem:partition_further} since
$m=\log \frac{1}{\mu^2}=\log (\poly(\secpar))=O(\log \secpar)$ and
$\frac{\gamma_0}{T}=2^{-3m}=\mu^{6}=1/\poly(\secpar)$.
Then we have
\begin{align*}
    &\Pr_{k,\td,y,c}\left[M_{k,\td,y,c}\circ U\ket{c}_{\regC}\ket{\psi}_{\regX,\regZ}= \top\right] \\
     &=\Pr_{k,\td,y,c,\hat{\gamma}}\left[M_{k,\td,y,c}\circ U\ket{c}_{\regC}\left(\sum_{i=1}^{m}\ket{\psi_{\bar{c}_1,\dots,\bar{c}_{i-1},c_i}}_{\regX,\regZ} + \ket{\psi_{\bar{c}_1,\dots,\bar{c}_m}}_{\regX,\regZ}+ \ket{\psi_{err}}_{\regX,\regZ}\right) = \top\right] \\
    &\leq (m+2) \underset{k,\td,y,c,\hat{\gamma}}{E}\Biggl[\sum_{i=1}^{m} \|\ket{\psi_{\bar{c}_1,\dots,\bar{c}_{i-1},c_i}}_{\regX,\regZ}\|^2\Pr\left[M_{k,\td,y,c}\circ U\frac{\ket{c}_{\regC}\ket{\psi_{\bar{c}_1,\dots,\bar{c}_{i-1},c_i}}_{\regX,\regZ}}{\|\ket{\psi_{\bar{c}_1,\dots,\bar{c}_{i-1},c_i}}_{\regX,\regZ}\|}=\top\right]\\
   &+\|\ket{\psi_{\bar{c}_1,\dots,\bar{c}_m}}_{\regX,\regZ}\|^2\Pr\left[M_{k,\td,y,c}\circ U\frac{\ket{c}_{\regC}\ket{\psi_{\bar{c}_1,\dots,\bar{c}_m}}_{\regX,\regZ}}{\|\ket{\psi_{\bar{c}_1,\dots,\bar{c}_m}}_{\regX,\regZ}} =\top\right]\\
    &+ \|\ket{\psi_{err}}_{\regX,\regZ}\|^2\Pr\left[M_{k,\td,y,c}\circ U\frac{\ket{c}_{\regC} \ket{\psi_{err}}_{\regX,\regZ}}{\|\ket{\psi_{err}}_{\regX,\regZ}\|}=\top\right]\Biggr]\\
        &\leq (m+2) \underset{k,\td,y,c,\hat{\gamma}}{E}\Biggl[\sum_{i=1}^{m} \|\ket{\psi_{\bar{c}_1,\dots,\bar{c}_{i-1},c_i}}_{\regX,\regZ}\|^2\Pr\left[M_{i,k_i,\td_i,y_i,c_i}\circ U\frac{\ket{c}_{\regC}\ket{\psi_{\bar{c}_1,\dots,\bar{c}_{i-1},c_i}}_{\regX,\regZ}}{\|\ket{\psi_{\bar{c}_1,\dots,\bar{c}_{i-1},c_i}}_{\regX,\regZ}\|}=\top\right]\\
   &+\|\ket{\psi_{\bar{c}_1,\dots,\bar{c}_m}}_{\regX,\regZ}\|^2+ \|\ket{\psi_{err}}_{\regX,\regZ}\|^2 \Biggr]\\
    &\leq (m+2)(m(2^{m-1}\gamma_0 +\negl(n))+ 2^{-m} + \frac{m^2}{T}+\negl(\secpar)) \\
    & \leq \mathsf{poly}(\log \mu^{-1}) \mu^{2}+\negl(\secpar). 
\end{align*}
The first equation follows from Lemma \ref{lem:partition_further}. The first inequality follows from the Cauchy-Schwarz inequality.
\takashi{More explanation may be useful.}
 The second inequality holds since considering the verification on a particular coordinate just increases the acceptance probability and probabilities are at most $1$.
The third inequality follows from Eq.~\ref{eq:Test} and \ref{eq:Hada}, which give an upper bound of the first term and Lemma~\ref{lem:partition_further}, which gives upper bounds of the second and third terms.
The last inequality follows from our choices of $\gamma_0$, $T$, and $m$.
For sufficiently large $\secpar$,  this can be upper bounded by $\mu$.
\end{proof}

%---------------------Older  Proof--------------------------------
\begin{comment}
According to Lemma~\ref{lem:partition_further}, we can decompose any state $\ket{\psi}$ above as 
\begin{align*}
    &\ket{\psi} = \ket{\psi_{c_1}} + \ket{\psi_{\bar{c}_1,c_2}} + \cdots +\ket{\psi_{\bar{c}_1,\dots,c_m}} + \ket{\psi_{\bar{c}_1,\dots,\bar{c}_m}}+ \ket{\psi_{err}}.
\end{align*}
To prove the theorem, we need to show that 
\begin{align*}
    \Pr[M\circ(V_{i,H}U_c\ket{\psi_{\bar{c}_1,\dots,\bar{c}_{i-1},1}}) = accept] = \negl(n), 
\end{align*}
where $V_{i,H}$ is the verification procedure in the $i$th round and $M$ is the measurement to check whether the prover wins the Hadamard round. 

When $i=0$, $\ket{\psi} = \ket{\psi_0}+ \ket{\psi_1}+\ket{\psi_{err_1}}$, and $\ket{\psi_1}$ wins the test round with high probability. We prove that the prover with internal state $\ket{\psi_1}$ wins the Hadamard round with only negligible probability by contradiction. Suppose $\ket{\psi_1}$ wins the Hadamard round with noticeable probability. Then, we can construct the following attack for $\ket{\psi}$ such that Lemma~\ref{lem:Mah_soundness} fails. Without loss of generality, we can assume $\|\ket{\psi_1}\|^2\geq 1/poly(n)$. The prover first applies the corresponding $G_{1,\gamma,\delta}$ and measure the register $ph,th,in$ to obtain $\ket{\psi_1}$ with noticeable probability, which implies that the prover can win the test round with noticeable probability by Lemma~\ref{lem:partition}. Then, consider the Hadamard round, if the prover does not obtain $\ket{\psi_1}$ from $G_{1,\gamma,\delta}$, it just randomly outputs $u,d$ to the verifier; otherwise, if the prover obtains $\ket{\psi_1}$, based on our hypothesis, it can win with noticeable probability. Overall, the prover can win the Hadamard round with noticeable probability, which violates Lemma~\ref{lem:Mah_soundness}. Therefore, the prover with internal state $\ket{\psi_1}$ wins the Hadamard round with only negligible probability.   

We can decompose $\ket{\psi}$ by using Procedure~\ref{fig:process_H}
\begin{align*}
    \ket{\psi} =  \ket{\psi_{c_1}}+ \ket{\psi_{\bar{c}_1,c_2}} + \cdots + \ket{\psi_{\bar{c}_1,\dots, \bar{c}_{m-1},c_{m}}} + \ket{\psi_{\bar{c}_1,\dots, \bar{c}_{m-1},\bar{c}_{m}}}+ \ket{\psi_{err}}.
\end{align*}
Similar to the case with only one trial, we can show that the prover with internal state $\ket{\psi_{\bar{c}_1,\dots, \bar{c}_{m-1},1}}$ wins the Hadamard round with negligible probability by contradiction.
Suppose the prover with $\ket{\psi_{\bar{c}_1,\dots, \bar{c}_{m-1},0}}$ can win the Hadamard round with noticeable probability. Without loss of generality, $\|\ket{\psi_{\bar{c}_1,\dots, \bar{c}_{m-1},1}}\|^2>1/\poly(n)$. This implies that the prover has noticeable probability to obtain $\ket{\psi_{\bar{c}_1,\dots, \bar{c}_{m-1},1}}$ and thus it can win the test round with high probability. Then, in the Hadamard round, the prover can again use $H_c$ to obtain $\ket{\psi_{\bar{c}_1,\dots, \bar{c}_{m-1},1}}$ and win the Hadamard round with noticeable probability, which fails Lemma~\ref{lem:Mah_soundness}. Hence, the prover with $\ket{\psi_{\bar{c}_1,\dots, \bar{c}_{m-1},1}}$ can only win the Hadamard wound with negligible probability. 


Now, we are ready to prove the theorem by using contradiction. Let $m' = \log^2n$. We suppose that there exists a prover can win with probability $\mu=1/\poly(n)$. We choose $m = \log \frac{1}{\mu^2}$, $\gamma_0 = 2^{-2m}$, and $T=2^{-m}$. Then, we choose the first $m$ trials to do parallel repetition and show that by our choices of parameters, the prover can only succeed with probability less than $\mu$. Note that the verifier has $c_1,\dots,c_m$ be chosen uniformly independently. Hence, $c_{m+1},\dots,c_{m'}$ can be viewed as some redundant information uncorrelated to $c_1,\dots,c_m$ given to the prover, which does not change our analysis in Lemma~\ref{lem:Mah_soundness}, Lemma~\ref{lem:partition}, and Lemma~\ref{lem:partition_further}. Let the verifier's verification be $V_{1,c_1},\dots,V_{m,c_m}$, where $V_{i,0}$ is doing the test round in the $i$th trial and $V_{i,1}$ is doing the Hadamard round. Then, 
\begin{align}
    &\Pr[M\circ\left(U_c\ket{\psi}\right) = accept] \\
    &\leq (m+2)(\Pr[M\circ\left(U_c\ket{\psi_{c_1}}\right)=accept] \\
    &+ \Pr[M\circ\left(U_c\ket{\psi_{\bar{c}_1,c_2}}\right)=accept]\\
    &+\cdots+\Pr[M\circ\left(U_c\ket{\psi_{\bar{c}_1,\dots,c_m}}\right)=accept]\label{eq:last_1}\\
    &+\Pr[M\circ\left(U_c\ket{\psi_{\bar{c}_1,\dots,\bar{c}_m}}\right)=accept] \label{eq:last_2}\\
    &+ \Pr[M\circ\left(U_c\ket{\psi_{err}}\right)=accept])\\
    &\leq (m2^m\gamma_0 + 2^{-m} + \frac{m}{T})(m+2) \leq \mu. 
\end{align}
The first inequality follows from the Cauchy-Schwarz inequality. The second inequality follows from the fact that $V_1,\dots,V_m$ commute, and thus we can choose $V_{i,c_i}$ to be the first operator operating on $\ket{\psi_{\bar{c}_1,\dots,\bar{c}_{i-1},c_{i}}}$; then, by our analysis, the probability that the prover can win is at most $2^m\gamma_0$. The states considered in Eq.~\ref{eq:last_1} and Eq.~\ref{eq:last_2} have norm at most $1/2^m$ and $m/T$ according to Lemma~\ref{lem:partition_further}. The last inequality follows from our choices of $\gamma_0$, $T$, and $m$.

For all noticeable $\mu$, we can find corresponding $m$, $\gamma_0$, and $T$ such that the prover can only win with probability less than $\mu$. Therefore, the probability the prover can win the test can only be negligible when $m=\poly(n)$.  
\end{comment}
\section{Two-Round Protocol via Fiat-Shamir Transform}\label{sec:tworound}
In this section, we show that if we apply the Fiat-Shamir transform to $m$-parallel version of the Mahadev's protocol, then we obtain two-round protocol in the QROM. 
That is, we prove the following theorem.
\begin{theorem}\label{thm:MahFS}
Assuming LWE assumption, there exists a two-round CVQC protocol with overwhelming completeness and negligible soundness error in the QROM.
\end{theorem}

\begin{proof}%(of Theorem~\ref{thm:MahFS})
Let $m>\secpar$ be a sufficiently large integer so that $m$-parallel version of the Mahadev's protocol has negligible soundness.
For notational simplicity, we abuse the notation to simply use $\ver_{i}$, $\pro_{i}$, and $\ver_{\out}$ to mean the $m$-parallel repetitions of them. 
Let $H:\calY\ra \bit^{m}$ be a hash function idealized as a quantum random oracle where $\calX$ is the space of the second message $y$ and $\calY=\bit^{m}$.
Our two-round protocol is described below:
\begin{description}
\item[First Message:] The verifier runs $\ver_1$ to generate $(k,\td)$. Then it sends $k$ to the prover and keeps $\td$ as its state.
\item[Second Message:] The prover runs $\pro_2$ on input $k$ to generate $y$ along with the prover's state $\ket{\st_\pro}$. Then set $c\defeq H(y)$, and runs $\pro_4$ on input $\ket{\st_\pro}$ and $y$ to generate $a$. Finally, it returns $(y,a)$ to the verifier.
\item[Verification:] The verifier computes $c=H(y)$, runs $\ver_{\out}(k,\td,y,c,a)$, and outputs as $\ver_{\out}$ outputs.
\end{description}

It is clear that the completeness is preserved given that $H$ is a random oracle.
We reduce the soundness of this protocol to the soundness of $m$-parallel version of the Mahadev's protocol.
For proving this, we borrow the following lemma shown in \cite{C:DFMS19}.

%\begin{lemma}[\cite{C:DFMS19}]\label{lem:FS}
%Let $\calY$ be finite non-empty sets. There exists a black-box polynomial-time two-stage quantum algorithm $\calS$ with the following property. Let $\A$ be an arbitrary oracle quantum algorithm that makes $q$ queries to a uniformly random $H:\calY\ra \bit^{m}$ and that outputs some $y\in \calY$ and output $a$. 
%Then, the two-stage algorithm $\calS^{\A}$ outputs $y\in\calY$ in the first stage
%and, upon a random $c^*\in \bit^{m}$ as input to the second stage, output $a$ so that for any $x_\circ\in \calX$ and any predicate $V$:
%\begin{align*}
%    \Pr_{c^*}\left[y=y_\circ \land V(y,c^*,a):(y,a)\sample \langle\calS^{\A},c^* \rangle \right]\lapprox \frac{1}{O(q^2)}\Pr_{H}\left[y=y_{\circ}\land V(y,H(y),a):(y,a)\sample \A^H\right],
%\end{align*}
%where 
%$(y,a)\sample \langle\calS^{\A},c^* \rangle$ means that $\calS^{\A}$ outputs $y$ and $a$ in the first and second stages respectively on the second stage input $c^*$, and  $\lapprox$ hides a term that is bounded by $\frac{1}{2^{m+1}q}$ when summed over all $y_{\circ}\in \calY$.
%\end{lemma}

\begin{lemma}[{\cite[Theorem 2]{C:DFMS19}}]\label{lem:FS}
Let $\calY$ be finite non-empty sets. There exists a black-box polynomial-time two-stage quantum algorithm $\calS$ with the following property. Let $\A$ be an arbitrary oracle quantum algorithm that makes $q$ queries to a uniformly random $H:\calY\ra \bit^{m}$ and that outputs some $y\in \calY$ and output $a$. 
Then, the two-stage algorithm $\calS^{\A}$ outputs $y\in\calY$ in the first stage
and, upon a random $c\in \bit^{m}$ as input to the second stage, output $a$ so that for any $x_\circ\in \calX$ and any predicate $V$:
\begin{align*}
    \Pr_{c}\left[V(y,c,a):(y,a)\sample \langle\calS^{\A},c \rangle \right]\leq \frac{1}{O(q^2)}\Pr_{H}\left[V(y,H(y),a):(y,a)\sample \A^H\right]-\frac{1}{2^{m+1}q},
\end{align*}
where 
$(y,a)\sample \langle\calS^{\A},c \rangle$ means that $\calS^{\A}$ outputs $y$ and $a$ in the first and second stages respectively on the second stage input $c$.
\end{lemma}

We assume that there exists an efficient adversary $\A$ that breaks the soundness of the above two-round protocol.
We fix $x\notin \lang$ on which $\A$ succeeds in cheating.
We fix $(k,\td)$ that is in the support of the verifier's first message.
We apply Lemma~\ref{lem:FS} for $\A=\A(k)$ and $V=\ver_\out(k,\td,\cdot,\cdot,\cdot)$, to obtain an algorithm $\calS^{\A(k)}$ that satisfies
\begin{align*}
    &\Pr_{c}\left[V_{\out}(k,\td,y,c,a):(y,a)\sample \langle\calS^{\A(k)},c \rangle \right]\\
    \leq &\frac{1}{O(q^2)}\Pr_{H}\left[V_{\out}(k,\td,y,H(y),a):(y,a)\sample \A^H(k)\right]-\frac{1}{2^{m+1}q}.
\end{align*}
Averaging over all possible $(k,\td)$, we have
\begin{align*}
    &\Pr_{k,\td,c}\left[V_{\out}(k,\td,y,c,a):(y,a)\sample \langle\calS^{\A(k)},c \rangle \right]\\
    \leq &\frac{1}{O(q^2)}\Pr_{k,\td,H}\left[V_{\out}(k,\td,y,H(y),a):(y,a)\sample \A^H(k)\right]-\frac{1}{2^{m+1}q}.
\end{align*}
Since we assume that $\A$ breaks the soundness of the above two-round protocol,
\[
\Pr_{k,\td,H}\left[V_{\out}(k,\td,y,H(y),a):(y,a)\sample \A^H(k)\right]
\]
is non-negligible in $\secpar$.
Therefore, as long as $q=\poly(\secpar)$, 
\[
\Pr_{k,\td,c^*}\left[V_{\out}(k,\td,y,c^*,a):(y,a)\sample \langle\calS^{\A(k)},c^* \rangle \right]\]
is also non-negligible in $\secpar$.
Then, we construct an adversary $\B$ that breaks the soundness of parallel version of Mahadev's protocol as follows:
\begin{description}
\item[Second Message:] Given the first message $k$, $\B$ runs the first stage of $\calS^{\A(k)}$ to obtain $y$. It sends $y$ to the verifier.
\item[Forth Message:] Given the third message $c$, $\B$ gives $c$ to $\calS^{\A(k)}$ as the second stage input, and let $a$ be the output of it.
Then $\B$ sends $a$ to the verifier.
\end{description}
Clearly, the probability that $\B$ succeeds in cheating is 
\[
\Pr_{k,\td,c^*}\left[V_{\out}(k,\td,y,c^*,a):(y,a)\sample \langle\calS^{\A(k)},c^* \rangle \right],\]
which is non-negligible in $\secpar$.
This contradicts the soundness of $m$-parallel version of Mahadev's protocol (Theorem~\ref{thm:rep_soundness}).
Therefore we conclude that there does not exists an adversary that succeeds in the two-round protocol with non-negligible probability assuming LWE in the QROM.
\end{proof}
\section{Making Verifier Efficient}\label{sec:efficient}
In this section, we construct a CVQC protocol with efficient verification in the CRS+QRO model where a classical common reference string is available for both prover and verifier in addition to quantum access to QRO.
Our main theorem in this section is stated as follows:
\begin{theorem}\label{thm:Eff}
Assuming LWE assumption and existence of post-quantum iO, post-quantum FHE, and two-round CVQC protocol in the standard model, there exists a two-round CVQC protocol for $\QTIME(T)$ with verification complexity $\poly(n, \log T)$ in the CRS+QRO model.
\end{theorem}
%assuming that a standard model instantiation of the two-round protocol given in Sec.~\ref{sec:tworound} is secure.
\begin{remark}
One may think that the underlying two-round CVQC protocol can be in the QROM instead of in the standard model since we rely on the QROM anyway.
However, this is not the case since we need to use the underlying two-round CVQC in a non-black box way, which cannot be done if that is in the QROM.
Since our two-round protocol given in Sec.~\ref{sec:tworound} is only proven secure in the QROM, we do not know any two-round CVQC protocol provably secure in the standard model.
On the other hand, it is widely used heuristic in cryptography that a scheme proven secure in the QROM is also secure in the standard model if the QRO is instantiated by a well-designed cryptographic hash function such as SHA-3. 
Therefore, we believe that it is reasonable to assume that a standard model instantiation of the scheme in  Sec.~\ref{sec:tworound} with a concrete hash function is sound.  
\end{remark}
\begin{remark}
One may think we need not assume CRS in addition to QRO since CRS may be replaced with an output of QRO.
This can be done if CRS is just a uniformly random string.
However, in our construction, CRS is non-uniform and has a certain structure.
Therefore we cannot implement CRS by QRO.
\end{remark}

\subsection{Preparation}
First, we prepare a lemma that is used in our security proof.
\begin{lemma}\label{lem:adaptive_program}
For any finite sets $\calX$ and $\calY$ and two-stage oracle-aided quantum algorithm $\A=(\A_1,\A_2)$, we have
\begin{align*}
\Pr\left[1 \sample \A_2^{H}(\ket{\st_\A},z):\ket{\st_\A}\sample\A_1^{H}()\right]-\Pr\left[1 \sample \A_2^{H[z,G]}(\ket{\st_\A},z):\ket{\st_\A}\sample\A_1^{H}()\right]\leq q_12^{-\frac{\ell}{2}+1}
\end{align*}
where 
$z\sample \bit^{\ell}$,
$H\sample \func(\bit^{\ell}\times \calX,\calY)$, $G\sample \func(\calX,\calY)$, $H[z,G]$ is defined by 
\begin{align*}
H[z,G](z',x)=
\begin{cases}
G(x) &\text{~if~}z'=z \\
H(z',x) &\text{~else}
\end{cases}.    
\end{align*}
where $q_1$ denotes the maximal number of queries by $\A_1$.
\end{lemma}
This can be proven similarly to \cite[Lemma 2.2]{EC:SaiXagYam18}. 
We give a proof in Appendix~\ref{sec:proof_adaptive_program} for completeness.

\subsection{Four-Round Protocol}\label{sec:efficient-four}
First, we construct a four-round scheme with efficient verification, which is transformed into two-round protocol in the next subsection.
Our construction is based on the following building blocks:
\begin{itemize}
\item A two-round CVQC protocol $\Pi=(\pro=\pro_2,\ver=(\ver_1,\ver_\out))$ in the standard model, which works as follows: 
\begin{description}
\item[$\ver_1$:] On input the security parameter $1^\secpar$ and $x$, it generates a pair $(\key,\td)$ of a``key" and ``trapdoor", sends $\key$ to $\pro$, and keeps $\td$ as its internal state.
\item[$\pro_2$:] On input $x$ and $\key$, it generates a response $e$ and sends it to $\ver$.
\item[$\ver_\out$:] On input $x$, $\key$, $\td$, $e$, it returns $\top$ indicating acceptance or $\bot$ indicating rejection.
\end{description}

\item A post-quantum PRG $\PRG:\bit^{\ell_s}\ra \bit^{\ell_r}$ where $\ell_r$ is the length of randomness for $\ver_1$.

%\item A family $\famCRH$ of post-quantum CRH $\CRH:\bit^*\ra \bit^{\Omega(\secpar)}$.

\item An FHE scheme $\Pi_\FHE=(\fhekeygen,\fheenc,\fheeval,\fhedec)$ with post-quantum CPA security.

\item A strong output compressing randomized encoding scheme $\Pi_\RE=(\rsetup,\renc,\rdec)$ with post-quantum security. We denote the simulator for $\Pi_\RE$ by $\calS_\re$.

\item A SNARK $\Pi_{\SNARK}=(\pro_{\snark},\ver_{\snark})$ in the QROM for an $\NP$ language $\lang_{\snark}$ defined below:  

We have $(x,\pk_{\fhe},\ct,\ct')\in \lang_{\snark}$
if and only if there exists $e$ such that
$\ct'= \fheeval(\pk_{\fhe},\allowbreak C[x,e],\ct)$ where $C[x,e]$ is a circuit that works as follows:
\begin{description}
\item[$C{[}x,e{]}(s)$:] Given input $s$, it computes $(k,\td)\sample \ver_1(1^\secpar,x;PRG(s))$, and returns $1$ if and only if $\ver_\out(x,k,\td,e)=\top$ and $0$ otherwise. 
\end{description}

%We have $(x,\pk_{\fhe},\ct,\ct')\in \lang_{\snark}$
%if and only if there exists $e$ such that
%$\crh(\crs_{\re})=t$,
%$k=\rdec(\crs_{\re},\Menc)$, and
%$\ct'= \fheeval(\pk_{\fhe},C[x,e],\ct)$ where $C[x,e]$ is a circuit that works as follows:
%\begin{description}
%\item[$C{[}x,e{]}(s)$:] Given input $s$, it computes $(k,\td)\sample \ver_1(1^\secpar,x;PRG(s))$, and returns $1$ if and only if $\ver_\out(x,k,\td,e)=\top$ and $0$ otherwise. 
%\end{description}
\end{itemize}



Let $\lang$ be a BPP language decided by a quantum Turing machine $\QTM$ (i.e., for any $x\in \bit^*$, $x\in \lang$ if and only if $\QTM$ accepts $x$),
and for any $T$, $\lang_T$ denotes the set consisting of $x\in \lang$ such that 
$\QTM$ accepts $x$ in $T$ steps.
Then we construct a 4-round CVQC protocol $(\setupeff,\proeff=(\proefftwo,\proefffour),\vereff=(\vereffone,\vereffthree,\vereffout))$ for $\lang_T$ in the CRS+QRO model where the verifier's efficiency only logarithmically depends on $T$.
Let $H:\bit^{2\secpar}\times \bit^{2\secpar}\ra \bit^{\secpar}$ be a quantum random oracle. 

\begin{description}
\item[$\setupeff(1^\secpar)$:]
The setup algorithm takes the security parameter $1^\secpar$ as input, generates $\crs_\re \sample \bit^{\ell}$ % and $\CRH\sample \famCRH$, 
and computes $\ek_\re\sample \rsetup(1^\secpar,1^\ell,\crs_\re)$ %and $t\defeq\crh(\crs_\re)$ 
where $\ell$ is a parameter specified later.
Then it outputs a CRS for verifier $\crs_{\vereff}\defeq \ek_\re$ and a CRS for prover $\crs_{\proeff}\defeq \crs_\re$.\footnote{We note that we divide the CRS into $\crs_{\vereff}$ and $\crs_{\proeff}$ just for the verifier efficiency and soundness still holds even if a cheating prover sees $\crs_{\vereff}$.} 
\item[$\vereffone^{H}$:] 
Given $\crs_{\vereff}= \ek_\re$ and $x$, 
it generates $s\sample \bit^{\ell_s}$ and $(\pk_{\fhe},\sk_{\fhe})\sample \fhekeygen(1^\secpar)$,
computes $\ct\sample \fheenc(\pk_{\fhe},s)$
and  $\Menc\sample \renc(\ek_\re,M,s,T')$
where $M$ is a Turing machine that works as follows:
\begin{description}
\item[$M(s)$:] Given an input $s\in \bit^{\ell_s}$, it computes $(k,\td)\sample \ver_1(1^\secpar,x;PRG(s))$ and outputs $k$
\end{description}
and $T'$ is specified later.
Then it sends $(\Menc,\pk_{\fhe},\ct)$ to $\proeff$ and keeps $\sk_{\fhe}$ as its internal state.

\item[$\proefftwo^{H}$:] Given $\crs_{\proeff}=\crs_\re$, $x$ and the message  $(\Menc,\pk_{\fhe},\ct)$ from the verifier, it computes $k\la \rdec(\crs_{\re},\Menc)$, $e\sample \pro_2(x,k)$, and $\ct'\la \fheeval(\pk_{\fhe},C[x,e],\ct)$ where $C[x,e]$ is a classical circuit defined above.
Then it sends $\ct'$ to $\pro$ and keeps $(\pk_{\fhe},\ct,\ct',e)$ as its state.
\item[$\vereffthree^{H}$] Upon receiving $\ct'$, it randomly picks $z\sample \bit^{2\secpar}$ and sends $z$ to $\proeff$.
\item[$\proefffour^{H}$] Upon receiving $z$, 
it computes $\pi_\snark \sample \pro_{\snark}^{H(z,\cdot)}((x,\pk_{\fhe},\ct,\ct'),e)$
and sends $\pi_\snark$ to $\vereff$.

\item[$\vereffout^{H}$:] 
It returns $\top$ if $\ver_{\snark}^{H(z,\cdot)}((x,\pk_{\fhe},\ct,\ct'),\pi_{\snark})=\top$ and $1 \la \fhedec(\sk_{\fhe},\ct')$ and $\bot$ otherwise.
\end{description}

\paragraph{Choice of parameters.}
\begin{itemize}
\item We set $\ell$ to be an upper bound of the length of $k$ where $(k,\td)\sample \ver_1(1^\secpar,x)$ for $x\in \lang_T$. We note that we have $\ell=\poly(\secpar,T)$.
\item We set $T'$ to be an upperbound of the running time of $M$ on input $s\in \bit^{\ell_s}$ when $x\in \lang_T$. We note that we have $T'=\poly(\secpar,T)$.
\end{itemize}

\paragraph{Verification Efficiency.}
By encoding efficiency of $\Pi_{RE}$ and verification efficiency of $\Pi_{\SNARK}$, $\vereff$ runs in time $\poly(\secpar,|x|,\log T)$.

\begin{theorem}[Completeness]\label{thm:eff_completeness}
%If the protocol is run honestly on input $x\in \lang$, then $\ver$ returns $\bot$ with overwhelming probability.
For any $x\in \lang_T$, 
\begin{align*}
\Pr\left[\langle\proeff^{H}(\crs_{\proeff}),\vereff^H(\crs_{\vereff})\rangle(x)=\bot \right]=\negl(\secpar)    
\end{align*}
where $(\crs_{\proeff},\crs_{\vereff})\sample \setupeff(1^\secpar)$.
\end{theorem}
\begin{proof}
This easily follows from completeness and correctness of the underlying primitives.
\end{proof}

\begin{theorem}[Soundness]\label{thm:eff_soundness}
For any $x\notin \lang_T$ any efficient quantum cheating prover $\A$, 
\begin{align*}
\Pr\left[\langle\A^H(\crs_{\proeff},\crs_{\vereff}),\vereff^H(\crs_{\vereff})\rangle(x)=\top\right]=\negl(\secpar)    
\end{align*}
where $(\crs_{\proeff},\crs_{\vereff})\sample \setupeff(1^\secpar)$.
\end{theorem}
\begin{proof}
We fix $T$ and $x\notin \lang_T$.
Let $\A$ be a cheating prover.
First, we divides $\A$ into the first stage $\A_1$, which is given $(\crs_{\proeff},\crs_{\vereff})$ and the first message and outputs the second message $\ct'$ and its internal state $\ket{\st_{\A}}$, and the second stage $\A_2$, which is given the internal state $\ket{\st_{\A}}$ and the third message and outputs the fourth message $\pi_{\snark}$.
We consider the following sequence of games between an adversary $\A=(\A_1,\A_2)$ and a challenger. 
Let $q_1$ and $q_2$ be an upper bound of number of random oracle queries by $\A_1$ and $\A_2$, respectively.
We denote the event that the challenger returns $1$ in $\game_i$ by $\TT_i$.
\begin{description}
\item[$\game_1$:] This is the original soundness game.
Specifically, the game runs as follows:
\begin{enumerate}
    \item The challenger generates 
    $H\sample \func(\bit^{2\secpar}\times \bit^{2\secpar},\bit^{\secpar})$, 
    $\crs_\re\sample \bit^{\ell}$, %$\CRH\sample \famCRH$, 
    $s\sample \bit^{\ell_s}$, and $(\pk_{\fhe},\sk_{\fhe})\sample \fhekeygen(1^\secpar)$, and computes $\ek_\re\sample \rsetup(1^\secpar,1^\ell,\crs_\re)$,  %$t\defeq\crh(\crs_\re)$, 
    $\ct\sample \fheenc(\pk_{\fhe},s)$, and  $\Menc\sample \renc(\ek_\re,M,s,T')$.
    \item $\A_1^{H}$ is given $\crs_{\proeff}\defeq \crs_\re$, $\crs_{\vereff}\defeq \ek_\re$ and the first message $(\Menc,\pk_{\fhe},\ct)$, and outputs the second message $\ct'$ and its internal state $\ket{\st_{\A}}$.
    \item The challenger randomly picks $z\sample \bit^{2\secpar}$.
    \item $\A_2^{H}$ is given the state $\ket{\st_{\A}}$ and the third message $z$ and outputs $\pi_{\snark}$.
    \item The challenger returns $1$ if  $\ver_{\snark}^{H(z,\cdot)}((x,\pk_{\fhe},\ct,\ct'),\pi_{\snark})=\top$ and $1 \la \fhedec(\sk_{\fhe},\ct')$ and $0$ otherwise.
\end{enumerate}

\item[$\game_2$:]
This game is identical to the previous game except that the oracles given to $\A_2$ and $V_{\snark}$ are replaced with $H[z,G]$ and $G$ in Step 4 and 5 respectively where $G\sample \func(\bit^{2\secpar},\bit^{\secpar})$ and $H[z,G]$ is as defined in Lemma~\ref{lem:adaptive_program}. 
We note that the oracle given to $\A_1$ in Step 2 is unchanged from $H$.

\item[$\game_3$:]
This game is identical to the previous game except that Step 4 and 5 are modified as follows:
\begin{description}
\item[4.] The challenger runs $e \sample \ext^{\A'_2[H,\ket{\st_{\A}},z]}((x,\pk_{\fhe},\ct,\ct'),1^{q_2},1^\secpar)$ where $\A'_2[H,\st_{\A},z]$ is an oracle-aided quantum algorithm that is given an oracle $G$ and emulates $\A_2^{H[z,G]}(\ket{\st_{\A}},z)$.
\item[5.] The challenger returns $1$ if 
$e$ is a valid witness for $(x,\pk_{\fhe},\ct,\ct')\in \lang_{\snark}$ and $1 \la \fhedec(\sk_{\fhe},\ct')$ and $0$ otherwise.
\end{description}

%\item[$\game_4$:]
%This game is identical to the previous game except that Step 5 is modified as follows:
%\begin{description}
%\item[5.] The challenger returns $1$ if 
%$k'=k$ where $(k,\td)\sample \ver_1(1^\secpar,x;PRG(s))$, 
%$(\crs'_{\re},k',e)$ is a valid witness for $(x,\pk_{\fhe},\ct,\ct')\in \lang_{\snark}$ and $1 \la \fhedec(\sk_{\fhe},\ct')$ and $0$ otherwise.
%\end{description}

\item[$\game_4$:]
This game is identical to the previous game except that Step 5 is modified as follows:
\begin{description}
\item[5.] The challenger returns $1$ if 
$e$ is a valid witness for $(x,\pk_{\fhe},\ct,\ct')\in \lang_{\snark}$, and $\ver_\out(x,k,\td,e)=\top$ where $(k,\td)\sample \ver_1(1^\secpar,x;PRG(s))$ and $0$ otherwise.
\end{description}

\item[$\game_5$:]
This game is identical to the previous game except that $\ct$ is generated as $\ct\sample \fheenc(\pk_{\fhe},\allowbreak 0^{2\secpar})$ in Step 1.

\item[$\game_6$:]
This game is identical to the previous game except that $\crs_\re$, $\ek_\re$, and $\Menc$ are generated in a different way. Specifically, in Step $1$, the challenger computes
$(k,\td)\sample \ver_1(1^\secpar,x;PRG(s))$, $(\crs_\re,\Menc)\sample \calS_\re(1^\secpar,1^{|M|},1^{\ell_s},k,T^*)$, and $\ek_\re \sample \rsetup(1^\secpar,1^\ell,\crs_\re)$ where $T^*$ is the running time of $M(\inp)$.
We note that the same $(k,\td)$ generated in this step is also used in Step 5.

\item[$\game_7$:]
This game is identical to the previous game except that $PRG(s)$ used for generating $(k,\td)$ in Step 1 is replaced with a true randomness.
\end{description}

This completes the descriptions of games. 
Our goal is to prove $\Pr[\TT_1]=\negl(\secpar)$. We prove this by the following lemmas. 
Since Lemmas~\ref{lem:eff_gamehop_five}, \ref{lem:eff_gamehop_six}, and \ref{lem:eff_gamehop_seven} can be proven by straightforward reductions, we only give proofs for the rest of lemmas.



\begin{lemma}\label{lem:eff_gamehop_one}
We have $|\Pr[\TT_2]-\Pr[\TT_1]|\leq q_12^{-(\secpar+1)}$.
\end{lemma}
\begin{proof}
This lemma is obtained by applying Lemma~\ref{lem:adaptive_program} for $\B=(\B_1,\B_2)$ described below:
\begin{description}
\item[$\B_1^{O_1}$():]   It generates 
    $\crs_\re\sample \bit^{\ell}$,  $s\sample \bit^{\ell_s}$, and $(\pk_{\fhe},\sk_{\fhe})\sample \fhekeygen(1^\secpar)$, computes $\ek_\re\sample \rsetup(1^\secpar,1^\ell,\crs_\re)$,  $\ct\sample \fheenc(\pk_{\fhe},s)$, $\Menc\sample \renc(\ek_\re,M,s,T')$,  and $\ct\sample \fheenc(\pk_{\fhe},s)$, and sets $\crs_{\proeff}=\crs_\re$ and $\crs_{\vereff}\defeq \ek_\re$.
Then it runs $(\ct',\ket{\st_{\A}})\sample \A_1^{O_1}(\crs_{\proeff},\crs_{\vereff},x,(\Menc,\pk_{\fhe},\ct))$, and outputs $\ket{\st_\B}\defeq(\ket{\st_\A},x,\Menc,\ct,\ct',\sk_\fhe)$.\footnote{Classical strings are encoded as quantum states in a trivial manner.}
\item[$\B_2^{O_2}(\ket{\st_\B},z)$:] 
It runs $\pi_{\snark}\sample \A_2^{O_2}(\ket{\st_\A})$, and outputs $1$ if $\ver_{\snark}^{O_2(z,\cdot)}((x,\pk_{\fhe},\ct,\ct'),\pi_{\snark})=\top$ and $1 \la \fhedec(\sk_{\fhe},\ct')$ and $0$ otherwise.
\end{description}
\end{proof}

\begin{lemma}\label{lem:eff_gamehop_two}
If $\Pi_{\SNARK}$ satisfies the extractability and $\Pr[\TT_2]$ is non-negligible, then $\Pr[\TT_3]$ is also non-negligible.
\end{lemma}
\begin{proof}
Let $\transcript_3$ be the transcript of the protocol before the forth message is sent (i.e., $\transcript_3=(\crs_{\proeff},\crs_{\vereff},\Menc,\pk_{\fhe},\ct',z)$).
We say that $(H,\sk_\fhe,\transcript_3,\ket{\st_\A})$ is good if we randomly choose $G\sample \func(\bit^{2\secpar},\bit^{\secpar})$ and run $\pi_{\snark}\sample \A_2^{H[z,G]}(\ket{\st_\A})$ to complete the transcript, then the transcript is accepted (i.e., we have $\ver_{\snark}^{G}((x,\pk_{\fhe},\ct,\ct'),\pi_{\snark})=\top$ and $1 \la \fhedec(\sk_{\fhe},\ct')$) with non-negligible probability.
By a standard averaging argument, if $\Pr[\TT_2]$ is non-negligible, then a non-negligible fraction of $(H,\sk_\fhe,\transcript_3,\ket{\st_\A})$ is good when they are generated as in $\game_2$.
We fix good $(\transcript_3,\sk_\fhe,\ket{\st_\A})$.
Then by the extractability of $\Pi_{\SNARK}$, $\ext$ succeeds in extracting a witness for $(x,\pk_{\fhe},\ct,\ct')\in \lang_{\snark}$ with non-negligible probability. Moreover, since we assume $(H,\sk_\fhe,\transcript_3,\ket{\st_\A})$ is good, we always have $1 \la \fhedec(\sk_{\fhe},\ct')$ (since otherwise a transcript with prefix $\transcript_3$ cannot be accepted).
Therefore we can conclude that $\Pr[\TT_3]$ is non-negligible.
\end{proof}

%\begin{lemma}\label{lem:eff_gamehop_three}
%If $\famCRH$ satisfies collision-resistance, then we have $|\Pr[\TT_4]-\Pr[\TT_3]|\leq \negl(\secpar)$.
%\end{lemma}
%\begin{proof}
%If $(\crs'_{\re},k',e)$ is a valid witness for $(x,\pk_{\fhe},\ct,\ct')\in \lang_{\snark}$, then we especially have   $\crh(\crs'_{\re})=t$ and
%$k'=\rdec(\Menc,\crs'_{\re})$.
%Conditioned on this happening, by the collision resistance of $\famCRH$, we have $\crs'_\re=\crs_\re$ with overwhelming probability.
%If this happens, $k'=k$ where $(k,\td)\sample \ver_1(1^\secpar,x;PRG(s))$ by the correctness of $\Pi_{\RE}$.
%Therefore, when the challenger returns $1$ in $\game_3$, we have $k'=k$ with overwhelming probability. Thus, the lemma follows. 
%\end{proof}

\begin{lemma}\label{lem:eff_gamehop_four}
We have $\Pr[\TT_4]=\Pr[\TT_3]$.
\end{lemma}
\begin{proof}
If $e$ is a valid witness for $(x,\pk_{\fhe}, \ct,\ct')\in \lang_{\snark}$, then we especially have $\ct'= \fheeval(\pk_{\fhe},\allowbreak C[x,e],\ct)$.
By the correctness of $\Pi_\FHE$, we have $\fhedec(\sk_\fhe,\ct')=C[x,e](s)=(\ver_\out(x,k,\td,e)\overset{?}{=}\top)$ where $(k,\td)\sample \ver_1(1^\secpar,x;PRG(s))$.
Therefore, the challenger returns $1$ in $\game_4$ if and only if it returns $1$ in $\game_3$.
\end{proof}

\begin{lemma}\label{lem:eff_gamehop_five}
If $\Pi_{\FHE}$ is CPA-secure, then we have $|\Pr[\TT_5]-\Pr[\TT_4]|\leq \negl(\secpar)$.
\end{lemma}

\begin{lemma}\label{lem:eff_gamehop_six}
If $\Pi_{\RE}$ is secure, then we have
$|\Pr[\TT_6]-\Pr[\TT_5]|\leq \negl(\secpar)$.
\end{lemma}

\begin{lemma}\label{lem:eff_gamehop_seven}
If $\PRG$ is secure, then we have  $|\Pr[\TT_7]-\Pr[\TT_6]|\leq \negl(\secpar)$.
\end{lemma}

\begin{lemma}\label{lem:eff_gamehop_eight}
If $(\pro,\ver)$ satisfies soundness, then we have $\Pr[\TT_7]\leq \negl(\secpar)$.
\end{lemma}
\begin{proof}
Suppose that $\Pr[\TT_7]$ is non-negligible. Then we construct an adversary $\B$ against the underlying two-round protocol as follows:
\begin{description}
\item[$\B(k)$:] Given the first message $k$, it generates 
    $H\sample\func(\bit^{2\secpar}\times \bit^{2\secpar},\bit^{\secpar})$, $G\sample \func(\bit^{2\secpar},\allowbreak \bit^{\secpar})$, $z\sample \bit^{2\secpar}$,
    $(k,\td)\sample \ver_1(1^\secpar,x;PRG(s))$, $(\crs_\re,\Menc)\sample \calS_\re(1^\secpar,1^{|M|},1^{\ell_s},k,T^*)$, $\ek_\re \sample \rsetup(1^\secpar,1^\ell,\crs_\re)$,   and $(\pk_{\fhe},\sk_{\fhe})\sample \fhekeygen(1^\secpar)$,  computes  $\ct\sample \fheenc(\allowbreak \pk_{\fhe},0^{2\secpar})$,
    and sets $\crs_{\proeff}=\crs_\re$ and $\crs_{\vereff}\defeq \ek_\re$.
Then it runs $(\ct',\ket{\st_{\A}})\sample \A_1^{H}(\crs_{\proeff},\crs_{\vereff},\allowbreak x,(\Menc,\pk_{\fhe},\ct))$ and $e\sample \ext^{\A'_2[H,\ket{\st_{\A}},z]}((x,\pk_{\fhe},\ct,\ct'),1^{q_2},1^\secpar)$ and outputs $e$.
\end{description}
Then we can easily see that the probability that we have $\ver_\out(x,k,\td,e)$ is at least $\Pr[\TT_7]$.
Therefore, if the underlying two-round protocol is sound, then $\Pr[\TT_7]=\negl(\secpar)$.
\end{proof}


By combining Lemmas~\ref{lem:eff_gamehop_one} to~\ref{lem:eff_gamehop_seven}, we can see that if $\Pr[\TT_1]$ is non-negligible, then $\Pr[\TT_7]$ is also non-negligible, which contradicts Lemma~\ref{lem:eff_gamehop_eight}.
Therefore we conclude that $\Pr[\TT_1]=\negl(\secpar)$.
\end{proof}

%\begin{remark}[On Reducing Rounds]
%One may think that we can shrink the scheme to two-round by letting the prover send the SNARK proof as a part of the second message.
%However, we do not know how to prove the soundness of this version since Chiesa et al.~\cite{TCC:ChiManSpo19} only proved non-adaptive extractability for SNARK in the QROM where the statement should be fixed in advance.  
%If we assume adaptive extractability for SNARK, then we would be able to prove the soundness for the two-round version.
%This motivates to construct SNARK with adaptive extractability in the QROM.
%\end{remark}




\subsection{Reducing to Two-Round via Fiat-Shamir}
Here, we show that the number of rounds can be reduced to $2$ relying on another random oracle.
Namely, we observe that the third message of the scheme is just a public coin, and so we can apply the Fiat-Shamir transform similarly to Sec.\ref{sec:tworound}.
In the following, we describe the protocol for completeness.

Our two-round CVQC protocol $(\setupefffs,\proefffs,\verefffs=(\verefffsone,\verefffsout))$ for $\lang_T$ in the CRS+QRO model is described as follows.
Let $H:\bit^{2\secpar}\times \bit^{2\secpar}\ra \bit^{\secpar}$ be a quantum random oracle and $H':\bit^{\ell_{ct'}}\ra \bit^{2\secpar}$ be another quantum random oracle where $\ell_{\ct'}$ is the maximal length of $\ct'$ in the four-round scheme and $\ell$ and $T'$ be as defined in the previous section.


\begin{description}
\item[$\setupefffs(1^\secpar)$:]
The setup algorithm takes the security parameter $1^\secpar$ as input, generates $\crs_\re \sample \bit^{\ell}$ % and $\CRH\sample \famCRH$, 
and computes $\ek_\re\sample \rsetup(1^\secpar,1^\ell,\crs_\re)$. %and $t\defeq\crh(\crs_\re)$ 
Then it outputs a CRS for verifier $\crs_{\verefffs}\defeq \ek_\re$ and a CRS for prover $\crs_{\proefffs}\defeq \crs_\re$.
\item[$\verefffsone^{H,H'}$:] 
Given $\crs_{\verefffs}= \ek_\re$ and $x$, 
it generates $s\sample \bit^{\ell_s}$ and $(\pk_{\fhe},\sk_{\fhe})\sample \fhekeygen(1^\secpar)$,
computes $\ct\sample \fheenc(\pk_{\fhe},s)$
and  $\Menc\sample \renc(\ek_\re,M,s,T')$
where $M$ is a Turing machine that works as follows:
\begin{description}
\item[$M(s)$:] Given an input $s\in \bit^{\ell_s}$, it computes $(k,\td)\sample \ver_1(1^\secpar,x;PRG(s))$ and outputs $k$.
\end{description}
Then it sends $(\Menc,\pk_{\fhe},\ct)$ to $\proefffs$ and keeps $\sk_{\fhe}$ as its internal state.

\item[$\proefffstwo^{H,H'}$:] Given $\crs_{\proefffs}=\crs_\re$, $x$ and the message  $(\Menc,\pk_{\fhe},\ct)$ from the verifier, it computes $k\la \rdec(\crs_{\re},\Menc)$, $e\sample \pro_2(x,k)$, and $\ct'\la \fheeval(\pk_{\fhe},C[x,e],\ct)$ where $C[x,e]$ is a classical circuit defined above.
Then it computes $z\defeq H'(\ct')$, computes $\pi_\snark \sample \pro_{\snark}^{H(z,\cdot)}((x,\pk_{\fhe},\ct,\ct'),e)$
and sends $(\ct',\pi_\snark)$ to $\verefffs$.

\item[$\verefffsout^{H,H'}$:] 
It computes $z\defeq H'(\ct')$ and returns $\top$ if $\ver_{\snark}^{H(z,\cdot)}((x,\pk_{\fhe},\ct,\ct'),\pi_{\snark})=\top$ and $1 \la \fhedec(\sk_{\fhe},\ct')$ and $\bot$ otherwise.
\end{description}

\paragraph{Verification Efficiency.}
Clearly, the verification efficiency is preserved from the protocol in Sec.~\ref{sec:efficient-four}

\begin{theorem}[Completeness]\label{thm:efffs_completeness}
%If the protocol is run honestly on input $x\in \lang$, then $\ver$ returns $\bot$ with overwhelming probability.
For any $x\in \lang_T$, 
\begin{align*}
\Pr\left[\langle\proefffs^{H,H'}(\crs_{\proefffs}),\verefffs^{H,H'}(\crs_{\verefffs})\rangle(x)=\bot \right]=\negl(\secpar)    
\end{align*}
where $(\crs_{\proefffs},\crs_{\verefffs})\sample \setupefffs(1^\secpar)$.
\end{theorem}


\begin{theorem}[Soundness]\label{thm:efffs_soundness}
For any $x\notin \lang_T$ any efficient quantum cheating prover $\A$, 
\begin{align*}
\Pr\left[\langle\A^{H,H'}(\crs_{\proefffs},\crs_{\vereff}),\verefffs^{H,H'}(\crs_{\verefffs})\rangle(x)=\top\right]=\negl(\secpar)    
\end{align*}
where $(\crs_{\proefffs},\crs_{\verefffs})\sample \setupefffs(1^\secpar)$.
\end{theorem}
This can be reduced to Theorem~\ref{thm:eff_soundness} similarly to the proof of soundness of the protocol in Sec.~\ref{sec:tworound}.


%\begin{remark}[Making the protocol non-interactive in designated verifier setting]
%In the designated verifier setting where the verifier can hold a secret verification key (in other words, the soundness is required only if $\crs_{\verefffs}$ is hidden from a cheating prover), we can make the protocol non-interactive. Moreover, we need not use strong output-compressing randomized encoding (and thus iO) in this setting.
%In the following, we briefly describe the construction.
%First, we observe that the first message of (a standard model instantiation of) the  two-round CVQC protocol given in Sec.~\ref{sec:tworound} only depends on the size of the problem instance $x$ and does not depend on $x$ itself thanks to the similar property of the Mahadev's protocol.
%Then we can construct a non-interactive protocol as follows:
%The setup algorithm generates $s\sample \bit^\secpar$ and 
%$(\pk_{\fhe},\sk_{\fhe})\sample \fhekeygen(1^\secpar)$,
%computes 
%$(k,\td)\sample \ver_1(1^\secpar,1^{|x|};PRG(s))$
%and $\ct\sample \fheenc(\pk_{\fhe},s)$,
%and sets $(k,\pk_{\fhe},\ct)$ as a CRS and $\sk_{\fhe}$ as a secret verification key. 
%The rest of the protocol works similaly to the above scheme.
%We can prove the soundness of the scheme similarly. A caveat is that we can only prove non-adaptive soundness where the statement $x$ should be fixed before the setup since the Mahadev's protocol is only shown to satisfy non-adaptive soundness. 
%\end{remark}








\bibliographystyle{alpha}
\bibliography{abbrev3,crypto,reference}

\appendix
\section{Proof of Lemma~\ref{lem:adaptive_program}}\label{sec:proof_adaptive_program}
Here, we give a proof of Lemma~\ref{lem:adaptive_program}.
We note that the proof is essentially the same as the proof of \cite[Lemma 2.2]{EC:SaiXagYam18}.

Before proving the lemma, we introduce another lemma, which gives a lower bound for a decisional variant of Grover's search problem. 

\begin{lemma}[{\cite[Lemma C.1]{SonYun17}}] \label{lem:Decision_Grover}
Let $g_z:\bit^\ell \rightarrow \bit$ denotes a function defined as $g_{z}(z):=1$ and $g_z(z'):=0$ for all $z'\not=z$, and $g_{\bot}:\bit^\ell \rightarrow \bit$ denotes a function that returns $0$ for all inputs. Then for any quantum adversary $\B=(\B_1,\B_2)$ we have 
\[
\left|\Pr[1\sample\B_2(\ket{\st_{\B}},z) \mid \ket{\st_{\B}}\sample \B_1^{g_z}()]- \Pr[1\sample\B_2(\ket{\st_{\B}},z) \mid \ket{\st_{\B}}\sample \B_1^{g_{\bot}}()] \right|\leq q_1 \cdot2^{-\frac{\ell}{2}+1}.
\]
where $z\sample \bit^{\ell}$ and $q_1$ denotes the maximal number of queries by $\B_1$.
\end{lemma}

Then we prove Lemma~\ref{lem:adaptive_program}.

\begin{proof}(of Lemma~\ref{lem:adaptive_program}.)
We consider the following sequence of games.
We denote the event that $\game_i$ returns $1$ by $\TT_i$.
\begin{description}
\item[$\game_1$:] This game simulates the environment of the first term of LHS in the inequality in the lemma. Namely, the challenger chooses $z\sample \bit^{\ell}$, $H\sample \func(\bit^{\ell}\times \calX,\calY)$, $\A_1$ runs with oracle $H$ to generate $\ket{\st_{\A}}$, $\A_2$ runs on input $(\ket{\st_{\A}},z)$ with oracle $H$ to generate a bit $b$, and the game returns $b$.
\item[$\game_2$:] This game is identical to the previous game except that the oracle given to $\A_1$ is replaced with $H[z,G]$ where $G\sample \func(\calX,\calY)$.
\item[$\game_3$:] This game is identical to the previous game except that the oracle given to $\A_1$ is replaced with $H$ and the oracle given to $\A_2$ is replaced with $H[z,G]$.
We note that this game simulates the environment as in the second term of the LHS in the inequality in the lemma.
\end{description}
What we need to prove is $|\Pr[\TT_1]-\Pr[\TT_3]|\leq q_12^{-\frac{\ell}{2}+1}$.
First we observe that the change from $\game_2$ to $\game_3$ is just conceptual and nothing changes from the adversary's view since in both games, the oracles given to $\A_1$ and $\A_2$ are random oracles that agrees on any input $(z',x)$ such that $z'\neq z$ and independent on any input $(z,x)$.
Therefore we have $\Pr[\TT_2]=\Pr[\TT_3]$.
What is left is to prove $|\Pr[\TT_1]-\Pr[\TT_2]|\leq q_12^{-\frac{\ell}{2}+1}$.
For proving this, we construct an algorithm $\B=(\B_1,\B_2)$ that breaks Lemma~\ref{lem:adaptive_program} with the advantage $|\Pr[\TT_1]-\Pr[\TT_2]$ as follows:

\begin{description}
\item[$\B_1^{g^*}()$:] It generates $H\sample \func(\bit^{\ell}\times \calX,\calY)$ and  $G\sample \func(\calX,\calY)$, implements an oracle $O_1$ as
\begin{align*}
O_1(z',x)=
\begin{cases}
G(x) &\text{~if~}g^*(z')=1 \\
H(z',x) &\text{~else}
\end{cases},    
\end{align*}
runs $\ket{\st_{\A}}\sample\A_1^{O_1}()$ and outputs $\ket{\st_{\B}}\defeq \ket{\st_{\A}}$
\item[$\B_2(\ket{\st_{\B}}=\ket{\st_{\A}},z)$:]
It runs $b \sample \A_2^{H}(\ket{\st_{\B}},z)$ and outputs $b$.
\end{description}
It is easy to see that if $g^*=g_{\bot}$, then $\B$ perfectly simulates $\game_1$ for $\A$ and if $g^*=g_z$, then $\B$ perfectly simulates $\game_2$ for $\A$.
Therefore, we have $|\Pr[\TT_1]-\Pr[\TT_2]|\leq q_12^{-\frac{\ell}{2}+1}$ by Lemma~\ref{lem:adaptive_program}.
\end{proof} 
\end{document}
