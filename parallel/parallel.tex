\section{Parallel Repetition of Mahadev's Protocol}

\subsection{Overview of Mahadev's Protocol}
Here, we recall the Mahadev's protocol \cite{FOCS:Mahadev18a}. We only give a high-level description of the protocol and properties of it and omit the details since they are not needed to show our result. 

The protocol is run between a quantum prover $\pro$ and a classical verifier $\ver$ on a common input $x$. The aim of the protocol is to enable a verifier to classically verify $x\in \lang$ for a BQP language $\lang$ with the help of interactions with a quantum prover.
The protocol is a 4-round protocol where the first message is sent from $\ver$ to $\pro$. 
We denote the $i$-th message generation algorithm by $\ver_i$ for $i\in\{1,3\}$ or $\pro_i$ for $i\in \{2,4\}$ and denote the verifier's final decision algorithm by $\ver_\out$.
Then a high-level description of the protocol is given below.
\begin{description}
\item[$\ver_1$:] On input the security parameter $1^\secpar$ and $x$, it generates a pair $(\key,\td)$ of a``key" and ``trapdoor", sends $\key$ to $\pro$, and keeps $\td$ as its internal state.
\item[$\pro_2$:] On input $x$ and $\key$, it generates a classical ``commitment" $\comy$ along with a quantum state $\ket{\st_\pro}$, sends $\comy$ to $\pro$, and keeps $\ket{\st_\pro}$ as its internal state.
\item[$\ver_3$:] It randomly picks $c\sample \bit$ and sends $c$ to $\pro$.\footnote{The third message is just a public-coin, and does not depend on the transcript so far or $x$.}
For a knowledgeable reader, the case of $c=0$ corresponds to the ``test round" and the case of $c=1$ corresponds to the ``Hadamard round" in the terminology in \cite{FOCS:Mahadev18a}.
\item[$\pro_4$:] On input $\ket{\st_\pro}$ and $c$, it generates a classical string $\ans$ and sends $\ans$ to $\pro$.
\item[$\ver_\out$:] On input $\key$, $\td$, $y$, $c$, and $\ans$, it returns $\top$ indicating acceptance or $\bot$ indicating rejection.
In case $c=0$, the verification can be done publicly, that is, $\ver_\out$ need not take $\td$ as input.
\end{description}

For the protocol, we have the following properties:\\
\noindent\textbf{Completeness:}
For all $x\in \lang$, we have $\Pr[\langle \pro,\ver \rangle(x)]=\bot]= \negl(\secpar)$.\\
\noindent\textbf{Soundness:}
If the LWE problem is hard for quantum polynomial-time algorithms, then for any $x\notin \lang$ and a quantum polynomial-time cheating prover $\pro^*$, we have  $\Pr[\langle \pro^*,\ver \rangle(x)]=\bot]\leq 3/4$.

We need a slightly different form of soundness implicitly shown in \cite{FOCS:Mahadev18a}, which roughly says that if a cheating prover can pass the ``test round" (i.e., the case of $c=0$) with overwhelming probability, then it can pass the ``Hadamard round" (i.e., the case of $c=1$) only with a negligible probability. 
\begin{lemma}[implicit in \cite{FOCS:Mahadev18a}]\label{lem:Mah_soundness}
If the LWE problem is hard for quantum polynomial-time algorithms, then for any $x\notin \lang$ and a quantum polynomial-time cheating prover $\pro^*$ such that  $\Pr[\langle \pro^*,\ver \rangle(x)]=\bot\mid c=0]=\negl(\secpar)$, we have $\Pr[\langle \pro^*,\ver \rangle(x)]=\top\mid c=1]=\negl(\secpar)$.
\end{lemma}

\subsection{Parallel Repetition}
Here, we prove that the parallel repetition of the Mahadev's protocol decrease the soundness bound to be negligible.
Let $\pro^m$ and $\ver^m$ be $m$-parallel repetitions of the honest prover $\pro$ and verifier $\ver$ in the Mahadev's protocol. Then we have the following:
\begin{theorem}[Completeness]\label{thm:rep_completeness}
For all $x\in \lang$ and $m= \poly(\secpar)$, we have $\Pr[\langle \pro^m,\ver^m \rangle(x)]=\bot]= \negl(\secpar)$.\\
\end{theorem}
\begin{theorem}[Soundness]\label{thm:rep_soundness}
If the LWE problem is hard for quantum polynomial-time algorithms, then for any $x\notin \lang$ and a quantum polynomial-time cheating prover $\pro^*$, we have  $\Pr[\langle \pro^*,\ver^m \rangle(x)]=\top]\leq \negl(\secpar)$ for $m=\poly(\secpar)$.
\end{theorem}

The completeness (Theorem~\ref{thm:rep_completeness}) easily follows from the completeness of the Mahadev's protocol.
In the next subsection, we prove the soundness (Theorem~\ref{thm:rep_soundness}).

\subsection{Proof of Soundness}
\noindent\textbf{Characterization of cheating prover.}
Any cheating prover can be characterized by a tuple $(U_0,\{U_c\}_{c\in \bit^n})$ of unitaries over Hilbert space $\hil_{\regX}\otimes \hil_{\regZ} \otimes \hil_{\regY}  \otimes \hil_{\regK} $. 
A prover characterized by $(U_0,\{U_{\bfc}\}_{\bfc\in \bit^n})$ works as follows.\footnote{Here, we hardwire into the cheating prover the instance $x\notin \lang$ on which it will cheat instead of giving it as an input.}
\begin{description}
\item[Second Message:] Upon receiving $k=(\key_1,...,\key_m)$, it applies $U_0$ to the state $\ket{0}\otimes\ket{0}\otimes\ket{0}\otimes  \ket{k}$, and then measures the $Y$ register to obtain $y=(\comy_1,...,\comy_m)$. Then it sends $\bfy$ to $\ver$ and keeps the resulting state $\ket{\psi_{k,y}}$ over  $\hil_{\regX,\regZ}$.
\item [Forth Message:] Upon receiving $c\in \bit^{m}$, it applies $U_c$ to $\ket{\psi}_{\pro^*}$ and then measures the $\regX$ register in computational basis to obtain $a=(a_1,...,a_m)$. We denote the designated register for $a_i$ by $\regX_i$ 
\end{description}

%Let $\ket{\psi_{k,y}}$ be the prover's state over $\hil_{\regX,\regZ}$ after sending $y=(y_1,...,y_m)$ conditioned on a fixed value of $k$ and $y$.
For $i\in[m]$ and $c_i\in \bit$, let $\Pi_{i,c_i}^{out}$ be the projection to the space spanned by states that 
contains a accepting message in $\regX_i$ w.r.t. the challenge $c_i$.
Then we have
\begin{align*}
\Pr[\langle P^*,V \rangle=accept]=\mathbb{E}_{k,y,c}[\|\Pi_{m,c_m}^{out}...\Pi_{1,c_1}^{out}U_{c}\ket{\psi_{k,y}}\|^{2}]
\end{align*}
We prove this is negligible in the following.

First, we observe that it suffices to show that for any polynomial $p$, there exists $m=O(\log(\secpar))$ such that we have 
\begin{align*}
\mathbb{E}_{k,y,c}[\|\Pi_{m,c_m}^{out}...\Pi_{1,c_1}^{out}U_{c}\ket{\psi_{k,y}}\|^{2}]\leq 1/p(\secpar).
\end{align*}
This is because if we set $m=O(\secpar)$, then we can consider the first $m'=O(\log \secpar)$ coordinates and do the same analysis ignoring the rest of coordinates.
\takashi{Is this correct?}

In the following, we fix a polynomial $p$, and set $m$ and $\epsilon$ so that 
$\epsilon=1/\poly(\secpar)$ and  $(m+1)m2^{m-1}2\epsilon+2^{-m}\leq 1/2p(\secpar)$.



%\begin{lemma}
%Let $U$ be an efficient unitary acting on registers $\regX$, $\regY$, and $\regZ$.
%Let $f:\calX \ra \bit$ be an efficiently computable function and $\Pi_f$ be a projector that projects $\regZ$ register onto the space spanned by $\{\ket{x}\}_{X\in S_f}$
%where $S_f\defeq {x\in \calX \text{~s.t.~}f(x)=1}$ (i.e., we let $\Pi\defeq \sum_{x\in S_f}I\ot I\ot\ket{x}\bra{x})$.
%Then for any $\epsilon=1/\poly(\secpar)$, there exists a decomposition of $\hil_\regX$ into two subspaces $\Sgood$ and $\Sbad$ and an efficient quantum algorithm $\ext$ such that the following is satisfied:
%\begin{enumerate}
%    \item $\ext$ outputs $x\in S_f$ with probability $1$ on any input $\ket{\psigood}\in \Sgood$.    
%    \item For any $\ket{\psibad}\in \Sbad$, we have $\|\Pi_f U\ket{\psibad}\ket{0}\ket{0}\|^2\leq \epsilon$
    %\item If there is an efficient quantum algorithm with input $z$ to generate $\ket{\psi}$ such that $\|\Pi_{\Sgood}\ket{\psi}\|$ is non-negligible where $\Pi_{\Sgood}$ denotes the projection onto $\Sgood$, then there exists an efficient quantum algorithm with input $z$ to generate $\Pi_{\Sgood}\ket{\psi}$ with overwhelming probability. 
%\end{enumerate}
%\end{lemma}

For proving the soundness, we prepare several lemmas.
\begin{lemma}\label{lem:decomp}
Let $\A_i$ be a quantum algorithm that takes a quantum state $\ket{\psi}\in \hil$, picks random $c\sample \bit^m$ such that $c_i=0$, applies a unitary $U_c$ to $\ket{\psi}$ and finally measures $\regX$ to output a classical string $a_i\in \calX$.
%Let $f_i:\calX \ra \bit$ be a function such that $f(x_i)=1$ if and only if $\ver_\out(k_i,y_i,0,x_i)=\top$. 
%Then for any $\epsilon=1/\poly(\secpar)$, 
Then there exist a decomposition of $\hil_{\regX,\regZ}$ into two subspaces $T_{i,0}$ and $T_{i,1}$ and an efficient quantum algorithm $\ext_i$ such that for any fixed $k,y$ and any state $\ket{\psi}$ the followings are satisfied: Let $\Pi_{i,b}^{in}$ be the projection to $T_{i,b}$.
\begin{enumerate}
    \item
    We have $\Pr[\ver_\out(k_i,y_i,0,a_i)=\top: a_i\sample \ext_i(\Pi_{i,0}^{in}\ket{\psi})]=1-\negl(\secpar)$.
    \item We have $\Pr[\ver_\out(k_i,y_i,0,a_i)=\top:a_i\sample \A_i(\Pi_{i,1}^{in}\ket{\psi})]\leq \epsilon$.
    \item Measurement w.r.t. $\{\Pi_{i,0}^{in}, \Pi_{i,1}^{in}\}$ can be done efficiently.
    \takashi{approximately?}
\end{enumerate}
\end{lemma}


\begin{lemma}\label{lem:decomp2}
%For any adversary's second stage strategy $\{U_c\}_{v\in\bit^{m}}$,
%we can define partition $\hil_{\regX,\regZ}$ into $T_{i,0}$ and $T_{i,1}$ for each $i\in[m]$, such that the following properties are satisfied.
For the partitions $(T_{i,0},T_{i,1})_{i\in[m]}$ as defined in Lemma~\ref{lem:decomp}, the following is satisfied:
Assuming LWE, for all $i\in[m]$ and $c\in \bit^{m}$, we have 
    \begin{align*}
     \mathbb{E}_{k,y}\left[\left\|\Pi_{i,c_i}^{out}U_c\Pi_{i,1-c_i}^{in}\Pi_{i-1,c_{i-1}}^{in}...\Pi_{1,c_1}^{in}\ket{\psi_{k,y}}\right\|^2\right]\leq  2^{m-1}\epsilon +\negl(\secpar).
    \end{align*}
%  \item We have 
%    \begin{align*}
%     \mathbb{E}_{k,y}\left[\left\|\Pi_{i,0}^{out}U_c\Pi_{i,1}^{in}\Pi_{i-1,c_{i-1}}^{in}...\Pi_{1,c_1}^{in}\ket{\psi_{k,y}}\right\|\right]\leq  (2^m\epsilon)^{1/2}
%    \end{align*}
    
%  \item Assuming LWE, we have 
%    \begin{align*}
%     \mathbb{E}_{k,y}\left[\left\|\Pi_{i,1}^{out}U_c\Pi_{i,0}^{in}\Pi_{i-1,c_{i-1}}^{in}...\Pi_{1,c_1}^{in}\ket{\psi_{k,y}}\right\|\right]\leq  \negl(\secpar)
%    \end{align*}
\end{lemma}

\begin{proof}
%For each $i\in [m]$, we apply Lemma~\ref{lem:decomp} w.r.t $\A_i$ 
%that applies the second stage $\A$ with a random $c$ such that $c_i=0$ and then measures $\regX_i$ and $f$ that performs the verification on the $i$-th coordinate in the test round. 
%More precisely, given an input $\ket{\psi}$, it picks random $c_{-i}$, set $c_i\defeq 0$ to define $c$, applies $U_c$ to $\ket{\psi}$, measures $\regX_i$, and outputs the measurement outcome. 
%We define $f(x)=1$ if and only if $\ver_{\out}(k,y,0,x)=accept$.
%By Lemma~\ref{lem:decomp}, there exist a partition ($T_{i,0},T_{i,1}$) of $\hil_{\regX,\regZ}$ and an efficient quantum algorithm $\ext_i$ such that for any state $\ket{\psi}$, 
%\begin{itemize} 
%  \item
%    We have $\Pr[f(x)=1: x\sample \ext_i(\Pi_{i,0}^{in}\ket{\psi})]=1-\negl(\secpar)$.
%    \item We have $\Pr[f(x)=1:x\sample \A_i(\Pi_{i,1}^{in}\ket{\psi})]\leq \epsilon$.
%    \item Measurement w.r.t. $\{\Pi_{i,0}^{in}, \Pi_{i,1}^{in}\}$ can be done efficiently.
%\end{itemize}
First, we show the inequality for the case of $c_i=0$.
For any fixed $k$, $y$, the second claim of Lemma~\ref{lem:decomp} implies 
\begin{align*}
    &~~~\mathbb{E}_{c_{-i}}\left[\left\|\Pi_{i,0}^{out}U_c\Pi_{i,1}^{in}\Pi_{i-1,c_{i-1}}^{in}...\Pi_{1,c_1}^{in}\ket{\psi_{k,y}}\right\|^2\right] \leq \epsilon,
\end{align*}
which implies 
\begin{align*}
    \left\|\Pi_{i,0}^{out}U_c\Pi_{i,1}^{in}\Pi_{i-1,c_{i-1}}^{in}...\Pi_{1,c_1}^{in}\ket{\psi_{k,y}}\right\|^2\leq 2^{m-1}\epsilon
\end{align*}
for all $c_{-i}\in \bit^{m-1}$.
Averaging over all $(k,y)$, we obtain the inequality with the case $c_i=0$.

We move on to the proof of the case of $c_i=1$.
To prove this, we assume that 
\begin{align*}
     \mathbb{E}_{k,y}\left[\left\|\Pi_{i,1}^{out}U_c\Pi_{i,0}^{in}\Pi_{i-1,c_{i-1}}^{in}...\Pi_{1,c_1}^{in}\ket{\psi_{k,y}}\right\|^2\right]\geq 1/\poly(\secpar). 
\end{align*}
for some polynomial $\poly$, and construct an adversary $\B$ that breaks Lemma~\ref{lem:Mah_soundness}.

$\B$ is given $k_i$ as the first round message, and repeats the following procedure $\Theta(\poly(\secpar))$ times:
\begin{enumerate}
\item Pick $k_{-i}$ according to the protocol to complete $k=(k_1,...,k_m)$
\item Pick $c\sample \bit^{m}$ such that $c_i=1$.
\item Run the first stage of $\A$ to obtain $y$ along with its internal state $\ket{\psi_{k,y}}$.
\item Apply a projector $\Pi_{i,0}^{in}\Pi_{i-1,c_{i-1}}^{in}...\Pi_{1,c_1}^{in}$ to $\ket{\psi_{k,y}}$ by applying iterative projective measurements and unitary $U_c$. We note that these projective measurements can be implemented efficiently by the third claim of  Lemma~\ref{lem:decomp}. 
\takashi{Here, we assume perfect projection. In the actual proof, we have to think about the approximate version.}
If it fails in any of the projection, then it immediately aborts this trial and goes back to the 1st step of the loop.
Otherwise, it goes out the loop to continue the attack.
\end{enumerate}
Since we assume \begin{align*}
    \mathbb{E}_{k,y}\left[\left\|\Pi_{i,1}^{out}U_c\Pi_{i,0}^{in}\Pi_{i-1,c_{i-1}}^{in}...\Pi_{1,c_1}^{in}\ket{\psi_{k,y}}\right\|^2\right]\geq 1/\poly(\secpar),
\end{align*}
we especially have  
\begin{align*}
\mathbb{E}_{k,y}\left[\left\|\Pi_{i,0}^{in}\Pi_{i-1,c_{i-1}}^{in}...\Pi_{1,c_1}^{in}\ket{\psi_{k,y}}\right\|^2\right]\geq 1/\poly(\secpar).
\end{align*}
Therefore, in each trial in the loop, the probability that $\B$ does not abort is at least $1/\poly(\secpar)$. 
Thus, $\B$ goes out the loop within $\Theta(\poly(\secpar))$ times repetitions with overwhelming probability.
Assuming this happens, $\B$ now has $k,y$ along with the state $\ket{\psi_{\B}}\defeq\Pi_{i,0}^{in}\Pi_{i-1,c_{i-1}}^{in}...\Pi_{1,c_1}^{in}\ket{\psi_{k,y}}/\left\|\Pi_{i,0}^{in}\Pi_{i-1,c_{i-1}}^{in}...\Pi_{1,c_1}^{in}\ket{\psi_{k,y}}\right\|$.
Then it sends $y_i$ to the verifier.
Then verifier returns $c'_i$.
\begin{itemize}
\item If $c'_i=0$, $\B$ runs $\ext_i(\ket{\psi_{\B}})$ to obtain $a_i$. This is an accepting answer with probability $1-\negl(\secpar)$ by the first claim of  Lemma~\ref{lem:decomp}.
\item If $c'_i=1$, $\B$ applies $U_c$ to $\ket{\psi_{\B}}$ and then measures $\regX_i$ to obtain $a_i$. 
Let $\fail$ be the event that $\B$ fails to generate $\ket{\psi_{\B}}$ within $\Theta(\poly(\secpar))$ times of trials.
Then we have 
%The probability that $\B$ fails to generate $\ket{\psi_{\B}}$ within $\Theta(\poly(\secpar))$ times of trials and $a_i$ is accepted is 
\begin{align*}
% &~~~\mathbb{E}_{k,y}\left[\left\|\Pi_{i,1}^{out}U_c\left(\Pi_{i,0}^{in}\Pi_{i-1,c_{i-1}}^{in}...\Pi_{1,c_1}^{in}\ket{\psi_{k,y}}/\left\|\Pi_{i,0}^{in}\Pi_{i-1,c_{i-1}}^{in}...\Pi_{1,c_1}^{in}\ket{\psi_{k,y}}\right\|\right)\right\|^2\right]\\
%&\leq 
 &~~~\Pr\left[\ver_{\out}(k_i,\td,y_i,1,a_i) \mid \overline{\fail}\right]\\
 &\leq \Pr\left[\ver_{\out}(k_i,\td,y_i,1,a_i) \wedge \overline{\fail}\right]-\Pr[\fail]\\
 &\leq \mathbb{E}_{k,y}\left[\left\|\Pi_{i,1}^{out}U_c\Pi_{i,0}^{in}\Pi_{i-1,c_{i-1}}^{in}...\Pi_{1,c_1}^{in}\ket{\psi_{k,y}}\right\|^2\right]-\negl(\secpar)\\
 &\geq 1/\poly(\secpar)-\negl(\secpar).
\end{align*}
\end{itemize}
\takashi{More careful analysis needed.}

Overall, $\B$ succeeds in answering a correct answer with overwhelming probability for the case of $c_i=0$ and with non-negligible probability for the case of $c_i=1$, which contradicts Lemma~\ref{lem:Mah_soundness}.
Therefore, under the LWE assumption, we have
\begin{align*}
   \mathbb{E}_{k,y}\left[\left\|\Pi_{i,1}^{out}U_c\Pi_{i,0}^{in}\Pi_{i-1,c_{i-1}}^{in}...\Pi_{1,c_1}^{in}\ket{\psi_{k,y}}\right\|^2\right]=\negl(\secpar), 
\end{align*}
which concludes the proof of the proof of Lemma~\ref{lem:decomp2}.
\end{proof}

\begin{lemma}\label{lem:decomp3}
For any $\ket{\psi}\in \hil_{\regX,\regZ}$ %and decompositions of $\hil_{\regX,\regZ}$ into $T_{i,0}$ and $T_{i,1}$ for $i\in[m]$ 
and $c\in[m]$, we define
\begin{align*}
 \ket{\psi_i^{c}}\defeq \Pi_{i,1-c_i}^{in}\Pi_{i-1,c_{i-1}}^{in}...\Pi_{1,c_1}^{in}\ket{\psi} 
\end{align*}
and 
\begin{align*}
 \ket{\psi_{m+1}^{c}}\defeq \ket{\psi}-\sum_{i=1}^{m}\ket{\psi_{i}^{c}}.
\end{align*}
Then we have
\begin{align*}
 \mathbb{E}_{c}\left[\left\|\ket{\psi_{m+1}^{c}}\right\|^{2}\right]=2^{-m}.
\end{align*}
\end{lemma}
\begin{proof}
\takashi{easy induction?}
\end{proof}

\begin{lemma}\label{lem:parallel_soundness}
\begin{align*}
\mathbb{E}_{k,y,c}\left[\left\|\Pi_{m,c_m}...\Pi_{1,c_1}U_{c}\ket{\psi_{k,y}}\right\|^{2}\right]\leq 2^{-m} + (m+1)m 2^{m-1}\epsilon +\negl(\secpar).
\end{align*}
\end{lemma}
\begin{proof}
Fix $c$.
We decompose $\ket{\psi_{k,y}}$ as in Lemma~\ref{lem:decomp3} to write 
\begin{align*}
\ket{\psi_{k,y}}=\sum_{i=1}^{m+1}\ket{\psi_{k,y,i}^{c}}.
\end{align*}
Then we have 
\begin{align*}
&~~~\mathbb{E}_{k,y}\left[\left\|\Pi_{m,c_m}^{out}...\Pi_{1,c_1}^{out}U_{c}\ket{\psi_{k,y}}\right\|^2\right]\\
&\leq \mathbb{E}_{k,y}\left[\left(\sum_{i=1}^{m+1}\left\|\Pi_{m,c_m}^{out}...\Pi_{1,c_1}^{out}U_{c}\ket{\psi_{k,y,i}}\right\|\right)^2\right]\\
&\leq \mathbb{E}_{k,y}\left[(m+1)\sum_{i=1}^{m+1}\left\|\Pi_{m,c_m}^{out}...\Pi_{1,c_1}^{out}U_{c}\ket{\psi_{k,y,i}}\right\|^2\right]\\
&\leq (m+1)\left(\sum_{i=1}^{m}\mathbb{E}_{k,y}\left[\left\|\Pi_{i,c_i}^{out}U_{c}\ket{\psi_{k,y,i}}\right\|^2\right]+\mathbb{E}_{k,y}\left[\left\|\ket{\psi_{m+1}^{c}}\right\|^2\right]\right)\\
&\leq (m+1)m 2^{m-1}\epsilon+\negl(\secpar)+ \mathbb{E}_{k,y}\left[\left\|\ket{\psi_{m+1}^{c}}\right\|^2\right]
\end{align*}
by Lemma~\ref{lem:decomp2}.
By taking average over all $c$, we have 
\begin{align*}
&~~~\mathbb{E}_{k,y,c}\left[\left\|\Pi_{m,c_m}...\Pi_{1,c_1}U_{c}\ket{\psi_{k,y}}\right\|^2\right]\\
&\leq (m+1)m 2^{m-1}\epsilon+\negl(\secpar) +\mathbb{E}_{k,y,c}\left[\left\|\ket{\psi_{m+1}^{c}}\right\|^2\right]\\
&\leq (m+1)m 2^{m-1}\epsilon+\negl(\secpar) +2^{-m}
\end{align*}
by Lemma~\ref{lem:decomp3}.
\end{proof}

By Lemma~\ref{lem:parallel_soundness} and $(m+1)m2^{m-1}2\epsilon+2^{-m}\leq 1/2p(\secpar)$, we have 
\begin{align*}
    \mathbb{E}_{k,y,c}\left[\left\|\Pi_{m,c_m}...\Pi_{1,c_1}U_{c}\ket{\psi_{k,y}}\right\|^{2}\right]\leq 1/p(\secpar).
\end{align*}
As discussed in the beginning of this subsection, this suffices to show the soundness of the parallel repetition of the Mahadev's protocol (Theorem~\ref{thm:rep_soundness}).

\subsection{Jordan's Lemma and the Subspace Supporting Test Round}


%In this section, for each trial $i\in [m]$, we will use Jordan's lemma to partition the the Prover's space $\hil$ into $T_i$ and $T^{\bot}_i$, which satisfies the following properties. 
%\begin{itemize}
%    \item $T^{\bot}_i$ is the orthogonal complement of $T_i$ in $\hil$.  
%    \item For $\ket{s} \in T_i$, there exists an efficient quantum algorithm $U_T$, such that $U_T\ket{s}$ can pass the test round with probability $1-\frac{1}{\poly(n)}$; but there is no efficient quantum algorithm $U_H$, such that $U_H\ket{s}$ can pass the Hadamard round. 
%\end{itemize}


Let $m$ be the number of repetitions. We consider the prover's state after sending $y_1,\dots,y_m$. We let the prover's state space be $\hil_{\regX,\regZ}$, where $\regX$ is the register corresponding to the last message and $\regZ$ is the prover's working qubits. We can decompose $\regX$ into $\regX_1,\dots,\regX_m$ without loss of generality, where $\regX_i$ is corresponding to the output message of the $i$th trial.     


\begin{lemma}\label{lem:test_subspace}
For $i\in [m]$, we can partition $\hil_{\regX,\regZ}$ into $T_i$, $T^{\bot}_i$, and $T^{err}_i$, such that the following properties are satisfied. 
\begin{itemize}
    \item $T^{\bot}_i$ is the orthogonal complement of $T_i$ in $\hil_{\regX,\regZ}$.
    \item For $\ket{\psi}\in T_i$, for any efficient quantum algorithm $A$ used in Hadamard round, 
    \begin{align*}
        &\Pr[V\left(A(\ket{\psi})\right)=accept]\leq \delta. 
    \end{align*} 
    \takashi{This holds only if $\ket{\psi}$ is generated efficiently.}
    \item For $\ket{\psi^{\bot}}\in T^{\bot}_i$, 
    \begin{align*}
        &\Pr[f^{-1}(y_i)=M_{\regX_i}\circ\ket{\psi^{\bot}}]\leq \epsilon,  
    \end{align*}
    where $M_{\regX_i}$ is the standard-basis measurement in the register $\regX_i$, 
\end{itemize}
\end{lemma}
\nai{Need to specify the relation between $\epsilon$ and $\delta$.}
%\nai{I want to say that $T_i$ is the subspace where the prover can pass the text round but must fail in the Hadamard round. Needs to be more formal.}

%\nai{For $T_i^{\bot}$, do we need to say it cannot pass the test round with high probability? Or, we only need to care $T_i$.}

\begin{lemma}[Jordan's lemma]~\label{lemma:Jordan}
Given any two projectors $\Pi_1$ and $\Pi_2$. There exists a decomposition of the Hilbert space into one-dimensional and two-dimensional subspaces, which satisfy the following properties: 
\begin{itemize}
    \item All subspaces are orthogonal to each other.
    \item For any two-dimensional subspace $S$, for all $\ket{s}\in S$, $\Pi_1 \ket{s} \in S$ and $\Pi_2 \ket{s} \in S$.
    \item For any two-dimensional subspace $S$, $\Pi_1$ and $\Pi_2$ are rank-one projectors, i.e., there exist two vectors $\ket{v_1}$ and $\ket{v_2}$ in $S$ such that for all $ \ket{s}\in S$,  $\Pi_1 \ket{s} = \ipro{v_1}{s}\ket{v_1}$ and $\Pi_2 |s\rangle =\ipro{v_2}{s}\ket{v_2}$. 
\end{itemize}
\end{lemma}
\begin{proof}
We defer the proof to the Appendix. 
\end{proof}

\begin{proof}[Proof of Lemma~\ref{lem:test_subspace}]

Let $\regC$ be the register for the messages $c_1,\dots,c_m$ from the verifier. We define two projectors
\begin{align*}
    &\Pi_{in} := \opro{0}{0}_{\regC}\otimes I_{\regX,\regZ}\quad and \\
    &\Pi_{out} := H_{\regC_{-i}}U_T^{\dag}\left(I_{\regC}\otimes(\opro{f^{-1}(y_i)}{f^{-1}(y_i)})_{\regX_i}\otimes I_{\regX_{-i},\regZ}\right)U_T H_{\regC_{-i}}.  
\end{align*}
\nai{Need to clarify why we set $\regC_{-i}$ be uniform superposition.}

By Jordan's lemma, we can decompose the space $\hil_{\regC,\regX,\regZ}$ into one-dimensional and two-dimensional subspaces. Let $S_1,\dots,S_{\ell}$ be the two-dimensional subspaces from decomposition. Then, $\Pi_{in}$ on $S_1,\dots,S_{\ell}$ can be represented as $\ket{v_1},\dots,\ket{v_{\ell}}$. We can represent $\ket{v_i}$ as $\ket{0}_{\regC}\ket{v'_j}_{\regX,\regZ}$ for all $j\in [\ell]$ since $\Pi_{in}$ has register $\regC$ be all-zero state. Similarly, we represent $\Pi_{out}$ on $S_1,\dots,S_{\ell}$ as $\ket{w}_1,\dots,\ket{w}_{\ell}$. 

Now, we define a basis 
\begin{align*}
    B_i := \{\ket{v'_j}_{\regX,\regZ}:\; |\ipro{v_j}{w_j}|^2\geq \epsilon\}. 
\end{align*}
We let $T_i := span(B_i)$ and $T_i^{\bot}$ be the orthogonal complement of $T_i$ in $\hil_{\regX,\regZ}$. It is obvious that $\ket{\psi^{\bot}}\in T_i^{\bot}$ accepted with probability at most $\epsilon$. 


Finally, for $\ket{\psi}\in T_i$, we can use amplitude amplification to amplify the accepting probability in the test round to $1$. This implies that the prover only has negligible probability to be rejected in the test round, the probability the prover passes the test round is negligible according to Lemma~\ref{lem:Mah_soundness}. 
\nai{Make it clear how to do amplitude amplification here. We need to ``call the verifier'' to implement the projector $\Pi_{out}$ or the rotation $I-2\Pi_{out}$ since only the verifier knows $f^{-1}(y_i)$.}
\takashi{Lemma~\ref{lem:Mah_soundness} can be applied only if $\ket{\psi}$ is generated by an efficient algorithm. On the other hand, we cannot implement the projection onto $T_i$ efficiently. So here is somehow tricky to argue.}

\end{proof}

\subsection{Approximate and efficient subspace}

For $i\in [k]$, we define $T_i$, $T_i^{err}$, and $T_i^{\bot}$ on space $\hil_{X,Z,C_{-i}}$. Here, $T_i$ is the subspace where  


The following procedure define a projector $G_{i\gamma}$. Let $\epsilon = 1/poly$, $T = poly$, and $T << 1/\epsilon$. 
\begin{itemize}
    \item Let $U = V_i P H_{C_{-i}}$. 
    \item $\Pi_{in} := \opro{0}{0}\otimes I_A$. 
    \item $\Pi_{out}$, $R_1$, $R_2$. 
    \item $Q = R_2R_1$. 
    \item Do phase estimation $E$ with precision $\epsilon/2$. 
    \item i.e. $Q\ket{u} = e^{i\hat{\theta}}\ket{u}$. 
    \begin{align*}
        E\ket{u}\ket{0} \rightarrow \sum_{\theta} \alpha_{\theta} \ket{u}\ket{\theta}. 
    \end{align*}
    such that 
    \begin{align*}
        \sum_{\theta\notin \hat{\theta}+\epsilon/2}|\alpha_{\theta}|^2\leq 2^{-n}. 
    \end{align*}
    \item Sample $\gamma$ from $j\epsilon$ for $j\in [T]$
    \begin{align*}
        \ket{u}\ket{\theta}\ket{0} \rightarrow \ket{u}\ket{\theta}\ket{b}
    \end{align*}
    $b=1$ iff $2\pi -(\gamma-\epsilon/2)>\theta>\gamma-\frac{\epsilon}{2}$. 
    \item Uncompute $E$. 
    \item Measure the last qubit (the index of the threshold). 
\end{itemize}

\begin{align*}
    S_{in}:= \opro{0}{0}\otimes I_A \mbox{ wrt }Q_i
\end{align*}
decompose to $\{\ket{u_j}, \theta_j\}$. Here $u_j$ is given by Jordan's Lemma. 

for $\ket{\psi}\in S_{in}$
let 
\begin{align*}
    \ket{\psi} = \sum_{j} \alpha_j \ket{u_j}. 
\end{align*}
Fix $\gamma$ define 
\begin{align*}
    &\ket{\psi_0} = \sum_{j: \theta_j <\gamma-\epsilon} \alpha_j\ket{u_j} = \Pi_{\geq \gamma} \ket{\psi}\\
    &\ket{\psi_1} = \sum_{j: \theta_j >\gamma }\alpha_j\ket{u_j} = \Pi_{\leq \gamma-\epsilon} \ket{\psi} \\
    &\ket{\psi_{err}} = \sum_{j: others} \alpha_j\ket{u_j} 
\end{align*}

\begin{align*}
    G_{i,\gamma} (\ket{\psi_0} + \ket{\psi_1}) 
\end{align*}
project to $\ket{\psi_0}$ or $\ket{\psi_1}$ with small error from phase estimation. 

\begin{align*}
    &G_{i,\gamma} \ket{\psi} \rightarrow \ket{\tilde{\psi}_0}\mbox{ or }\ket{\tilde{\psi}_1}.\\ 
    &\ket{\tilde{\psi}_b} = \ket{\psi_b}+G_{i,\gamma}\ket{\psi_{err}}.
\end{align*}

Need to prove the following inequality. 
\begin{align*}
    \|\ket{\tilde{\psi}_b} - \ket{\psi_b}\|<\|\ket{\psi_{err}}\|.
\end{align*}

\begin{align*}
    E[\|\ket{\psi_{err}}\|] \leq \sqrt{1/T} \mbox{ or } 1/T. 
\end{align*}

Note: 
\begin{align*}
    \ket{\tilde{\psi}_1}\mbox{ reject by Hadamard round w.h.p}. 
\end{align*}
It seems that we can just analyze $\ket{\tilde{\psi}_b}$ instead of $\ket{\psi_b}$. 



\subsection{older proof}
\begin{lemma}
For $i\in [m]$, we can partition $\hil_{\regX,\regZ}$ into $T_{i}$ and $T_{i}^{\bot}$, such that the following properties are satisfied. 
For all $i\in[m]$, $c_{<i}\in \bit^{i-1}$, and efficiently implementable unitary $U$,
\begin{itemize}
    \item $T_{i}^{\bot}$ is the orthogonal complement of $T_{i}$ in $\hil_{\regX,\regZ}$.

    \item We have 
    \begin{align*}
     \mathbb{E}_{k,y}\left[\left\|\Pi_{i,0}\Pi_{i-1,c_{i-1}}...\Pi_{1,c_1}U\ket{\psi_{k,y}}\right\|^2\right]\leq   \mathbb{E}_{k,y}\left[\left\|\Pi_{T_{i}}U^{\dagger}\Pi_{i-1,c_{i-1}}...\Pi_{1,c_1}U\ket{\psi_{k,y}}\right\|^{2}\right]+\left(2^{m+4}\epsilon\right)^{1/2}
    \end{align*}
    
     \item Assuming LWE, we have 
    \begin{align*}
     \mathbb{E}_{k,y}\left[\left\|\Pi_{i,1}\Pi_{i-1,c_{i-1}}...\Pi_{1,c_1}U\ket{\psi_{k,y}}\right\|^{2}\right]\leq   \mathbb{E}_{k,y}\left[\left\|\Pi_{T_{i}^{\bot}}U^{\dagger}\Pi_{i-1,c_{i-1}}...\Pi_{1,c_1}U\ket{\psi_{k,y}}\right\|^{2}\right]+\negl(\secpar)
    \end{align*}
\end{itemize}
\end{lemma}
\begin{proof}
In the following, we denote $\ket{\psi_{k,y,i}}\defeq U^{\dagger}\Pi_{i-1,c_{i-1}}...\Pi_{1,c_1}U\ket{\psi_{k,y}}$.
For each $i\in [m]$, we apply Lemma~\ref{lem:decomp} w.r.t $\A_i$ 
%that applies the second stage $\A$ with a random $c$ and then measures $\regX_i$. 
%More precisely, 
that, given an input $\ket{\psi}$, applies $U$ to $\ket{\psi}$, measures $\regX_i$, and outputs the measurement outcome. 
By Lemma~\ref{lem:decomp}, there exist a partition ($T_i,T_i^{\bot}$) of $\hil_{\regX,\regZ}$ and an efficient quantum algorithm $\ext_i$ such that for any state $\ket{\psi}$, 
\begin{itemize} 
  \item
    We have $\Pr[f(x)=1: x\sample \ext_i(\Pi_{T_i}\ket{\psi})]=1$.
    \item We have $\Pr[f(x)=1:x\sample \A_i(\Pi_{T_i^{\bot}}\ket{\psi})]\leq \epsilon$.
    \item Measurement w.r.t. $\{\Pi_{T_i}, \Pi_{T_i^{\bot}}\}$ can be done efficiently.
\end{itemize}
For any fixed $k$, $y$, and $c$, we have
\begin{align*}
    &~~~\left\|\Pi_{i,1}\Pi_{i-1,c_{i-1}}...\Pi_{1,c_1}U\ket{\psi_{k,y}}\right\|\\
    &=\left\|\Pi_{i,1}U\ket{\psi_{k,y,i}}\right\|\\
    &\leq \left\|\Pi_{i,1}U\Pi_{T_i}\ket{\psi_{k,y,i}}\right\|+\left\|\Pi_{i,1}U\Pi_{T_i^\bot}\ket{\psi_{k,y,i}}\right\|\\
    &\leq  \left\|\Pi_{T_i}\ket{\psi_{k,y,i}}\right\|+\left\|\Pi_{i,1}U\Pi_{T_i^\bot}\ket{\psi_{k,y,i}}\right\|.
\end{align*}
By squaring both sides of the inequality, we obtain
\begin{align*}
    &~~~\left\|\Pi_{i,1}\Pi_{i-1,c_{i-1}}...\Pi_{1,c_1}U\ket{\psi_{k,y}}\right\|^2\\
    &\leq  \left\|\Pi_{T_i}\ket{\psi_{k,y,i}}\right\|^2+\left\|\Pi_{i,1}U\Pi_{T_i^\bot}\ket{\psi_{k,y,i}}\right\|^2+2\left\|\Pi_{T_i}\ket{\psi_{k,y,i}}\right\|\left\|\Pi_{i,1}U\Pi_{T_i^\bot}\ket{\psi_{k,y,i}}\right\|.
\end{align*}

By the second property above, we have
\begin{align*}
    \mathbb{E}_{c}\left[\left\|\Pi_{i,1}U\Pi_{T_i^\bot}\ket{\psi_{k,y,i}}\right\|^{2}\right]\leq \epsilon
\end{align*}
which implies 
\begin{align*}
    \left\|\Pi_{i,1}U\Pi_{T_i^\bot}\ket{\psi_{k,y,i}}\right\|^{2}\leq 2^{m}\epsilon
\end{align*}
for all $c\in \bit^{m}$.
In addition, it is trivial that we have
\begin{align*}
    \left\|\Pi_{T_i}\ket{\psi_{k,y,i}}\right\|\leq 1.
\end{align*}
Then we obtain
\begin{align*}
    &~~~\left\|\Pi_{i,1}\Pi_{i-1,c_{i-1}}...\Pi_{1,c_1}U\ket{\psi_{k,y}}\right\|^2\\
    &\leq  \left\|\Pi_{T_i}\ket{\psi_{k,y,i}}\right\|^2+2^{m}\epsilon+2\left(2^{m}\epsilon\right)^{1/2}\\
    &\leq \left\|\Pi_{T_i}\ket{\psi_{k,y,i}}\right\|^2+\left(2^{m+4}\epsilon\right)^{1/2}
\end{align*}

Summing up over all $(k,y)$, we obtain the first inequality.

We move on to the proof of the second inequality.
%First, if $\mathbb{E}_{k,y}[\|\ket{\psi_{k,y,i}}\|]=\negl(\secpar)$, then we have 
%\begin{align*}
%\mathbb{E}_{k,y}[\|\Pi_{i,1}\Pi_{i-1,c_{i-1}}...\Pi_{1,c_1}U_{c}\ket{\psi_{k,y}}\|]\leq \mathbb{E}_{k,y}[\|\ket{\psi_{k,y,i}}\|]=\negl(\secpar)   
%\end{align*}
%and thus the inequality trivially holds.
%In the following, we assume that $\mathbb{E}_{k,y}[\|\ket{\psi_{k,y,i}}\|]$ is non-negligible.
Similarly to the above, we have 
\begin{align*}
    &~~~~\mathbb{E}_{k,y}\left[\left\|\Pi_{i,0}\Pi_{i-1,c_{i-1}}...\Pi_{1,c_1}U\ket{\psi_{k,y}}\right\|\right]\\
    &=\mathbb{E}_{k,y}\left[\left\|\Pi_{i,0}U\ket{\psi_{k,y,i}}\right\|\right]\\
    &\leq \mathbb{E}_{k,y}\left[\left\|\Pi_{i,0}U\Pi_{T_i}\ket{\psi_{k,y,i}}\right\|\right]+\mathbb{E}_{k,y}\left[\left\|\Pi_{i,0}U\Pi_{T_i^\bot}\ket{\psi_{k,y,i}}\right\|\right]\\
    &\leq \mathbb{E}_{k,y}\left[\left\|\Pi_{i,0}U\Pi_{T_i}\ket{\psi_{k,y,i}}\right\|\right]+\mathbb{E}_{k,y}\left[\left\|\Pi_{T_i^\bot}\ket{\psi_{k,y,i}}\right\|\right]
\end{align*}
Thus it suffices to show 
\begin{align*}
   \mathbb{E}_{k,y}\left[\left\|\Pi_{i,0}U\Pi_{T_i}\ket{\psi_{k,y,i}}\right\|\right]=\negl(\secpar). 
\end{align*}
To prove this, we assume that 
\begin{align*}
   \mathbb{E}_{k,y}\left[\left\|\Pi_{i,0}U\Pi_{T_i}\ket{\psi_{k,y,i}}\right\|\right]\geq 1/\poly(\secpar). 
\end{align*}
for some polynomial $\poly$, and construct an adversary $\B$ that breaks Lemma~\ref{lem:Mah_soundness}.

$\B$ is given $k_i$ as the first round message, and repeats the following procedure $\Theta(\poly(\secpar))$ times:
\begin{enumerate}
\item Pick $k_{-i}$ according to the protocol to complete $k=(k_1,...,k_m)$
\item Pick $c_{<i}\sample \bit^{i-1}$.
\item Run the first stage of $\A$ to obtain $y$ along with its internal state $\ket{\psi_{k,y}}$.
\item Apply $U$ to $\ket{\psi_{k,y}}$ and measures $\regX_1,..,\regX_{i-1}$ to obtain $a_{1}$,...,$a_{i-1}$.
\item If at least one of $a_1$,...,$a_{i-1}$ is not an accepting answer w.r.t. the transcript of the corresponding coordinate, then immediately abort this trial and go back to the 1st step of the loop. 
\item Apply $U^{\dagger}$ to the post-measurement state to obtain $\ket{\psi_{k,y,i}}$.
\item Apply a measurement $\{\Pi_{T_i},\Pi_{T_i^{\bot}}\}$ to $\ket{\psi_{k,y,i}}$. 
If the measurement gives  $\Pi_{T_i^{\bot}}\ket{\psi_{k,y,i}}$, then it immediately aborts this trial and goes back to the 1st step of the loop. Otherwise, it goes out the loop to continue the attack.
\end{enumerate}
Since we assume \begin{align*}
   \mathbb{E}_{k,y}\left[\left\|\Pi_{i,0}U\Pi_{T_i}\ket{\psi_{k,y,i}}\right\|\right]\geq 1/\poly(\secpar),
\end{align*}
we especially have 
\begin{align*}
   \mathbb{E}_{k,y}\left[\left\|\Pi_{T_i}\ket{\psi_{k,y,i}}\right\|\right]\geq 1/\poly(\secpar).    
\end{align*}
Therefore in each trial in the loop, the probability that $\B$ does not abort is at least $1/\poly(\secpar)$. 
Thus, $\B$ goes out the loop within $\Theta(\poly(\secpar))$ times repetitions with overwhelming probability.
Assuming this happens, $\B$ now has $k,y$ along with the corresponding state $\ket{\psi_{\B}}\defeq \Pi_{T_i}\ket{\psi_{k,y,i}}/\left\|\Pi_{T_i}\ket{\psi_{k,y,i}}\right\|$.
Then it sends $y_i$ to the verifier.
Then verifier returns $c_i$.
\begin{itemize}
\item If $c_i=0$, $\B$ runs $\ext_i(\ket{\psi_{\B}})$ to obtain $a_i$. This is an accepting answer with probability 1 by the first property above.
\item If $c_i=1$, $\B$ applies $U$ to  $\ket{\psi_{\B}}$ and then measures $\regX_i$ to obtain $a_i$. The probability that $a_i$ is accepted is 
\begin{align*}
   &~~~~\mathbb{E}_{k,y}\left[\left\|\Pi_{i,0}U\left(\Pi_{T_i}\ket{\psi_{k,y,i}}/\|\Pi_{T_i}\ket{\psi_{k,y,i}}\|\right)\right\|\right]\\
    &\leq \mathbb{E}_{k,y}\left[\left\|\Pi_{i,0}U\Pi_{T_i}\ket{\psi_{k,y,i}}\right\|\right],
\end{align*}
which is assumed to be non-negligible.

\takashi{More careful analysis may be needed here since non-aborting probability in the loop depends on $(k,y)$.}
\end{itemize}

Overall, $\B$ succeeds in answering a correct answer with overwhelming probability for the case of $c_i=0$ and with non-negligible probability for the case of $c_i=1$, which contradicts Lemma~\ref{lem:Mah_soundness}.
Therefore, under the LWE assumption, we have
\begin{align*}
   \mathbb{E}_{k,y}[\|\Pi_{i,0}U\Pi_{T_i}\ket{\psi_{k,y,i}}\|]=\negl(\secpar), 
\end{align*}
which concludes the proof of the proof of Lemma~\ref{lem:decomp2}.





\end{proof}

\begin{theorem}
\begin{align*}
\Pr[\langle P^*,V \rangle=accept]=\mathbb{E}_{k,y,c}\left[\left\|\Pi_{m,c_m}...\Pi_{1,c_1}U_{c}\ket{\psi_{k,y}}\right\|^{2}\right]\leq 2^{-m}+(2^{m+4}\epsilon)^{1/2} +\negl(\secpar).
\end{align*}
\end{theorem}
\begin{proof}
Let $N_{k,y,i,c}\defeq \left\|\Pi_{i,c_i}...\Pi_{1,c_1}U_{c}\ket{\psi_{k,y}}\right\|^{2}$ for notational simplicity.
Summing up the two inequalities in Lemma~\ref{lem:decomp2} and dividing it by $2$, under the LWE assumption, for any $i$, $c_{<i}$, and $U$, we have 
\begin{align*}
     \mathbb{E}_{k,y,c_i}\left[\left\|\Pi_{i,c_i}\Pi_{i-1,c_{i-1}}...\Pi_{1,c_1}U\ket{\psi_{k,y}}\right\|^{2}\right]\leq \frac{1}{2}\left(\mathbb{E}_{k,y}\left[\left\|\Pi_{i-1,c_{i-1}}...\Pi_{1,c_1}U\ket{\psi_{k,y}}\right\|^{2}\right]+\left(2^{m+4}\epsilon\right)^{1/2}+\negl(\secpar)\right).  
\end{align*}
By setting $U\defeq U_c$, for any fixed choice of $c_{-i}$, we have
\begin{align*}
\mathbb{E}_{k,y,c_{i}}\left[N_{k,y,i,c}\right]\leq \frac{1}{2}\left(\mathbb{E}_{k,y,c_i}\left[N_{k,y,i-1,c}\right]+\left(2^{m+4}\epsilon\right)^{1/2}+\negl(\secpar)\right).
\end{align*}
\takashi{Should check if $U$ can depend on $c$. I guess the current argument is wrong.}

Then we have
\begin{align*}
 &~~~~\mathbb{E}_{k,y,c}\left[N_{k,y,i,c}\right]\\
 &=  \mathbb{E}_{c_{-i}}\left[\mathbb{E}_{k,y,c_i}\left[N_{k,y,m,c}\right]\right]\\
 &\leq \mathbb{E}_{c_{-i}}\left[\frac{1}{2}\left(\mathbb{E}_{k,y,c_i}\left[N_{k,y,m-1,c}\right]+\left(2^{m+4}\epsilon\right)^{1/2}+\negl(\secpar)\right)\right]\\
 &=\frac{1}{2}\mathbb{E}_{k,y,c}\left[N_{k,y,i-1,c}\right]+\frac{1}{2}\left(\left(2^{m+4}\epsilon\right)^{1/2}+\negl(\secpar)\right)
\end{align*}

Applying this ineuqality for each $i=m,...,1$, we obtain
\begin{align*}
\mathbb{E}_{k,y,c}\left[N_{k,y,m,c}\right]\leq 2^{-m}+\left(2^{m+4}\epsilon\right)^{1/2}+\negl(\secpar).
\end{align*}
\end{proof}


