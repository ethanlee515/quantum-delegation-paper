

\section{Parallel Repetition of Mahadev's Protocol}

\subsection{Overview of Mahadev's Protocol}\label{sec:mahadev_overview}
Here, we recall the Mahadev's protocol \cite{FOCS:Mahadev18a}. We only give a high-level description of the protocol and properties of it and omit the details since they are not needed to show our result. 

The protocol is run between a quantum prover $\pro$ and a classical verifier $\ver$ on a common input $x$. The aim of the protocol is to enable a verifier to classically verify $x\in \lang$ for a BQP language $\lang$ with the help of interactions with a quantum prover.
The protocol is a 4-round protocol where the first message is sent from $\ver$ to $\pro$. 
We denote the $i$-th message generation algorithm by $\ver_i$ for $i\in\{1,3\}$ or $\pro_i$ for $i\in \{2,4\}$ and denote the verifier's final decision algorithm by $\ver_\out$.
Then a high-level description of the protocol is given below.
\begin{description}
\item[$\ver_1$:] On input the security parameter $1^\secpar$ and $x$, it generates a pair $(\key,\td)$ of a``key" and ``trapdoor", sends $\key$ to $\pro$, and keeps $\td$ as its internal state.
\item[$\pro_2$:] On input $x$ and $\key$, it generates a classical ``commitment" $\comy$ along with a quantum state $\ket{\st_\pro}$, sends $\comy$ to $\pro$, and keeps $\ket{\st_\pro}$ as its internal state.
\item[$\ver_3$:] It randomly picks $c\sample \bit$ and sends $c$ to $\pro$.\footnote{The third message is just a public-coin, and does not depend on the transcript so far or $x$.}
Following the terminology in \cite{FOCS:Mahadev18a}, we call the case of $c=0$ the ``test round" and the case of $c=1$ the ``Hadamard round".
\item[$\pro_4$:] On input $\ket{\st_\pro}$ and $c$, it generates a classical string $\ans$ and sends $\ans$ to $\pro$.
\item[$\ver_\out$:] On input $\key$, $\td$, $y$, $c$, and $\ans$, it returns $\top$ indicating acceptance or $\bot$ indicating rejection.
In case $c=0$, the verification can be done publicly, that is, $\ver_\out$ need not take $\td$ as input.
\end{description}

For the protocol, we have the following properties:\\
\noindent\textbf{Completeness:}
For all $x\in \lang$, we have $\Pr[\langle \pro,\ver \rangle(x)]=\bot]= \negl(\secpar)$.\\
\noindent\textbf{Soundness:}
If the LWE problem is hard for quantum polynomial-time algorithms, then for any $x\notin \lang$ and a quantum polynomial-time cheating prover $\pro^*$, we have  $\Pr[\langle \pro^*,\ver \rangle(x)]=\bot]\leq 3/4$.

We need a slightly different form of soundness implicitly shown in \cite{FOCS:Mahadev18a}, which roughly says that if a cheating prover can pass the ``test round" (i.e., the case of $c=0$) with overwhelming probability, then it can pass the ``Hadamard round" (i.e., the case of $c=1$) only with a negligible probability. 
\begin{lemma}[implicit in \cite{FOCS:Mahadev18a}]\label{lem:Mah_soundness}
If the LWE problem is hard for quantum polynomial-time algorithms, then for any $x\notin \lang$ and a quantum polynomial-time cheating prover $\pro^*$ such that  $\Pr[\langle \pro^*,\ver \rangle(x)]=\bot\mid c=0]=\negl(\secpar)$, we have $\Pr[\langle \pro^*,\ver \rangle(x)]=\top\mid c=1]=\negl(\secpar)$.
\end{lemma}

We will also use the following simple fact:
\begin{fact}\label{fact:perfectly_pass_test}
There exists an efficient prover that passes the test round with probability $1$ (but passes the Hadamard round with probability $0$) even if $x\notin \lang$. 
\end{fact}

\subsection{Parallel Repetition}
Here, we prove that the parallel repetition of the Mahadev's protocol decrease the soundness bound to be negligible.
Let $\pro^m$ and $\ver^m$ be $m$-parallel repetitions of the honest prover $\pro$ and verifier $\ver$ in the Mahadev's protocol. Then we have the following:
\begin{theorem}[Completeness]\label{thm:rep_completeness}
For all $m= \Omega(\log^2(\secpar))$, for all $x\in \lang$, we have $\Pr[\langle \pro^m,\ver^m \rangle(x)]=\bot]= \negl(\secpar)$.\\
\end{theorem}
\begin{theorem}[Soundness]\label{thm:rep_soundness}
For all $m= \Omega(\log^2(\secpar))$, if the LWE problem is hard for quantum polynomial-time algorithms, then for any $x\notin \lang$ and a quantum polynomial-time cheating prover $\pro^*$, we have  $\Pr[\langle \pro^*,\ver^m \rangle(x)]=\top]\leq \negl(\secpar)$.
\end{theorem}
The completeness (Theorem~\ref{thm:rep_completeness}) easily follows from the completeness of the Mahadev's protocol.
In the next subsection, we prove the soundness (Theorem~\ref{thm:rep_soundness}).

\subsection{Proof of Soundness}\label{sec:proof_of_soundness}
First, we remark that it suffices to show that for any $\mu=1/\poly(n)$, there exists $m=O(\log(n))$ such that the success probability of the cheating prover is at most $\mu$.
This is because we are considering $\omega(\log(n))$-parallel repetition, in which case the number of repetitions is larger than   any $m=O(\log(n))$ for sufficiently large $n$, and thus we can just focus on  the first $m$ coordinates ignoring the rest of the coordinates.  
Thus, we prove the above claim in this section.
%We prove the soundness by showing that for any noticeable $\mu=1/\poly(\secpar)$, there exists a number $m=O(\log \secpar)$ such that by parallelly repeating the protocol $m$ times, the soundness error can be reduced to less than $\mu$. 


\noindent\textbf{Characterization of cheating prover.}
Any cheating prover can be characterized by a tuple $(U_0,U)$ of unitaries over Hilbert space $\hil_{\regC}\otimes \hil_{\regX}\otimes \hil_{\regZ} \otimes \hil_{\regY}  \otimes \hil_{\regK}$. 
A prover characterized by $(U_0,U)$ works as follows.\footnote{Here, we hardwire into the cheating prover the instance $x\notin \lang$ on which it will cheat instead of giving it as an input.}
\begin{description}
\item[Second Message:] Upon receiving $k=(\key_1,...,\key_m)$, it applies $U_0$ to the state $\ket{0}_{\regX}\otimes\ket{0}_{\regZ}\otimes\ket{0}_{\regY}\otimes  \ket{k}_{\regK}$, and then measures the $Y$ register to obtain $y=(\comy_1,...,\comy_m)$. Then it sends $\bfy$ to $\ver$ and keeps the resulting state $\ket{\psi(k,y)}_{\regX,\regZ}$ over  $\hil_{\regX,\regZ}$.
\item [Forth Message:] Upon receiving $c\in \bit^{m}$, it applies $U$ to $\ket{c}_{\regC}\ket{\psi(k,y)}_{\regX,\regZ}$ and then measures the $\regX$ register in computational basis to obtain $a=(a_1,...,a_m)$. We denote the designated register for $a_i$ by $\regX_i$. %Then, we can view the verifier's verification procedure on $i$th coordinate as a unitary $V_i$. 
\end{description}




Here, we first introduce a general lemma about two projectors that was shown by Nagaj, Wocjan, and Zhang \cite{NWZ09} by using the Jordan's lemma.
\begin{lemma}[{\cite[Appendix A]{NWZ09}}]\label{lem:decomposition}
Let $\Pi_0$ and $\Pi_1$ be projectors on an $N$-dimensional Hilbert space $\hil$ and let $R_0:= 2\Pi_0-I$, $R_1:= 2\Pi_1-I$, and $Q:=R_1R_0$. $\hil$ can be decomposed into two-dimensional subspaces $S_j$ for $j\in[\ell]$ and $N-2\ell$ one-dimensional subspaces $T_j^{(bc)}$ for $b,c\in \bit$ that satisfies the following properties:
\begin{enumerate}
\item For each two-dimensional subspace $S_j$, there exist two orthonormal bases $(\ket{\alpha_j},\ket{\alpha_j^{\bot}})$ and $(\ket{\beta_j},\ket{\beta_j^{\bot}})$ of $S_j$ such that $\ipro{\alpha_j}{\beta_j}$ is a positive real and for all $ \ket{s}\in S_j$,  $\Pi_0 \ket{s} = \ipro{\alpha_j}{s}\ket{\alpha_j}$ and $\Pi_1 |s\rangle =\ipro{\beta_j}{s}\ket{\beta_j}$.  Moreover, $Q$ is a rotation with eigenvalues $e^{\pm i\theta_j}$ in $S_j$ corresponding to eigenvectors $\ket{\phi_j^+}=\frac{1}{\sqrt{2}}(\ket{\alpha_j}+i\ket{\alpha_j^\bot})$ and $\ket{\phi_j^-}=\frac{1}{\sqrt{2}}(\ket{\alpha_j}-i\ket{\alpha_j^\bot})$
where $\theta_j=2\arccos \ipro{\alpha_j}{\beta_j}=2\arccos\sqrt{\bra{\alpha_j}\Pi_1\ket{\alpha_j}}$.
\item  Each one-dimensional subspace $T_j^{(bc)}$ is spanned by a vector $\ket{\alpha_j^{(bc)}}$ such that $\Pi_0 \ket{\alpha_j^{(bc)}}=b \ket{\alpha_j^{(bc)}}$ and $\Pi_1 \ket{\alpha_j^{(bc)}}=c \ket{\alpha_j^{(bc)}}$.
\end{enumerate}
\end{lemma}


Fix $k$ and $y$ for now (until we finish the proof of Lemma~\ref{lem:partition_further}). For $i\in [m]$, we consider two projectors 
\begin{align*}
    &\Pi_{in}:= \opro{0^m}{0^m}_{\regC}\otimes I_{\regX,\regZ}\\
    %&\Pi_{i,out} := (UH_{\regC_{-i}})^{\dag}(\sum_{b,x: f_{k}(b,x)=y}\opro{b,x}{b,x}_{\regX_i})\otimes I_{\regC,\regX_{-i},\regZ} (UH_{\regC_{-i}}),
   & \Pi_{i,out} := (UH_{\regC_{-i}})^{\dag}(\sum_{a_i\in \Acc_{k_i,y_i}}\opro{a_i}{a_i}_{\regX_i}\otimes I_{\regC,\regX_{-i},\regZ}) (UH_{\regC_{-i}}),
\end{align*}
where 
$\regX_{-i}:= \regX_1,\dots,\regX_{i-1}, \regX_{i+1},\dots, \regX_{m}$,  $H_{\regC_{-i}}$ means applying Hadamard operators to registers $\regC_1,\dots,\regC_{i-1}, \regC_{i+1},\dots, \regC_{m}$, and $\Acc_{k_i,y_i}$ denotes the set of $a_i$ such that the verifier accepts $a_i$ in the test round on the $i$-th coordinate when the first and second messages are $k_i$ and $y_i$, respectively.
Note that one can efficiently check if $a_i\in \Acc_{k_i,y_i}$ without knowing the trapdoor behind $k_i$ since verification in the test round can be done publicly as explained in Sec. \ref{sec:mahadev_overview}. 
We apply Lemma~\ref{lem:decomposition} for $\Pi_0= \Pi_{in}$ and $\Pi_1=\Pi_{i,out}$ to decompose the space  $\hil_{\regC,\regX,\regZ}$ into the two-dimensional subspaces $\{S_j\}_{j}$ and one-dimensional subspaces $\{T_{j}^{bc}\}_{j,b,c}$.
In the following, we use notations defined in Lemma~\ref{lem:decomposition} for this particular application.
We can write $\ket{\alpha_j}_{\regC,\regX,\regZ}=\ket{0}\ot \ket{\hat{\alpha}_j}_{\regX,\regZ}$  since $\Pi_{in}\ket{\alpha_j}=\ipro{\alpha_j}{\alpha_j}\ket{\alpha_j}=\ket{\alpha_j}$.
Similarly, we can write  $\ket{\alpha_j^{(10)}}_{\regC,\regX,\regZ}=\ket{0}\ot \ket{\hat{\alpha}_j^{(10)}}_{\regX,\regZ}$ 
and  $\ket{\alpha_j^{(11)}}_{\regC,\regX,\regZ}=\ket{0}\ot \ket{\hat{\alpha}_j^{(11)}}_{\regX,\regZ}$. 
Then  $\{\ket{\hat{\alpha}_j}\}_{j}$ and $\{\ket{\hat{\alpha}_{j}^{(1c)}}\}_{j,c}$ span $\hil_{\regX,\regZ}$. 

First, we prove a lemma which gives a decomposition of a cheating prover's state.

\begin{lemma}\label{lem:partition}
Let $(U_0,U)$ be any prover's strategy. Let $m=O(\log \secpar)$, $i\in[m]$, 
$\gamma_0 \in[0,1]$, and $T\in \mathbb{N}$ such that $\frac{\gamma_0}{T}=1/\poly(\secpar)$. Let $\gamma$ be sampled uniformly randomly from $[\frac{\gamma_0}{T},\frac{2\gamma_0}{T},\dots,\frac{T\gamma_0}{T}]$. Then, there exists an efficient quantum procedure $G_{i,\gamma}$ such that for any (possibly sub-normalized) quantum state $\ket{\psi}_{\regX,\regZ}$,  
\begin{align*}
    G_{i,\gamma} \ket{0^m}_{\regC}\ket{\psi}_{\regX,\regZ}\ket{0^t}_{ph}\ket{0}_{th}\ket{0}_{in} = z_0\ket{0^m}_{\regC}\ket{\psi_{0}}_{\regX,\regZ}\ket{0^t01}_{ph,th,in}+ z_1\ket{0^m}_{\regC}\ket{\psi_{1}}_{\regX,\regZ}\ket{0^t11}_{ph,th,in} + \ket{\psi'_{err}}
\end{align*}
where $t$ is the number of qubits in the register $ph$, $z_0,z_1\in \mathbb{C}$ such that $|z_0|=|z_1|=1$, and 
$z_0$, $z_1$, $\ket{\psi_0}_{\regX,\regZ}$, $\ket{\psi_1}_{\regX,\regZ}$, and $\ket{\psi_{err}}_{\regX,\regZ}$ may depend on $\gamma$.

Furthermore, the following properties are satisfied.
%
\begin{enumerate}
%\item There exist (sub-normalized) states $\ket{\psi_0}_{\regX,\regZ}$ and $\ket{\psi_1}_{\regX,\regZ}$ such that when $G_{i,\gamma}$ runs on input $\ket{\psi}_{\regX,\regZ}$, the output is  $(0,\ket{\psi_0}_{\regX,\regZ}/\|\ket{\psi_0}_{\regX,\regZ}\|)$ with probability $\|\ket{\psi_0}_{\regX,\regZ}\|^2$, $(1,\ket{\psi_1}_{\regX,\regZ}/\|\ket{\psi_1}_{\regX,\regZ}\|)$ with probability $\|\ket{\psi_1}_{\regX,\regZ}\|^2$, and $\bot$ with probability  $1-\|\ket{\psi_0}_{\regX,\regZ}\|^2- \|\ket{\psi_1}_{\regX,\regZ}\|^2$,
    \item If we define $\ket{\psi_{err}}_{\regX,\regZ}\defeq \ket{\psi}_{\regX,\regZ} - \ket{\psi_{0}}_{\regX,\regZ}- \ket{\psi_{1}}_{\regX,\regZ}$, then we have  $E_{\gamma}[\|\ket{\psi_{err}}_{\regX,\regZ}\|^2]\leq \frac{6}{T}+\negl(n)$.
\item For any fixed $\gamma$, $\Pr[M_{ph,th,in}\circ \ket{\psi'_{err}} \in \{0^t01,0^t11\}] =0$. %where $M_{ph,th,in}$ is the computational-basis measurement in the register $(ph,th,in)$. %and $t$ is the number of qubits in $ph$.   
This implies that if we apply the measurement $M_{ph,th,in}$ on $\frac{G_{i,\gamma} \ket{0^m}_{\regC}\ket{\psi}_{\regX,\regZ}\ket{0^t}_{ph}\ket{0}_{th}\ket{0}_{in}}{\|\ket{\psi}_{\regX,\regZ}\|}$, then the outcome is $0^tb1$ with probability $\|\ket{\psi_b}_{\regX,\regZ}\|^2$ and the resulting state in the register $(\regX,\regZ)$  is $\frac{\ket{\psi_b}_{\regX,\regZ}}{\|\ket{\psi_b}_{\regX,\regZ}\|}$ ignoring a global phase factor.
    \item For any fixed $\gamma$, $E_{b\in \{0,1\}} [\|\ket{\psi_b}_{\regX,\regZ}\|^2]\leq \frac{1}{2}\|\ket{\psi}_{\regX,\regZ}\|^2$. 
        \item 
For any fixed $\gamma$ and $c\in \bit^m$ such that $c_i=0$, we have 
\begin{align*}
\Pr\left[M_{\regX_i}\circ U\frac{\ket{c}_{\regC}\ket{\psi_0}_{\regX,\regZ}}{\|\ket{\psi_0}_{\regX,\regZ}\|}\in \Acc_{k_i,y_i}\right]\leq 2^{m-1}\gamma+\negl(\secpar).
\end{align*}
    \item 
For any fixed $\gamma$, there exists an efficient quantum algorithm $\ext_i$ such that 
\begin{align*}  
  \Pr\left[\ext_i\left(\frac{\ket{0^m}_{\regC}\ket{\psi_1}_{\regX,\regZ}}{\|\ket{\psi_1}_{\regX,\regZ}\|}\right)\in \Acc_{k_i,y_i}\right]=1-\negl(\secpar).
  \end{align*}   
\end{enumerate}
\end{lemma}

%\begin{remark}
%To prove Theorem~\ref{thm:rep_soundness}, we only need $m$ to be at most $\log(n)$. Hence, $\gamma_0$ and $T$ can be $1/\poly(n)$. 
%\end{remark}

\begin{proof}[Proof of Lemma~\ref{lem:partition}]
Procedure~\ref{fig:process_G} defines an efficient process $G_{i,\gamma}$, which decomposes $\ket{\psi}_{\regX,\regZ}$ into $\ket{\psi_0}_{\regX,\regZ}$ , $\ket{\psi_1}_{\regX,\regZ}$ , and $\ket{\psi_{err}}_{\regX,\regZ}$  described in Lemma~\ref{lem:partition}.    
Here, $G_{i,\gamma} := U_{in}U^{\dag}_{est}U_{th}U_{est}$ operates on register $\regC$, $\regX$, $\regZ$, and additional registers $ph$, $th$, and $in$, and we let $\delta:=\frac{\gamma_0}{3T}$.


\floatname{algorithm}{Procedure}
\begin{algorithm}[h]
    \begin{mdframed}[style=figstyle,innerleftmargin=10pt,innerrightmargin=10pt]
    \begin{enumerate}
   % \item Uniformly choose $\gamma$ from $[\frac{\gamma_0}{T},\frac{2\gamma_0}{T},\dots,\frac{T\gamma_0}{T}]$ 
    \item Do quantum phase estimation $U_{est}$ on $Q=(2\Pi_{in}-I)(2\Pi_{i,out}-I)$ with input state $\ket{0^m}_{\regC}\ket{\psi}_{\regX,\regZ}$ and $\tau$-bit precision and failure probability $2^{-n}$ where the parameter $\tau$ will be specified later, i.e.,  
    \begin{align*}
        U_{est}\ket{u}_{\regC,\regX,\regZ}\ket{0^t}_{ph} \rightarrow \sum_{\theta\in(-\pi,\pi]} \alpha_{\theta} \ket{u}_{\regC,\regX,\regZ}\ket{\theta}_{ph}.
    \end{align*}
 such that $\sum_{\theta\notin \bar{\theta}\pm 2^{-\tau}}|\alpha_{\theta}|^2\leq 2^{-\secpar}$ for any eigenvector $\ket{u}_{\regC,\regX,\regZ}$ of $Q$ with eigenvalue $e^{i\bar{\theta}}$.
 %\item Apply $U_{cos}:\ket{u}_{\regC,\regX,\regZ}\ket{\theta'}_{ph}\ket{0}_{pr} \xrightarrow{U_{cos}} \ket{u}_{\regC,\regX,\regZ}\ket{\theta'}_{ph}\ket{\cos^2(\theta'/2)}_{pr}$. 
    \item Apply $U_{th}:\ket{u}_{\regC,\regX,\regZ}\ket{\theta}_{ph}\ket{0}_{th} \xrightarrow{U_{th}} \ket{u}_{\regC,\regX,\regZ}\ket{\theta}_{ph}\ket{b}_{th} $, 
    where $b=1$ if $\cos^2 (\theta/2)\geq \gamma-\delta$.
    %$\cos^2 \theta\in [\gamma-\delta/2, \gamma + \delta/2]$. 
    \item Apply $U_{est}^\dag$. 
   % \item Measure $ph$ and $th$ registers and let $m$ be the measurement outcome. If $m=0^{\tau}b$ for $b\in \bit$, then it returns $b$ and the quantum state in the registers $(\regX,\regZ)$, and $\bot$ otherwise where $\tau$ is the number of qubits in the $ph$ register.
   \item Apply $U_{in}: \ket{c}_{\regC}\ket{0}_{in} \xrightarrow{U_{in}}  \ket{c}_{\regC}\ket{b'}_{in}$,
    where $b'=1$ if $c= 0^m$. 
    %\item Measure the single-qubit registers $th$ and $err$. 
\end{enumerate}
    \caption{$G_{i,\gamma}$}
    \label{fig:process_G}
    \end{mdframed}
\end{algorithm}

%We can consider $U_c$ as a unitary $U$ operating on registers $\regC$, $\regX$, and $\regZ$. 



%First, we remark that $\ket{\alpha_1},...,\ket{\alpha_\ell}$ and $\{\ket{\alpha_{j}^{(1c)}}\}$ spans $\ket{0^m}_{\regC}\ot \hil_{\regX,\regZ}$. 

In the procedure, we choose $\tau$ so that for any $\theta$ and $\theta'$ such that $|\theta'-\theta|\leq 2^{-\tau}$, we have $|\cos^2(\theta'/2)-\cos^2(\theta/2)|\leq \delta/2$.
Since we can upper and lower bound $\cos^2(\theta'/2)-\cos^2(\theta/2)$ by polynomials in $\theta'-\theta$ by considering the Taylor series, we can set $\tau=O(-\log(\delta))$ for satisfying this property.\takashi{More explanation may be needed?}
Since phase estimation with $\tau$-bit precision and failure probability $2^{-\secpar}$ can be done in time $\poly(\secpar,2^{\tau})$ \cite{NWZ09} and $\delta=\frac{\gamma_0}{3T}=1/\poly(\secpar)$ by the assumption, the procedure runs in time $\poly(\secpar)$.

For each $j\in [\ell]$, we define $p_j:= \cos^2(\theta_j/2)=\bra{\alpha_j} \Pi_{i,out} \ket{\alpha_j}$.
We define the following projections on $\hil_{\regX,\regZ}$:
\begin{align*}
    &\Pi_{in, \leq \gamma-2\delta} := \sum_{j: p_j\leq \gamma-2\delta}\opro{\hat{\alpha}_j}{\hat{\alpha}_j}_{\regX,\regZ}+ \sum_{j} \opro{\hat{\alpha}_j^{10}}{\hat{\alpha}_j^{10}}_{\regX,\regZ},\\
     &\Pi_{in, \geq \gamma} := \sum_{j: p_j\geq  \gamma}\opro{\hat{\alpha}_j}{\hat{\alpha}_j}_{\regX,\regZ}+ \sum_{j} \opro{\hat{\alpha}_j^{11}}{\hat{\alpha}_j^{11}}_{\regX,\regZ},\\
    &\Pi_{in, mid} := \sum_{j: p_j \in (\gamma-2\delta,\gamma)}\opro{\hat{\alpha}_j}{\hat{\alpha}_j}_{\regX,\regZ}.
\end{align*}
 
We let $\ket{\psi_{\leq \gamma-2\delta}}_{\regX,\regZ}:=\Pi_{in, \leq \gamma-2\delta}\ket{\psi}_{\regX,\regZ}$,  $\ket{\psi_{\geq \gamma}}_{\regX,\regZ}:=\Pi_{in, \geq \gamma}\ket{\psi}_{\regX,\regZ}$, and $\ket{\psi_{mid}}_{\regX,\regZ}:=\Pi_{in, mid}\ket{\psi}_{\regX,\regZ}$.
Then we have 
\begin{align}
\ket{\psi}_{\regX,\regZ}=\ket{\psi_{\leq \gamma-2\delta}}_{\regX,\regZ}+\ket{\psi_{\geq\gamma}}_{\regX,\regZ}+\ket{\psi_{mid}}_{\regX,\regZ}. \label{eq:psi}
\end{align}
Roughly speaking, $\ket{\psi_{\leq \gamma-2\delta}}_{\regX,\regZ}$, $\ket{\psi_{\geq\gamma}}_{\regX,\regZ}$, $\ket{\psi_{mid}}_{\regX,\regZ}$ will correspond to $\ket{\psi_0}$, $\ket{\psi_1}$, and $\ket{\psi_{err}}$ with some error terms as explained in the following.

It is easy to see that $E_{\gamma}[\|\ket{\psi_{mid}}\|^2]\leq \frac{1}{T}$ since $\Pi_{in, mid}$ with different choice of $\gamma$ are disjoint. \takashi{Explanation may be needed.}
In the following, we analyze how the first two terms of Eq. \ref{eq:psi} develops by $G_{i,\gamma}$. 

$\ket{\psi_{\leq\gamma-2\delta}}_{\regX,\regZ}$ is a superposition of states $\{\ket{\hat{\alpha}_j}_{\regX,\regZ}\}_{j:p_j\leq \gamma-2\delta}$ and $\{\ket{\hat{\alpha}_j^{11}}_{\regX,\regZ}\}_{j}$.
By Lemma~\ref{lem:decomposition}, $\ket{\alpha_j}_{\regC,\regX,\regZ}=\ket{0^m}_{\regC}\ot \ket{\hat{\alpha}_j}_{\regX,\regZ}$ can be written as $\ket{\alpha_j}_{\regC,\regX,\regZ}=\frac{1}{\sqrt{2}}(\ket{\phi_j^+}_{\regC,\regX,\regZ}+\ket{\phi_j^-}_{\regC,\regX,\regZ})$ where $\ket{\phi_j^\pm}_{\regC,\regX,\regZ}$ is an eigenvector of $Q$ with eigenvalue $e^{\pm i\theta_j}$ where $\theta_j=2\arccos(\sqrt{p_j})\geq 2\arccos (\sqrt{\gamma-2\delta})$. 
Moreover,  $\ket{\alpha_j^{(10)}}_{\regC,\regX,\regZ}=\ket{0^m}_{\regC}\ot \ket{\hat{\alpha}_j^{(10)}}_{\regX,\regZ}$ is an eigenvector of $Q$ with eigenvalue $-1=e^{i\pi}$. 
Here, we remark that $\pi\geq 2\arccos x$ for any $0\leq x \leq 1$.
Thus, after applying $U_{est}$ to  $\ket{\psi_{\leq\gamma-2\delta}}_{\regX,\regZ}$, $(1-2^{-\secpar})$-fraction of the state contains $\theta$ in the register $ph$ such that $|\theta|\geq 2\arccos(\sqrt{\gamma-2\delta}) -2^{-\tau}$. which implies $\cos^2(\theta/2)\leq \gamma-\frac{3}{2}\delta<\gamma-\delta$ by our choice of $\tau$.
For this fraction of the state, $U_{th}$ does nothing.
 Thus, we have 
 \begin{align*}
 \TD(U_{est}\ket{0^m}_{\regC}\ket{\psi_{\leq\gamma-2\delta}}_{\regX,\regZ}\ket{0^t00}_{ph,th,in},U_{th}U_{est}\ket{0^m}_{\regC}\ket{\psi_{\leq\gamma-2\delta}}_{\regX,\regZ}\ket{0^t00}_{ph,th,in})\leq 2^{-n}
 \end{align*}
  and thus
  \begin{align*}
  \TD(\ket{0^m}_{\regC}\ket{\psi_{\leq\gamma-2\delta}}_{\regX,\regZ}\ket{0^t00}_{ph,th,in},U_{est}^\dag U_{th}U_{est}\ket{0^m}_{\regC}\ket{\psi_{\leq\gamma-2\delta}}_{\regX,\regZ}\ket{0^t00}_{ph,th,in})\leq 2^{-n}
  \end{align*}
   where $\TD$ denotes the trace distance. 
 \takashi{The explanation here makes sense? If not, we may need to write some expressions as in the previous manuscript.}

Therefore we can write
\begin{align}
 G_{i,\gamma}\ket{0^m}_{\regC}\ket{\psi_{\leq\gamma-2\delta}}_{\regX,\regZ}\ket{0^t00}_{ph,th,in}=z_{\leq \gamma-2\delta}\ket{0^m}_{\regC}\ket{\psi_{\leq\gamma-2\delta}}_{\regX,\regZ}\ket{0^t01}_{ph,th,in}+\ket{\psi'_{err,\leq \gamma-2\delta}} \label{eq:leq}
\end{align}

by using $z_{\leq \gamma-2\delta}$ such that $|z_{\leq \gamma-2\delta}|^2\geq 1-2^{-\secpar}$ and a state  $\ket{\psi'_{err,\leq \gamma-2\delta}}$ that is orthogonal to $\ket{0^m}_{\regC}\ket{\psi_{\leq\gamma-2\delta}}_{\regX,\regZ}\ket{0^t01}_{ph,th,in}$ such that $\|\ket{\psi'_{err,\leq \gamma-2\delta}}\|^2 \leq 2^{-n}$.

\begin{comment}
$\ket{\psi_{\geq\gamma}}_{\regX,\regZ}$ is a superposition of $\ket{\hat{\alpha}_j}_{\regX,\regZ}$ such that $p_j\geq \gamma$ and $\ket{\hat{\alpha}_j^{11}}_{\regX,\regZ}$.
By Lemma~??? $\ket{\alpha_j}_{\regC,\regX,\regZ}=\ket{0^m}_{\regC}\ot \ket{\hat{\alpha}_j}_{\regX,\regZ}$ can be written as $\ket{\alpha_j}_{\regC,\regX,\regZ}=\frac{1}{\sqrt{2}}(\ket{\phi_j^+}_{\regC,\regX,\regZ}+\ket{\phi_j^-}_{\regC,\regX,\regZ})$ where $\ket{\phi_j^\pm}_{\regC,\regX,\regZ}$ is an eigenvector of $Q$ with eigenvalue $e^{-i\theta_j}$. 
Moreover,  $\ket{\alpha_j^{(11)}}_{\regC,\regX,\regZ}=\ket{0^m}_{\regC}\ot \ket{\hat{\alpha}_j^{(11)}}_{\regX,\regZ}$ is an eigenvector of $Q$ with eigenvalue $1=e^{0}$. 
Thus, after applying $U_{est}$ to  $\ket{\psi_{\geq\gamma}}_{\regX,\regZ}$, $(1-2^{-\secpar})$-fraction of the state contains $\theta$ in the register $ph$ such that $\theta\leq 2\arccos(\sqrt{\gamma}) +2^{-t}$. which implies $\cos^2(\theta/2)\geq \gamma-\delta$ by our choice of $t$.
%For this fraction of the state, $U_{th}$ does nothing, and then $U_{\est}^\dag$ returns the state to the original state.  
 Thus, we have $\TD(U_{est}\ket{0^m}_{\regC}\ket{\psi_{\geq\gamma}}_{\regX,\regZ}\ket{0^t00}_{ph,th,in},U_{th}U_{est}\ket{0^m}_{\regC}\ket{\psi_{\geq\gamma}}_{\regX,\regZ}\ket{0^t00}_{ph,th,in})\leq 2^{-n}$ and thus$\TD(\ket{0^m}_{\regC}\ket{\psi_{\geq\gamma}}_{\regX,\regZ}\ket{0^t00}_{ph,th,in},U_{est}^\dag U_{th}U_{est}\ket{0^m}_{\regC}\ket{\psi_{\geq\gamma}}_{\regX,\regZ}\ket{0^t00}_{ph,th,in})\leq 2^{-n}$ where $\TD$ denotes the trace distance. 
 \takashi{The explanation here makes sense? If not, we may need to write some expressions as in the previous manuscript.}

Therefore we can write
\end{comment}

By a similar analysis, we can write 
\begin{align}
 G_{i,\gamma}\ket{0^m}_{\regC}\ket{\psi_{\geq\gamma}}_{\regX,\regZ}\ket{0^t00}_{ph,th,in}=z_{\geq \gamma}\ket{0^m}_{\regC}\ket{\psi_{\geq\gamma}}_{\regX,\regZ}\ket{0^t11}_{ph,th,in}+\ket{\psi'_{err,\geq \gamma}} \label{eq:geq}
\end{align}

by using $z_{\geq \gamma}$ such that $|z_{\geq \gamma}|^2\geq 1-2^{-\secpar}$ and a state  $\ket{\psi'_{err,\geq \gamma}}$ that is orthogonal to $\ket{0^m}_{\regC}\ket{\psi_{\geq\gamma}}_{\regX,\regZ}\ket{0^t01}_{ph,th,in}$ such that $\|\ket{\psi'_{err,\geq \gamma}}\|^2 \leq 2^{-n}$.


Combining Eq. \ref{eq:psi}, \ref{eq:leq}, and \ref{eq:geq}, we have
%we have 
%\begin{align*}
 %   G_{i,\gamma} \ket{0^m}_{\regC}\ket{\psi}_{\regX,\regZ}\ket{0^t}_{ph}\ket{0}_{th}\ket{0}_{in} = \ket{0^m}_{\regC}\ket{\psi_{0}}_{\regX,\regZ}\ket{0^t01}_{ph,th,in}+ \ket{0^m}_{\regC}\ket{\psi_{1}}_{\regX,\regZ}\ket{0^t11}_{ph,th,in} + \ket{\psi'_{err}}.
%\end{align*}
%where 
%\begin{align*}
%\ket{\psi_{0}}_{\regX,\regZ}=z_{\geq \gamma}\ket{\psi_{\geq\gamma}}_{\regX,\regZ}+
%\end{align*}
\begin{align}
\begin{split}
& G_{i,\gamma}\ket{0^m}_{\regC}\ket{\psi}_{\regX,\regZ}\ket{0^t00}_{ph,th,in} \\
&=\ket{0^m}_{\regC}(z_{\geq \gamma}\ket{\psi_{\geq\gamma}}_{\regX,\regZ}+\ket{\eta_{mid,0}}_{\regX,\regZ}+\ket{\eta_{other,0}}_{\regX,\regZ})\ket{0^t01}_{ph,th,in} \\
&+\ket{0^m}_{\regC}(z_{\leq \gamma-2\delta}\ket{\psi_{\leq\gamma-2\delta}}_{\regX,\regZ}+\ket{\eta_{mid,1}}_{\regX,\regZ}+\ket{\eta_{other,1}}_{\regX,\regZ})\ket{0^t11}_{ph,th,in} \\
&+ \ket{\psi'_{err}}. 
\end{split} \label{eq:resulting_state}
\end{align}

where $\ket{\eta_{mid,0}}_{\regX,\regZ}$, $\ket{\eta_{other,0}}_{\regX,\regZ}$, $\ket{\eta_{mid,1}}_{\regX,\regZ}$, and $\ket{\eta_{other,1}}_{\regX,\regZ}$ are defined so that
\begin{align}
& I_{\regC,\regX,\regZ} \ot \opro{0^t01}{0^t01}_{ph,th,in}G_{i,\gamma}\ket{0^m}_{\regC}\ket{\psi_{mid}}_{\regX,\regZ}\ket{0^t00}_{ph,th,in}=\ket{0^m}_{\regC}{\ket{\eta_{mid,0}}_{\regX,\regZ}}\ket{0^t01}_{ph,th,in}, \label{eq:etamid} \\
&I_{\regC,\regX,\regZ} \ot \opro{0^t11}{0^t11}_{ph,th,in}G_{i,\gamma}\ket{0^m}_{\regC}\ket{\psi_{mid}}_{\regX,\regZ}\ket{0^t00}_{ph,th,in}=\ket{0^m}_{\regC}{\ket{\eta_{mid,1}}_{\regX,\regZ}}\ket{0^t11}_{ph,th,in}, \label{eq:etamidprime} \\
&I_{\regC,\regX,\regZ} \ot \opro{0^t01}{0^t01}_{ph,th,in}(\ket{\psi'_{err,\leq \gamma-2\delta}}+\ket{\psi'_{err,\geq \gamma}})=\ket{0^m}_{\regC}{\ket{\eta_{other,0}}_{\regX,\regZ}}\ket{0^t01}_{ph,th,in}, \label{eq:etaother} \\
&I_{\regC,\regX,\regZ} \ot \opro{0^t11}{0^t11}_{ph,th,in}(\ket{\psi'_{err,\leq \gamma-2\delta}}+\ket{\psi'_{err,\geq \gamma}})=\ket{0^m}_{\regC}{\ket{\eta_{other,1}}_{\regX,\regZ}}\ket{0^t11}_{ph,th,in}. \label{eq:etaotherprime} 
\end{align}
and $\ket{\psi'_{err}}$  is defined by 
\begin{align}
%\begin{split}
\ket{\psi'_{err}}&:=
\sum_{s\notin \{0^t01,0^t11\}} I_{\regC,\regX,\regZ} \ot \opro{s}{s}_{ph,th,in}(G_{i,\gamma}\ket{0^m}_{\regC}\ket{\psi_{mid}}_{\regX,\regZ}\ket{0^t00}_{ph,th,in}+\ket{\psi'_{err,\leq \gamma-2\delta}}+\ket{\psi'_{err,\geq \gamma}}) \label{eq:psierrprime}
%&-\ket{0^m}_{\regC}(\ket{\eta_{mid,0}}_{\regX,\regZ}+\ket{\eta_{other,0}}_{\regX,\regZ})\ket{0^t01}_{ph,th,in} \\
%&-\ket{0^m}_{\regC}(\ket{\eta_{mid,1}}_{\regX,\regZ}+\ket{\eta_{other,1}}_{\regX,\regZ})\ket{0^t11}_{ph,th,in}. 
%\end{split} 
\end{align}

We remark that $\ket{\eta_{mid,0}}_{\regX,\regZ}$, $\ket{\eta_{other,0}}_{\regX,\regZ}$, $\ket{\eta_{mid,1}}_{\regX,\regZ}$, and $\ket{\eta_{other,1}}_{\regX,\regZ}$ are well-defined since after applying $G_{i,\gamma}$, the value in the register $in$ is $1$ if and only if the value in the register $\regC$ is $0^m$.

We let $z_0:=\frac{z_{\leq \gamma-2\delta}}{|z_{\leq \gamma-2\delta}|}$, $z_1:=\frac{z_{\geq \gamma}}{|z_{\geq \gamma}|}$, and 
\begin{align}
&\ket{\psi_0}_{\regX,\regZ}:=|z_{\leq \gamma-2\delta}|\ket{\psi_{\leq\gamma-2\delta}}_{\regX,\regZ}+ \overline{z}_0(\ket{\eta_{mid,0}}_{\regX,\regZ}+\ket{\eta_{other,0}}_{\regX,\regZ}), \label{eq:psizero} \\
&\ket{\psi_1}_{\regX,\regZ}:=|z_{\geq \gamma}|\ket{\psi_{\geq\gamma}}_{\regX,\regZ}+ \overline{z}_1(\ket{\eta_{mid,1}}_{\regX,\regZ}+\ket{\eta_{other,1}}_{\regX,\regZ}), \label{eq:psione}
\end{align}
where $\overline{z}_0$ and $\overline{z}_1$ denotes complex conjugates of $z_0$ and $z_1$. By Eq~\ref{eq:resulting_state}, \ref{eq:psizero}, and \ref{eq:psione}, we have
\begin{align*}
    G_{i,\gamma} \ket{0^m}_{\regC}\ket{\psi}_{\regX,\regZ}\ket{0^t}_{ph}\ket{0}_{th}\ket{0}_{in} = z_0\ket{0^m}_{\regC}\ket{\psi_{0}}_{\regX,\regZ}\ket{0^t01}_{ph,th,in}+ z_1\ket{0^m}_{\regC}\ket{\psi_{1}}_{\regX,\regZ}\ket{0^t11}_{ph,th,in} + \ket{\psi'_{err}}.
\end{align*}

Now, we are ready to prove the five claims in Lemma~\ref{lem:partition}.

\paragraph{Proof of the first claim.}
By Eq.~\ref{eq:psi}, \ref{eq:psizero}, and \ref{eq:psione}, we have
\begin{align*}
\ket{\psi_{err}}_{\regX,\regZ}&=\ket{\psi}_{\regX,\regZ}-\ket{\psi_0}_{\regX,\regZ}-\ket{\psi_1}_{\regX,\regZ}\\
&=(1-|z_{\leq \gamma-2\delta}|)\ket{\psi_{\leq\gamma-2\delta}}_{\regX,\regZ}+(1-|z_{\geq \gamma}|)\ket{\psi_{\geq\gamma}}_{\regX,\regZ}+\ket{\psi_{mid}}_{\regX,\regZ}\\
&-\overline{z}_0(\ket{\eta_{mid,0}}_{\regX,\regZ}+\ket{\eta_{other,0}}_{\regX,\regZ})
-\overline{z}_1(\ket{\eta_{mid,1}}_{\regX,\regZ}+\ket{\eta_{other,1}}_{\regX,\regZ}),
\end{align*}

Since $|z_{\leq \gamma-2\delta}|$ and $|z_{\geq \gamma}|$ are $1-\negl(\secpar)$, the norms of the first two terms are negligible.
By Eq.~\ref{eq:etaother} and \ref{eq:etaotherprime}, we have $\|\ket{\eta_{other,0}}_{\regX,\regZ}\|^2 + \|\ket{\eta_{other,1}}_{\regX,\regZ}\|^2\leq \|\ket{\psi'_{err,\leq \gamma-2\delta}}+\ket{\psi'_{err,\geq \gamma}}\|^2 \leq \negl(\secpar)$.
Therefore we have
\begin{align*}
\|\ket{\psi_{err}}_{\regX,\regZ}\|^2 &\leq \|\ket{\psi_{mid}}_{\regX,\regZ}-\overline{z}_0\ket{\eta_{mid,0}}_{\regX,\regZ}-\overline{z}_1\ket{\eta_{mid,1}}_{\regX,\regZ}\|^2+\negl(\secpar)\\
&\leq 3(\|\ket{\psi_{mid}}_{\regX,\regZ}\|^2+\|\ket{\eta_{mid,0}}_{\regX,\regZ}\|^2 + \|\ket{\eta_{mid,1}}_{\regX,\regZ}\|^2)+\negl(\secpar)
\end{align*}
where the latter inequality follows from the Cauchy-Schwarz inequality.
As already noted, we have $E_{\gamma}[\|\ket{\psi_{mid}}_{\regX,\regZ}\|^2]\leq \frac{1}{T}$.
By Eq.~\ref{eq:etamid} and \ref{eq:etamidprime}, we have $E_{\gamma}[\|\ket{\eta_{mid,0}}_{\regX,\regZ}\|^2 + \|\ket{\eta_{mid,1}}_{\regX,\regZ}\|^2]\leq E_{\gamma}[\|\ket{\psi_{mid}}_{\regX,\regZ}\|^2]\leq \frac{1}{T}$.
Therefore, we have $E_{\gamma}[\|\ket{\psi_{err}}_{\regX,\regZ}\|^2]\leq \frac{6}{T}+\negl(\secpar)$ and the first claim is proven.
\takashi{The constant $6$ is somewhat ugly. Is there a better analysis?}

\paragraph{Proof of the second claim.}
By Eq~\ref{eq:psierrprime}, we can see that 
\begin{align*}
I_{\regC,\regX,\regZ} \ot (\opro{001}{001}_{ph,th,in}+ \opro{011}{011}_{ph,th,in})\ket{\psi'_{err}}=0.
\end{align*}
This immediately implies the second claim.
%
\paragraph{Proof of the third claim.}
By the second claim,  $\ket{0^m}_{\regC}\ket{\psi_{0}}_{\regX,\regZ}\ket{0^t01}_{ph,th,in}$,  $\ket{0^m}_{\regC}\ket{\psi_{1}}_{\regX,\regZ}\ket{0^t11}_{ph,th,in}$, and $\ket{\psi'_{err}}$ are orthogonal with one another.
Therefore we have 
\begin{align*}
&\|G_{i,\gamma}\ket{0^m}_{\regC}\ket{\psi}_{\regX,\regZ}\ket{0^t00}_{ph,th,in}\|^2\\
=&\|z_0\ket{0^m}_{\regC}\ket{\psi_{0}}_{\regX,\regZ}\ket{0^t01}_{ph,th,in}\|^2+\|z_1\ket{0^m}_{\regC}\ket{\psi_{1}}_{\regX,\regZ}\ket{0^t11}_{ph,th,in}\|^2+\|\ket{\psi'_{err}}\|^2.
\end{align*}
Since we have 
$\|G_{i,\gamma}\ket{0^m}_{\regC}\ket{\psi}_{\regX,\regZ}\ket{0^t00}_{ph,th,in}\|^2=\|\ket{\psi}_{\regX,\regZ}\|^2$ and $\|z_b\ket{0^m}_{\regC}\ket{\psi_{b}}_{\regX,\regZ}\ket{0^tb1}_{ph,th,in}\|^2=\|\ket{\psi_{b}}_{\regX,\regZ}\|^2$, the above implies $\|\ket{\psi_{0}}_{\regX,\regZ}\|^2+\|\ket{\psi_{1}}_{\regX,\regZ}\|^2\leq \|\ket{\psi}_{\regX,\regZ}\|^2$, which implies the third claim.
%
\paragraph{Proof of the forth claim.}
By the definition of $\ket{\psi_{mid}}_{\regX,\regZ}$, the state $\ket{0^m}_{\regC}\ket{\psi_{mid}}_{\regX,\regZ}$ is in the subspace $S_{mid}$, which is the subspace spanned by $\{S_j\}_{j:p_j\in (\gamma-2\delta,\gamma)}$.
We define $\ket{\psi''_{mid,s}}_{\regC,\regX,\regZ}$ so that 
\begin{align*}
G_{i,\gamma}\ket{0^m}_{\regC}\ket{\psi_{mid}}\ket{0^t00}_{ph,th,in}=\sum_{s\in \bit^{t+2}} \ket{\psi''_{mid,s}}_{\regC,\regX,\regZ}\ket{s}_{ph,th,in}.
\end{align*}
Since each subspace $S_j$ is invariant under the projections $\Pi_{in}$ and $\Pi_{i,out}$, each $\ket{\psi''_{mid,s}}_{\regC,\regX,\regZ}$ is also in the subspace $S_{mid}$. \takashi{More explanation needed?}
In particular, $\ket{0^m}_{\regC}\ket{\eta_{mid,0}}_{\regX,\regZ}=\ket{\psi''_{mid,0^t01}}_{\regC,\regX,\regZ}$ is in the subspace $S_{mid}$.
%Then by Eq~\ref{eq:etamid} and that $\Pi_{in}S_{mid}$ is spanned by $\{\hat{\alpha}_j\}_{j: p_j\in (\gamma-2\delta,\gamma)}$, we can express $\ket{\eta_{mid,0}}_{\regX,\regZ}$ as a superposition of states $\{\hat{\alpha}_j\}_{j: p_j\in (\gamma-2\delta,\gamma)}$,
%\begin{align*}
%\ket{\eta_{mid,0}}_{\regX,\regZ}=\sum_{j: p_j\in (\gamma-2\delta,\gamma)} d_j \ket{\hat{\alpha}_j}. 
%\end{align*}
Therefore, by Eq. \ref{eq:psizero}, we can write 
\begin{align}
\ket{0^m}_{\regC}\ket{\psi_0}_{\regX,\regZ}=\sum_{j:p_j<\gamma} d_j \ket{\alpha_j}_{\regC,\regX,\regZ}+\sum_{j} d_j^{(10)} \ket{\alpha_j^{(10)}}_{\regC,\regX,\regZ}+\ket{\psi''_{err,0}}_{\regC,\regX,\regZ}  \label{eq:psizero_decompose} 
\end{align}
where $\ket{\psi''_{err,0}}_{\regC,\regX,\regZ}:=\overline{z}_0 \ket{0^m}_{\regC}\ket{\eta_{other,0}}_{\regX,\regZ}$.
We remark that $\|\ket{\psi''_{err,0}}_{\regC,\regX,\regZ}\|=\negl(\secpar)$ since we have $\|\ket{\psi''_{err,0}}_{\regC,\regX,\regZ}\|=\|\ket{\eta_{other,0}}_{\regX,\regZ}\|\leq \|\ket{\psi_{err,\leq \gamma-2\delta}}+\ket{\psi_{err,\geq \gamma}})\|=\negl(\secpar)$ by Eq. \ref{eq:etamid}.

By the definition of $\Pi_{i,out}$, we have 
\begin{align}
 \Pr_{c_{-i}}\left[M_{\regX_i}\circ U\frac{\ket{c_1...c_{i-1}0c_{i+1}...c_m}_{\regC}\ket{\psi_0}_{\regX,\regZ}}{\|\ket{\psi_0}_{\regX,\regZ}\|}\in \Acc_{k_i,y_i}\right]=\frac{\bra{0^m}_{\regC}\bra{\psi_0}_{\regX,\regZ} \Pi_{i,out} \ket{0^m}_{\regC}\ket{\psi_0}_{\regX,\regZ}}{\|\ket{\psi_0}_{\regX,\regZ}\|^2} \label{eq:accept_probability}
\end{align}
where $c_{-i}$ denotes $c_1...c_{i-1}c_{i+1}...c_{m}$.

By Lemma~\ref{lem:decomposition}, we can see that $\bra{\alpha_j}\Pi_{i,out} \ket{\alpha_{j'}}=0$ for all $j\neq j'$ and $\Pi_{i,out} \ket{\alpha_{j}^{(10)}}=0$ for all $j$.
By substituting Eq. \ref{eq:psizero_decompose} for Eq. \ref{eq:accept_probability}, we have
\begin{align*}
&~~~\Pr_{c_{-i}}\left[M_{\regX_i}\circ U\frac{\ket{c_1...c_{i-1}0c_{i+1}...c_m}_{\regC}\ket{\psi_0}_{\regX,\regZ}}{\|\ket{\psi_0}_{\regX,\regZ}\|}\in \Acc_{k_i,y_i}\right]\\
%&=(\sum_{j:p_j<\gamma} \overline{a}_j \bra{\alpha_j}_{\regC,\regX,\regZ}+\bra{\psi''_{err,0}}_{\regC,\regX,\regZ}) \Pi_{i,out} (\sum_{j:p_j<\gamma} a_j \ket{\alpha_j}_{\regC,\regX,\regZ}+\ket{\psi''_{err,0}}_{\regC,\regX,\regZ})\\
&=\frac{1}{\|\ket{\psi_0}_{\regX,\regZ}\|^2}\left(\sum_{j:p_j<\gamma} |d_j|^2 \bra{\alpha_j} \Pi_{i,out} \ket{\alpha_j}+ \sum_{j:p_j<\gamma} (\overline{d}_j \bra{\alpha_j}\Pi_{i,out} \ket{\psi''_{err,0}} + d_j \bra{\psi''_{err,0}}\Pi_{i,out} \ket{\alpha_j})\right)\\
%&\leq \sum_{j:p_j<\gamma} |a_j|^2 \gamma + \negl(\secpar)\\
&\leq \gamma+\negl(\secpar)
\end{align*}
where the last inequality follows from $\sum_{j:p_j<\gamma}|d_j|^2\leq \|\ket{\psi_0}_{\regX,\regZ}\|^2$ and $\|\ket{\psi''_{err,0}}_{\regC,\regX,\regZ}\|=\negl(\secpar)$. 
This immediately implies the forth claim considering that the number of possible $c_{-i}$ is $2^{m-1}$ and $m=O(\log \secpar)$.
%
\paragraph{Proof of the fifth claim.}
By a similar argument to the one in the proof of the forth claim, we can write  
\begin{align}
\ket{0^m}_{\regC}\ket{\psi_1}_{\regX,\regZ}=\sum_{j:p_j>\gamma-2\delta} d_j \ket{\alpha_j}_{\regC,\regX,\regZ}+\sum_{j} `d_j^{(11)} \ket{\alpha_j^{(11)}}_{\regC,\regX,\regZ}+\ket{\psi''_{err,1}}_{\regC,\regX,\regZ}  \label{eq:psione_decompose} 
\end{align}
where $\ket{\psi''_{err,1}}_{\regC,\regX,\regZ}$ is a state such that $\|\ket{\psi''_{err,1}}_{\regC,\regX,\regZ}\|=\negl(\secpar)$.

The algorithm $\ext_{i}$ is described below:

\begin{description}
\item[$\ext_{i}\left(\frac{\ket{0^m}_{\regC}\ket{\psi_1}_{\regX,\regZ}}{\|\ket{\psi_1}_{\regX,\regZ}\|}\right)$:]
Given $\frac{\ket{0^m}_{\regC}\ket{\psi_1}_{\regX,\regZ}}{\|\ket{\psi_1}_{\regX,\regZ}\|}$ as input, $\ext_{i}$ works as follows:
\begin{itemize}
%\item Append $\ket{0^m}_{\regC}$ to prepare $\ket{0^m}_{\regC}\ket{\psi_1}_{\regX,\regZ}$.
\item Repeat the following procedure $N=\poly(\secpar)$ times where $N$ is specified later:
\begin{enumerate}
\item Perform a measurement $\{\Pi_{i,out},I_{\regC,\regX,\regZ}-\Pi_{i,out}\}$. If the outcome is $0$, i,e, $\Pi_{i,out}$ is applied, then measure the register $\regX_i$ in computational basis to obtain $a_i$, outputs $a_i$, and immediately halts.
\item  Perform a measurement $\{\Pi_{in},I_{\regC,\regX,\regZ}-\Pi_{in}\}$.
\end{enumerate}
\item If it does not halts within $N$ trials in the previous step, output $\bot$.
\end{itemize}
\end{description}

By the definition of $\Pi_{i,out}$, it is clear that $\ext_{i}$ succeeds, (i.e., returns $a_i\in \Acc_{k_i,y_i}$) if it does not output $\bot$.
Since the algorithm $\ext_{i}$ just alternately applies measurements $\{\Pi_{i,out},I_{\regC,\regX,\regZ}-\Pi_{i,out}\}$ and $\{\Pi_{in},I_{\regC,\regX,\regZ}-\Pi_{in}\}$ and each subspaces $S_j$ and $T_j^{(11)}$ are invariant under $\Pi_{in}$ and $\Pi_{i,out}$, we can analyze the success probability of the algorithm separately on each subspace.
Therefore, it suffices to show that $\ext_{i}$ succeeds with probability $1-\negl(\secpar)$ on any input $\ket{\alpha_j}_{\regX,\regZ}$ such that  $p_j> \gamma-2\delta$ or $\ket{\alpha_j^{(11)}}$ for any $j$.
First, it is easy to see that on input $\ket{\alpha_j^{(11)}}$, $\ext_{i}$ returns $a_i\in \Acc_{k_i,y_i}$ at the first trial with probability $1$ since we have $\bra{\alpha_{j}^{(11)}} \Pi_{i,out} \ket{\alpha_{j}^{11}}=1$.
What is left is to prove that $\ext_{i}$ succeeds with probability $1-\negl(\secpar)$ on any input $\ket{\alpha_j}_{\regX,\regZ}$ such that  $p_j> \gamma-2\delta$. 

By Lemma~\ref{lem:decomposition}, it is easy to see that we have
\begin{align*}
&\ket{\alpha_j}_{\regX,\regZ}=\sqrt{p_j}\ket{\beta_j}_{\regX,\regZ}+\sqrt{1-p_j}\ket{\beta_j^\bot}_{\regX,\regZ},\\
&\ket{\beta_j}_{\regX,\regZ}=\sqrt{p_j}\ket{\alpha_j}_{\regX,\regZ}+\sqrt{1-p_j}\ket{\alpha_j^\bot}_{\regX,\regZ}.
\end{align*}

Let  $P_k$ and $P_k^\bot$ be the probability that $\ext_i$ succeeds within $k$ trials starting from the initial state $\ket{\alpha_j}_{\regX,\regZ}$   and $\ket{\alpha_j^\bot}_{\regX,\regZ}$, respectively.
Then by the above equations, it is easy to see that we have $P_0=P_0^\bot=0$ and
\begin{align*}
&P_{k+1}=p_j+(1-p_j)^2 P_{k}+ (1-p_j)p_j P_{k}^\bot, \\
&P_{k+1}^\bot=(1-p_j)+ p_j(1-p_j) P_{k}+ p_j^2 P_{k}^\bot.
\end{align*}

Solving this, we have 
\begin{align*}
P_N=1-(1-2p_j+2p_j^2)^{N-1}(1-p_j).
\end{align*}
\takashi{Just a back-of-envelope calculation and not super confident. Should double check. Or is there any better way of analysis?}

Since we assume $p_j> \gamma-2\delta>\frac{\gamma_0}{3T}=1/\poly(\secpar)$, we have $1-2p_j+2p_j^2=1-1/\poly(\secpar)$.
Therefore if we take $N=\poly(\secpar)$ sufficiently large, then $P_N=1-\negl(\secpar)$.
This means that $\ext_i$ succeeds within $N$ steps with probability $1-\negl(\secpar)$ starting from the initial state $\ket{\alpha_j}_{\regX,\regZ}$.
This completes the proof of the fifth claim and the proof of Lemma~\ref{lem:partition}. 
%\end{align*}
\end{proof}

%\begin{remark}\label{remark:one_half}
%Note that $\ket{\psi_0}$ and $\ket{\psi_1}$ may not be orthogonal. However, $\|\ket{\psi_0}\|^2+ \|\ket{\psi_1}\|^2\leq 1$ since $\|\ket{0^m}_{\regC}\ket{\psi_{0}}_{\regX,\regZ}\ket{001}+ \ket{0^m}_{\regC}\ket{\psi_{1}}_{\regX,\regZ}\ket{011} + \ket{\psi_{err}}\|^2 =1$ and  $\ket{0^m}_{\regC}\ket{\psi_{0}}_{\regX,\regZ}\ket{001}$ and $ \ket{0^m}_{\regC}\ket{\psi_{1}}_{\regX,\regZ}\ket{011}$ are orthogonal. This implies that 
%\begin{align*}
%    E_{c_i}[ \|\ket{\psi_{c_i}}\|^2 ] \leq 1/2
%\end{align*}
%\end{remark}

In Lemma~\ref{lem:partition}, we showed that by fixing any $i\in [m]$, we can partition any prover's state $\ket{\psi}_{\regX,\regZ}$ into $\ket{\psi_0}_{\regX,\regZ}$, $\ket{\psi_1}_{\regX,\regZ}$, and $\ket{\psi_{err}}_{\regX,\regZ}$ with certain properties. %such that $\ket{\psi_0}$ and $\ket{\psi_1}$ will be rejected and accepted in the test round with high probability.
%In the following, we show another procedure that further decompose the prover's state according to any given $c\in \{0,1\}^n$. 
In the following, we sequentially apply Lemma~\ref{lem:partition} for each $i\in[m]$ to further decompose the prover's state.




\begin{lemma}\label{lem:partition_further}
%Fix $c\in \{0,1\}^m$. 
Let $m$, $\gamma_0$, $T$ be as in Lemma~\ref{lem:partition}, and let $\gamma_i\sample [\frac{\gamma_0}{T},\frac{2\gamma_0}{T},\dots,\frac{T\gamma_0}{T}]$ for each $i\in [m]$.
For any $c\in \bit^m$, a state $\ket{\psi}_{\regX,\regZ}$ can be partitioned as follows.% by using Procedure~\ref{fig:process_H}. 
\begin{align*}
    & \ket{\psi}_{\regX,\regZ} = \ket{\psi_{c_1}}_{\regX,\regZ} + \ket{\psi_{\bar{c}_1,c_2}}_{\regX,\regZ} + \cdots +\ket{\psi_{\bar{c}_1,\dots,\bar{c}_{m-1},c_m}}_{\regX,\regZ} + \ket{\psi_{\bar{c}_1,\dots,\bar{c}_m}}_{\regX,\regZ}+ \ket{\psi_{err}}_{\regX,\regZ}
\end{align*}
where the way of partition may depend on the choice of $\hat{\gamma}=\gamma_1...\gamma_m$.
Further, the following properties are satisfied. 
\begin{enumerate}
    \item For any fixed $\hat{\gamma}$ and any $c$, $i\in [m]$ such that $c_i=0$, we have 
    \begin{align*}
    \Pr\left[M_{\regX_i}\circ U \frac{\ket{0^m}_{\regC}\ket{\psi_{\bar{c}_1,\dots,\bar{c}_{i-1},0}}_{\regX,\regZ}}{|\ket{\psi_{\bar{c}_1,\dots,\bar{c}_{i-1},0}}_{\regX,\regZ}|}\in \Acc_{k_i,y_i}\right]\leq 2^{m-1}\gamma_0+ \negl(\secpar).
    \end{align*}
    
    \item For any fixed $\hat{\gamma}$ and any $c$, $i\in[m]$ such that $c_i=1$, there exists an efficient algorithm $\ext_i$ such that 
    \begin{align*}  
  \Pr\left[\ext_i\left(\frac{\ket{0^m}_{\regC}\ket{\psi_{\bar{c}_1,\dots,\bar{c}_{i-1},1}}_{\regX,\regZ}}{\|\ket{\psi_{\bar{c}_1,\dots,\bar{c}_{i-1},1}}_{\regX,\regZ}\|}\right)\in \Acc_{k_i,y_i}\right]=1-\negl(\secpar).
  \end{align*}   
  \item For any fixed $\hat{\gamma}$, we have $E_c[\|\ket{\psi_{\bar{c}_1,\dots,\bar{c}_m}}_{\regX,\regZ}\|^2] \leq 2^{-m}$.
\item For any fixed $c$, we have $E_{\hat{\gamma}}[\|\ket{\psi_{err}}_{\regX,\regZ}\|^2]\leq \frac{6m^2}{T}+\negl(\secpar)$.
    \item For any fixed $\hat{\gamma}$ and $c$ there exists an efficient quantum algorithm $H_{\hat{\gamma},c}$ that is given $\ket{\psi}_{\regX,\regZ}$ as input and produces  $\frac{\ket{\psi_{\bar{c}_1,\dots,\bar{c}_{i-1},c_i}}_{\regX,\regZ}}{\|\ket{\psi_{\bar{c}_1,\dots,\bar{c}_{i-1},c_i}}_{\regX,\regZ}\|}$ with probability $\|\ket{\psi_{\bar{c}_1,\dots,\bar{c}_{i-1},c_i}}_{\regX,\regZ}\|^2$ ignoring a global phase factor.
%    \item Measuring the register $(ph_1,th_1,in_1),\dots,(ph_i,th_i,in_i)$ gives error (i.e., $\exists i$, such that $(ph_i,th_i,in_i)\neq (0^t,0,1)$ or $(0^t,1,1)$) with probability at most $\frac{m}{T}$. 
 %   \item For any fixed $\gamma$, $E_c[\|\ket{\psi_{\bar{c}_1,\dots,\bar{c}_m}}\|^2] \leq 2^{-m}$.
\end{enumerate}
\end{lemma}
\begin{proof}
We inductively define $\ket{\psi_{c_1}}_{\regX,\regZ}$,...,$\ket{\psi_{\bar{c}_1,...,\bar{c}_m}}_{\regX,\regZ}$ as follows.

First, we apply Lemma \ref{lem:partition} for the state $\ket{\psi}_{\regX,\regZ}$ with $\gamma=\gamma_1$ to give a decomposition
\begin{align*}
\ket{\psi}_{\regX,\regZ}=\ket{\psi_0}_{\regX,\regZ}+\ket{\psi_1}_{\regX,\regZ} + \ket{\psi_{err,1}}_{\regX,\regZ} 
\end{align*}
where $\ket{\psi_{err,1}}_{\regX,\regZ}$ corresponds to $\ket{\psi_{err}}_{\regX,\regZ}$ in Lemma \ref{lem:partition}.

For each $i=2,...,m$, we apply  Lemma \ref{lem:partition} for the state $\ket{\psi_{\bar{c}_1,...,\bar{c}_{i-1}}}_{\regX,\regZ}$ with $\gamma=\gamma_i$ to give a decomposition 
\begin{align*}
\ket{\psi_{\bar{c}_1,...,\bar{c}_{i-1}}}_{\regX,\regZ}=\ket{\psi_{\bar{c}_1,...,\bar{c}_{i-1},0}}_{\regX,\regZ}+\ket{\psi_{\bar{c}_1,...,\bar{c}_{i-1},1}}_{\regX,\regZ} + \ket{\psi_{err,i}}_{\regX,\regZ} 
\end{align*}  
where 
$\ket{\psi_{\bar{c}_1,...,\bar{c}_{i-1},0}}_{\regX,\regZ}$, $\ket{\psi_{\bar{c}_1,...,\bar{c}_{i-1},1}}_{\regX,\regZ}$, and $\ket{\psi_{err,i}}_{\regX,\regZ}$ corresponds to $\ket{\psi_{0}}_{\regX,\regZ}$, $\ket{\psi_{1}}_{\regX,\regZ}$, and $\ket{\psi_{err}}_{\regX,\regZ}$ in Lemma \ref{lem:partition}, respectively.
  
 Then it is easy to see that we have
 \begin{align*}
    & \ket{\psi}_{\regX,\regZ} = \ket{\psi_{c_1}}_{\regX,\regZ} + \ket{\psi_{\bar{c}_1,c_2}}_{\regX,\regZ} + \cdots +\ket{\psi_{\bar{c}_1,\dots,\bar{c}_{m-1},c_m}}_{\regX,\regZ} + \ket{\psi_{\bar{c}_1,\dots,\bar{c}_m}}_{\regX,\regZ}+ \ket{\psi_{err}}_{\regX,\regZ}
\end{align*}
where we define $\ket{\psi_{err}}_{\regX,\regZ}\defeq \sum_{i=1}^{m}\ket{\psi_{err,i}}_{\regX,\regZ}$. 
  
The first and second claims immediately follow from the forth and fifth claims of Lemma~\ref{lem:partition} and $\gamma_i\leq \gamma_0$ for each $i\in[m]$.  

By the third claim of  Lemma~\ref{lem:partition}, we have $E_{c_1...c_{i}}[\|\ket{\psi_{\bar{c}_1,...,\bar{c}_{i}}}_{\regX,\regZ}\|]\leq \frac{1}{2}E_{c_1...c_{i-1}}[\|\ket{\psi_{\bar{c}_1,...,\bar{c}_{i-1}}}_{\regX,\regZ}\|]$.
Ths implies the third claim.

 
By the first claim of  Lemma~\ref{lem:partition}, we have $E_{\gamma_i}[\|\ket{\psi_{err,i}}_{\regX,\regZ}\|^2]\leq \frac{6}{T}+\negl(\secpar)$.
The forth claim follows from this and the Cauchy-Schwarz inequality. 

Finally, for proving the fifth claim, we define the procedure $H_{\hat{\gamma},c}$ as described in Procedure~\ref{fig:process_H}
We can easily see that $H_{\hat{\gamma},c}$ satisfies the desired property by the second claim of  Lemma~\ref{lem:partition}.

\floatname{algorithm}{Procedure}
\begin{algorithm}[h]
    \begin{mdframed}[style=figstyle,innerleftmargin=10pt,innerrightmargin=10pt]
   On input $\ket{\psi}_{\regX,\regZ}$, it works as follows:
   
   For each $i=1,...,m$, it applies 
    \begin{enumerate}
    \item Prepare registers $\regC$, $(ph_1,th_1,in_1)$,..., $(ph_m,th_m,in_m)$ all of which are initialized to be $\ket{0}$.
    \item For each $i=1,...,m$, do the following: 
      \begin{enumerate}
      \item Apply $G_{i,\gamma_i}$ on the quantum state in the registers $(\regC,\regX,\regZ,ph_i,th_i,in_i)$.
      \item Measure the registers $(ph_i,th_i,in_i)$ in the computational basis.
      \item If the outcome is $0^tc_{i}1$, then it halts and returns the state in the register $(\regX,\regZ)$. If the outcome is $0^t\bar{c}_{i}1$, continue to run. Otherwise, immediately halt and abort.
      \end{enumerate}   
    \end{enumerate}

    \caption{$H_{\hat{\gamma},c}$}
    \label{fig:process_H}
    \end{mdframed}
\end{algorithm}
\end{proof}

Given Lemma~\ref{lem:partition_further}, we can start proving Theorem~\ref{thm:rep_soundness}. 

\begin{proof}[Proof of Theorem~\ref{thm:rep_soundness}]

%According to Lemma~\ref{lem:Mah_soundness}, we know that
%\begin{align*}
 %   \Pr_{k,y}[U_{0}\ket{\psi(k,y)}\mbox{ wins test round}]\geq 1-\negl(n) \Rightarrow \Pr_{k,y}[U_{0}\ket{\psi(k,y)}\mbox{ wins Hadamard round}]\leq \negl(n),   
%\end{align*}
%Here,  
First, we recall how a cheating prover characterized by $(U_0,U)$ works.
When the first message $k$ is given, it first applies 
\begin{align*}
    &U_0\ket{0}_{\regX,\regZ}\ket{0}_{\regY}\ket{k}_{\regK} \xrightarrow{\mbox{measure }\regY} \ket{\psi(k,y)}_{\regX,\regZ}\ket{k}_{\regK}.
\end{align*}
to generate the second message $y$ and $\ket{\psi(k,y)}_{\regX,\regZ}$.
Then after receiving the third message $c$, it applies $U$ on $\ket{c}_{\regC}\ket{\psi(k,y)}_{\regX,\regZ}$ and measures the register $\regX$ in the computational basis to obtain the forth message $a$.
In the following, we just write $\ket{\psi}_{\regX,\regZ}$ to mean $\ket{\psi(k,y)}_{\regX,\regZ}$ for notational simplicity.
%let $V_{i,b}$ be a unitary over $\hil_{\regC,\regX,\regZ}$ that runs the verification procedure for $c_i=b$ on the $i$-th coordinate and write the verification result in a designated register (say, $i$-th qubit of $\regZ$), and $M_i$ be the measurement on the designated register that contains the verification result on the $i$-th coordinate.
%Note that $V_{i,1}$ cannot be applied publicly without knowing the trapdoor,  but this does not affect our analysis below.
Let $M_{i,k_i,\td_i,y_i,c_i}$ be the measurement that outputs the verification result of the value in the register $\regX_i$ w.r.t.  $k_i,\td_i,y_i,c_i$, and let $M_{k,\td,y,c}$ be the measurement that returns $\top$ if and only if $M_{i,k_i,\td_i,y_i,c_i}$ returns $\top$ for all $i\in[m]$ where $k=(k_1,...,k_m)$, $\td=(\td_1,...,\td_m)$, $y=(y_1,...,y_m)$ and $c=(c_1,...,c_m)$.
%For any state $\ket{\phi}_{\regC,\regX,\regZ}$, we denote $M\circ \ket{\phi}_{\regC,\regX,\regZ}=\top$ to mean $M_i\circ \ket{\phi}_{\regC,\regX,\regZ}=\top$ for all $i\in[m]$ for notational simplicity. 
With this notation, a cheating prover's success probability can be written as 
\begin{align*}
    \Pr_{k,\td,y,c}[M_{k,\td,y,c}U\ket{c}_{\regC}\ket{\psi}_{\regX,\regZ} = \top].
\end{align*}

Let $\gamma_0$, $\hat{\gamma}$, and $T$ be as in Lemma~\ref{lem:partition_further}.
According to Lemma~\ref{lem:partition_further}, for any fixed $\hat{\gamma}$ and $c\in \bit^{m}$, we can decompose $\ket{\psi}_{\regX,\regZ}$ as 
\begin{align*}
    \ket{\psi}_{\regX,\regZ} =  \ket{\psi_{c_1}}_{\regX,\regZ}+ \ket{\psi_{\bar{c}_1,c_2}}_{\regX,\regZ} + \cdots + \ket{\psi_{\bar{c}_1,\dots, \bar{c}_{m-1},c_{m}}}_{\regX,\regZ} + \ket{\psi_{\bar{c}_1,\dots, \bar{c}_{m-1},\bar{c}_{m}}}_{\regX,\regZ}+ \ket{\psi_{err}}_{\regX,\regZ}.
\end{align*}

To prove the theorem, we show the following two inequalities.
First,  for any  fixed $\hat{\gamma}$, $i\in[m]$, $c\in \bit^{m}$ such that $c_i=0$, $k_i$, $\td_i$, and $y_i$, we have
\begin{align}
 \Pr\left[M_{i,k_i,\td_i,y_i,0} \circ \frac{U\ket{c}_{\regC}\ket{\psi_{\bar{c}_1,\ldots,\bar{c}_{i-1},0}}_{\regX,\regZ}}{\|\ket{\psi_{\bar{c}_1,\ldots,\bar{c}_{i-1},0}}_{\regX,\regZ}\|}=\top\right]\leq 2^{m-1}\gamma_0+\negl(\secpar). \label{eq:Test}
\end{align}
This easily follows from the first claim of Lemma~\ref{lem:partition_further}

Second, for any  fixed $\hat{\gamma}$, $i\in[m]$, and $c\in \bit^{m}$ such that $c_i=1$,
\begin{align}
    \underset{k,\td,y}{E}\left[\|\ket{\psi_{\bar{c}_1,\dots,\bar{c}_{i-1},1}}_{\regX,\regZ}\|^2\Pr\left[M_{i,k_i,\td_i,y_i,1}\circ U\frac{\ket{c}_{\regC}\ket{\psi_{\bar{c}_1,\dots,\bar{c}_{i-1},1}}_{\regX,\regZ}}{\|\ket{\psi_{\bar{c}_1,\dots,\bar{c}_{i-1},1}}_{\regX,\regZ}\|} = \top\right]\right] = \negl(n) \label{eq:Hada}
\end{align}
assuming the quantum hardness of LWE problem.

For proving Eq.~\ref{eq:Hada}, we consider a cheating prover against the original Mahadev's protocol on the $i$-th corrdinate described below:

\begin{enumerate}
    \item Given $k_i$, it picks $k_{-i}=k_1...k_{i-1},k_{i+1},...,k_{m}$ as in the protocol and computes $U_0\ket{0}_{\regX,\regZ}\ket{0}_{\regY}\ket{k}_{\regK}$ and measure the register $\regY$ to obtain $y=(y_1,...,y_m)$ along with the corresponding state $\ket{\psi}_{\regX,\regZ}=\ket{\psi(k,y)}_{\regX,\regZ}$.
    \item Apply $H_{\hat{\gamma},c}$ to generate the state $\frac{\ket{\psi_{\bar{c}_1,\dots,\bar{c}_{i-1},1}}_{\regX,\regZ}}{\|\ket{\psi_{\bar{c}_1,\dots,\bar{c}_{i-1},1}}_{\regX,\regZ}\|}$, which succeeds with probability $\|\ket{\psi_{\bar{c}_1,\dots,\bar{c}_{i-1},1}}_{\regX,\regZ}\|^2$ (ignoring a global phase factor).
    We denote by $\Succ$ the event that it succeeds in generating the state.
    If it fails to generate the state, then it overrides $y_i$ by picking it in a way such that it can pass the test round with probability $1$, which can be done according to Fact~\ref{fact:perfectly_pass_test}.
    Then it sends $y_i$ to the verifier.
    \item Given a challenge $c'_i$, it works as follows:
    \begin{itemize}
     \item When $c'_i=0$ (i.e., Test round), if $\Succ$ occurred, then it runs $\ext_i$ in the second claim of Lemma~\ref{lem:partition_further} on input $\frac{\ket{0^m}_{\regC}\ket{\psi_{\bar{c}_1,\dots,\bar{c}_{i-1},1}}_{\regX,\regZ}}{\|\ket{\psi_{\bar{c}_1,\dots,\bar{c}_{i-1},1}}_{\regX,\regZ}\|}$ to generate a forth message accepted with probability $1-\negl(\secpar)$. 
     If $\Succ$ did not occur, then it returns a forth message accepted with probability $1$, which is possible by Fact~\ref{fact:perfectly_pass_test}.
    \item When $c'_i=1$ (i.e., Hadamard round), if $\Succ$ occurred, then it computes  $U\frac{\ket{c}_{\regC}\ket{\psi_{\bar{c}_1,\dots,\bar{c}_{i-1},1}}_{\regX,\regZ}}{\|\ket{\psi_{\bar{c}_1,\dots,\bar{c}_{i-1},1}}_{\regX,\regZ}\|}$ and measure the register $\regX_i$ to obtain the forth message $a_i$.
    If $\Succ$ did not occur, it just aborts.
    \end{itemize}
\end{enumerate}
Then we can see that this cheating adversary passes the test round with overwhelming probability and passes the Hadamard round with the probability equal to the LHS of Eq.~\ref{eq:Hada}.
Therefore, Eq.~\ref{eq:Hada} follows from Lemma~\ref{lem:Mah_soundness} assuming the quantum hardness of LWE problem.

Now, we are ready to prove the theorem. 
As remarked at the beginning of Sec. \ref{sec:proof_of_soundness}, it suffices to show that for any $\mu=1/\poly(n)$, there exists $m=O(\log(n))$ such that the success probability of the cheating prover is at most $\mu$.
Here we set $m = \log \frac{1}{\mu^2}$, $\gamma_0 = 2^{-2m}$, and $T=2^{m}$. 
Note that this parameter setting satisfies the requirement for Lemma~\ref{lem:partition_further} since
$m=\log \frac{1}{\mu^2}=\log (\poly(\secpar))=O(\log \secpar)$ and
$\frac{\gamma_0}{T}=2^{-3m}=\mu^{6}=1/\poly(\secpar)$.
Then we have
\begin{align*}
    &\Pr_{k,\td,y,c}\left[M_{k,\td,y,c}\circ U\ket{c}_{\regC}\ket{\psi}_{\regX,\regZ}= \top\right] \\
     &=\Pr_{k,\td,y,c,\hat{\gamma}}\left[M_{k,\td,y,c}\circ U\ket{c}_{\regC}\left(\sum_{i=1}^{m}\ket{\psi_{\bar{c}_1,\dots,\bar{c}_{i-1},c_i}}_{\regX,\regZ} + \ket{\psi_{\bar{c}_1,\dots,\bar{c}_m}}_{\regX,\regZ}+ \ket{\psi_{err}}_{\regX,\regZ}\right) = \top\right] \\
    &\leq (m+2) \underset{k,\td,y,c,\hat{\gamma}}{E}\Biggl[\sum_{i=1}^{m} \|\ket{\psi_{\bar{c}_1,\dots,\bar{c}_{i-1},c_i}}_{\regX,\regZ}\|^2\Pr\left[M_{k,\td,y,c}\circ U\frac{\ket{c}_{\regC}\ket{\psi_{\bar{c}_1,\dots,\bar{c}_{i-1},c_i}}_{\regX,\regZ}}{\|\ket{\psi_{\bar{c}_1,\dots,\bar{c}_{i-1},c_i}}_{\regX,\regZ}\|}=\top\right]\\
   &+\|\ket{\psi_{\bar{c}_1,\dots,\bar{c}_m}}_{\regX,\regZ}\|^2\Pr\left[M_{k,\td,y,c}\circ U\frac{\ket{c}_{\regC}\ket{\psi_{\bar{c}_1,\dots,\bar{c}_m}}_{\regX,\regZ}}{\|\ket{\psi_{\bar{c}_1,\dots,\bar{c}_m}}_{\regX,\regZ}} =\top\right]\\
    &+ \|\ket{\psi_{err}}_{\regX,\regZ}\|^2\Pr\left[M_{k,\td,y,c}\circ U\frac{\ket{c}_{\regC} \ket{\psi_{err}}_{\regX,\regZ}}{\|\ket{\psi_{err}}_{\regX,\regZ}\|}=\top\right]\Biggr]\\
        &\leq (m+2) \underset{k,\td,y,c,\hat{\gamma}}{E}\Biggl[\sum_{i=1}^{m} \|\ket{\psi_{\bar{c}_1,\dots,\bar{c}_{i-1},c_i}}_{\regX,\regZ}\|^2\Pr\left[M_{i,k_i,\td_i,y_i,c_i}\circ U\frac{\ket{c}_{\regC}\ket{\psi_{\bar{c}_1,\dots,\bar{c}_{i-1},c_i}}_{\regX,\regZ}}{\|\ket{\psi_{\bar{c}_1,\dots,\bar{c}_{i-1},c_i}}_{\regX,\regZ}\|}=\top\right]\\
   &+\|\ket{\psi_{\bar{c}_1,\dots,\bar{c}_m}}_{\regX,\regZ}\|^2+ \|\ket{\psi_{err}}_{\regX,\regZ}\|^2 \Biggr]\\
    &\leq (m+2)(m(2^{m-1}\gamma_0 +\negl(n))+ 2^{-m} + \frac{m^2}{T}+\negl(\secpar)) \\
    & \leq \mathsf{poly}(\log \mu^{-1}) \mu^{2}+\negl(\secpar). 
\end{align*}
The first equation follows from Lemma \ref{lem:partition_further}. The first inequality follows from the Cauchy-Schwarz inequality.
\takashi{More explanation may be useful.}
 The second inequality holds since considering the verification on a particular coordinate just increases the acceptance probability and probabilities are at most $1$.
The third inequality follows from Eq.~\ref{eq:Test} and \ref{eq:Hada}, which give an upper bound of the first term and Lemma~\ref{lem:partition_further}, which gives upper bounds of the second and third terms.
The last inequality follows from our choices of $\gamma_0$, $T$, and $m$.
For sufficiently large $\secpar$,  this can be upper bounded by $\mu$.
\end{proof}

%---------------------Older  Proof--------------------------------
\begin{comment}
According to Lemma~\ref{lem:partition_further}, we can decompose any state $\ket{\psi}$ above as 
\begin{align*}
    &\ket{\psi} = \ket{\psi_{c_1}} + \ket{\psi_{\bar{c}_1,c_2}} + \cdots +\ket{\psi_{\bar{c}_1,\dots,c_m}} + \ket{\psi_{\bar{c}_1,\dots,\bar{c}_m}}+ \ket{\psi_{err}}.
\end{align*}
To prove the theorem, we need to show that 
\begin{align*}
    \Pr[M\circ(V_{i,H}U_c\ket{\psi_{\bar{c}_1,\dots,\bar{c}_{i-1},1}}) = accept] = \negl(n), 
\end{align*}
where $V_{i,H}$ is the verification procedure in the $i$th round and $M$ is the measurement to check whether the prover wins the Hadamard round. 

When $i=0$, $\ket{\psi} = \ket{\psi_0}+ \ket{\psi_1}+\ket{\psi_{err_1}}$, and $\ket{\psi_1}$ wins the test round with high probability. We prove that the prover with internal state $\ket{\psi_1}$ wins the Hadamard round with only negligible probability by contradiction. Suppose $\ket{\psi_1}$ wins the Hadamard round with noticeable probability. Then, we can construct the following attack for $\ket{\psi}$ such that Lemma~\ref{lem:Mah_soundness} fails. Without loss of generality, we can assume $\|\ket{\psi_1}\|^2\geq 1/poly(n)$. The prover first applies the corresponding $G_{1,\gamma,\delta}$ and measure the register $ph,th,in$ to obtain $\ket{\psi_1}$ with noticeable probability, which implies that the prover can win the test round with noticeable probability by Lemma~\ref{lem:partition}. Then, consider the Hadamard round, if the prover does not obtain $\ket{\psi_1}$ from $G_{1,\gamma,\delta}$, it just randomly outputs $u,d$ to the verifier; otherwise, if the prover obtains $\ket{\psi_1}$, based on our hypothesis, it can win with noticeable probability. Overall, the prover can win the Hadamard round with noticeable probability, which violates Lemma~\ref{lem:Mah_soundness}. Therefore, the prover with internal state $\ket{\psi_1}$ wins the Hadamard round with only negligible probability.   

We can decompose $\ket{\psi}$ by using Procedure~\ref{fig:process_H}
\begin{align*}
    \ket{\psi} =  \ket{\psi_{c_1}}+ \ket{\psi_{\bar{c}_1,c_2}} + \cdots + \ket{\psi_{\bar{c}_1,\dots, \bar{c}_{m-1},c_{m}}} + \ket{\psi_{\bar{c}_1,\dots, \bar{c}_{m-1},\bar{c}_{m}}}+ \ket{\psi_{err}}.
\end{align*}
Similar to the case with only one trial, we can show that the prover with internal state $\ket{\psi_{\bar{c}_1,\dots, \bar{c}_{m-1},1}}$ wins the Hadamard round with negligible probability by contradiction.
Suppose the prover with $\ket{\psi_{\bar{c}_1,\dots, \bar{c}_{m-1},0}}$ can win the Hadamard round with noticeable probability. Without loss of generality, $\|\ket{\psi_{\bar{c}_1,\dots, \bar{c}_{m-1},1}}\|^2>1/\poly(n)$. This implies that the prover has noticeable probability to obtain $\ket{\psi_{\bar{c}_1,\dots, \bar{c}_{m-1},1}}$ and thus it can win the test round with high probability. Then, in the Hadamard round, the prover can again use $H_c$ to obtain $\ket{\psi_{\bar{c}_1,\dots, \bar{c}_{m-1},1}}$ and win the Hadamard round with noticeable probability, which fails Lemma~\ref{lem:Mah_soundness}. Hence, the prover with $\ket{\psi_{\bar{c}_1,\dots, \bar{c}_{m-1},1}}$ can only win the Hadamard wound with negligible probability. 


Now, we are ready to prove the theorem by using contradiction. Let $m' = \log^2n$. We suppose that there exists a prover can win with probability $\mu=1/\poly(n)$. We choose $m = \log \frac{1}{\mu^2}$, $\gamma_0 = 2^{-2m}$, and $T=2^{-m}$. Then, we choose the first $m$ trials to do parallel repetition and show that by our choices of parameters, the prover can only succeed with probability less than $\mu$. Note that the verifier has $c_1,\dots,c_m$ be chosen uniformly independently. Hence, $c_{m+1},\dots,c_{m'}$ can be viewed as some redundant information uncorrelated to $c_1,\dots,c_m$ given to the prover, which does not change our analysis in Lemma~\ref{lem:Mah_soundness}, Lemma~\ref{lem:partition}, and Lemma~\ref{lem:partition_further}. Let the verifier's verification be $V_{1,c_1},\dots,V_{m,c_m}$, where $V_{i,0}$ is doing the test round in the $i$th trial and $V_{i,1}$ is doing the Hadamard round. Then, 
\begin{align}
    &\Pr[M\circ\left(U_c\ket{\psi}\right) = accept] \\
    &\leq (m+2)(\Pr[M\circ\left(U_c\ket{\psi_{c_1}}\right)=accept] \\
    &+ \Pr[M\circ\left(U_c\ket{\psi_{\bar{c}_1,c_2}}\right)=accept]\\
    &+\cdots+\Pr[M\circ\left(U_c\ket{\psi_{\bar{c}_1,\dots,c_m}}\right)=accept]\label{eq:last_1}\\
    &+\Pr[M\circ\left(U_c\ket{\psi_{\bar{c}_1,\dots,\bar{c}_m}}\right)=accept] \label{eq:last_2}\\
    &+ \Pr[M\circ\left(U_c\ket{\psi_{err}}\right)=accept])\\
    &\leq (m2^m\gamma_0 + 2^{-m} + \frac{m}{T})(m+2) \leq \mu. 
\end{align}
The first inequality follows from the Cauchy-Schwarz inequality. The second inequality follows from the fact that $V_1,\dots,V_m$ commute, and thus we can choose $V_{i,c_i}$ to be the first operator operating on $\ket{\psi_{\bar{c}_1,\dots,\bar{c}_{i-1},c_{i}}}$; then, by our analysis, the probability that the prover can win is at most $2^m\gamma_0$. The states considered in Eq.~\ref{eq:last_1} and Eq.~\ref{eq:last_2} have norm at most $1/2^m$ and $m/T$ according to Lemma~\ref{lem:partition_further}. The last inequality follows from our choices of $\gamma_0$, $T$, and $m$.

For all noticeable $\mu$, we can find corresponding $m$, $\gamma_0$, and $T$ such that the prover can only win with probability less than $\mu$. Therefore, the probability the prover can win the test can only be negligible when $m=\poly(n)$.  
\end{comment}