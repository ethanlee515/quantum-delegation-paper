\section{Construction of the $\QPIP_1$ Protocol for $\SampBQP$}
\label{sec:sampbqp_short}

As we mentioned in this introduction, we will employ the circuit \emph{history} state in the original construction of the Local Hamiltonian problem~\cite{kitaev2002classical} to encode the circuit information for $\SampBQP$.
However, there are distinct requirements between certifying the computation for $\BQP$ and $\SampBQP$ based on the history state.
For any quantum circuit $C$ on input $x$, the original construction for certifying $\BQP$\footnote{The original construction is for the purpose of certifying problems in QMA. We consider its simple restriction to problems inside BQP.} consists of local Hamiltonian $\Hin, \Hclock, \Hprop$, $\Hout$ where $\Hin$ is used to certify the initial input $x$, $\Hclock$ to certify the validness of the clock register,  $\Hprop$ to certify the gate-by-gate evolution according to the circuit description, and $\Hout$ to certify the final output.
In particular, the corresponding history state is in the ground space of $\Hin$, $\Hclock$, and $\Hprop$. Note that $\BQP$ is a decision problem and its outcome (0/1) can be easily encoded into the energy $\Hout$ on the single output qubit.
As a result, the outcome of $\BQP$ can simply be encoded by the \emph{ground energy} of $\Hin + \Hclock+\Hprop + \Hout$.

To deal with $\SampBQP$, we will still employ $\Hin, \Hclock$, and $\Hprop$ to certify the circuit's input, the clock register, and gate-by-gate evolution. However, in $\SampBQP$, we care about the entire final state of the circuit, rather than the energy on the output qubit.
%The history state remains in the ground space of $\Hin + \Hprop$.  
Our approach to certify the entire final state (which is encoded inside the history state) is to make sure that the history state is the unique ground state of $\Hin + \Hclock+ \Hprop$ and all other orthogonal states will have much higher energies.
Namely, we need to construct some $\Hin'+ \Hclock'+ \Hprop'$ with the history state as the unique ground state and with a large \emph{spectral} gap between the ground energy and excited energies.
It is hence guaranteed that any state with close-to-ground energy must also be close to the history state.
We remark that this is a different requirement from most local Hamiltonian constructions that focus on the ground energy.
We achieve so by using the \emph{perturbation} technique developed in~\cite{kempe_kitaev_regev_2006} for reducing the locality of Hamiltonian.
Another example of local Hamiltonian construction with a focus on the spectral gap can be found in~\cite{adiabatic}, where the purpose is to simulate quantum circuits by adiabatic quantum computation.

We need two more twists for our purpose.
First, as we will eventually measure the final state in order to obtain classical samples, we need that the final state occupies a large fraction of the history state. We can simply add dummy identity gates.
Second, as we are only able to perform $X$ or $Z$ measurement by techniques from~\cite{FOCS:Mahadev18a},
we need to construct X-Z only local Hamiltonians.
Indeed, this has been shown possible in, e.g.,~\cite{PhysRevA.78.012352}, which serves as the starting point of our construction.


We present the formal construction of our $\QPIP_1$ protocol $\PiSamp$ for $\SampBQP$ in \Cref{sec:sampbqp}, \Cref{ProtoQPIP1}. The soundness and completeness of \Cref{ProtoQPIP1} is stated in the following theorem, whose proof is also deferred to \Cref{sec:sampbqp}.% The proof of 


 %For more details of the construction of our $\QPIP_1$ Protocol for $\SampBQP$, \Cref{ProtoQPIP1}, see \Cref{sec:sampbqp}. The soundness and completeness of \Cref{ProtoQPIP1} is stated in the following theorem.

\begin{thm}
    \label{QPIP1thm}
	$\PiSamp$ is a $\QPIP_1$ protocol for the $\SampBQP$ problem  $(D_x)_{x\in\set{0,1}^*}$ with negligible completeness error and is statistically sound\footnote{The soundness and completeness of a $\SampBQP$ protocol is defined in \Cref{dfn:stats-secure-proto-sampbqp}.} where the verifier only needs to do non-adaptive $X/Z$ measurements.
	%\XW{do we want precise error bound?}	
\end{thm}