\section{Delegation Protocol for $\QPIP_1$ Client}

In this section, we construct a one-message $\QPIP_1$ delegation protocol for $\SampBQP$.
At a high level, it is a cut-and-choose protocol.
The server constructs multiple copies of the ground state as certificates,
then the client randomly chooses a copy to output and checks the rest.
Unfortunately our approach incurs inverse polynomial soundness errors.

There are two main challenges to this cut-and-choose approach.
First, the client needs to reliably extract the circuit output from its corresponding ground state.
We accomplish this by padding the circuit with identity gates at the end.
By doing so, the clock register collapses to after the last non-identity gate with high probability.

The other challenge is that a cheating prover's certificates can be arbitrarily entangled,
so common techniques such as Chernoff bounds can't be applied as-is.
This wasn't a challenge for $\Piblind$ for $\BQP$ earlier because...
\Ethan{TODO; need brief and intuitive explanation}
In our case with $\SampBQP$, we overcome this challenge instead by using de Finetti's theorem
to approximate our measurement results with that of some unentangled copies,
within inverse polynomial errors.

\def\GS{\mathsf{Ham}}
\nc{\PiGS}{\ensuremath{\Pi_\GS}}
\nc{\VGS}{\ensuremath{V_\GS}}
\nc{\PGS}{\ensuremath{P_\GS}}
\nc{\PGSstar}{\ensuremath{P_\GS^*}}
\nc{\cVGS}[1]{\ensuremath{\cV_{\GS,#1}}}
\nc{\cPGS}[1]{\ensuremath{\cP_{\GS,#1}}}

\def\Samp{\mathsf{Samp}}
\nc{\PiSamp}{\ensuremath{\Pi_\Samp}}
\nc{\VSamp}{\ensuremath{V_\Samp}}
\nc{\PSamp}{\ensuremath{P_\Samp}}
\nc{\PSampstar}{\ensuremath{P_\Samp^*}}
\nc{\cVSamp}[1]{\ensuremath{\cV_{\Samp,#1}}}
\nc{\cPSamp}[1]{\ensuremath{\cP_{\Samp,#1}}}

\Ethan{Again, look at other papers...}

\Ethan{TODO Recall variables from the last section}

\begin{protocol}{$\QPIP_1$ protocol $\PiSamp$ for $\SampBQP$}\label{ProtoQPIP1}
    \Ethan{Two action items; define T. Don't set eps this early.}
	Pick $\varepsilon=\poly(T^{-1})$.
	
	Let $C'$ be $C$ padded with $\poly(\varepsilon^{-1})$ identity gates at the end.
	Pick $h=\poly(\varepsilon)$, $m=O(h^{-3})$, $M=\poly(m)$.
	\Ethan{Solve instead of using ``poly"; ``order" is fine.
	For running time, poly is fine though.}
	\Ethan{get style from pg 22 on other paper}
	
	\Ethan{Also see 102, p2, 5}
	
	\Ethan{201 p12 p23 two different styles of protocol-writing}
	
	\Ethan{106 p31}
	
	\Ethan{Specify size of h, m, M when they appear instead of at the top}
	
	\Ethan{Use gamma or whatever Greek letter instead of h for threshold}

	\begin{enumerate}
		\item The honest prover prepares $M$ copies of $\ket{\psi_{C'(x)}}$ and sends all of it to the verifier qubit-by-qubit.
		\item The verifier privately samples $I\subset[M]$ s.t. $\abs{I}=m$, and $k\xleftarrow{\$}[M]\setminus I$.
			For $i$ from $1$ to $M$, it chooses what to do to the $i$-th copy, $\rho_i$, as follows:
		\begin{enumerate}
			\item If $i\in I$, run $z_i\leftarrow\cV_{\GS}(\rho_i)$.
			\item Else if $i=k$, measure the data register \Ethan{Revisit this name} of $\rho_i$ and save the outcome as $y$.
			\item Else, discard $\rho_i$.
		\end{enumerate}
			Let $Z=\sum_{i\in I} z_i$. If $Z>\frac{m}{2}-hm$ then the verifier accepts and outputs $y$. Otherwise, it rejects.
	\end{enumerate}
\end{protocol}

Note that $\VSamp$ only needs to apply $X$ and $Z$ measurements, and is classical otherwise. We now show the completeness and soundness of $\PiSamp$.

\begin{thm}
    \label{QPIP1thm}
	$\PiSamp$ has negligible completeness error and $\poly(\varepsilon)$ soundness error.
\end{thm}
\begin{prf}
	For completeness, notice $z_i$ are i.i.d. Bernoulli trials with success probability $\frac{1}{2}$.
	So we can apply the Chernoff bound \cref{thm:Chernoff} with $\mu=\frac{m}{2}$, $\delta=2h$ to get
	$$\Prob{\frac{m}{2}-Z\geq hm}\leq2e^{-\frac{\mu\delta^2}{3}}=\negl(T).$$
	\Ethan{Just match the protocol above; write $Z>m/2-hm$.
	This is Chernoff bound; nobody will get confused over it not matching the theorem statement precisely.}

	Now we show soundness.
	Suppose $\PSampstar$ is a cheating prover that sends some $\sigma$ to the verifier.

    \Ethan{``First step of our analysis is to use..."; use common sense?}
    
	We first show that de Finetti's theorem can indeed be used here to achieve independence.
	Randomly picking $m+1$ out of $M$ registers is equivalent to first applying a random permutation to get some $\sigma'$ then taking the first $m+1$ registers.
	A random permutation, in turn, is a classical mix over all possible permutations:
	$$\sigma'=\frac{1}{\abs{\Sym(M)}}\sum_{\Pi\in\Sym(M)}\Pi\sigma\Pi^\dagger.$$
	It is simple to check that $\sigma'$ is permutation-invariant:
	fix $\tilde{\Pi}\in\Sym(M)$, then
	$$\tilde{\Pi}\sigma'\tilde{\Pi}^\dagger
	=\frac{1}{\abs{\Sym(M)}}\sum_{\Pi\in\Sym(M)}\tilde{\Pi}\Pi\sigma\Pi^\dagger\tilde{\Pi}^\dagger
	=\frac{1}{\abs{\Sym(M)}}\sum_{\hat{\Pi}\in\Sym(M)}\hat{\Pi}\sigma\hat{\Pi}^\dagger
	=\sigma'$$
	where the second equality is by relabeling $\tilde{\Pi}\Pi=\hat{\Pi}$.

	Now we apply \cref{deFinetti} \Ethan{Theorem is always uppercase. Same with Lemma.} to approximate $\sigma'$ with a classical mix over tensors of independent states.
	\Ethan{Say this is measurement results, and state typeof(Lambda) while at it. Define specific Lambdas instead of max over; or actually, use for all}
	That is, $\exists\rho=\sum_j w_j\rho_j^{\otimes m+1}$ \Ethan{Double check that registers of rhos are all identical} such that:
	$$\max_{\Lambda_i}\norm{\Lambda_1\otimes\ldots\otimes\Lambda_{m+1}(\sigma'-\rho)}=O(\varepsilon).$$

	Now we separate the $\rho_j$ into two categories: $J=\set{j\in I|p_j=\cV_\GS(\rho_j)<\frac{1}{2}-2h}$ and its complement.

	First suppose $j\in J$, we use a standard Chernoff bound argument as follows to show that this case has negligible acceptance probabilities,
	hence incurs negligible soundness errors.
	Let $Y=\sum y_j$ be i.i.d. Bernoulli trials with success probabilities $p_y=\frac{1}{2}-2h$,
	then clearly $\Prob{Z\geq\frac{m}{2}-hm}<\Prob{Y\geq\frac{m}{2}-hm}$.
	By Chernoff bound, we have
	$$\Prob{Y\geq\frac{m}{2}-hm}=\Prob{Y-\mu\geq hm}\leq2e^{-\frac{\mu\delta^2}{3}}=\negl(T)$$
	where $\mu=mp_y=\frac{m}{2}-2hm$ and $\delta=\frac{2h}{1-4h}$.
	Summing up all the $\rho_j$ terms in $J$, we have
	$$\sum_{j\in J} w_j q_j<2^{-\lambda}=\negl(\lambda).$$

	Now, for the other case, $j\notin J$, we have
	$$\Prob{\cV_\GS(\rho_i)=acc}>\frac{1}{2}-2h\Rightarrow\tr\left(H_{C'(x)}\rho_j\right)<\poly(\varepsilon).$$
	We also know that the least nonzero eigenvalue of $H_{C'(x)}$ is lower-bounded by $\frac{3}{4}$, so we obtain \Ethan{Need notation for fidelity}
	$$D(\rho_j, \ket{\phi_{C'(x)}})<\poly(\varepsilon)$$

	The probability of measuring $t<T$ on the clock register of $\ket{\phi_{C'(x)}}$ is $\poly(\varepsilon)$ due to the padding,
	so the data register has $1-\poly(\varepsilon)$ probability to be $C(x)$ at this point.

	The soundness errors at each step is $\poly(\varepsilon)$, so the conclusion follows.
\end{prf}
