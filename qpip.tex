%\section{Delegation Protocol for $\QPIP_1$ Client}

% In this section, we construct a one-message $\QPIP_1$ delegation protocol for $\SampBQP$.
% At a high level, it is a cut-and-choose protocol.
% The server constructs multiple copies of the ground state as certificates,
% then the client randomly chooses a copy to output and checks the rest.
% Unfortunately our approach incurs inverse polynomial soundness errors.

% There are two main challenges to this cut-and-choose approach.
% First, the client needs to reliably extract the circuit output from its corresponding ground state.
% We accomplish this by padding the circuit with identity gates at the end.
% By doing so, the clock register collapses to after the last non-identity gate with high probability.

% The other challenge is that a cheating prover's certificates can be arbitrarily entangled,
% so common techniques such as Chernoff bounds can't be applied as-is.
% This wasn't a challenge for $\Piblind$ for $\BQP$ earlier because...
% \Ethan{TODO; need brief and intuitive explanation}
% In our case with $\SampBQP$, we overcome this challenge instead by using de Finetti's theorem
% to approximate our measurement results with that of some unentangled copies,
% within inverse polynomial errors.

\begin{protocol}{$\QPIP_1$ protocol $\PiSamp=(\PSamp, \VSamp)$ for the $\SampBQP$ problem $(D_x)_{x\in\set{0, 1}^*}$}
	\label{ProtoQPIP1}

	Inputs:
	\begin{itemize}
		\item Security parameter $1^\lambda$ where $\lambda\in\bbN$
		\item Accuracy parameter $1^{1/\eps}$ where $\eps\in(0, 1)$
		\item Classical input $x\in\zo^n$ to the $\SampBQP$ instance
		%	\Ethan{I switched the order of the inputs for consistency}
	\end{itemize}

	Ingredients:
	\begin{itemize}
		\item Let $C$ be a quantum circuit consisting of only Hadamard and Toffoli gates, which on input $x$ samples from some $C_x$ such that $\norm{C_x-D_x}_{\mathrm{TV}}\leq\eps$.
		\item Let $T$ be the number of gates in $C$.
		\item Let $C'$ be $C$ padded with $\frac{6T}{\eps}$ identity gates at the end.
		\item Let $H_{C'(x)}=\sum_i \alpha_i H_i$ be the X-Z local Hamiltonian instance associated with the computation of $C'(x)$ from \Cref{thm:LHReduction}.
		\item Let $\histpsi{{C'(x)}}$ be the ground state of $H_{C'(x)}$.
		\item Let $s\leq 3T+\frac{6T}{\eps}$ be the number of qubits in $\histpsi{C'(x)}$.
		%Why this number?} \Ethan{This is used in de Finetti theorem. Btw I forgot to count ancilla, so it wasn't correct before.}
		\item Let $\cVGS$ be the verification algorithm for $H_{C'(x)}$ as defined in \Cref{AlgGroundStateCheck}.
	\end{itemize}

	Protocol:
	\begin{enumerate}
		\item\label{step:qpip1-state-gen} The honest prover prepares $M=649\frac{T^{41}\lambda^2}{\eps^{51}}$ copies of $\histpsi{C'(x)}$ and sends all of it to the verifier qubit-by-qubit.
		\item\label{step:qpip1-verify} The verifier samples $I\subset[M]$ s.t. $\abs{I}=m$ where $m=\frac{T^{20}\lambda}{\eps^{24}}$, and $k\xleftarrow{\$}[M]\setminus I$.
			For $i$ from $1$ to $M$, it chooses what to do to the $i$-th copy, $\rho_i$, as follows:
		\begin{enumerate}
			\item If $i\in I$, run $y_i\leftarrow\cV_{\GS}(\rho_i)$.
			\item Else if $i=k$, measure the data register 
			%\Ethan{Revisit this name. Nobody has defined/used this register before.} \XW{this is the Z measurement?} \Ethan{Yeah} 
			of $\rho_i$ under the computational (i.e., $Z$) basis and save the outcome as $z$.
			\item Else, discard $\rho_i$.
		\end{enumerate}
			Let $Y=\sum_{i\in I} y_i$. If $Y>\frac{m}{2}-\kappa m$ where $\kappa=\frac{1}{192}\frac{\eps^2}{\sum_i\abs{\alpha_i}}$, then the verifier outputs $(\Acc, z)$.
			Else, it outputs $(\Rej, \bot)$.
	\end{enumerate}
\end{protocol}

Note that $\VSamp$ only needs to apply $X$ and $Z$ measurements, and is classical otherwise.
It is simple to check that $\VSamp$ runs in $\poly(n, \lambda, \frac{1}{\epsilon})$ time.
We now show its completeness and soundness.
  
\begin{thm}
    \label{QPIP1thm}
	$\PiSamp$ is a $\QPIP_1$ protocol for the $\SampBQP$ problem $(D_x)_{x\in\set{0,1}^*}$ with negligible completeness error and is statistically sound.\footnote{The soundness and completeness of a $\SampBQP$ protocol is defined in \Cref{dfn:stats-secure-proto-sampbqp}.}
	%\XW{do we want precise error bound?}
\end{thm}
\begin{prf}
	For completeness, notice $y_i$ are i.i.d. Bernoulli trials with success probability $\frac{1}{2}$.
	So we can apply Chernoff bound to get
	$$\Prob{Y>\frac{m}{2}-\eps m}\leq\negl(\lambda).$$
	So $\VSamp$ has negligible probability to reject.
	
	Now we show soundness.
	Suppose $\PSampstar$ is a cheating prover that, on inputs $x, 1^{1/\eps}, 1^\lambda$, sends some $\sigma$ to the verifier.

	The first step of our analysis is to use de Finetti's theorem.
	Randomly picking $m+1$ out of $M$ registers is equivalent to first applying a random permutation then taking the first $m+1$ registers.
	A random permutation, in turn, is a classical mix over all possible permutations:
	$$\sigma'=\frac{1}{\abs{\Sym(M)}}\sum_{\Pi\in\Sym(M)}\Pi\sigma\Pi^\dagger.$$
	It is simple to check that $\sigma'$ is permutation-invariant:
	fix $\tilde{\Pi}\in\Sym(M)$, then
	$$\tilde{\Pi}\sigma'\tilde{\Pi}^\dagger
	=\frac{1}{\abs{\Sym(M)}}\sum_{\Pi\in\Sym(M)}\tilde{\Pi}\Pi\sigma\Pi^\dagger\tilde{\Pi}^\dagger
	=\frac{1}{\abs{\Sym(M)}}\sum_{\hat{\Pi}\in\Sym(M)}\hat{\Pi}\sigma\hat{\Pi}^\dagger
	=\sigma',$$
	where the second equality is by relabeling $\tilde{\Pi}\Pi=\hat{\Pi}$.

	Now we apply de Finetti's theorem (\Cref{deFinetti}) to approximate our measurement result on $\sigma$ with that of a classical mix over $(m+1)$-fold tensor products of identical copies of some states $\tau_j$.
	That is, there exists some $\rho=\sum_j w_j\tau_j^{\otimes m+1}$
	such that for all quantum-classical channels $\Lambda_i$ acting on registers of size $s$:
	$$\norm{\Lambda_1\otimes\ldots\otimes\Lambda_{m+1}(\sigma'_{\leq m+1}-\rho)}_1
	\leq\sqrt{\frac{2m^2s}{M-m}}
	\leq\frac{\eps}{6},$$
	where $\sigma'_{\leq m+1}$ is the first $m+1$ registers of $\sigma'$.
	In our context, for $1\leq i\leq m$, $\Lambda_i$ corresponds to measurements chosen by $\cVGS$.
	$\Lambda_{m+1}$ measures the data register under the standard basis.
	
	We now analyze the verifier's output for each $\tau_j$ in $\rho=\sum_j w_j\tau_j^{\otimes m+1}$. Let $(d_j, z_j)$ be the verifier's output corresponding to $\tau_j$, and define $z_{j, ideal}$ accordingly.
	We claim that $\norm{(d_j, z_j)-(d_j, z_{j, ideal})}_{\mathrm{TV}}<\frac{2\eps}{6}$.
	Let $p_j$ be the probability that $\cVGS(\tau_j)=\Acc$, and consider the following two cases.

	First suppose $p_j<\frac{1}{2}-2\kappa$,
	then a standard Chernoff bound argument can be applied show that this case has negligible acceptance probabilities.
	As a result, $\norm{(d_j, z_j)-(d_j, z_{j, ideal})}_{\mathrm{TV}}=\negl(\lambda)$.

	Now suppose $p_j\geq\frac{1}{2}-2\kappa$.
	\iffalse
	By \Cref{cor:HamCheck} we have a lower bound for the fidelity between $\tau_j$ and $\ket{\psi_{C'(x)}}$:
	$$p_j>\frac{1}{2}-2\kappa\Rightarrow F(\tau_j, \ket{\psi_{C'(x)}}\bra{\psi_{C'(x)}})\geq1-\frac{\eps^2}{36}.$$
	which in turn implies a lower bound on the respective trace distance:
	$$\norm{\tau_j - \ket{\psi_{C'(x)}}\bra{\psi_{C'(x)}}}_{\tr}<\frac{\eps}{6}.$$
	\fi
	By \Cref{thm:HamCheckClose} we have
	$$\norm{\tau_j - \ket{\psi^{\mathrm{hist}}_{C'(x)}}\bra{\psi^{\mathrm{hist}}_{C'(x)}}}_{\tr}\leq\frac{2}{\sqrt{3}}\sqrt{2\kappa\cdot2\sum_i\abs{\alpha_i}}=\frac{\eps}{6}.$$
	Observe that when $d_j=\Acc$, $z_j$ is the measurement results on $\tau_j$'s data register,
	which is $\frac{\eps}{6}$-close to that of $\ket{\psi_{C'(x)}}$.
	As the clock register is traced out from $\ket{\psi_{C'(x)}}$, the data register has at least $1-\frac{\eps}{6}$ probability to contain $C(x)$ due to the padding.
	So $z_j$ is $\frac{2\eps}{6}$-close to $z_{ideal}$ when $d=\Acc$,
	which implies that $(d_j, z_j)$ is $\frac{2\eps}{6}$-close to $(d_j, z_{j, ideal})$.

    Finally, let $(d_\rho, z_\rho)$ denote the verifier's output distribution corresponding to $\rho=\sum_j w_j\tau_j^{\otimes m+1}$,
	and define $z_{\rho, ideal}$ accordingly.
	We have $\norm{(d_\rho, z_\rho) - (d, z_{\rho, ideal})}_{\mathrm{TV}} < \frac{2\eps}{6}$ since the same is true for all components $\tau_j$.

	As $(d, z)$ is $\frac{\eps}{6}$-close to $(d_\rho, z_\rho)$ by the data processing inequality,
	using the triangle inequality we have $\norm{(d, z) - (d_\rho, z_{\rho, ideal})}_{\mathrm{TV}}\leq\frac{\eps}{2}$.
	Hence $\norm{d - d_\rho}_{\mathrm{TV}}\leq\frac{\eps}{2}$, which implies $\norm{(d_\rho, z_{\rho, ideal}) - (d, z_{ideal})}_{\mathrm{TV}}$ $\leq$ $\frac{\eps}{2}$.
	It then follows from the triangle inequality that
	$\norm{(d, z) - (d, z_{ideal})}_{\mathrm{TV}}\leq\eps$.
\end{prf}
