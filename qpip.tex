\section{Our QPIP Scheme}

Here we create a QPIP scheme for...  In the case for decision problems, it would be a simple Chernoff bound as in \cite{kitaev2002classical}. A prover, honest or not, couldn't do better than sending $n$ identical copies of some state. The same reasoning does not generalize to our case, so instead we give the following construction.

We first give an alternative characterization of $\QPIP_0$.

\begin{definition}
	A sampling problem $S=(D_x)$ is said to be in $\QPIP_{XZ}$, or has a $\QPIP_{XZ}$ protocol,  with completeness $c$ and soundness $s$ if there exists a pair of algorithms $(\bbP, \bbV)$ with the following properties with input $x$:
	\begin{itemize}
		\item $\bbP$ is a $\BQP$ machine.
		\item $\bbV$ is a hybrid quantum-classical machine. Its classical part is a $BPP$ machine. Its quantum part is a single-qubit register, on which the verifier can apply only $X$ or $Z$ measurements.
		\item There is a quantum channel between $\bbP$ and $\bbV$ that can transmit one qubit at a time.
		\item There is a classical communication channel between $\bbP$ and $\bbV$.
		\item (Completeness) After interacting with $\bbP$, $\bbV$ accepts with probability $\geq c$.
		\item (Soundness) Given $\bbP'$, if $(\bbP', \bbV)$ accepts with probability greater than $\delta$, then conditioned on this acceptance, $\bbV$'s outputs has distribution $\tilde{D}$ with $\norm{\tilde{D}-D_x}\leq s$.
	\end{itemize}
\end{definition}

\begin{theorem}
	$\QPIP_{XZ}=\QPIP_0$
\end{theorem}
\begin{proof}
	See \cite{mahadev_delegation}.
\end{proof}

\begin{theorem}
	$\SampBQP$ is in $\QPIP_{XZ}$.
\end{theorem}

\begin{theorem}
	There is a $\QPIP_{XZ}$ protocol for TODO some ref for estimating ground state
\end{theorem}
\begin{proof}
	TODO The prover sends whatever-it-is qubit-by-qubit...
\end{proof}

Suppose $\mathcal{F}$ accepts $\ket{\psi}$ with probability $\frac{1}{2}$, and any states perpendicular to it with probability $\frac{1}{2}-\Delta$.

Let $\varepsilon>0$, $p>0$. Pick $n\in\mathbb{N}$ so that by Chernoff bound:
\begin{itemize}
	\item $\ket{\psi^{\otimes n}}$ is rejected with probability at most $p$.
	\item If $\mathcal{F}(\ket{\phi})$ has less than $\frac{1}{2}-\varepsilon$ accept probability, then $\ket{\phi^{\otimes n}}$ is accepted with probability at most $p$.
\end{itemize}

\floatname{algorithm}{Protocol}
TODO Do something about this
\begin{algorithm}
	\caption{Amplification with simple input}
	\label{AlgAmp1}
	\begin{algorithmic}[1]
		\State The prover sends $\poly(n)$ copies of...
		\State The verifier picks $m+n$ copies...
		\State The verifier runs Chernoff bound on...
		\If{number of accepted copies $\geq (\frac{1}{2}-\frac{\varepsilon}{2})n$}
			\State Measure time step...
			\State Output the...
		\Else
			\State Reject
		\EndIf
	\end{algorithmic}
\end{algorithm}
TODO what's $n$?

TODO Talk about de Finetti's here.

Now we consider the general case. We use \autoref{deFinetti}. Let $n=l$ and $k=\poly(n)$, so $\sqrt{\frac{2l^2\ln\abs{A}}{k-l}}$ can be arbitrarily inverse-poly small. That is, with some small error in measurement results, we can assume without the loss of generality that the rest of the algorithm acts on an input of the form... TODO.

\begin{theorem}
	When \hyperref[AlgAmp1]{Protocol~\ref*{AlgAmp1}} accepts an input with probability greater than $\delta$, conditioned on this acceptance, $\mathcal{F}$ has probability at least $1-\frac{p}{\delta}$ to accept each register of the output.
\end{theorem}
\begin{proof}
	Note that the input is a classical probability distribution over inputs $\ket{\rho_i^{\otimes n+m}}$ with $w_i$ as weights.

	Let $q_j$ be the accept probability of $\ket{\rho_j^{\otimes n+m}}$ under \autoref{AlgAmp1}. By Bayes' theorem, conditioned on acceptance, the probability of the input being $\ket{\rho_j^{\otimes n+m}}$ is then $\frac{w_j q_j}{\sum_i w_i q_i}$.

	Let $J=\set{j:q_j<p}$.

	$\Rightarrow\sum_{j\in J} w_j q_j<p$, since $\sum_i w_i=1$.
	
	$\Rightarrow\frac{\sum_{j\in J} w_j q_j}{\sum_i w_i q_i}<\frac{p}{\delta}$.
\end{proof}

So we pad the circuit with identity matrices at the end before constructing... TODO, if we measured $t>\frac{T}{2}$ on the time register, we would know that the other registers include the output of the required quantum computation. So simply take $m$ to be high enough that it happens at least once with overwhelming probability. When it does, measure the entire register. All the errors up to this point are $O(\varepsilon)$, so will be the distance between this output distribution and the true one.

\Ethan{The above is incredibly handwavy.}
