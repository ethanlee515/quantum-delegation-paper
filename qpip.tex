\section{Delegation Protocol for $\QPIP_1$ Client}

In this section, we construct a one-message $\QPIP_1$ delegation protocol for $\SampBQP$.
At a high level, it is a cut-and-choose protocol.
The server constructs multiple copies of the ground state as certificates,
then the client randomly chooses a copy to output and checks the rest.
Unfortunately our approach incurs inverse polynomial soundness errors.

There are two main challenges to this cut-and-choose approach.
First, the client needs to reliably extract the circuit output from its corresponding ground state.
We accomplish this by padding the circuit with identity gates at the end.
By doing so, the clock register collapses to after the last non-identity gate with high probability.

The other challenge is that a cheating prover's certificates can be arbitrarily entangled,
so common techniques such as Chernoff bounds can't be applied as-is.
This wasn't a challenge for $\Piblind$ for $\BQP$ earlier because...
\Ethan{TODO; need brief and intuitive explanation}
In our case with $\SampBQP$, we overcome this challenge instead by using de Finetti's theorem
to approximate our measurement results with that of some unentangled copies,
within inverse polynomial errors.


\Ethan{Again, look at other papers...}

We start with a $\SampBQP$ problem $(D_x)_{x\in\set{0, 1}^*}$ and an input $x$,
then there exists some quantum circuit $C$ that samples from $D_x$ within desired accuracy.
We assume without loss of generality that $C$ comprises of only Hadamard and Toffoli gates.
We now present our $\QPIP_1$ protocol that allows a client to verifiably obtain the measurement outcome of $C(x)$.

\begin{protocol}{$\QPIP_1$ protocol $\PiSamp$ for computation of $C(x)$ and error parameter $\varepsilon$}\label{ProtoQPIP1}

	Let $C'$ be $C$ padded with $\frac{T}{\varepsilon}$ identity gates at the end.
	Let $\ket{\psi_{C'(x)}}$ be the local Hamiltonian ground state associated with $C'$ as defined in \Cref{thm:LHReduction},
	and $\cVGS$ be the way to check this ground state as defined in \Cref{AlgGroundStateCheck}.
	Recall that by \Cref{thm:HamCheck} there exists some $c\in\bbN$
	\Ethan{Maybe figure out the exact value of $c$ later? It's really messy due to multiple uses of projection lemma among other things}
	so that if $\cVGS$ accepts $\rho$ with probability at least $\frac{1}{2}-\kappa$ where $\kappa=\frac{\varepsilon}{(T+\frac{T}{\varepsilon})^c}$ then $F(\rho, \ket{\psi_{C'(x)}})\geq1-\varepsilon^2$
	\Ethan{TODO: That theorem currently doesn't even say that. At least write a corollary for it.}

	\Ethan{get style from pg 22 on other paper}
	
	\Ethan{Also see 102, p2, 5}
	
	\Ethan{201 p12 p23 two different styles of protocol-writing}
	
	\Ethan{106 p31}

	\begin{enumerate}
		\item The honest prover prepares $M=\kappa^{-4}\varepsilon^{-4}\ln(\frac{T}{\varepsilon})$ copies \Ethan{That isn't an integer...?} of $\ket{\psi_{C'(x)}}$ and sends all of it to the verifier qubit-by-qubit.
		\item The verifier privately samples $I\subset[M]$ s.t. $\abs{I}=m$ where $m=\kappa^{-2}\varepsilon^{-1}$, and $k\xleftarrow{\$}[M]\setminus I$.
			For $i$ from $1$ to $M$, it chooses what to do to the $i$-th copy, $\rho_i$, as follows:
		\begin{enumerate}
			\item If $i\in I$, run $z_i\leftarrow\cV_{\GS}(\rho_i)$.
			\item Else if $i=k$, measure the data register \Ethan{Revisit this name} of $\rho_i$ and save the outcome as $y$.
			\item Else, discard $\rho_i$.
		\end{enumerate}
			Let $Z=\sum_{i\in I} z_i$. If $Z>\frac{m}{2}-\varepsilon m$ then the verifier accepts and outputs $y$. Otherwise, it rejects.
	\end{enumerate}
\end{protocol}

Note that $\VSamp$ only needs to apply $X$ and $Z$ measurements, and is classical otherwise. We now show the completeness and soundness of $\PiSamp$.

\begin{thm}
    \label{QPIP1thm}
	$\PiSamp$ has negligible completeness error and $\poly(\varepsilon^{-1})$ soundness error.
\end{thm}
\begin{prf}
	For completeness, notice $z_i$ are i.i.d. Bernoulli trials with success probability $\frac{1}{2}$.
	So we can apply the Chernoff bound \cref{thm:Chernoff} with $\mu=\frac{m}{2}$ and $\delta=2\varepsilon$ to get
	$$\Prob{Z>\frac{m}{2}-\varepsilon m}\leq2e^{-\frac{\mu\delta^2}{3}}=\negl(\varepsilon^{-1}).$$

	Now we show soundness.
	Suppose $\PSampstar$ is a cheating prover that sends some $\sigma$ to the verifier.

	The first step of our analysis is to use de Finetti's theorem.
	Randomly picking $m+1$ out of $M$ registers is equivalent to first applying a random permutation to get some $\sigma'$ then taking the first $m+1$ registers.
	A random permutation, in turn, is a classical mix over all possible permutations:
	$$\sigma'=\frac{1}{\abs{\Sym(M)}}\sum_{\Pi\in\Sym(M)}\Pi\sigma\Pi^\dagger.$$
	It is simple to check that $\sigma'$ is permutation-invariant:
	fix $\tilde{\Pi}\in\Sym(M)$, then
	$$\tilde{\Pi}\sigma'\tilde{\Pi}^\dagger
	=\frac{1}{\abs{\Sym(M)}}\sum_{\Pi\in\Sym(M)}\tilde{\Pi}\Pi\sigma\Pi^\dagger\tilde{\Pi}^\dagger
	=\frac{1}{\abs{\Sym(M)}}\sum_{\hat{\Pi}\in\Sym(M)}\hat{\Pi}\sigma\hat{\Pi}^\dagger
	=\sigma'$$
	where the second equality is by relabeling $\tilde{\Pi}\Pi=\hat{\Pi}$.

	Now we apply de Finetti's theorem (\Cref{deFinetti}) to approximate $\sigma'$ with a classical mix over tensors of independent states.
	Note that the ground state has $\abs{A}=n+T+\frac{T}{e}$ qubits.
	That is, $\exists\rho=\sum_j w_j\rho_j^{\otimes m+1}$ such that for all quantum-classical channels $\Lambda_i$ on $\abs{A}$ qubits:
	$$\norm{\Lambda_1\otimes\ldots\otimes\Lambda_{m+1}(\sigma'-\rho)}=\frac{todo}{todo}=O(\varepsilon).$$

	Now we separate the $\rho_j$ into two categories: $J=\set{j\in I|p_j=\cV_\GS(\rho_j)<\frac{1}{2}-2\varepsilon}$ and its complement.

	First suppose $j\in J$, we use a standard Chernoff bound argument as follows to show that this case has negligible acceptance probabilities,
	hence incurs negligible soundness errors.
	Let $Y=\sum y_j$ be i.i.d. Bernoulli trials with success probabilities $p_y=\frac{1}{2}-2h$,
	then clearly $\Prob{Z\geq\frac{m}{2}-hm}<\Prob{Y\geq\frac{m}{2}-hm}$.
	By Chernoff bound, we have
	$$\Prob{Y\geq\frac{m}{2}-hm}=\Prob{Y-\mu\geq hm}\leq2e^{-\frac{\mu\delta^2}{3}}=\negl(T)$$
	where $\mu=mp_y=\frac{m}{2}-2hm$ and $\delta=\frac{2h}{1-4h}$.
	Summing up all the $\rho_j$ terms in $J$, we have
	$$\sum_{j\in J} w_j q_j<2^{-\lambda}=\negl(\lambda).$$

	Now, for the other case, $j\notin J$, we have
	$$\Prob{\cV_\GS(\rho_i)=acc}>\frac{1}{2}-2\varepsilon\Rightarrow\sqrt{\tr H_{C'(x)}\rho_j}<\varepsilon.$$
	We also know that the least nonzero eigenvalue of $H_{C'(x)}$ is lower-bounded by $\frac{3}{4}$, so we obtain \Ethan{Need notation for fidelity}
	$$D(\rho_j, \ket{\phi_{C'(x)}})<\poly(\varepsilon)$$

	The probability of measuring $t<T$ on the clock register of $\ket{\phi_{C'(x)}}$ is $\poly(\varepsilon)$ due to the padding,
	so the data register has $1-\poly(\varepsilon)$ probability to be $C(x)$ at this point.

	The soundness errors at each step is $\poly(\varepsilon)$, so the conclusion follows.
\end{prf}
