\section{Delegation Protocol for Hybrid Client}

In this section, we construct a one-message $\QPIP_1$ delegation protocol for $\SampBQP$.
At a high level, it is a cut-and-choose protocol.
The server constructs multiple copies of the ground state as certificates,
then the client randomly chooses a copy to output and checks the rest.
Unfortunately our approach incurs inverse polynomial soundness errors.

There are two main challenges to this cut-and-choose approach.
First, the client needs to reliably extract the circuit output from its corresponding ground state.
We accomplish this by padding the circuit with identity gates at the end.
By doing so, the clock register collapses to after the last non-identity gate with high probability.

The other challenge is that a cheating prover's certificates can be arbitrarily entangled,
so common techniques such as Chernoff bounds can't be applied as-is.
This wasn't a challenge for $\Piblind$ for $\BQP$ earlier because...
\Ethan{TODO; need brief and intuitive explanation}
In our case with $\SampBQP$, we overcome this challenge instead by using de Finetti's theorem
to approximate our measurement results with that of some unentangled copies,
within inverse polynomial errors.

\def\GS{\mathsf{GS}}
\nc{\PiGS}{\ensuremath{\Pi_\GS}}
\nc{\VGS}{\ensuremath{V_\GS}}
\nc{\PGS}{\ensuremath{P_\GS}}
\nc{\PGSstar}{\ensuremath{P_\GS^*}}
\nc{\cVGS}[1]{\ensuremath{\cV_{\GS,#1}}}
\nc{\cPGS}[1]{\ensuremath{\cP_{\GS,#1}}}

\def\Samp{\mathsf{Samp}}
\nc{\PiSamp}{\ensuremath{\Pi_\Samp}}
\nc{\VSamp}{\ensuremath{V_\Samp}}
\nc{\PSamp}{\ensuremath{P_\Samp}}
\nc{\PSampstar}{\ensuremath{P_\Samp^*}}
\nc{\cVSamp}[1]{\ensuremath{\cV_{\Samp,#1}}}
\nc{\cPSamp}[1]{\ensuremath{\cP_{\Samp,#1}}}

\def\GS{\mathsf{GS}}

Now we present our $\QPIP_1$ protocol. It is parametrized by $\lambda$, the degree of its inverse polynomial soundness.

\begin{protocol}{$\QPIP_1$ protocol $\PiSamp$ for $\SampBQP$}\label{ProtoQPIP1}
	Let $C'$ be $C$ padded with $T^{\lambda + 1}$ identity gates at the end.

	Pick $\varepsilon$ small enough so it holds for all $\ket\phi$ that:
	\begin{equation}
		\label{QPIP1eps}
		P[\cA_\GS(\ket{\phi})=acc]>\frac{1}{2}-2\varepsilon\Rightarrow\braket{\phi|H_{C'(x)}|\phi}<\frac{1}{T^\lambda}
	\end{equation}

	Pick $n$ large enough so by Chernoff bound:
	\begin{equation}
		\label{QPIP1Chernoff1}
		P\left[Bin(n, \frac{1}{2}-2\varepsilon)\geq\left(\frac{1}{2}-\varepsilon\right)n\right]<2^{-\lambda}
	\end{equation}
	\begin{equation}
		\label{QPIP1Chernoff2}
		P\left[Bin(n, \frac{1}{2})\leq\left(\frac{1}{2}-\varepsilon\right)n\right]<2^{-\lambda}
	\end{equation}

	Let $N$ be large enough so that de Finetti's theorem can be applied at the cost of $T^{-\lambda}$ errors if one randomly picks $n+1$ subsystems out of $N$ permutation-invariant subsystems.
		That is,
		$$\sqrt{\frac{2(n+1)^2\ln\abs{A}}{N-(n+1)}}<T^{-\lambda}$$
		where $\abs{A}$ is the number of qubits in $\ket{\psi_{C'(x)}}$

	\begin{enumerate}
		\item The verifier privately samples $I\subset[N]$ s.t. $\abs{I}=n$.
			It then privately samples $k\xleftarrow{\$}[N]\setminus I$.
		\item The honest prover prepares $N$ copies of $\ket{\psi_{C'(x)}}$ and sends all of them to the verifier qubit-by-qubit.
		\item For $i$ from $1$ to $N$, the verifier chooses what to do to the $i$-th copy, $\rho_i$, as follows:
		\begin{enumerate}
			\item If $i\in I$, run $\cA_{\GS}(\rho_i)$.
			\item Otherwise, if $i=k$, measure the data register of $\rho_i$ and save the outcome as $y$.
			\item Otherwise, discard $\rho_i$.
		\end{enumerate}
		\item If the proportion of copies accepted by $\cA_{\GS}$ is greater than $\frac{1}{2}-\varepsilon$ then the verifier accepts and outputs $y$. Otherwise, it rejects.
	\end{enumerate}
\end{protocol}

Note that $\VSamp$ only needs to apply $X$ and $Z$ measurements, and is classical otherwise. We now show the completeness and soundness of $\PiSamp$.

\begin{thm}
    \label{QPIP1thm}
	$\PiSamp$ has negligible completeness error and $O(T^-\lambda)$ soundness error.
\end{thm}
\begin{prf}
	The completeness follows from \cref{QPIP1Chernoff2} and inspection.

	Now we show soundness.
	Suppose $\PSampstar$ is a cheating prover that sends some $\sigma$ to the verifier.

	We first show that de Finetti's theorem can indeed be used here to achieve independence.
	Randomly picking $n+1$ out of $N$ registers is equivalent to first applying a random permutation then taking the first $n+1$ registers.
	A random permutation, in turn, is a classical mix over all possible permutations:
	$$\sigma'=\frac{1}{\abs{\Sym(N)}}\sum_{\Pi\in\Sym(N)}\Pi\sigma\Pi^\dagger$$
	We then verify that $\sigma'$ is permutation-invariant.
	Fix $\tilde{\Pi}\in\Sym(N)$, then
	$$\tilde{\Pi}\sigma'\tilde{\Pi}^\dagger
	=\frac{1}{\abs{\Sym(N)}}\sum_{\Pi\in\Sym(N)}\tilde{\Pi}\Pi\sigma\Pi^\dagger\tilde{\Pi}^\dagger
	=\frac{1}{\abs{\Sym(N)}}\sum_{\hat{\Pi}\in\Sym(N)}\hat{\Pi}\sigma\hat{\Pi}^\dagger
	=\sigma'$$
	where the second equality is by relabeling $\tilde{\Pi}\Pi=\hat{\Pi}$, which is allowed since $\Sym(N)$ is a group.

	Now we apply \cref{deFinetti} to approximate $\sigma'$ with a classical mix over tensors of independent states.
	That is, $\exists\rho=\sum_i w_i\rho_i^{\otimes n+1}$ such that:
	$$\max_{\Lambda_i}\norm{\Lambda_1\otimes\ldots\otimes\Lambda_{n+1}(\sigma'-\rho)}=O(\varepsilon)$$

	Let $p_i$ be the accept probability of at least $(\frac{1}{2}-\varepsilon)n$ copies from $\rho_i^{\otimes n}$ accept under $\cA_{\GS}$.
	We now partition the terms $\rho_i$ into two categories according to the Chernoff bound in \cref{QPIP1Chernoff1}:
	$$I=\set{i:p_i\geq 2^{-\lambda}}$$

	For $i\notin I$ we have
	$$\sum_{i\in I} w_i p_i<2^{-\lambda}=\negl(\lambda)$$
	so we can approximate with negligible error that all $\rho_i$ where $i\notin I$ are rejected.

	For $i\in I$ we have
	$$P[\cA_\GS(\rho_i)=acc]>\frac{1}{2}-2\varepsilon$$
	and hence
	$$\braket{\rho_i|H_{C'(x)}|\rho_i}<\frac{1}{T^\lambda}$$
	by \cref{QPIP1eps}.
	We also know that the least nonzero eigenvalue of $H_{C'(x)}$ is lower-bounded by $\frac{3}{4}$, so we obtain... \Ethan{TODO not done yet. Probably some squares or square roots missing here.}
	$$\braket{\rho_i|\phi_{C'(x)}}>1-O(T^{-\lambda})$$
	\Ethan{Now we need some kinda standard fidelity argument to show that the measurement results will be close to as if the ground state is measured.}

	The probability of measuring $t<T$ on the clock register of $\ket{\psi_{C'(x)}}$ is $\frac{T}{T+T^{\lambda+1}}<T^{-\lambda}$,
	so the data register has $1-\varepsilon$ probability to be $C(x)$ at this point.

	The soundness errors incurred at each step is at most $O(T^{-\lambda})$, so the conclusion follows.
\end{prf}

