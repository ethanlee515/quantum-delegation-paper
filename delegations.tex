\section{Quantum Delegation Schemes}

Here we describe schemes that allows a classical client to interact with a quantum server, and various properties that such schemes can have.

\begin{definition}
	A \emph{quantum delegation protocol} is an polynomial-round interactive protocol between a BPP verifier and a BQP prover. The BPP verifier has an arbitrary quantum circuit with classical input that it needs to evaluate and measure the result of by interacting with the prover.
\end{definition}

\begin{definition}
	A quantum delegation protocol has \emph{long output} if the output can have multiple bits. In this case, the verifier should end with a measurement result of the output register.
\end{definition}

\begin{definition}
	A quantum delegation protocol is \emph{secure} if a malicious prover cannot learn the input or the output of the computation. In particular, security notions such as \emph{CPA} can be applied here. (TODO Just define CPA right away since there are no other security notions around here)
\end{definition}

\begin{definition}
	A quantum delegation protocol is \emph{$\delta,\epsilon$-verifiable} if the verifier's output distribution is within $\delta$ of the true distribution whenever it interacts with a prover whose probability to be accepted is greater than $\epsilon$.
\end{definition}

In \cite{mahadev_delegation}, a verifiable scheme is proposed. In \cite{mahadev_qfhe}, a secure scheme is proposed. (TODO just make this CPA or whatever it actually is...) We propose a verifiable scheme with long output which can also be made secure.

TODO check where does \cite{1904.06320} fit in.

\subsection{Our contributions}

We start by constructing a verifiable long output scheme. Then, we encrypt the scheme as given in \cite{mahadev_delegation} to achieve security. TODO Does it actually work though? There are multiple steps in the scheme... We encrypting each step separately or what?
