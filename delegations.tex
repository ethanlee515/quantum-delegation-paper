\section{Quantum Delegation Schemes}

Here we describe schemes that allows a classical client to interact with a quantum server, and various properties that such schemes can have.

\begin{definition}
	A \emph{quantum delegation protocol} is an polynomial-round interactive protocol between a BPP verifier and a BQP prover. The BPP verifier has an arbitrary quantum circuit with classical input that it needs to evaluate and measure the result of by interacting with the prover.
\end{definition}

\begin{definition}
	A quantum delegation protocol has \emph{long output} if the output register contains multiple qubits.
\end{definition}

\begin{definition}
	A quantum delegation protocol is \emph{CPA-secure} if both the input and output of the circuit are CPA-secure as ciphertexts. In this case, the scheme should be seen as a homomorphic encryption scheme.
\end{definition}

\begin{definition}
	A quantum delegation protocol is \emph{$\delta(\epsilon)$-verifiable} if the verifier's output distribution is within $\epsilon$ of the true distribution whenever it interacts with a prover whose probability to be accepted is greater than $\delta$.
\end{definition}

In \cite{mahadev_delegation}, a verifiable scheme is proposed. In \cite{mahadev_qfhe}, a secure scheme is proposed. \Ethan{just make this CPA or whatever it actually is...}

\Ethan{check where does \cite{1904.06320} fit in.}

\subsection{Our contributions}

We propose a verifiable scheme with long output which can also be made secure. We start by constructing a verifiable long output scheme. Then, we use \cite{mahadev_delegation} to achieve security. That is, we the techniques in \cite{mahadev_delegation} to encrypt the circuit and the input. We also tack on the evaluation key as extra input, so the entire computation can still be encoded as a single circuit.
