\section{The History XZ Local Hamiltonian}
\label{sec:Hamiltonian}

In this section, we give a reduction from $\SampBQP$ to a local Hamiltonian problem. \hannote{put proofs back. Adiabatic ref?}

\Ethan{Adiabatic and spectral gap ref = \cite{adiabatic}}

%Here we lay our theoretical foundation by describing a reduction from $\SampBQP$ that we'll use throughout this paper. \hannote{?} Here we'll only state the theorems, and prove them later in the subsections.

First note that $\set{H, T}$ is a universal gate set.


\hannote{citation}
\begin{thm}
	Given a circuit $C$ and classical input $x$, a $\BPP$ machine can reduce it to another circuit $C'$ and classical input $x'$, where $C'$ contains only Hadamard and Toffoli gates.
\end{thm}

%We want to define and construct an instance of Local Hamiltonian that encodes the history of the computation.
 
 
 We denote the set of $5$-local tensor products of $X$ and $Z$ gates as $\mathcal{G}_{XZ}$.% and define the following that implicitly depends on $L$.

\begin{definition}
	$$\mathcal{G}_{XZ}=\set{U_0\otimes U_1\otimes\ldots\otimes U_n: U_i\in\set{I,X,Z}, \abs{\set{i: U_i\ne I}}\leq 5}$$
\end{definition}

%We fix $L=5$ for this paper.For any $L\in\bbN$, 

\hannote{something?}

\begin{thm}
	Let $C$ be a circuit consisting of $T$ Hadamard or Toffoli gates. For any input string $x$,  there exists a local Hamiltonian $H_{C(x)}$  with the following properties:
	\begin{itemize}
		\item $H_{C(x)}$ can be written as $\sum_{S\in\mathcal{G}_{XZ}} d_S S$ with at most $O(T)$ terms
		\item $H_{C(x)}$ has a unique ground state  $\ket{\psi_{C(x)}}=\sum_{t=0}^TU_t\ldots U_1\ket{x}\otimes\ket{\hat{t}}$ with eigenvalue $0$. We call the first register of $\ket{\psi_{C(x)}}$ the data register, and the second register of $\ket{\psi_{C(x)}}$ the time register.
		\item For any state $\ket\phi$ such that $\braket{\phi|\psi_{C(x)}}=0$,  $\braket{\phi|H_{C(x)}|\phi}>\frac{3}{4}$.
	\end{itemize}
	
	Furthermore, there exists a $\BPP$ machine that can construct a description of $H_{C(x)}$ in polynomial time.
\end{thm}

\iffalse
\begin{definition}
	We call the first registers of $\ket{\psi_{C(x)}}$ the data register, and the later registers of $\psi_{C(x)}$ the time register.
\end{definition}


\begin{thm}
	Given a circuit $C$ consisting of Hadamard and Toffoli gates and an input string $x$, there exists a $\BPP$ machine that can construct a description of $H_{C(x)}$ in polynomial time.
\end{thm}
\fi


\begin{thm}
	Given circuit $C$ consisting of Hadamard and Toffoli gates and an input string $x\in\set{0,1}^*$, there exists an quantum algorithm that accepts $\ket{\phi}$ with probability $\frac{1}{2}-\Omega(\frac{1}{\poly(T)})\braket{\phi|H_{C(x)}|\phi}$. In particular, it accepts the $\ket{psi_{C(x)}}$ with probability $\fot$. Furthermore, this algorithm doesn't apply any quantum gates, using only $X$ and $Z$ measurements.
\end{thm}

Lastly, before we present proofs for the above theorems, we make an observation.

\begin{observation}
	\label{idpadding}
	Given a circuit $C$ and an input string $x$, one can first pad $C$ with identity matrices at the end, doubling its number of gates and obtaining another circuit $C'$. One can then measure the time register of $\ket{\psi_{C'(x)}}$ and obtain with probability $\frac{1}{2}$ the state $C(x)$.
\end{observation}

\subsection{Circuits of Toffoli and Hadamard gates}

\hannote{throw this into appendix?}
 Here we restate the result from from \cite{quant-ph/0301040} showing that $\set{H, \Lambda^2(X)}$ is a universal gate set. Because of this result, we cam assume without loss of generality that all quantum circuits consists of only Toffoli and Hadamard gates.
 
 %After this subsection, we'll be able to assume without loss of generality that all quantum circuits consists of only Toffoli and Hadamard gates. %After doing that, we make a further observation that we'll use later.

\hannote{theorem?} We know that Hadamard gate and controlled phase gate is universal from \cite{kitaev_1997}
$$H=\frac{1}{\sqrt{2}}\begin{pmatrix}1&1\\1&-1\end{pmatrix}$$
	$$\Lambda(P(i))=\begin{pmatrix}1&0&0&0\\0&1&0&0\\0&0&1&0\\0&0&0&i\end{pmatrix}$$

Now we present the construction from \cite{quant-ph/0301040}.

\begin{thm}
	Let $C$ be a circuit that:
	\begin{itemize}
		\item consists of either $H$ or $\Lambda(P(i))$, and the number of gates is $T$.\hannote{confusing?}
		\item uses $n$ qubits
	\end{itemize}
	Then a classical machine given $C$ can compute a circuit $C'$ that:
	\begin{itemize}
		\item consists of at most $4T$ gates, each either $H$ or $\Lambda^2(X)$ \hannote{don't like this notation}.
		\item uses $n+1$ qubits
		\item Let $x\in\set{0,1}^n$. Let $x|| 0\in\set{0,1}^{n+1}$ be $x$ concatenated with $0$.
		The measurement result of a tensor product of $X$ and $Z$ operators on the first $n$ qubits of $C'(x||0)$ has the same distribution as that on $C(x)$.
	\end{itemize}
\end{thm}

\begin{proof}
	Consider the transform on quantum states
	$$\mathcal{F}(\ket{\phi})=(\Re\ket{\phi})\otimes\ket{0}+(\Im\ket{\phi})\otimes\ket{1}$$
	where $\Re$ and $\Im$ denote real and imaginary parts respectively.

	This transform commutes with Hadamard gates on the respective qubit. On the other hand, exchanging $\mathcal{F}$ with a controlled phase gate turn it into a combination of Hadamard and Toffoli gates. Mathematically,
	$$\mathcal{F}\circ H_s=H_s\circ\mathcal{F}$$
	$$\mathcal{F}\circ\Lambda_f(P(i)_s)=\Lambda^2_{f,s}(X_0Z_0)\circ\mathcal{F}=\Lambda^2_{f,s}(X_0)H_0\Lambda^2_{f,s}(X_0)H_0\circ\mathcal{F}$$

	We construct $C'$ so that $\mathcal{F}\circ C=C'\circ\mathcal{F}$ by following the computation above. It satisfies the required properties by construction.
	\begin{itemize}
		\item $C'$ uses only Hadamard and Toffoli gates.
		\item Exchanging $\mathcal{F}$ with $H$ doesn't change the circuit size. Exchanging $\mathcal{F}$ with controlled phase gate turns it into $4$ gates. So the final result is at most $4T$ gates.
		\item When $x$ is classical, $\mathcal{F}(x)=x||0$. So $F\circ C(x)=C'(x||0)$. It is simple \hannote{need a little more work?}to verify that $\mathcal{F}$ preserves $X$ and $Z$ measurement results on the first $n$ qubits.
	\end{itemize}
\end{proof}

Note that our new gate set can be written as linear combinations of $X$ and $Z$ matrices.

\begin{thm}
	$H,\Lambda^2(X)\in\spn\mathcal{G}_{XZ}$ with $O(1)$ nonzero components.
\end{thm}

\begin{proof}
	$$H=\frac{1}{\sqrt{2}}(X+Z)$$
	$$\Lambda^2_{1,2}(X_3)=\ket{11}\bra{11}\otimes X+(I-\ket{11}\bra{11})\otimes I$$
	$$\ket{11}\bra{11}=\frac{1}{4}(I\otimes I+Z\otimes Z-I\otimes Z-Z\otimes I)$$
\end{proof}

\subsection{Reducing quantum circuit to local Hamiltonian}

In this section, we show how to reduce a $\SampBQP$ circuit to a  local-$XZ$-Hamiltonian.

\begin{thm}
	Let $C=U_T\ldots U_1$ be a circuit consisting of Hadamard and Toffoli gates, and let $x\in\set{0, 1}^n$. Then there exists a $\BPP$  machine which generates the  description of a $H_{C(x)}\in\spn\mathcal{G}_{XZ}$ such that:
	\begin{itemize}
		\item its unique ground state is $$\ket{\psi_{C(x)}}=\sum_{t=0}^TU_t...U_1\ket{x}\otimes\ket{\hat{t}}$$
		\item $\braket{\psi_{C(x)}|H_{C(x)}|\psi_{C(x)}}=0$
		\item $\forall\ket\phi,\braket{\phi|\psi_{C(x)}}=0\Rightarrow\braket{\phi|H_{C(x)}|\phi}\geq\frac{3}{4}$
		\item $H_{C(x)}$ has $O(T)$ nonzero terms whose coefficients are  at most $O(T^3)$.
	\end{itemize}
\end{thm}

\begin{proof}
	We need to ensure the excited states of $H_{C(x)}$ have high eigenvalues. The base construction comes from \cite{kitaev2002classical}. The simplification to $\spn\mathcal{G}_{XZ}$ is taken from \cite{PhysRevA.78.012352}.

	Let $x_i$ denote the $i$-th bit of $x$, and let $n$ be the number of qubits in $C$.

	First, we ensure that the invalid clock states have high eigenvalues by applying the following Hamiltonian to the time register.
	$$H_{clock}=\sum_{t=1}^{T-1}\ket{01}\bra{01}_{t,t+1}$$
	As a sum of projections, clearly $H_{clock}\geq 0$. We shall also show that $H_{clock}\in\spn\mathcal{G}_{XZ}$.
	$$H_{clock}=\frac{1}{4}(Z_1 - Z_T) + \frac{1}{4}\sum_{t=1}^{T-1}(I-Z_tZ_{t+1}) $$
	This can be checked by fixing the first and last qubit, then doing induction on the number of switches.

	Then, we ensure that the initial condition is set up correctly.
	$$H_{in}=\sum_{i=1}^n(I-\ket{x_i}\bra{x_i})\otimes\ket{0}\bra{0}_1$$
	The kernel of this is precisely where everything is set up consistently with $\phi$ in time step $t=0$. Furthermore, $H_{in}\in\spn\mathcal{G}_{XZ}$
	$$H_{in}=\frac{1}{4}\sum_{i=1}^n(I-(-1)^{x_i}Z_i)\otimes(I+Z_1)$$

	Then, we ensure that the gates are applied correctly.
	$$H_{prop}=\sum_{t\in T_1}H_{prop,t}$$
	$$H_{prop,t}=I\otimes\ket{\widehat{t}}\bra{\widehat{t}}
	+I\otimes\ket{\widehat{t-1}}\bra{\widehat{t-1}}
	-U_t\otimes\ket{\widehat{t}}\bra{\widehat{t-1}}
	-U_t^\dagger\otimes\ket{\widehat{t-1}}\bra{\widehat{t}}$$

	Next, we check that $H_{prop}\geq0$ by the following transform.

	$$W=\sum_{j=0}^L U_j\ldots U_1\otimes\ket{j}\bra{j}$$
	$$W^\dagger\phi=\sum_{t=0}^T\ket{x}\otimes\ket{\hat{t}}$$
	$$W^\dagger H_{prop} W=
	\begin{pmatrix}
		\frac{1}{2} & -\frac{1}{2} & & & &  \\
		-\frac{1}{2} & 1 & -\frac{1}{2} & & & \\
		& -\frac{1}{2} & 1 & \ddots & & \\
		& & \ddots & \ddots & -\frac{1}{2} & \\
		& & & -\frac{1}{2} & 1 & -\frac{1}{2} \\
		& & & & -\frac{1}{2} & \frac{1}{2}
	\end{pmatrix}$$

	According to \cite{2002quant.ph.10077A}, $H_{prop}\geq 0$ due to the above form being relevant to random walks. Furthermore, the last nonzero eigenvalue of $H_{prop}$ is at least $\frac{1}{2(T+1)^2}$.

	We can also write $H_{prop}\in\spn\mathcal{G}_{XZ}$. Note that $U^\dagger=U$, since the gates are either Hadamard or Toffoli. Additionally, $\frac{1}{2}(I-Z_{t-1})$ annihilates time steps before $t-1$. $\frac{1}{2}(I+Z_{t+1})$ similarly annihilates steps $t+1$ and after.
	$$H_{prop,t}=\frac{I}{4}\otimes(I-Z_{t-1})(I+Z_{t+1})-\frac{U}{4}\otimes(I-Z_{t-1})X_t(I+Z_{t+1})$$
	Extra care must be taken for boundary cases.
	$$H_{prop,1}=\frac{1}{2}(I+Z_2)-U_1\otimes\frac{1}{2}(X_1+X_1Z_2)$$
	$$H_{prop,T}=\frac{1}{2}(I-Z_{t-1})-U_T\otimes\frac{1}{2}(X_T-Z_{T-1}X_T)$$

	To properly combine the three Hamiltonians defined and analyze it using the projection lemma (\cref{thm:projection}), we consider the kernels of the Hamiltonians we defined.
	$$K_{clock}=\ker H_{clock}$$
	$$K_{in}=\ker H_{in}$$
	$$K_{prop}=\ker H_{prop}$$

	Clearly,
	$$K_{clock}=\set{\sum_{t=1}^T \ket{\phi_t}\otimes\ket{\hat{t}}:\ket{\phi_t}\in\cB^{\otimes n}}$$
	Let $\ket\phi\in K_{clock}\cap K_{prop}$.
	Consider some $\widetilde{t}$ such that $\braket{\widetilde{t}|\phi}\ne0$.
	As $\ket\phi\in K_{prop}$, we can then do induction using $\widetilde{t}$ as base case to get the relations between the $\ket{\phi_t}$s and conclude that $\ket\phi$ must have the form:
	$$\ket\phi=\sum_{t=0}^TU_t\ldots U_1\ket{y}\otimes\ket{\hat{t}}$$
	$$\Rightarrow K_{clock}\cap K_{prop}=\set{\sum_{t=0}^TU_t\ldots U_1\ket{y}\otimes\ket{\hat{t}}: \ket{y}\in\mathcal{B}^{\otimes n}}$$

	As a direct result, we get
	$$K_{clock}\cap K_{in}\cap K_{prop}=\spn\set{\ket{\psi_{C(x)}}}$$

We now consider the space outside the desired state.
$$S=(\spn\set{\ket{\psi_{C(x)}}})^\bot$$
$$H_{clock}\big|_S,H_{in}\big|_S,H_{prop}\big|_S$$
To combine the Hamiltonians, we apply the projection lemma twice.
$$\exists J_{clock}
=\frac{\poly\left(\norm{H_{in}\big|_S}\right)}{\lambda\left(H_{clock}\big|_{S\cap K^\bot_{clock}}\right)}
=O(n)=O(T)$$
$$\lambda(H_{in}\big|_S+J_{clock}H_{clock}\big|_S)\geq
\lambda(H_{in}\big|_{S\cap K_{clock}})-\frac{1}{8}$$
$$\exists J_{prop}=\frac{\poly\left(\norm{H_{in}\big|_S+J_{clock}H_{clock}\big|_S}\right)}{\lambda\left(H_{prop}\big|_{S\cap K^\bot_{prop}}\right)}
=\frac{O(n+T)}{\Omega(T^{-2})}=O(T^3)$$
$$\lambda(H_{in}\big|_S+J_{clock}H_{clock}\big|_S+J_{prop}H_{prop}\big|_S)\geq
\lambda(H_{in}\big|_{S\cap K_{clock}\cap K_{prop}})-\frac{1}{4}$$
\begin{align*}
	S\cap K_{clock}\cap K_{prop}&=S\cap\set{\sum_{t=0}^TU_t\ldots U_1\ket{y}\otimes\ket{\hat{t}}|\ket{y}\in\mathcal{B}^{\otimes n}}\\
	&=\set{\sum_{t=0}^TU_t\ldots U_1\ket{y}\otimes\ket{\hat{t}}:\braket{\psi_{C(x)}|y}=0}
\end{align*}
$$\Rightarrow\lambda((H_{in}+J_{clock}H_{clock}+J_{prop}H_{prop})\big|_S)\geq\frac{3}{4}$$
So we set $H_{C(x)}=H_{in}+J_{clock}H_{clock}+J_{prop}H_{prop}$, which satisfies the required properties by construction.
\end{proof}

\subsection{Checking the ground state of the local Hamiltonian}

%Now we put the above results in term of measurement outputs.\hannote{???}

We present an algorithm that checks whether a given state is the ground state of some fixed $H_{C(x)}$ following~\cite{PhysRevA.93.022326}.

\begin{algorithm}
	\caption{Check for ground state}
	\label{AlgGroundStateCheck}
		Let $H=\sum_{S\in\mathcal{G}_{XZ}} d_S S$.
		Let $\ket\phi$ be the potential ground state to check
		\begin{itemize}
			\item Set $D = \sum_{S\in\mathcal{G}_{XZ}}|d_S|$
			\item Set $p_S = \frac{|d_S|}{D}$
			\item Sample $\widetilde{S}$ from $\mathcal{G}_{XZ}$, weighted by $p_S$.
			\item Measure $\ket\phi$ in the $\widetilde{S}$ basis, recording the result as $\lambda_{\widetilde{S}}$.
			\item If $\sgn(d_{\widetilde{S}})\lambda_{\widetilde{S}}=-1$, accept. Otherwise, reject.
		\end{itemize}
\end{algorithm}

\begin{thm}\label{ThmXZCheck}
	Let $H$ have $O(\poly(T))$ nonzero components whose coefficients are each at most $O(\poly(T))$.
	Then \autoref{AlgGroundStateCheck} accepts $\ket\phi$ with probability $\frac{1}{2}-\Omega(\frac{1}{\poly(T)})\braket{\phi|H|\phi}$.
\end{thm}
\begin{proof}

	Here we follow \cite{PhysRevA.93.022326}.
	\begin{align*}
		\frac{1}{D}\braket{\phi|H|\phi}&=\sum_{S\in\mathcal{G}_{XZ}} p_S\sgn(d_S)\braket{\phi|S|\phi}\\
		&=\sum_{S\in\mathcal{G}_{XZ}} p_S\sgn(d_S)\E[\lambda_S]\\
		&=\E_{\widetilde{S}}[\sgn(d_{\widetilde{S}})\E[\lambda_{\widetilde{S}}]]\\
		&=\E_{\widetilde{S}}[\sgn(d_{\widetilde{S}})\lambda_{\widetilde{S}}]
	\end{align*}

	Note that $\sgn(d_{\widetilde{S}})\lambda_{\widetilde{S}}=\pm1$. Let $p$ be the probability that $\sgn(d_{\widetilde{S}})\lambda_{\widetilde{S}}=-1$.
	$$\Rightarrow \frac{1}{D}\braket{\phi|H|\phi}=\E_{\widetilde{S}}[\sgn(d_{\widetilde{S}})\lambda_{\widetilde{S}}]=-p+(1-p)$$
	\begin{align*}
		\Rightarrow p&=\frac{1}{2}-\frac{1}{2D}\braket{\phi|H|\phi}\\
		&=\frac{1}{2}-\Omega\left(\frac{1}{\poly(T)}\right)\braket{\phi|H|\phi}
	\end{align*}

\end{proof}
