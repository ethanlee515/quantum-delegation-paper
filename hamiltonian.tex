\section{Construction of the $\QPIP_1$ Protocol for $\SampBQP$}
\label{sec:sampbqp}

\subsection{Reducing $\SampBQP$ to the X-Z Local Hamiltonian} \label{sec:LHXZ}

Recall the definition of the history state which serves as a transcript of the circuit evolution~\cite{kitaev2002classical}:

\begin{dfn}[History-state]
    \label{dfn:groundstate}    
    Given any quantum circuit $C=U_T\ldots U_1$ of $T$ elementary gates and input $x\in\{0,1\}^n$, the \emph{history}-state $\histpsi{C(x)}$ is defined by
    \begin{equation}
        \histpsi{C(x)} \equiv \frac{1}{\sqrt{T}}\sum_{t=0}^{T-1}U_t\ldots U_1\ket{x,0}\otimes\ket{\hat{t}},
    \end{equation}
    where the first register of $n$-qubit refers to the input, the second of $m$-qubit refers to the work space ($\mathrm{poly}(n)$, w.l.o.g, $\leq T$ size) which is initialized to $\ket{0}$, and the last refers to the clock space which encodes the time information. Note that $\hat{t}$ could be some representation of $t$=0,..., $T-1$.
\end{dfn}

We define the X-Z local Hamiltonian of interest as follows:

\begin{dfn}[$k$-local X-Z Hamiltonian] For any $n$-qubit system, the set of $k$-local X-Z terms, denoted by $\LHXZ{k}$, contains Hermitian matrices that apply non-trivially on at most $k$ qubits as a product of Pauli $X$ and $Z$ terms. Namely,
\begin{equation}
  \LHXZ{k} = \left\{h_1 \otimes h_2 \otimes \ldots \otimes h_n: \forall i \in [n], h_i \in \{I, X, Z\}, \text{and}, \abs{\set{i: h_i=X \text{ or }Z}} \leq k. \right \}.
\end{equation}
A $k$-local X-Z Hamiltonian $H$ is a linear combination of terms from $\LHXZ{k}$. Namely,
\begin{equation}
  H = \sum_i \alpha_{i} H_i,  \quad \forall i, \alpha_i \in R,  H_i \in \LHXZ{k}.
\end{equation}
\end{dfn}

As our starting point, we will include the existing result of constructing X-Z local Hamitonians for general $\BQP$ computation.
First, we note the fact Toffoli and Hadamard gates form a universal gate set for quantum computation~\cite{Shi03, quant-ph/0301040}.
It is easy to see that both Toffoli and Hadamard gates can be represented as linear combinations of terms from $\LHXZ{}$. For example,
 \begin{equation}
     \mathrm{(Hadamard)} \quad H \equiv \frac{1}{\sqrt{2}} \begin{pmatrix}1&1\\1&-1\end{pmatrix} = \frac{1}{\sqrt{2}}\left (X+Z\right).
 \end{equation}
 The Toffoli gate maps bits $(a,b,c)$ to $(a,b, c \oplus (a \text{ and } b))$, which can be decomposed as $\ket{11}\bra{11}\otimes X+(I-\ket{11}\bra{11})\otimes I$ where $\ket{11}\bra{11}=\frac{1}{4}(I\otimes I+Z\otimes Z-I\otimes Z-Z\otimes I)$.
 
We will follow the unary clock register design from Kitaev's original 5-local Hamiltonian construction~\cite{kitaev2002classical}. Namely, valid unary clock states ($T$-qubit) are $\ket{00\ldots0}$, $\ket{10\dots0}$, $\ket{110\ldots0}$, etc, which span the ground energy space of the following local Hamiltonian:
\begin{equation}
    \Hclock= \sum_{t=1}^{T-1}\proj{01}_{t,t+1},
\end{equation}
where $\proj{01}_{t,t+1}$ stands for a projection on the $t$th and $(t+1)$th qubit in the clock register.
It is observed in~\cite{PhysRevA.78.012352} that $\Hclock$ can be reformulated as a linear combination of terms from $\LHXZ{}$ as follows,
\begin{equation} \label{eqn:Hclock}
   \Hclock= \frac{1}{4}(Z_1 - Z_T) + \frac{1}{4}\sum_{t=1}^{T-1}(I-Z_t\otimes Z_{t+1}),
\end{equation}
where $Z_t$ refers to Pauli $Z$ operated on the $t$th qubit in the clock register.

One can achieve so similarly for $\Hin$ and $\Hprop$. For the input condition, we want to make sure $x=(x_1, \ldots, x_n) \in \{0,1\}^n$ is in the input space and the workspace is initialized to $\ket{0}$ for all qubits at the time $0$. Namely, one can set $\Hin$ to be
\begin{equation}
    \Hin = \sum_{i=1}^n(I- \proj{x_i}_i)\otimes \proj{0}_1 + \sum_{i=1}^m \proj{1}_{n+i} \otimes \proj{0}_1,
\end{equation}
where the last part $\proj{0}_1$ applies on to the first qubit in the clock register. One can rewrite $\Hin$ as
\begin{equation}\label{eqn:Hin}
 \Hin=\frac{1}{4}\sum_{i=1}^n(I-(-1)^{x_i}Z_i)\otimes(I+Z_1) + \frac{1}{4} \sum_{i=1}^m (I - Z_{n+i}) \otimes (I + Z_1).
\end{equation}

For the propagation of the quantum state through the circuit, one uses the $\Hprop$ as follows:
\begin{equation} \label{eqn:Hprop}
    \Hprop=\sum_{t=1}^T \Hprop^t,
\end{equation}
where
\begin{equation}
    \Hprop^t=\frac{1}{2}I\otimes\proj{\widehat{t}}
    +\frac{1}{2}I\otimes\proj{\widehat{t-1}}
    -\frac{1}{2} U_t\otimes\ket{\widehat{t}}\bra{\widehat{t-1}}
    -\frac{1}{2}U_t^\dagger\otimes\ket{\widehat{t-1}}\bra{\widehat{t}}.
\end{equation}
Note that $\ket{\widehat{t}}\bra{\widehat{t-1}}=\ket{110}\bra{100}_{(t-1,t,t+1)}$ and similarly for $\ket{\widehat{t-1}}\bra{\widehat{t}}$.
Note that $U_t^\dagger=U_t$, since our gates are either Hadamard or Toffoli. It is observed in~\cite{PhysRevA.78.012352} that
\begin{equation}
   \Hprop^t=\frac{I}{4}\otimes(I-Z_{t-1})\otimes (I+Z_{t+1})-\frac{U_t}{4}\otimes(I-Z_{t-1})\otimes X_t\otimes (I+Z_{t+1}), \forall t=2, \ldots, T-1,
\end{equation}
and
\begin{eqnarray}
  \Hprop^1 &= & \frac{1}{2}(I+Z_2)-U_1\otimes\frac{1}{2}(X_1+X_1\otimes Z_2), \\
  \Hprop^T &= & \frac{1}{2}(I-Z_{t-1})-U_T\otimes\frac{1}{2}(X_T-Z_{T-1}\otimes X_T).
\end{eqnarray}
Combining with the fact that each $U_t$ can be written as a linear combination of terms from $\LHXZ{}$, we conclude that $\Hclock$, $\Hin$, $\Hprop$ are 6-local X-Z Hamiltonian.

We will employ the perturbation technique to amplify the spectral gap of $\Hclock + \Hin + \Hprop$.
Let $\ground{H}$ denote the ground energy of any Hamiltonian $H$.
The projection lemma from \cite{kempe_kitaev_regev_2006} approximates $\ground{H_1 + H_2}$ in terms of $\ground{H_1\big|_{\ker H_2}}$, where $\ker H_2$ denotes the \emph{kernel} space of $H_2$.

\begin{lem}[Lemma 1 in \cite{kempe_kitaev_regev_2006}]
    \label{thm:proj1}
    Let $H=H_1+H_2$ be the sum of two Hamiltonians operating on Hilbert space $\cH=\cS+\cS^\bot$.
    The Hamiltonian $H_2$ is such that $\cS$ is a zero eigenspace and the eigenvectors in $\cS^\bot$ have eigenvalues at least $J>2\norm{H_1}$. Then,
    $$\lambda\left(H_1\big|_\cS\right)-\frac{\norm{H_1}^2}{J-2\norm{H_1}}\leq\lambda(H)\leq\lambda\left(H_1\big|_\cS\right).$$
\end{lem}

We will use the following simple reformulation instead.

\begin{lem}
    \label{lem:projection}
    Let $H_1, H_2$ be local Hamiltonians where $H_2\geq0$. Let $K=\ker H_2$ and
    $$J=\frac{8\norm{H_1}^2+ 2\norm{H_1}}{\lambda\left(H_2\big|_{K^\bot}\right)},$$
    then we have
    $$\lambda(H_1+JH_2)\geq\lambda\left(H_1\big|_K\right)-\frac{1}{8}.$$
\end{lem}
\begin{proof}
    Apply \Cref{thm:proj1} to $H=H_1+JH_2$. Note that the least non zero eigenvalue of $JH_2$ is greater than $2\norm{H_1}$.
\end{proof}

\begin{thm}
    \label{thm:LHReduction}
    Given any quantum circuit $C = U_T \ldots U_1$ of  $T$ elementary gates and input $x \in \{0,1\}^n$, one can construct a 6-local X-Z Hamiltonian $H_{C(x)}$ in polynomial time such that
    \begin{enumerate}
        \item[(1)] $H_{C(x)} = \sum_i \alpha_i H_i$ where
        each $H_i \in \LHXZ{6}$ and $|\alpha_i| \in O(T^9)$. Moreover, there are at most $O(T)$ non-zero terms.
        \item[(2)] $H_{C(x)}$ has $\histpsi{C(x)}$ as the unique ground state with eigenvalue $0$ and has a spectral gap at least $\frac{3}{4}$. Namely,  for any state $\ket\phi$ that is orthogonal to $\histpsi{C(x)}$, we have $\braket{\phi|H_{C(x)}|\phi}\geq \frac{3}{4}$.
    \end{enumerate}
\end{thm}

\begin{proof}
We will use the above construction $\Hclock$ (\cref{eqn:Hclock}), $\Hin$ (\cref{eqn:Hin}), $\Hprop$ (\cref{eqn:Hprop}) as our starting point, which are already 6-local X-Z Hamiltonian constructable in polynomial time. However, $H_{\mathrm{old}}=\Hin + \Hclock + \Hprop$ does not have the desired spectral gap. To that end, our construction will be a weighted sum of $\Hin$, $\Hclock$, and $\Hprop$ as follows,
\begin{equation}
    H_{\mathrm{new}}= \Hin + \Jclock \Hclock + \Jprop \Hprop,
\end{equation}
where $\Jclock$ and $\Jprop$ will be obtained using \Cref{lem:projection}.

Let $\Kin=\ker \Hin$, $\Kclock=\ker \Hclock$, and $\Kprop=\ker \Hprop$. It is known from e.g.,~\cite{kitaev2002classical}, that
\[
   \Kin \cap \Kclock \cap \Kprop = \spn\set{\histpsi{C(x)}}.
\]
Thus $\histpsi{C(x)}$ remains in the ground space of $H_{\mathrm{new}}$. Let $S$ denote its orthogonal space. Namely, $S=(\spn\set{\ket{\psi_{C(x)}}})^\bot$.
Denote by $\Hin\big|_S$ the restriction of $\Hin$ on space $S$ and similarly for others.

Consider $\Hin + \Jclock\Hclock$ first. According to \cref{lem:projection}, by choosing
\[
  \Jclock = \frac{8\norm{\Hin\big|_S}^2 + 2 \norm{\Hin\big|_S}}{\ground{\Hclock\big|_{S\cap \Kclock^\bot}}} = O(T^2),
\]
where we use the fact $\norm{\Hin|_S}\leq T$ and $\ground{\Hclock\big|_{S\cap \Kclock^\bot}}\geq \ground{\Hclock\big|_{\Kclock^\bot}}=1$,
we have
\[
 \ground{\Hin\big|_S + \Jclock\Hclock\big|_S}\geq \ground{\Hin\big|_{S\cap \Kclock}} - \frac{1}{8}.
\]
Consider further adding $\Jprop\Hprop$ term.  By choosing
\[
 \Jprop= \frac{O(\norm{\Hin\big|_S + \Jclock\Hclock\big|_S}^2)}{\ground{\Hprop\big|_{S \cap \Kprop^\bot}}}= O(T^8),
\]
where we use the fact $\ground{\Hprop\big|_{S\cap \Kprop^\bot}}\geq \ground{\Hprop\big|_{\Kprop^\bot}}=\Omega(T^{-2})$~\cite{kitaev2002classical}, we have  
\[
 \ground{\Hin\big|_S + \Jclock\Hclock\big|_S + \Jprop\Hprop\big|_S} \geq \ground{\Hin\big|_{S \cap \Kclock \cap \Kprop}} - \frac{1}{4}.
\]
A simple observation here is that $S \cap \Kclock \cap \Kprop$ is the span of history states with different inputs or different initialization of the work space. Namely, $\ground{\Hin\big|_{S \cap \Kclock \cap \Kprop}} \geq 1$. Thus,
\[
  \ground{(\Hin + \Jclock\Hclock + \Jprop\Hprop)\big|_S} \geq 1- \frac{1}{4}=\frac{3}{4}.
\]
Given that $\histpsi{C(x)}$ is the ground state of $H_{\mathrm{new}}$ with eigenvalue 0, and any orthogonal state to $\histpsi{C(x)}$ has eigenvalue at least $\ground{H_{\mathrm{new}}}\geq 3/4$, the spectral gap of $H_{\mathrm{new}}$ at at least $3/4$.

Note that $H_{\mathrm{new}}$ is a 6-local X-Z Hamiltonian by construction. It suffices to check the bound of $\abs{\alpha_i}$ and the number of terms. The former is one more than the order of $\Jprop$ since each $\Hprop^t$ contributes $\frac{1}{4}$ to the $I$ term, creating an extra factor of $T$. The latter is by counting the number of terms from $\Hin, \Hclock, \Hprop$, each of which is bounded by $O(T)$.
\end{proof}

\noindent \textbf{Remark.} We believe that the specific parameter dependence above can be tightened by a more careful analysis. However, as our focus is on the feasibility, we keep the above slightly loose analysis which might be more intuitive. 


\subsection{Delegation Protocol for $\QPIP_1$ client}
\label{sec:qpip1}
In this subsection, we construct a one-message $\QPIP_1$ delegation protocol for $\SampBQP$. By definition of $\QPIP_1$, we assume the client has limited quantum power, e.g., performing single qubit $X$ or $Z$ measurement one by one.
Intuitively, one should expect the one-message from the server to the client is something like the history state so that the client can measure to sample.  

At a high-level, the design of such protocols should consist of at least two components: (1) the first component should test whether the message is indeed a valid history state; (2) the second component should simulate the last step of any $\SampBQP$ computation by measuring the final state in the computational basis.

Our construction of X-Z local Hamiltonian $H$ from \Cref{thm:LHReduction} will help serve the first purpose.
In particular, we adopt a variant of the energy verification protocol for local Hamiltonian (e.g., ~\cite{mf16, PhysRevA.93.022326}) to certify the energy of $H$ with only $X$ or $Z$ measurements.
Moreover, because of the large spectral gap, when the energy is small, the underlying state must also be close to the history state.
Precisely, consider the following protocol $\cVGS$:

\begin{protocol}{Energy verification for X-Z local Hamiltonian $\cVGS$} \label{AlgGroundStateCheck}
Given a $k$-local X-Z Hamiltonian
$H=\sum_i \alpha_{i} H_i$ (i.e., $\forall i, H_i \in \LHXZ{k}$) and any state $\ket{\phi}$.
%Let $\ket\phi$ be the potential ground state to check.

\begin{itemize}
\item Let $p_i= \abs{\alpha_i}/\sum_i \abs{\alpha_i}$ for each $i$. Sample $i^*$ according to $p_{i^*}$.
\item Pick $H_{i^*}$ which acts non-trivially on at most $k$ qubits of $\ket{\phi}$. Measure the corresponding single-qubit Pauli X or Z operator.
Record the list of the results $x_j=\pm 1$ for $j=1, \ldots k$.
\item Let $r=x_1x_2\cdots x_k$. The protocol \emph{accepts} if $r$ and $\alpha_{i^*}$ have different signs, i.e., $\sgn(\alpha_{i^*})r=-1$. Otherwise, the protocol \emph{rejects}.
\end{itemize}
\end{protocol}
% In this subsection, we prove \cref{ThmXZCheck}.
% That is, we present and analyze an algorithm that checks whether a given state is the ground state of some fixed $H_{C(x)}$, following~\cite{PhysRevA.93.022326}.

\begin{lem}[\cite{PhysRevA.93.022326}]
    \label{thm:HamCheck}
    For any $k$-local X-Z Hamiltonian $H=\sum_i \alpha_{i} H_i$ and any state $\ket{\phi}$,
    the protocol $\cVGS$ in Protocol~\ref{AlgGroundStateCheck} accepts with
    probability
\begin{equation}
 \mathrm{Prob}[ \cVGS \text{ accepts } \ket{\psi}] = \frac{1}{2} - \frac{1}{2 \sum_i \abs{\alpha_i}}\braket{\phi|H|\phi}.
\end{equation}
\end{lem}

\begin{theorem} \label{thm:HamCheckClose}
Given any quantum circuit $C$ and input $x$, consider using $H_{C(x)}$ from \Cref{thm:LHReduction} in Protocol~\ref{AlgGroundStateCheck} ($\cVGS$).
For any state $\rho$, and $0< \epsilon < 1$, if $\cVGS$ accepts $\rho$ with probability
\[
 \mathrm{Prob}[\cVGS \text{ accepts } \rho] \geq \frac{1}{2} - \frac{\epsilon}{2 \sum_i \abs{\alpha_i}},
\]
then the trace distance between $\rho$ and $\histpsi{C(x)}$ is at most $\frac{2}{\sqrt{3}}\sqrt{\epsilon}$.
\end{theorem}

\begin{prf} Consider the pure state case $\rho=\proj{\phi}$ first. By \Cref{thm:HamCheck} and our assumption, we have $\braket{\phi|H|\phi} \leq \epsilon$.
Decompose $\ket{\phi}= \alpha \histpsi{C(x)} + \beta \histpsi{C(x)}^\bot$. Note that $\histpsi{C(x)}$ is an eigenvector $H_{C(x)}$ of eigenvalue 0 and all other eigenvalues are at least $3/4$. Thus, we have $\abs{\alpha}^2 \geq 1-\frac{4}{3}\epsilon$. Thus,
\[
   \norm{\proj{\psi^{\mathrm{hist}}_{C(x)}}- \proj{\phi}}_{\tr} = \sqrt{1- |\braket{\psi^{\mathrm{hist}}_{C(x)}|\phi}|^2}
   \leq \frac{2}{\sqrt{3}}\sqrt{\epsilon}.
\]
For any mixed state $\rho=\sum_i p_i \proj{\phi_i}$, by the triangle inequality, we have
\[
    \norm{\proj{\psi^{\mathrm{hist}}_{C(x)}}- \rho}_{\tr} \leq \sum_i p_i    \norm{\proj{\psi^{\mathrm{hist}}_{C(x)}}- \proj{\phi_i}}_{\tr} \leq \sum_i p_i \frac{2}{\sqrt{3}} \sqrt{\epsilon} =\frac{2}{\sqrt{3}}\sqrt{\epsilon}.
\]
\end{prf}

To serve the second purpose, one needs to combine the test and the output on multiple copies of the history states, where we construct the following cut-and-choose protocol.  
The challenge comes from the fact that a cheating prover might send something rather than copies of the history state.
In particular, the prover can entangle between different copies in order to cheat.
In the case of certifying a $\BQP$ computation, the goal is to verify the ground energy of any local Hamiltonian.
A cheating strategy with potential entanglement won't create any witness state with an energy lower than the actual ground energy.
Thus, this kind of attack won't work for $\BQP$ computation.

However, in the context of $\SampBQP$, one needs to certify the ground energy (known to be zero in this case) and to output a good copy.
While the statistical test as before can be used to certify the ground energy,
it has less control on the shape of the output copy.
In fact, the prover can always prepare a bad copy along with many good copies as a plain attack.
This attack will succeed when the bad copy is chosen to output, the probability of which is non-negligible in terms of the total number of copies assuming some symmetry of the protocol.
The potential entanglement among different copies could further complicate the analysis.
We employ the quantum \emph{de Finetti}'s theorem to address this technical challenge.
Specifically, given any permutation-invariant $k$-register state, it is known that the reduced state on many subsets of $k$-register will be close to a separable state.
This helps establish some sort of independence between different copies in the analysis.
To serve our purpose, we adopt the following version of quantum de Finetti's theorem from~\cite{Brandao2017} where the error depends nicely  on the number of qubits, rather than the dimension of quantum systems.

\begin{thm}[\cite{Brandao2017}]
    \label{deFinetti}
    Let $\rho^{A_1\ldots A_k}$ be a permutation-invariant state on registers $A_1,\ldots,A_k$ where each register contains $s$ qubits.
    For any $0\leq l\leq k$,  there exists states $\set{\rho_i}$ and $\set{p_i}\subset\bbR$ such that
    $$\max_{\Lambda_1,\ldots,\Lambda_l}
    \norm{(\Lambda_1\otimes\ldots\otimes\Lambda_l)\left(\rho^{A_1\ldots A_l}-\sum_ip_i\rho_i^{A_1}\otimes\ldots\otimes\rho_i^{A_l}\right)}_1
    \leq\sqrt{\frac{2l^2s}{k-l}}$$
    where $\Lambda_i$ are quantum-classical channels.
\end{thm}

\section{Delegation Protocol for Hybrid Client}

In this section, we construct a one-message $\QPIP_1$ delegation protocol for $\SampBQP$.
At a high level, it is a cut-and-choose protocol.
The server constructs multiple copies of the ground state as certificates,
then the client randomly chooses a copy to output and checks the rest.
Unfortunately we have inverse polynomial soundness errors here.

There are two main challenges to this cut-and-choose approach.
First, the client needs to reliably extract the circuit output from its corresponding ground state.
We accomplish this by padding the circuit with identity gates at the end.
By doing so, the clock register collapses to after the last non-identity gate with high probability.

The other challenge is that a cheating prover's certificates can be arbitrarily entangled,
so common techniques such as Chernoff bounds can't be applied as-is.
This wasn't a challenge for $\Piblind$ for $\BQP$ earlier because its soundness is one-sided:
we only guarantee that a client cannot accept a no-instance;
a cheating server is allowed to cause the client to reject a yes-instance.
Because of this asymmetry, a cheating prover's for $\Piblind$ optimal strategy is to maximize its acceptance probability,
which corresponds to sending $n$ identical copies of some state.
In our case with $\SampBQP$, we overcome this challenge by using de Finetti's theorem
to approximate our measurement results with that of measurement results on some unentangled copies.

\def\GS{\mathsf{GS}}
\nc{\PiGS}{\ensuremath{\Pi_\GS}}
\nc{\VGS}{\ensuremath{V_\GS}}
\nc{\PGS}{\ensuremath{P_\GS}}
\nc{\PGSstar}{\ensuremath{P_\GS^*}}
\nc{\cVGS}[1]{\ensuremath{\cV_{\GS,#1}}}
\nc{\cPGS}[1]{\ensuremath{\cP_{\GS,#1}}}

\def\Samp{\mathsf{Samp}}
\nc{\PiSamp}{\ensuremath{\Pi_\Samp}}
\nc{\VSamp}{\ensuremath{V_\Samp}}
\nc{\PSamp}{\ensuremath{P_\Samp}}
\nc{\PSampstar}{\ensuremath{P_\Samp^*}}
\nc{\cVSamp}[1]{\ensuremath{\cV_{\Samp,#1}}}
\nc{\cPSamp}[1]{\ensuremath{\cP_{\Samp,#1}}}

\def\GS{\mathsf{GS}}

\begin{protocol}{$\QPIP_1$ protocol $\PiSamp$ for $\SampBQP$}\label{ProtoQPIP1}
	Pick $\varepsilon=\poly(T^{-1})$
	Let $C'$ be $C$ padded with $\poly(\varepsilon^{-1})$ identity gates at the end.
	Pick $h=\poly(\varepsilon)$, $m=O(h^{-3})$, $M=\poly(m)$.

	\begin{enumerate}
		\item The honest prover prepares $M$ copies of $\ket{\psi_{C'(x)}}$ and sends all of it to the verifier qubit-by-qubit.
		\item The verifier privately samples $I\subset[M]$ s.t. $\abs{I}=m$, and $k\xleftarrow{\$}[M]\setminus I$.
			For $i$ from $1$ to $N$, it chooses what to do to the $i$-th copy, $\rho_i$, as follows:
		\begin{enumerate}
			\item If $i\in I$, run $z_i\leftarrow\cA_{\GS}(\rho_i)$.
			\item Otherwise, if $i=k$, measure the data register of $\rho_i$ and save the outcome as $y$.
			\item Otherwise, discard $\rho_i$.
		\end{enumerate}
			Let $Z=\sum_{i\in I} z_i$. If $Z>\frac{m}{2}-hm$ then the verifier accepts and outputs $y$. Otherwise, it rejects.
	\end{enumerate}
\end{protocol}

Note that $\VSamp$ only needs to apply $X$ and $Z$ measurements, and is classical otherwise. We now show the completeness and soundness of $\PiSamp$.

\begin{thm}
    \label{QPIP1thm}
	$\PiSamp$ has negligible completeness error and $O(T^-\lambda)$ soundness error.
\end{thm}
\begin{prf}
	For completeness, notice $z_i$ are i.i.d. Bernoulli trials with success probability $\frac{1}{2}$.
	So we can apply the Chernoff bound \cref{thm:Chernoff} with $\mu=\frac{m}{2}$, $\delta=2h$ to get
	$$\Prob{\frac{m}{2}-Z\geq hm}\leq2e^{-\frac{\mu\delta^2}{3}}=\negl(T)$$

	Now we show soundness.
	Suppose $\PSampstar$ is a cheating prover that sends some $\sigma$ to the verifier.

	We first show that de Finetti's theorem can indeed be used here to achieve independence.
	Randomly picking $m+1$ out of $M$ registers is equivalent to first applying a random permutation to get some $\sigma'$ then taking the first $m+1$ registers.
	A random permutation, in turn, is a classical mix over all possible permutations:
	$$\sigma'=\frac{1}{\abs{\Sym(M)}}\sum_{\Pi\in\Sym(M)}\Pi\sigma\Pi^\dagger$$
	It is simple to check that $\sigma'$ is permutation-invariant.
	Fix $\tilde{\Pi}\in\Sym(M)$, then
	$$\tilde{\Pi}\sigma'\tilde{\Pi}^\dagger
	=\frac{1}{\abs{\Sym(M)}}\sum_{\Pi\in\Sym(M)}\tilde{\Pi}\Pi\sigma\Pi^\dagger\tilde{\Pi}^\dagger
	=\frac{1}{\abs{\Sym(M)}}\sum_{\hat{\Pi}\in\Sym(M)}\hat{\Pi}\sigma\hat{\Pi}^\dagger
	=\sigma'$$
	where the second equality is by relabeling $\tilde{\Pi}\Pi=\hat{\Pi}$, which is allowed since $\Sym(M)$ is a group.

	Now we apply \cref{deFinetti} to approximate $\sigma'$ with a classical mix over tensors of independent states.
	That is, $\exists\rho=\sum_j w_j\rho_j^{\otimes m+1}$ such that:
	$$\max_{\Lambda_i}\norm{\Lambda_1\otimes\ldots\otimes\Lambda_{m+1}(\sigma'-\rho)}=O(\varepsilon)$$

	Now we separate the $\rho_j$ into two categories: $J=\set{j\in I:p_j=\cA_\GS(\rho_j)<\frac{1}{2}-2h}$ and its complement. \Ethan{TODO import and use the right ``given" command.}

	First suppose $j\in J$
	Let $Y=\sum y_j$ be i.i.d. Bernoulli trials with success probabilities $p_y=\frac{1}{2}-2h$,
	then clearly $\Prob{Z>\frac{m}{2}-hm}<\Prob{Y>\frac{m}{2}-hm}$.
	By Chernoff bound, we have
	$$\Prob{Y-(\frac{1}{2}-2\varepsilon)m\geq hm}\leq2e^{-\frac{\mu\delta^2}{3}}=\negl(T)$$
	where $\mu_y=mp_y$ and $\delta=\frac{2h}{1-4h}$.

	Summing up all these terms, we get
	$$\sum_{j\in J} w_j q_j<2^{-\lambda}=\negl(\lambda)$$
	So this case doesn't get accepted often enough to matter.

	Now, for the other case, $j\notin J$.
	$$P[\cA_\GS(\phi)=acc]>\frac{1}{2}-2\varepsilon\Rightarrow\braket{\phi|H_{C'(x)}|\phi}<\frac{1}{T^\lambda}$$
	we then have
	$$\braket{\rho_i|H_{C'(x)}|\rho_i}<\frac{1}{T^\lambda}.$$

	We also know that the least nonzero eigenvalue of $H_{C'(x)}$ is lower-bounded by $\frac{3}{4}$, so we obtain \Ethan{at least I think we do. Need theorem.}
	$$\braket{\rho_i|\phi_{C'(x)}}>1-O(T^{-\lambda})$$
	\Ethan{Now we need some kinda standard fidelity argument to show that the measurement results will be close too. I'm not familiar with them at the moment.}

	The probability of measuring $t<T$ on the clock register of $\ket{\psi_{C'(x)}}$ is $\varepsilon$,
	so the data register has $1-\varepsilon$ probability to be $C(x)$ at this point.

	The soundness errors incurred at each step is at most $O(T^{-\lambda})$, so the conclusion follows.
\end{prf}




