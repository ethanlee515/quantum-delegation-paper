\section{$\SampBQP$ Delegation Protocol for $\QPIP_1$ Client}
\label{sec:sampbqp}

\def \Hin {H_{\mathrm{in}}}
\def \Hout {H_{\mathrm{out}}}
\def \Hprop {H_{\mathrm{prop}}}
\def \Hclock {H_{\mathrm{clock}}}
\def \Jclock {J_{\mathrm{clock}}}
\def \Jprop {J_{\mathrm{prop}}}
\def \Kin {K_{\mathrm{in}}}
\def \Kclock {K_{\mathrm{clock}}}
\def \Kprop {K_{\mathrm{prop}}}

\newcommand{\histpsi}[1]{\ket{\psi_{#1}^{\mathrm{hist}}}}
\newcommand{\LHXZ}[1]{\mathrm{LH}_{\mathrm{XZ}}^{#1}}
\newcommand{\ground}[1]{{\lambda \left (#1 \right)}}

\XW{merge Section 3 and 4; also easy the discussion;}

\subsection{Construction of the X-Z Local Hamiltonian for $\SampBQP$} \label{sec:LHXZ}
As we mentioned in this introduction, we will employ the circuit \emph{history} state in the original construction of the Local Hamiltonian problem~\cite{kitaev2002classical} to encode the circuit information for $\SampBQP$. 
However, there are distinct requirements between certifying the computation for $\BQP$ and $\SampBQP$ based on the history state.
For any quantum circuit $C$ on input $x$, the original construction for certifying $\BQP$\footnote{The original construction is for the purpose of certifying problems in QMA. We consider its simple restriction to problems inside BQP.} consists of local Hamiltonian $\Hin, \Hclock, \Hprop$, $\Hout$ where $\Hin$ is used to certify the initial input $x$, $\Hclock$ to certify the validness of the clock register,  $\Hprop$ to certify the gate-by-gate evolution according to the circuit description, and $\Hout$ to certify the final output. 
In particular, the corresponding history state is in the ground space of $\Hin$, $\Hclock$, and $\Hprop$. Note that $\BQP$ is a decision problem and its outcome (0/1) can be easily encoded into the energy $\Hout$ on the single output qubit. 
As a result, the outcome of $\BQP$ can simply be encoded by the \emph{ground energy} of $\Hin + \Hclock+\Hprop + \Hout$. 

To deal with $\SampBQP$, we will still employ $\Hin, \Hclock$, and $\Hprop$ to certify the circuit's input, the clock register, and gate-by-gate evolution. However, in $\SampBQP$, we care about the entire final state of the circuit, rather than the energy on the output qubit. 
%The history state remains in the ground space of $\Hin + \Hprop$.  
Our approach to certify the entire final state (which is encoded inside the history state) is to make sure that the history state is the unique ground state of $\Hin + \Hclock+ \Hprop$ and all other orthogonal states will have much higher energies. 
Namely, we need to construct some $\Hin'+ \Hclock'+ \Hprop'$ with the history state as the unique ground state and with a large \emph{spectral} gap between the ground energy and excited energies. 
It is hence guaranteed that any state with close-to-ground energy must also be close to the history state. 
We remark that this is a different requirement from most local Hamiltonian constructions that focus on the ground energy. 
We achieve so by using the \emph{perturbation} technique developed in~\cite{kempe_kitaev_regev_2006} for reducing the locality of Hamiltonian. 
Another example of local Hamiltonian construction with a focus on the spectral gap can be found in~\cite{adiabatic}, where the purpose is to simulate quantum circuits by adiabatic quantum computation. 

We need two more twists for our purpose. 
First, as we will eventually measure the final state in order to obtain classical samples, we need that the final state occupies a large fraction of the history state. We can simply add dummy identity gates. 
Second, as we are only able to perform $X$ or $Z$ measurement by techniques from~\cite{FOCS:Mahadev18a},
we need to construct X-Z only local Hamiltonians. 
Indeed, this has been shown possible in, e.g.,~\cite{PhysRevA.78.012352}, which serves as the starting point of our construction. 

%\XW{TOADD: some orgainization}
%\XW{highlight the different requirement from the normal Local Hamiltonian reduction; and highlight a bit about the solution idea}

% In this section, we give a reduction from $\SampBQP$ to a local Hamiltonian instance.
% The local Hamiltonian problem is $\QMA$-complete, which means it's $\BQP$-hard.
% Protocols to delegate $\BQP$ computations such as \cite{FOCS:Mahadev18a} \Ethan{Are there any more?} take advantage of this fact and require the prover to send the corresponding $\QMA$ certificate.
% We take a similar approach to delegate $\SampBQP$ computations; however, we need to extract more information from the certificates.
% Local Hamiltonians have been extensively studied in contexts of adiabatic quantum computations, so we will summarize known results from sources such as \cite{adiabatic}.
% \Ethan{This source was given to us by Yu-Ching; I haven't read it. Also I probably should cite more sources here.}

Recall the definition of the history state which serves as a transcript of the circuit evolution~\cite{kitaev2002classical}:

\begin{dfn}[History-state]
	\label{dfn:groundstate}	
    Given any quantum circuit $C=U_T\ldots U_1$ of $T$ elementary gates and input $x\in\{0,1\}^n$, the \emph{history}-state $\histpsi{C(x)}$ is defined by
    \begin{equation}
        \histpsi{C(x)} \equiv \frac{1}{\sqrt{T}}\sum_{t=0}^TU_t\ldots U_1\ket{x,0}\otimes\ket{\hat{t}},
    \end{equation}
    where the first register of $n$-qubit refers to the input, the second of $m$-qubit refers to the work space ($\mathrm{poly}(n)$, w.l.o.g, $\leq T$ size) which is initialized to $\ket{0}$, and the last refers to the clock space which encodes the time information. Note that $\hat{t}$ could be some representation of $t$=0,..., $T$. 
\end{dfn}

% \begin{rmk}
% 	\label{idpadding}
% 	One can first pad $C$ with identity gates at the end, doubling its number of gates and obtaining another circuit $C'$.
% 	Then by measuring the time register of $\ket{\psi_{C'(x)}}$ one obtains with probability $\frac{1}{2}$ the state $C(x)$.
% \end{rmk}

% \XW{define $k$-local XZ Hamiltonian}
% We denote the set of $5$-local tensor products of $X$ and $Z$ gates as $\mathcal{G}_{XZ}$ as below; note that this spans a subspace of $5$-local Hamiltonians.

We define the X-Z local Hamiltonian of interest as follows: 

\begin{dfn}[$k$-local X-Z Hamiltonian] For any $n$-qubit system, the set of $k$-local X-Z terms, denoted by $\LHXZ{k}$, contains Hermitian matrices that apply non-trivially on at most $k$ qubits as a product of Pauli $X$ and $Z$ terms. Namely, 
\begin{equation}
  \LHXZ{k} = \left\{h_1 \otimes h_2 \otimes \ldots \otimes h_n: \forall i \in [n], h_i \in \{I, X, Z\}, \text{and}, \abs{\set{i: h_i=X \text{ or }Z}} \leq k. \right \}. 
\end{equation}
A $k$-local X-Z Hamiltonian $H$ is a linear combination of terms from $\LHXZ{k}$. Namely, 
\begin{equation}
  H = \sum_i \alpha_{i} H_i,  \quad \forall i, \alpha_i \in R,  H_i \in \LHXZ{k}.
\end{equation}
\end{dfn}

As our starting point, we will include the existing result of constructing X-Z local Hamitonians for general $\BQP$ computation. 
First, we note the fact Toffoli and Hadamard gates form a universal gate set for quantum computation~\cite{Shi03, quant-ph/0301040}. 
It is easy to see that both Toffoli and Hadamard gates can be represented as linear combinations of terms from $\LHXZ{}$. For example, 
 \begin{equation}
     \mathrm{(Hadamard)} \quad H \equiv \frac{1}{\sqrt{2}} \begin{pmatrix}1&1\\1&-1\end{pmatrix} = \frac{1}{\sqrt{2}}\left (X+Z\right).
 \end{equation}
 The Toffoli gate maps bits $(a,b,c)$ to $(a,b, c \oplus (a \text{ and } b))$, which can be decomposed as $\ket{11}\bra{11}\otimes X+(I-\ket{11}\bra{11})\otimes I$ where $\ket{11}\bra{11}=\frac{1}{4}(I\otimes I+Z\otimes Z-I\otimes Z-Z\otimes I)$.
 %Here we restate the result from from \cite{quant-ph/0301040} showing that $\set{H, \Lambda^2(X)}$ is a universal gate set.
 %Because of this result, we can assume without loss of generality that our quantum circuits consists of only Toffoli and Hadamard gates.
% Which is significant because this gate set is a linear combinations of tensor products of $X$ and $Z$ operators:
% \begin{Fact}
%     	$H,\Lambda^2(X)\in\spn\mathcal{G}_{XZ}$ with $O(1)$ nonzero components.
% \end{Fact}
% \begin{prf}
% 	$$H=\frac{1}{\sqrt{2}}(X+Z)$$
% 	$$\Lambda^2_{1,2}(X_3)=\ket{11}\bra{11}\otimes X+(I-\ket{11}\bra{11})\otimes I$$
% 	$$\ket{11}\bra{11}=\frac{1}{4}(I\otimes I+Z\otimes Z-I\otimes Z-Z\otimes I)$$
% \end{prf}

We will follow the unary clock register design from Kitaev's original 5-local Hamiltonian construction~\cite{kitaev2002classical}. Namely, valid unary clock states ($T$-qubit) are $\ket{00\ldots0}$, $\ket{10\dots0}$, $\ket{110\ldots0}$, etc, which span the ground energy space of the following local Hamiltonian: 
\begin{equation}
    \Hclock= \sum_{t=1}^{T-1}\proj{01}_{t,t+1},
\end{equation}
where $\proj{01}_{t,t+1}$ stands for a projection on the $t$th and $(t+1)$th qubit in the clock register. 
It is observed in~\cite{PhysRevA.78.012352} that $\Hclock$ can be reformulated as a linear combination of terms from $\LHXZ{}$ as follows, 
\begin{equation} \label{eqn:Hclock}
   \Hclock= \frac{1}{4}(Z_1 - Z_T) + \frac{1}{4}\sum_{t=1}^{T-1}(I-Z_t\otimes Z_{t+1}),
\end{equation}
where $Z_t$ refers to Pauli $Z$ operated on the $t$th qubit in the clock register. 

One can achieve so similarly for $\Hin$ and $\Hprop$. For the input condition, we want to make sure $x=(x_1, \ldots, x_n) \in \{0,1\}^n$ is in the input space and the workspace is initialized to $\ket{0}$ for all qubits at the time $0$. Namely, one can set $\Hin$ to be
\begin{equation} 
    \Hin = \sum_{i=1}^n(I- \proj{x_i}_i)\otimes \proj{0}_1 + \sum_{i=1}^m \proj{1}_{n+i} \otimes \proj{0}_1, 
\end{equation}
where the last part $\proj{0}_1$ applies on to the first qubit in the clock register. One can rewrite $\Hin$ as 
\begin{equation}\label{eqn:Hin}
 \Hin=\frac{1}{4}\sum_{i=1}^n(I-(-1)^{x_i}Z_i)\otimes(I+Z_1) + \frac{1}{4} \sum_{i=1}^m (I - Z_{n+i}) \otimes (I + Z_1). 
\end{equation}
% Then, we ensure that the initial condition is set up correctly.
% 	$$H_{in}=\sum_{i=1}^n(I-\ket{x_i}\bra{x_i})\otimes\ket{0}\bra{0}_1$$
% 	The kernel of this is precisely where everything is set up consistently with $\phi$ in time step $t=0$. Furthermore, $H_{in}\in\spn\mathcal{G}_{XZ}$

For the propagation of the quantum state through the circuit, one uses the $\Hprop$ as follows: 
\begin{equation*} \label{eqn:Hprop}
    \Hprop=\sum_{t=1}^T \Hprop^t, 
\end{equation*}
where 
\begin{equation}
    \Hprop^t=\frac{1}{2}I\otimes\proj{\widehat{t}}
	+\frac{1}{2}I\otimes\proj{\widehat{t-1}}
	-\frac{1}{2} U_t\otimes\ket{\widehat{t}}\bra{\widehat{t-1}}
	-\frac{1}{2}U_t^\dagger\otimes\ket{\widehat{t-1}}\bra{\widehat{t}}.
\end{equation}
Note that $\ket{\widehat{t}}\bra{\widehat{t-1}}=\ket{110}\bra{100}_{(t-1,t,t+1)}$ and similarly for $\ket{\widehat{t-1}}\bra{\widehat{t}}$.
% 	Then, we ensure that the gates are applied correctly.
% 	$$H_{prop}=\sum_{t\in T_1}H_{prop,t}$$
% 	$$H_{prop,t}=I\otimes\ket{\widehat{t}}\bra{\widehat{t}}
% 	+I\otimes\ket{\widehat{t-1}}\bra{\widehat{t-1}}
% 	-U_t\otimes\ket{\widehat{t}}\bra{\widehat{t-1}}
% 	-U_t^\dagger\otimes\ket{\widehat{t-1}}\bra{\widehat{t}}$$
% 	We have $H_{prop}\geq 0$ according to \cite{2002quant.ph.10077A}; in fact, the least nonzero eigenvalue of $H_{prop}$ is lower bounded by $\frac{1}{2(T+1)^2}$.
%One can similarly rewrite $\Hprop$ in terms of elements from $\LHXZ{}$. 
Note that $U_t^\dagger=U_t$, since our gates are either Hadamard or Toffoli. It is observed in~\cite{PhysRevA.78.012352} that 
\begin{equation}
   \Hprop^t=\frac{I}{4}\otimes(I-Z_{t-1})\otimes (I+Z_{t+1})-\frac{U_t}{4}\otimes(I-Z_{t-1})\otimes X_t\otimes (I+Z_{t+1}), \forall t=2, \ldots, T-1,
\end{equation}
and 
\begin{eqnarray}
  \Hprop^1 &= & \frac{1}{2}(I+Z_2)-U_1\otimes\frac{1}{2}(X_1+X_1\otimes Z_2) \\
  \Hprop^T &= & \frac{1}{2}(I-Z_{t-1})-U_T\otimes\frac{1}{2}(X_T-Z_{T-1}\otimes X_T).
\end{eqnarray}
% 	We can also write $H_{prop}\in\spn\mathcal{G}_{XZ}$.
% 	Additionally, $\frac{1}{2}(I-Z_{t-1})$ annihilates time steps before $t-1$. $\frac{1}{2}(I+Z_{t+1})$ similarly annihilates steps $t+1$ and after.
% 	
% $$H_{prop,t}=\frac{I}{4}\otimes(I-Z_{t-1})(I+Z_{t+1})-\frac{U}{4}\otimes(I-Z_{t-1})X_t(I+Z_{t+1})$$
% 	Extra care must be taken for boundary cases.
% 	$$H_{prop,1}=\frac{1}{2}(I+Z_2)-U_1\otimes\frac{1}{2}(X_1+X_1Z_2)$$
% 	$$H_{prop,T}=\frac{1}{2}(I-Z_{t-1})-U_T\otimes\frac{1}{2}(X_T-Z_{T-1}X_T)$$
Combining with the fact that each $U_t$ can be written as a linear combination of terms from $\LHXZ{}$, we conclude that $\Hclock$, $\Hin$, $\Hprop$ are 6-local X-Z Hamiltonian. 

We will employ the perturbation technique to amplify the spectral gap of $\Hclock + \Hin + \Hprop$. 
Let $\ground{H}$ denote the ground energy of any Hamiltonian $H$. 
The projection lemma from \cite{kempe_kitaev_regev_2006} approximates $\ground{H_1 + H_2}$ in terms of $\ground{H_1\big|_{\ker H_2}}$, where $\ker H_2$ denotes the \emph{kernel} space of $H_2$. 

\begin{lem}[Lemma 1 in \cite{kempe_kitaev_regev_2006}]
	\label{thm:proj1}
	Let $H=H_1+H_2$ be the sum of two Hamiltonians operating on Hilbert space $\cH=\cS+\cS^\bot$.
	The Hamiltonian $H_2$ is such that $\cS$ is a zero eigenspace and the eigenvectors in $\cS^\bot$ have eigenvalues at least $J>2\norm{H_1}$. Then,
	$$\lambda\left(H_1\big|_\cS\right)-\frac{\norm{H_1}^2}{J-2\norm{H_1}}\leq\lambda(H)\leq\lambda\left(H_1\big|_\cS\right)$$
\end{lem}

We will use the following simple reformulation instead. 
%We will instead use the following formulation, which can be obtained by relabeling variables from above.

\begin{lem}
	\label{lem:projection}
	Let $H_1, H_2$ be local Hamiltonians where $H_2\geq0$. Let $K=\ker H_2$ and
	$$J=\frac{8\norm{H_1}^2+ 2\norm{H_1}}{\lambda\left(H_2\big|_{K^\bot}\right)}$$
	then we have
	$$\lambda(H_1+JH_2)\geq\lambda\left(H_1\big|_K\right)-\frac{1}{8}$$
\end{lem}
\begin{proof}
	Apply \Cref{thm:proj1} to $H=H_1+JH_2$. Note that the least nonzero eigenvalue of $JH_2$ is greater than $2\norm{H_1}$.
\end{proof}

\begin{thm}
	\label{thm:LHReduction}
	Given any quantum circuit $C = U_T \ldots U_1$ of  $T$ elementary gates and input $x \in \{0,1\}^n$, one can construct a 6-local X-Z Hamiltonian $H_{C(x)}$ in polynomial time such that 
	\begin{enumerate}
		\item[(1)] $H_{C(x)} = \sum_i \alpha_i H_i$ where 
		each $H_i \in \LHXZ{6}$ and $|\alpha_i| \in O(T^9)$. Moreover, there are at most $O(T)$ non-zero terms. 
		\item[(2)] $H_{C(x)}$ has $\histpsi{C(x)}$ as the unique ground state with eigenvalue $0$ and has a spectral gap at least $\frac{3}{4}$. Namely,  for any state $\ket\phi$ that is orthogonal to $\histpsi{C(x)}$, we have $\braket{\phi|H_{C(x)}|\phi}\geq \frac{3}{4}$.
	\end{enumerate}
\end{thm}

\begin{proof}
We will use the above construction $\Hclock$ (\cref{eqn:Hclock}), $\Hin$ (\cref{eqn:Hin}), $\Hprop$ (\cref{eqn:Hprop}) as our starting point, which are already 6-local X-Z Hamiltonian constructable in polynomial time. However, $H_{\mathrm{old}}=\Hin + \Hclock + \Hprop$ does not have the desired spectral gap. To that end, our construction will be a weighted sum of $\Hin$, $\Hclock$, and $\Hprop$ as follows, 
\begin{equation}
    H_{\mathrm{new}}= \Hin + \Jclock \Hclock + \Jprop \Hprop,
\end{equation}
where $\Jclock$ and $\Jprop$ will be obtained using \Cref{lem:projection}. 

Let $\Kin=\ker \Hin$, $\Kclock=\ker \Hclock$, and $\Kprop=\ker \Hprop$. It is known from e.g.,~\cite{kitaev2002classical}, that 
\[
   \Kin \cap \Kclock \cap \Kprop = \spn\set{\histpsi{C(x)}}.
\]
Thus $\histpsi{C(x)}$ remains in the ground space of $H_{\mathrm{new}}$. Let $S$ denote its orthogonal space. Namely, $S=(\spn\set{\ket{\psi_{C(x)}}})^\bot$. 
Denote by $\Hin\big|_S$ the restriction of $\Hin$ on space $S$ and similarly for others. 

%\XW{need to check the following calculation and modify the theorem statement respectively.}
%We will apply \Cref{lem:projection} twice to obtain $\Jclock$ and $\Jprop$ respectively. 
Consider $\Hin + \Jclock\Hclock$ first. According to \cref{lem:projection}, by choosing 
\[
  \Jclock = \frac{8\norm{\Hin\big|_S}^2 + 2 \norm{\Hin\big|_S}}{\ground{\Hclock\big|_{S\cap \Kclock^\bot}}} = O(T^2),
\]
where we use the fact $\norm{\Hin|_S}\leq T$ and $\ground{\Hclock\big|_{S\cap \Kclock^\bot}}\geq \ground{\Hclock\big|_{\Kclock^\bot}}=1$, 
we have 
\[
 \ground{\Hin\big|_S + \Jclock\Hclock\big|_S}\geq \ground{\Hin\big|_{S\cap \Kclock}} - \frac{1}{8}. 
\]
Consider further adding $\Jprop\Hprop$ term.  By choosing 
\[
 \Jprop= \frac{O(\norm{\Hin\big|_S + \Jclock\Hclock\big|_S}^2)}{\ground{\Hprop\big|_{S \cap \Kprop^\bot}}}= O(T^8),
\]
where we use the fact $\ground{\Hprop\big|_{S\cap \Kprop^\bot}}\geq \ground{\Hprop\big|_{\Kprop^\bot}}=\Omega(T^{-2})$~\cite{kitaev2002classical}, we have  
% $$\exists J_{prop}=\frac{\poly\left(\norm{H_{in}\big|_S+J_{clock}H_{clock}\big|_S}\right)}{\lambda\left(H_{prop}\big|_{S\cap K^\bot_{prop}}\right)}
% =\frac{O(n+T)}{\Omega(T^{-2})}=O(T^3)$$
% $$\lambda(H_{in}\big|_S+J_{clock}H_{clock}\big|_S+J_{prop}H_{prop}\big|_S)\geq
% \lambda(H_{in}\big|_{S\cap K_{clock}\cap K_{prop}})-\frac{1}{4}$$
\[
 \ground{\Hin\big|_S + \Jclock\Hclock\big|_S + \Jprop\Hprop\big|_S} \geq \ground{\Hin\big|_{S \cap \Kclock \cap \Kprop}} - \frac{1}{4}. 
\]
A simple observation here is that $S \cap \Kclock \cap \Kprop$ is the span of history states with different inputs or different initialization of the work space. Namely, $\ground{\Hin\big|_{S \cap \Kclock \cap \Kprop}} \geq 1$. Thus, 
\[
  \ground{(\Hin + \Jclock\Hclock + \Jprop\Hprop)\big|_S} \geq 1- \frac{1}{4}=\frac{3}{4}. 
\]
Given that $\histpsi{C(x)}$ is the ground state of $H_{\mathrm{new}}$ with eigenvalue 0, and any orthogonal state to $\histpsi{C(x)}$ has eigenvalue at least $\ground{H_{\mathrm{new}}}\geq 3/4$, the spectral gap of $H_{\mathrm{new}}$ at at least $3/4$. 

Note that $H_{\mathrm{new}}$ is a 6-local X-Z Hamiltonian by construction. It suffices to check the bound of $\abs{\alpha_i}$ and the number of terms. The former is one more than the order of $\Jprop$ since each $\Hprop^t$ contributes $\frac{1}{4}$ to the $I$ term, creating an extra factor of $T$. The latter is by counting the number of terms from $\Hin, \Hclock, \Hprop$, each of which is bounded by $O(T)$.
\end{proof}
% 	To properly combine the three Hamiltonians defined and analyze it using the projection lemma (\cref{thm:projection}), we consider the kernels of the Hamiltonians we defined.
% 	$$K_{clock}=\ker H_{clock}$$
% 	$$K_{in}=\ker H_{in}$$
% 	$$K_{prop}=\ker H_{prop}$$

% 	Clearly,
% 	$$K_{clock}=\set{\sum_{t=1}^T \ket{\phi_t}\otimes\ket{\hat{t}}:\ket{\phi_t}\in\cB^{\otimes n}}$$
% 	Let $\ket\phi\in K_{clock}\cap K_{prop}$.
% 	Consider some $\widetilde{t}$ such that $\braket{\widetilde{t}|\phi}\ne0$.
% 	As $\ket\phi\in K_{prop}$, we can then do induction using $\widetilde{t}$ as base case to get the relations between the $\ket{\phi_t}$s and conclude that $\ket\phi$ must have the form:
% 	$$\ket\phi=\sum_{t=0}^TU_t\ldots U_1\ket{y}\otimes\ket{\hat{t}}$$
% 	$$\Rightarrow K_{clock}\cap K_{prop}=\set{\sum_{t=0}^TU_t\ldots U_1\ket{y}\otimes\ket{\hat{t}}: \ket{y}\in\mathcal{B}^{\otimes n}}$$
% 	As a direct result, we get
% 	$$K_{clock}\cap K_{in}\cap K_{prop}=\spn\set{\ket{\psi_{C(x)}}}$$
% \begin{align*}
% 	S\cap K_{clock}\cap K_{prop}&=S\cap\set{\sum_{t=0}^TU_t\ldots U_1\ket{y}\otimes\ket{\hat{t}}|\ket{y}\in\mathcal{B}^{\otimes n}}\\
% 	&=\set{\sum_{t=0}^TU_t\ldots U_1\ket{y}\otimes\ket{\hat{t}}:\braket{\psi_{C(x)}|y}=0}
% \end{align*}
% $$\Rightarrow\lambda((H_{in}+J_{clock}H_{clock}+J_{prop}H_{prop})\big|_S)\geq\frac{3}{4}$$
% So we set $H_{C(x)}=H_{in}+J_{clock}H_{clock}+J_{prop}H_{prop}$, which satisfies the required properties by construction.

\subsection{Delegation Protocol for $\QPIP_1$ client}
\label{sec:qpip1}
In this subsection, we construct a one-message $\QPIP_1$ delegation protocol for $\SampBQP$. By definition of $\QPIP_1$, we assume the client has limited quantum power, e.g., performing single qubit $X$ or $Z$ measurement one by one. 
Intuitively, one should expect the one-message from the server to the client is something like the history state so that the client can measure to sample.  

At a high-level, the design of such protocol should consist of at least two components: (1) the first component should test whether the message is indeed a valid history state; (2) the second component should simulate the last step of a $\SampBQP$ computation by measuring the final state in the computational basis. 

Our construction of X-Z local Hamiltonian $H$ from \Cref{thm:LHReduction} will help serve the first purpose. 
In particular, we adopt a variant of the energy verification protocol for local Hamiltonian (e.g., ~\cite{mf16, PhysRevA.93.022326}) to certify the energy of $H$ with only $X$ or $Z$ measurements.
Moreover, because of the large spectral gap, when the energy is small, the underlying state must also be close to the history state. 
Precisely, consider the following protocol $\cVGS$: 

\begin{protocol}{Energy verification for X-Z local Hamiltonian $\cVGS$} \label{AlgGroundStateCheck}
Given a $k$-local X-Z Hamiltonian 
$H=\sum_i \alpha_{i} H_i$ (i.e., $\forall i, H_i \in \LHXZ{k}$) and any state $\ket{\phi}$.
%Let $\ket\phi$ be the potential ground state to check.

\begin{itemize}
\item Let $p_i= \abs{\alpha_i}/\sum_i \abs{\alpha_i}$ for each $i$. Sample $i^*$ according to $p_{i^*}$. 
\item Pick $H_{i^*}$ which acts non-trivially on at most $k$ qubits of $\ket{\phi}$. Measure the corresponding single-qubit Pauli X or Z operator. 
Record the list of the results $x_j=\pm 1$ for $j=1, \ldots k$. 
\item Let $r=x_1x_2\cdots x_k$. The protocol \emph{accepts} if $r$ and $\alpha_{i^*}$ have different signs, i.e., $\sgn(\alpha_{i^*})r=-1$. Otherwise, the protocol \emph{rejects}. 
\end{itemize}
\end{protocol}
% In this subsection, we prove \cref{ThmXZCheck}.
% That is, we present and analyze an algorithm that checks whether a given state is the ground state of some fixed $H_{C(x)}$, following~\cite{PhysRevA.93.022326}.

\begin{lem}[\cite{PhysRevA.93.022326}]
	\label{thm:HamCheck}
	For any $k$-local X-Z Hamiltonian $H=\sum_i \alpha_{i} H_i$ and any state $\ket{\phi}$, 
	the protocol $\cVGS$ in Protocol~\ref{AlgGroundStateCheck} accepts with 
	probability
\begin{equation}
 \mathrm{Prob}[ \cVGS \text{ accepts } \ket{\psi}] = \frac{1}{2} - \frac{1}{2 \sum_i \abs{\alpha_i}}\braket{\phi|H|\phi}.
\end{equation}
\end{lem}

\begin{theorem} \label{thm:HamCheckClose}
Given any quantum circuit $C$ and input $x$, consider using $H_{C(x)}$ from \Cref{thm:LHReduction} in Protocol~\ref{AlgGroundStateCheck} ($\cVGS$). 
For any state $\rho$, and $0< \epsilon < 1$, if $\cVGS$ accepts $\rho$ with probability, 
\[
 \mathrm{Prob}[\cVGS \text{ accepts } \rho] \geq \frac{1}{2} - \frac{\epsilon}{2 \sum_i \abs{\alpha_i}},
\]
then the trace distance between $\rho$ and $\histpsi{C(x)}$ is at most $\frac{2}{\sqrt{3}}\sqrt{\epsilon}$. 
\end{theorem}

\begin{prf} Consider the pure state case $\rho=\proj{\phi}$ first. By \Cref{thm:HamCheck} and our assumption, we have $\braket{\phi|H|\phi} \leq \epsilon$. 
Decompose $\ket{\phi}= \alpha \histpsi{C(x)} + \beta \histpsi{C(x)}^\bot$. Note that $\histpsi{C(x)}$ is a eigenvector $H_{C(x)}$ of eigenvalue 0 and all other eigenvalues are at least $3/4$. Thus, we have $\abs{\alpha}^2 \geq 1-\frac{4}{3}\epsilon$. Thus, 
\[
   \norm{\proj{\psi^{\mathrm{hist}}_{C(x)}}- \proj{\phi}}_{tr} = \sqrt{1- |\braket{\psi^{\mathrm{hist}}_{C(x)}|\phi}|^2}
   \leq \frac{2}{\sqrt{3}}\sqrt{\epsilon}.
\]
For any mixed state $\rho=\sum_i p_i \proj{\phi_i}$, by the triangle inequality, we have 
\[
    \norm{\proj{\psi^{\mathrm{hist}}_{C(x)}}- \rho}_{tr} \leq \sum_i p_i    \norm{\proj{\psi^{\mathrm{hist}}_{C(x)}}- \proj{\phi_i}}_{tr} \leq \sum_i p_i \frac{2}{\sqrt{3}} \sqrt{\epsilon} =\frac{2}{\sqrt{3}}\sqrt{\epsilon}.
\]
% $$\ket{\phi}=\braket{\psi_{C(x)}|\phi}\ket{\phi_K}+\sqrt{1-\abs{\braket{\psi_{C(x)}|\phi}}^2}\ket{\phi_{K^\bot}}$$
% 	We can then see
% 	$$\braket{\phi|H|\phi}=1-\abs{\braket{\psi_{C(x)}|\phi}}^2=1-F(\ket{\psi_{C(x)}}, \ket{\phi})$$
% 	which can be generalized for mixed states since we have for all $\ket\psi$ that
% 	$$F(\sum_i p_i \ket{\phi_i}\bra{\phi_i}, \ket\psi)=\sum_i p_i F(\ket{\phi_i}, \ket\psi).$$
% 	As a result, we have
% 	$$p_{acc}=\frac{1}{2}-\frac{c}{T^5}(1-F(\ket{\psi_{C(x)}}, \rho))\geq\frac{1}{2}-\frac{c\eps}{T^5}$$
% 	which implies our claim.
\end{prf}

% \begin{algorithm}
% 	\caption{Check for ground state $\cVGS$}
% 	\label{AlgGroundStateCheck}
% 		Let $H=\sum_{S\in\mathcal{G}_{XZ}} d_S S$.
% 		Let $\ket\phi$ be the potential ground state to check.
% 		\begin{itemize}
% 			\item Set $D = \sum_{S\in\mathcal{G}_{XZ}}|d_S|$
% 			\item Set $p_S = \frac{|d_S|}{D}$
% 			\item Sample $\widetilde{S}$ from $\mathcal{G}_{XZ}$, weighted by $p_S$.
% 			\item Measure $\ket\phi$ in the $\widetilde{S}$ basis, recording the result as $\lambda_{\widetilde{S}}$.
% 			\item If $\sgn(d_{\widetilde{S}})\lambda_{\widetilde{S}}=-1$, accept by outputting $1$. Otherwise, reject by outputting $0$.
% 		\end{itemize}
% \end{algorithm}
% \begin{prf}
% 	Here we follow \cite{PhysRevA.93.022326}.
% 	\begin{align*}
% 		\frac{1}{D}\braket{\phi|H|\phi}&=\sum_{S\in\mathcal{G}_{XZ}} p_S\sgn(d_S)\braket{\phi|S|\phi}\\
% 		&=\sum_{S\in\mathcal{G}_{XZ}} p_S\sgn(d_S)\E[\lambda_S]\\
% 		&=\E_{\widetilde{S}}[\sgn(d_{\widetilde{S}})\E[\lambda_{\widetilde{S}}]]\\
% 		&=\E_{\widetilde{S}}[\sgn(d_{\widetilde{S}})\lambda_{\widetilde{S}}]
% 	\end{align*}

% 	Note that $\sgn(d_{\widetilde{S}})\lambda_{\widetilde{S}}=\pm1$. Let $p$ be the probability that $\sgn(d_{\widetilde{S}})\lambda_{\widetilde{S}}=-1$.
% 	$$\Rightarrow \frac{1}{D}\braket{\phi|H|\phi}=\E_{\widetilde{S}}[\sgn(d_{\widetilde{S}})\lambda_{\widetilde{S}}]=-p+(1-p)$$
% 	\begin{align*}
% 		\Rightarrow p&=\frac{1}{2}-\frac{1}{2D}\braket{\phi|H|\phi}\\
% 		&=\frac{1}{2}-\frac{O(1)}{T^5}\braket{\phi|H|\phi}
% 	\end{align*}
% \end{prf}

To serve the second purpose, one needs to combine the test and the output on multiple copies of the history states, where we construct the following cut-and-choose protocol.  
% At a high level, it is a cut-and-choose protocol.
% The server constructs multiple copies of the ground state as certificates,
% then the client randomly chooses a copy to output and checks the rest.
% Unfortunately our approach incurs inverse polynomial soundness errors.
% There are two main challenges to this cut-and-choose approach.
% First, the client needs to reliably extract the circuit output from its corresponding ground state.
% We accomplish this by padding the circuit with identity gates at the end.
% By doing so, the clock register collapses to after the last non-identity gate with high probability.
The challenge comes from the fact that a cheating prover might send something rather than copies of the history state. 
In particular, the prover can entangle between different copies in order to cheat. 
In the case of certifying a $\BQP$ computation, the goal is to verify the ground energy of any local Hamiltonian. 
A cheating strategy with potential entanglement won't create any witness state with an energy lower than the actual ground energy. 
Thus, this kind of attack won't work for $\BQP$ computation. 

However, in the context of $\SampBQP$, one needs to certify the ground energy (known to be zero in this case) and to output a good copy. 
%%% not sure about this high-level argument. 
While the statistical test as before can be used to certify the ground energy, 
it has less control on the shape of the output copy.
In fact, the prover can always prepare a bad copy among with many good copies as a plain attack. 
This attack will succeed when the bad copy is chosen to output, the probability of which is non-negligible in terms of the total number of copies assuming some symmetry of the protocol. 
The potential entanglement among different copies could further complicate the analysis. 
We employ the quantum \emph{de Finetti}'s theorem to address this technical challenge. 
Specifically, given any permutation-invariant $k$-register state, it is known that the reduced state on many subsets of $k$-register will be close to a separable state. 
This helps establish some sort of independence between different copies in the analysis. 
To serve our purpose, we adopt the following version of quantum de Finetti's theorem from~\cite{Brandão2017} where the error depends nicely  on the number of qubits, rather than the dimension of quantum systems. 
% one always needs to send the best certificate state (i.e., the perfect history state) in order to pass the local Hamiltonian ground energy test. 
% The other challenge is that a cheating prover's certificates can be arbitrarily entangled,
% so common techniques such as Chernoff bounds can't be applied as-is.
%This wasn't a challenge for $\Piblind$ for $\BQP$ earlier because...
% In our case with $\SampBQP$, we overcome this challenge instead by using de Finetti's theorem
% to approximate our measurement results with that of some unentangled copies,
% within inverse polynomial errors.
% De Finetti theorem provides a way to obtain close to independent samples by taking random subsystems of a quantum system.
% There are many formulations; we use the one from \cite{Brandão2017} because we need to avoid exponential dependence on number of qubits in each subsystem.
\begin{thm}[\cite{Brandão2017}]
	\label{deFinetti}
	Let $\rho^{A_1\ldots A_k}$ be a permutation-invariant state on registers $A_1,\ldots,A_k$ where each register contains $s$ qubits. 
	For any $0\leq l\leq k$,  there exists states $\set{\rho_i}$ and $\set{p_i}\subset\bbR$ such that
	$$\max_{\Lambda_1,\ldots,\Lambda_l}
	\norm{(\Lambda_1\otimes\ldots\otimes\Lambda_l)\left(\rho^{A_1\ldots A_l}-\sum_ip_i\rho_i^{A_1}\otimes\ldots\otimes\rho_i^{A_l}\right)}_1
	\leq\sqrt{\frac{2l^2s}{k-l}}$$
	where $\Lambda_i$ are quantum-classical channels.
\end{thm}

\XW{comment on saturating the error bound?}

\section{Delegation Protocol for Hybrid Client}

In this section, we construct a one-message $\QPIP_1$ delegation protocol for $\SampBQP$.
At a high level, it is a cut-and-choose protocol.
The server constructs multiple copies of the ground state as certificates,
then the client randomly chooses a copy to output and checks the rest.
Unfortunately we have inverse polynomial soundness errors here.

There are two main challenges to this cut-and-choose approach.
First, the client needs to reliably extract the circuit output from its corresponding ground state.
We accomplish this by padding the circuit with identity gates at the end.
By doing so, the clock register collapses to after the last non-identity gate with high probability.

The other challenge is that a cheating prover's certificates can be arbitrarily entangled,
so common techniques such as Chernoff bounds can't be applied as-is.
This wasn't a challenge for $\Piblind$ for $\BQP$ earlier because its soundness is one-sided:
we only guarantee that a client cannot accept a no-instance;
a cheating server is allowed to cause the client to reject a yes-instance.
Because of this asymmetry, a cheating prover's for $\Piblind$ optimal strategy is to maximize its acceptance probability,
which corresponds to sending $n$ identical copies of some state.
In our case with $\SampBQP$, we overcome this challenge by using de Finetti's theorem
to approximate our measurement results with that of measurement results on some unentangled copies.

\def\GS{\mathsf{GS}}
\nc{\PiGS}{\ensuremath{\Pi_\GS}}
\nc{\VGS}{\ensuremath{V_\GS}}
\nc{\PGS}{\ensuremath{P_\GS}}
\nc{\PGSstar}{\ensuremath{P_\GS^*}}
\nc{\cVGS}[1]{\ensuremath{\cV_{\GS,#1}}}
\nc{\cPGS}[1]{\ensuremath{\cP_{\GS,#1}}}

\def\Samp{\mathsf{Samp}}
\nc{\PiSamp}{\ensuremath{\Pi_\Samp}}
\nc{\VSamp}{\ensuremath{V_\Samp}}
\nc{\PSamp}{\ensuremath{P_\Samp}}
\nc{\PSampstar}{\ensuremath{P_\Samp^*}}
\nc{\cVSamp}[1]{\ensuremath{\cV_{\Samp,#1}}}
\nc{\cPSamp}[1]{\ensuremath{\cP_{\Samp,#1}}}

\def\GS{\mathsf{GS}}

\begin{protocol}{$\QPIP_1$ protocol $\PiSamp$ for $\SampBQP$}\label{ProtoQPIP1}
	Pick $\varepsilon=\poly(T^{-1})$
	Let $C'$ be $C$ padded with $\poly(\varepsilon^{-1})$ identity gates at the end.
	Pick $h=\poly(\varepsilon)$, $m=O(h^{-3})$, $M=\poly(m)$.

	\begin{enumerate}
		\item The honest prover prepares $M$ copies of $\ket{\psi_{C'(x)}}$ and sends all of it to the verifier qubit-by-qubit.
		\item The verifier privately samples $I\subset[M]$ s.t. $\abs{I}=m$, and $k\xleftarrow{\$}[M]\setminus I$.
			For $i$ from $1$ to $N$, it chooses what to do to the $i$-th copy, $\rho_i$, as follows:
		\begin{enumerate}
			\item If $i\in I$, run $z_i\leftarrow\cA_{\GS}(\rho_i)$.
			\item Otherwise, if $i=k$, measure the data register of $\rho_i$ and save the outcome as $y$.
			\item Otherwise, discard $\rho_i$.
		\end{enumerate}
			Let $Z=\sum_{i\in I} z_i$. If $Z>\frac{m}{2}-hm$ then the verifier accepts and outputs $y$. Otherwise, it rejects.
	\end{enumerate}
\end{protocol}

Note that $\VSamp$ only needs to apply $X$ and $Z$ measurements, and is classical otherwise. We now show the completeness and soundness of $\PiSamp$.

\begin{thm}
    \label{QPIP1thm}
	$\PiSamp$ has negligible completeness error and $O(T^-\lambda)$ soundness error.
\end{thm}
\begin{prf}
	For completeness, notice $z_i$ are i.i.d. Bernoulli trials with success probability $\frac{1}{2}$.
	So we can apply the Chernoff bound \cref{thm:Chernoff} with $\mu=\frac{m}{2}$, $\delta=2h$ to get
	$$\Prob{\frac{m}{2}-Z\geq hm}\leq2e^{-\frac{\mu\delta^2}{3}}=\negl(T)$$

	Now we show soundness.
	Suppose $\PSampstar$ is a cheating prover that sends some $\sigma$ to the verifier.

	We first show that de Finetti's theorem can indeed be used here to achieve independence.
	Randomly picking $m+1$ out of $M$ registers is equivalent to first applying a random permutation to get some $\sigma'$ then taking the first $m+1$ registers.
	A random permutation, in turn, is a classical mix over all possible permutations:
	$$\sigma'=\frac{1}{\abs{\Sym(M)}}\sum_{\Pi\in\Sym(M)}\Pi\sigma\Pi^\dagger$$
	It is simple to check that $\sigma'$ is permutation-invariant.
	Fix $\tilde{\Pi}\in\Sym(M)$, then
	$$\tilde{\Pi}\sigma'\tilde{\Pi}^\dagger
	=\frac{1}{\abs{\Sym(M)}}\sum_{\Pi\in\Sym(M)}\tilde{\Pi}\Pi\sigma\Pi^\dagger\tilde{\Pi}^\dagger
	=\frac{1}{\abs{\Sym(M)}}\sum_{\hat{\Pi}\in\Sym(M)}\hat{\Pi}\sigma\hat{\Pi}^\dagger
	=\sigma'$$
	where the second equality is by relabeling $\tilde{\Pi}\Pi=\hat{\Pi}$, which is allowed since $\Sym(M)$ is a group.

	Now we apply \cref{deFinetti} to approximate $\sigma'$ with a classical mix over tensors of independent states.
	That is, $\exists\rho=\sum_j w_j\rho_j^{\otimes m+1}$ such that:
	$$\max_{\Lambda_i}\norm{\Lambda_1\otimes\ldots\otimes\Lambda_{m+1}(\sigma'-\rho)}=O(\varepsilon)$$

	Now we separate the $\rho_j$ into two categories: $J=\set{j\in I:p_j=\cA_\GS(\rho_j)<\frac{1}{2}-2h}$ and its complement. \Ethan{TODO import and use the right ``given" command.}

	First suppose $j\in J$
	Let $Y=\sum y_j$ be i.i.d. Bernoulli trials with success probabilities $p_y=\frac{1}{2}-2h$,
	then clearly $\Prob{Z>\frac{m}{2}-hm}<\Prob{Y>\frac{m}{2}-hm}$.
	By Chernoff bound, we have
	$$\Prob{Y-(\frac{1}{2}-2\varepsilon)m\geq hm}\leq2e^{-\frac{\mu\delta^2}{3}}=\negl(T)$$
	where $\mu_y=mp_y$ and $\delta=\frac{2h}{1-4h}$.

	Summing up all these terms, we get
	$$\sum_{j\in J} w_j q_j<2^{-\lambda}=\negl(\lambda)$$
	So this case doesn't get accepted often enough to matter.

	Now, for the other case, $j\notin J$.
	$$P[\cA_\GS(\phi)=acc]>\frac{1}{2}-2\varepsilon\Rightarrow\braket{\phi|H_{C'(x)}|\phi}<\frac{1}{T^\lambda}$$
	we then have
	$$\braket{\rho_i|H_{C'(x)}|\rho_i}<\frac{1}{T^\lambda}.$$

	We also know that the least nonzero eigenvalue of $H_{C'(x)}$ is lower-bounded by $\frac{3}{4}$, so we obtain \Ethan{at least I think we do. Need theorem.}
	$$\braket{\rho_i|\phi_{C'(x)}}>1-O(T^{-\lambda})$$
	\Ethan{Now we need some kinda standard fidelity argument to show that the measurement results will be close too. I'm not familiar with them at the moment.}

	The probability of measuring $t<T$ on the clock register of $\ket{\psi_{C'(x)}}$ is $\varepsilon$,
	so the data register has $1-\varepsilon$ probability to be $C(x)$ at this point.

	The soundness errors incurred at each step is at most $O(T^{-\lambda})$, so the conclusion follows.
\end{prf}



%%%%
%%%% Keep the old version here for the record. 
%%%%

% \section{The History XZ Local Hamiltonian}
% \label{sec:Hamiltonian}

% In this section, we give a reduction from $\SampBQP$ to a local Hamiltonian instance.

% The local Hamiltonian problem is $\QMA$-complete, which means it's $\BQP$-hard.
% Protocols to delegate $\BQP$ computations such as \cite{FOCS:Mahadev18a} \Ethan{Are there any more?} take advantage of this fact and require the prover to send the corresponding $\QMA$ certificate.
% We take a similar approach to delegate $\SampBQP$ computations; however, we need to extract more information from the certificates.
% Local Hamiltonians have been extensively studied in contexts of adiabatic quantum computations, so we will summarize known results from sources such as \cite{adiabatic}.
% \Ethan{This source was given to us by Yu-Ching; I haven't read it. Also I probably should cite more sources here.}

% We begin by recalling the form that local Hamiltonian certificates:

% \begin{dfn}
% 	\label{dfn:groundstate}	
% 	Let $C=U_T\ldots U_1$ be a quantum circuit consisting of $T$ gates, and let $x\in\set{0,1}^n$. Then we denote
% 	$$\ket{\psi_{C(x)}}=\sum_{t=0}^TU_t\ldots U_1\ket{x}\otimes\ket{\hat{t}}$$
% 	We call the first register of $\ket{\psi_{C(x)}}$ the data register, and the second register of $\ket{\psi_{C(x)}}$ the time register.
% \end{dfn}

% \begin{rmk}
% 	\label{idpadding}
% 	One can first pad $C$ with identity gates at the end, doubling its number of gates and obtaining another circuit $C'$.
% 	Then by measuring the time register of $\ket{\psi_{C'(x)}}$ one obtains with probability $\frac{1}{2}$ the state $C(x)$.
% \end{rmk}

% We denote the set of $5$-local tensor products of $X$ and $Z$ gates as $\mathcal{G}_{XZ}$ as below; note that this spans a subspace of $5$-local Hamiltonians.
% \begin{dfn}
% 	$$\mathcal{G}_{XZ}=\set{U_0\otimes U_1\otimes\ldots\otimes U_n: U_i\in\set{I,X,Z}, \abs{\set{i: U_i\ne I}}\leq 5}$$
% \end{dfn}

% Now we state the main theorems that we'll work towards in this section.
% The first one is the reduction from $\SampBQP$ to a local Hamiltonian instance, with some additional properties that will be convenient for us:
% \begin{thm}
% 	\label{thm:LHReduction}
% 	Let $C$ be a circuit of size $T$. For any input string $x$, there exists a local Hamiltonian $H_{C(x)}$  with the following properties:
% 	\begin{itemize}
% 		\item $H_{C(x)}$ can be written as $\sum_{S\in\mathcal{G}_{XZ}} d_S S$ with at most $O(T)$ terms
% 		\item $H_{C(x)}$ has $\ket{\psi_{C(x)}}$ as the unique ground state with eigenvalue $0$.
% 		\item For any state $\ket\phi$ such that $\braket{\phi|\psi_{C(x)}}=0$,  $\braket{\phi|H_{C(x)}|\phi}>\frac{3}{4}$.
% 		\item Its descriptions can be constructed by a $\BPP$ machine in polynomial time.
% 	\end{itemize}
% \end{thm}

% The second one is a way to verify ground states of local Hamiltonians, again with some extra properties:
% \begin{thm}
% 	\label{ThmXZCheck}
% 	Given circuit $C$ and an input string $x\in\set{0,1}^*$, there exists a $\BQP$ algorithm that accepts $\ket{\phi}$ with probability $\frac{1}{2}-\Omega(\frac{1}{\poly(T)})\braket{\phi|H_{C(x)}|\phi}$. (In particular, it accepts the $\ket{\psi_{C(x)}}$ with probability $\fot$.) Furthermore, this algorithm doesn't apply any quantum gates, using only $X$ and $Z$ measurements.
% \end{thm}

% \subsection{Circuits of Toffoli and Hadamard gates}

% Here we restate the result from from \cite{quant-ph/0301040} showing that $\set{H, \Lambda^2(X)}$ is a universal gate set.
% Because of this result, we can assume without loss of generality that our quantum circuits consists of only Toffoli and Hadamard gates.
% Which is significant because this gate set is a linear combinations of tensor products of $X$ and $Z$ operators:
% \begin{thm}
% 	$H,\Lambda^2(X)\in\spn\mathcal{G}_{XZ}$ with $O(1)$ nonzero components.
% \end{thm}
% \begin{prf}
% 	$$H=\frac{1}{\sqrt{2}}(X+Z)$$
% 	$$\Lambda^2_{1,2}(X_3)=\ket{11}\bra{11}\otimes X+(I-\ket{11}\bra{11})\otimes I$$
% 	$$\ket{11}\bra{11}=\frac{1}{4}(I\otimes I+Z\otimes Z-I\otimes Z-Z\otimes I)$$
% \end{prf}

% Now we present the proof that $\set{H, \Lambda^2(X)}$ is indeed a universal gate set from \cite{quant-ph/0301040}: \Ethan{This definitely belongs in the appendix instead...}
% \begin{thm}
% 	Let $C$ be a circuit that:
% 	\begin{itemize}
% 		\item consists of $T$ gates, each either $H$ and $\Lambda(P(i))$
% 		\item uses $n$ qubits
% 	\end{itemize}
% 	Then a classical machine given $C$ can compute a circuit $C'$ that:
% 	\begin{itemize}
% 		\item consists of at most $4T$ gates, each either $H$ or $\Lambda^2(X)$ \hannote{don't like this notation}.
% 		\item uses $n+1$ qubits
% 		\item Let $x\in\set{0,1}^n$. Let $x|| 0\in\set{0,1}^{n+1}$ be $x$ concatenated with $0$.
% 		The measurement result of a tensor product of $X$ and $Z$ operators on the first $n$ qubits of $C'(x||0)$ has the same distribution as that on $C(x)$.
% 	\end{itemize}
% \end{thm}
% \begin{prf}
% 	We know that Hadamard gate and controlled phase gate is universal from \cite{kitaev_1997}
% 	$$H=\frac{1}{\sqrt{2}}\begin{pmatrix}1&1\\1&-1\end{pmatrix}$$
% 		$$\Lambda(P(i))=\begin{pmatrix}1&0&0&0\\0&1&0&0\\0&0&1&0\\0&0&0&i\end{pmatrix}$$

% 	Now consider the transform on quantum states
% 	$$\mathcal{F}(\ket{\phi})=(\Re\ket{\phi})\otimes\ket{0}+(\Im\ket{\phi})\otimes\ket{1}$$
% 	where $\Re$ and $\Im$ denote real and imaginary parts respectively.

% 	This transform commutes with Hadamard gates on the respective qubit. On the other hand, exchanging $\mathcal{F}$ with a controlled phase gate turn it into a combination of Hadamard and Toffoli gates. Mathematically,
% 	$$\mathcal{F}\circ H_s=H_s\circ\mathcal{F}$$
% 	$$\mathcal{F}\circ\Lambda_f(P(i)_s)=\Lambda^2_{f,s}(X_0Z_0)\circ\mathcal{F}=\Lambda^2_{f,s}(X_0)H_0\Lambda^2_{f,s}(X_0)H_0\circ\mathcal{F}$$

% 	We construct $C'$ so that $\mathcal{F}\circ C=C'\circ\mathcal{F}$ by following the computation above. It satisfies the required properties by construction.
% 	\begin{itemize}
% 		\item $C'$ uses only Hadamard and Toffoli gates.
% 		\item Exchanging $\mathcal{F}$ with $H$ doesn't change the circuit size. Exchanging $\mathcal{F}$ with controlled phase gate turns it into $4$ gates. So the final result is at most $4T$ gates.
% 		\item When $x$ is classical, $\mathcal{F}(x)=x||0$. So $F\circ C(x)=C'(x||0)$. It is simple \hannote{need a little more work?}to verify that $\mathcal{F}$ preserves $X$ and $Z$ measurement results on the first $n$ qubits.
% 	\end{itemize}
% \end{prf}

% \subsection{Reducing quantum circuit to local Hamiltonian}

% In this subsection, we prove \cref{thm:LHReduction} by showing how to reduce a $\SampBQP$ circuit to a local-$XZ$-Hamiltonian.

% 	The base construction comes from \cite{kitaev2002classical}. The simplification to $\spn\mathcal{G}_{XZ}$ is taken from \cite{PhysRevA.78.012352}.

% 	Let $x_i$ denote the $i$-th bit of $x$, and let $n$ be the number of qubits in $C$.

% 	We need to ensure the excited states of $H_{C(x)}$ have high eigenvalues.

% 	First, we ensure that the invalid clock states have high eigenvalues by applying the following Hamiltonian to the time register.
% 	$$H_{clock}=\sum_{t=1}^{T-1}\ket{01}\bra{01}_{t,t+1}$$
% 	As a sum of projections, clearly $H_{clock}\geq 0$. We shall also show that $H_{clock}\in\spn\mathcal{G}_{XZ}$.
% 	$$H_{clock}=\frac{1}{4}(Z_1 - Z_T) + \frac{1}{4}\sum_{t=1}^{T-1}(I-Z_tZ_{t+1}) $$
% 	This can be checked by fixing the first and last qubit, then doing induction on the number of switches.

% 	Then, we ensure that the initial condition is set up correctly.
% 	$$H_{in}=\sum_{i=1}^n(I-\ket{x_i}\bra{x_i})\otimes\ket{0}\bra{0}_1$$
% 	The kernel of this is precisely where everything is set up consistently with $\phi$ in time step $t=0$. Furthermore, $H_{in}\in\spn\mathcal{G}_{XZ}$
% 	$$H_{in}=\frac{1}{4}\sum_{i=1}^n(I-(-1)^{x_i}Z_i)\otimes(I+Z_1)$$

% 	Then, we ensure that the gates are applied correctly.
% 	$$H_{prop}=\sum_{t\in T_1}H_{prop,t}$$
% 	$$H_{prop,t}=I\otimes\ket{\widehat{t}}\bra{\widehat{t}}
% 	+I\otimes\ket{\widehat{t-1}}\bra{\widehat{t-1}}
% 	-U_t\otimes\ket{\widehat{t}}\bra{\widehat{t-1}}
% 	-U_t^\dagger\otimes\ket{\widehat{t-1}}\bra{\widehat{t}}$$

% 	We have $H_{prop}\geq 0$ according to \cite{2002quant.ph.10077A}; in fact, the least nonzero eigenvalue of $H_{prop}$ is lower bounded by $\frac{1}{2(T+1)^2}$.

% 	We can also write $H_{prop}\in\spn\mathcal{G}_{XZ}$. Note that $U^\dagger=U$, since our gates are either Hadamard or Toffoli.
% 	Additionally, $\frac{1}{2}(I-Z_{t-1})$ annihilates time steps before $t-1$. $\frac{1}{2}(I+Z_{t+1})$ similarly annihilates steps $t+1$ and after.
% 	$$H_{prop,t}=\frac{I}{4}\otimes(I-Z_{t-1})(I+Z_{t+1})-\frac{U}{4}\otimes(I-Z_{t-1})X_t(I+Z_{t+1})$$
% 	Extra care must be taken for boundary cases.
% 	$$H_{prop,1}=\frac{1}{2}(I+Z_2)-U_1\otimes\frac{1}{2}(X_1+X_1Z_2)$$
% 	$$H_{prop,T}=\frac{1}{2}(I-Z_{t-1})-U_T\otimes\frac{1}{2}(X_T-Z_{T-1}X_T)$$

% 	To properly combine the three Hamiltonians defined and analyze it using the projection lemma (\cref{thm:projection}), we consider the kernels of the Hamiltonians we defined.
% 	$$K_{clock}=\ker H_{clock}$$
% 	$$K_{in}=\ker H_{in}$$
% 	$$K_{prop}=\ker H_{prop}$$

% 	Clearly,
% 	$$K_{clock}=\set{\sum_{t=1}^T \ket{\phi_t}\otimes\ket{\hat{t}}:\ket{\phi_t}\in\cB^{\otimes n}}$$
% 	Let $\ket\phi\in K_{clock}\cap K_{prop}$.
% 	Consider some $\widetilde{t}$ such that $\braket{\widetilde{t}|\phi}\ne0$.
% 	As $\ket\phi\in K_{prop}$, we can then do induction using $\widetilde{t}$ as base case to get the relations between the $\ket{\phi_t}$s and conclude that $\ket\phi$ must have the form:
% 	$$\ket\phi=\sum_{t=0}^TU_t\ldots U_1\ket{y}\otimes\ket{\hat{t}}$$
% 	$$\Rightarrow K_{clock}\cap K_{prop}=\set{\sum_{t=0}^TU_t\ldots U_1\ket{y}\otimes\ket{\hat{t}}: \ket{y}\in\mathcal{B}^{\otimes n}}$$

% 	As a direct result, we get
% 	$$K_{clock}\cap K_{in}\cap K_{prop}=\spn\set{\ket{\psi_{C(x)}}}$$

% We now consider the space outside the desired state.
% $$S=(\spn\set{\ket{\psi_{C(x)}}})^\bot$$
% $$H_{clock}\big|_S,H_{in}\big|_S,H_{prop}\big|_S$$
% To combine the Hamiltonians, we apply the projection lemma twice.
% $$\exists J_{clock}
% =\frac{\poly\left(\norm{H_{in}\big|_S}\right)}{\lambda\left(H_{clock}\big|_{S\cap K^\bot_{clock}}\right)}
% =O(n)=O(T)$$
% $$\lambda(H_{in}\big|_S+J_{clock}H_{clock}\big|_S)\geq
% \lambda(H_{in}\big|_{S\cap K_{clock}})-\frac{1}{8}$$
% $$\exists J_{prop}=\frac{\poly\left(\norm{H_{in}\big|_S+J_{clock}H_{clock}\big|_S}\right)}{\lambda\left(H_{prop}\big|_{S\cap K^\bot_{prop}}\right)}
% =\frac{O(n+T)}{\Omega(T^{-2})}=O(T^3)$$
% $$\lambda(H_{in}\big|_S+J_{clock}H_{clock}\big|_S+J_{prop}H_{prop}\big|_S)\geq
% \lambda(H_{in}\big|_{S\cap K_{clock}\cap K_{prop}})-\frac{1}{4}$$
% \begin{align*}
% 	S\cap K_{clock}\cap K_{prop}&=S\cap\set{\sum_{t=0}^TU_t\ldots U_1\ket{y}\otimes\ket{\hat{t}}|\ket{y}\in\mathcal{B}^{\otimes n}}\\
% 	&=\set{\sum_{t=0}^TU_t\ldots U_1\ket{y}\otimes\ket{\hat{t}}:\braket{\psi_{C(x)}|y}=0}
% \end{align*}
% $$\Rightarrow\lambda((H_{in}+J_{clock}H_{clock}+J_{prop}H_{prop})\big|_S)\geq\frac{3}{4}$$
% So we set $H_{C(x)}=H_{in}+J_{clock}H_{clock}+J_{prop}H_{prop}$, which satisfies the required properties by construction.

% \subsection{Checking the ground state of the local Hamiltonian}

% In this subsection, we prove \cref{ThmXZCheck}.
% That is, we present and analyze an algorithm that checks whether a given state is the ground state of some fixed $H_{C(x)}$, following~\cite{PhysRevA.93.022326}.

% \begin{algorithm}
% 	\caption{Check for ground state}
% 	\label{AlgGroundStateCheck}
% 		Let $H=\sum_{S\in\mathcal{G}_{XZ}} d_S S$.
% 		Let $\ket\phi$ be the potential ground state to check.
% 		\begin{itemize}
% 			\item Set $D = \sum_{S\in\mathcal{G}_{XZ}}|d_S|$
% 			\item Set $p_S = \frac{|d_S|}{D}$
% 			\item Sample $\widetilde{S}$ from $\mathcal{G}_{XZ}$, weighted by $p_S$.
% 			\item Measure $\ket\phi$ in the $\widetilde{S}$ basis, recording the result as $\lambda_{\widetilde{S}}$.
% 			\item If $\sgn(d_{\widetilde{S}})\lambda_{\widetilde{S}}=-1$, accept. Otherwise, reject.
% 		\end{itemize}
% \end{algorithm}

% \begin{thm}
% 	Let $H$ have $O(\poly(T))$ nonzero components whose coefficients at most $O(\poly(T))$.
% 	Then \cref{AlgGroundStateCheck} accepts $\ket{\phi}$ with probability $\frac{1}{2}-\Omega(\frac{1}{\poly(T)})\braket{\phi|H_{C(x)}|\phi}$.
% \end{thm}
% \begin{prf}
% 	Here we follow \cite{PhysRevA.93.022326}.
% 	\begin{align*}
% 		\frac{1}{D}\braket{\phi|H|\phi}&=\sum_{S\in\mathcal{G}_{XZ}} p_S\sgn(d_S)\braket{\phi|S|\phi}\\
% 		&=\sum_{S\in\mathcal{G}_{XZ}} p_S\sgn(d_S)\E[\lambda_S]\\
% 		&=\E_{\widetilde{S}}[\sgn(d_{\widetilde{S}})\E[\lambda_{\widetilde{S}}]]\\
% 		&=\E_{\widetilde{S}}[\sgn(d_{\widetilde{S}})\lambda_{\widetilde{S}}]
% 	\end{align*}

% 	Note that $\sgn(d_{\widetilde{S}})\lambda_{\widetilde{S}}=\pm1$. Let $p$ be the probability that $\sgn(d_{\widetilde{S}})\lambda_{\widetilde{S}}=-1$.
% 	$$\Rightarrow \frac{1}{D}\braket{\phi|H|\phi}=\E_{\widetilde{S}}[\sgn(d_{\widetilde{S}})\lambda_{\widetilde{S}}]=-p+(1-p)$$
% 	\begin{align*}
% 		\Rightarrow p&=\frac{1}{2}-\frac{1}{2D}\braket{\phi|H|\phi}\\
% 		&=\frac{1}{2}-\Omega\left(\frac{1}{\poly(T)}\right)\braket{\phi|H|\phi}
% 	\end{align*}
% \end{prf}

% The second one is a way to verify ground states of local Hamiltonians, again with some extra properties:
% \begin{thm}
% 	\label{ThmXZCheck}
% 	Given circuit $C$ and an input string $x\in\set{0,1}^*$, there exists a $\BQP$ algorithm that accepts $\ket{\phi}$ with probability $\frac{1}{2}-\Omega(\frac{1}{\poly(T)})\braket{\phi|H_{C(x)}|\phi}$. (In particular, it accepts the $\ket{\psi_{C(x)}}$ with probability $\fot$.) Furthermore, this algorithm doesn't apply any quantum gates, using only $X$ and $Z$ measurements.
% \end{thm}

%\subsection{Circuits of Toffoli and Hadamard gates}

% Old notation: Let $\Lambda_c(U)$ denote the gate $U$ controlled on qubit $c$, and $\Lambda^2_{f, s}(U)$ denote the gate $U$ controlled by both $f$ and $s$. I.e. $\Lambda^2_{1, 2}(X_3)$ would be a Toffoli gate.
% \Ethan{TODO: replace with something that looks better and doesn't conflict with the use of $\Lambda$ as measurement.}


% Now we present the proof that $\set{H, \Lambda^2(X)}$ is indeed a universal gate set from \cite{quant-ph/0301040}: \Ethan{This definitely belongs in the appendix instead...}
% \begin{thm}
% 	Let $C$ be a circuit that:
% 	\begin{itemize}
% 		\item consists of $T$ gates, each either $H$ and $\Lambda(P(i))$
% 		\item uses $n$ qubits
% 	\end{itemize}
% 	Then a classical machine given $C$ can compute a circuit $C'$ that:
% 	\begin{itemize}
% 		\item consists of at most $4T$ gates, each either $H$ or $\Lambda^2(X)$ \hannote{don't like this notation}.
% 		\item uses $n+1$ qubits
% 		\item Let $x\in\set{0,1}^n$. Let $x|| 0\in\set{0,1}^{n+1}$ be $x$ concatenated with $0$.
% 		The measurement result of a tensor product of $X$ and $Z$ operators on the first $n$ qubits of $C'(x||0)$ has the same distribution as that on $C(x)$.
% 	\end{itemize}
% \end{thm}
% \begin{prf}
% 	We know that Hadamard gate and controlled phase gate is universal from \cite{kitaev_1997}
% 	$$H=\frac{1}{\sqrt{2}}\begin{pmatrix}1&1\\1&-1\end{pmatrix}$$
% 		$$\Lambda(P(i))=\begin{pmatrix}1&0&0&0\\0&1&0&0\\0&0&1&0\\0&0&0&i\end{pmatrix}$$

% 	Now consider the transform on quantum states
% 	$$\mathcal{F}(\ket{\phi})=(\Re\ket{\phi})\otimes\ket{0}+(\Im\ket{\phi})\otimes\ket{1}$$
% 	where $\Re$ and $\Im$ denote real and imaginary parts respectively.

% 	This transform commutes with Hadamard gates on the respective qubit. On the other hand, exchanging $\mathcal{F}$ with a controlled phase gate turn it into a combination of Hadamard and Toffoli gates. Mathematically,
% 	$$\mathcal{F}\circ H_s=H_s\circ\mathcal{F}$$
% 	$$\mathcal{F}\circ\Lambda_f(P(i)_s)=\Lambda^2_{f,s}(X_0Z_0)\circ\mathcal{F}=\Lambda^2_{f,s}(X_0)H_0\Lambda^2_{f,s}(X_0)H_0\circ\mathcal{F}$$

% 	We construct $C'$ so that $\mathcal{F}\circ C=C'\circ\mathcal{F}$ by following the computation above. It satisfies the required properties by construction.
% 	\begin{itemize}
% 		\item $C'$ uses only Hadamard and Toffoli gates.
% 		\item Exchanging $\mathcal{F}$ with $H$ doesn't change the circuit size. Exchanging $\mathcal{F}$ with controlled phase gate turns it into $4$ gates. So the final result is at most $4T$ gates.
% 		\item When $x$ is classical, $\mathcal{F}(x)=x||0$. So $F\circ C(x)=C'(x||0)$. It is simple \hannote{need a little more work?}to verify that $\mathcal{F}$ preserves $X$ and $Z$ measurement results on the first $n$ qubits.
% 	\end{itemize}
% \end{prf}
