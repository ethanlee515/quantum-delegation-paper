\section{Delegation of Quantum Sampling Problems} \label{sec:samp_definition}

In this section, we formally introduce the task of delegation for quantum sampling problems. We start by recalling the complexity class $\SampBQP$ defined by Aaronson~\cite{aaronson_2013, Boson}, which captures the class of sampling problems that
are approximately solvable by polynomial-time quantum algorithms.


\begin{definition} [Sampling Problem]
    \label{dfn:sampling-problem}
    A \emph{sampling problem} is a collection of probability distributions $(D_x)_{x\in\set{0, 1}^*}$, one for each input string $x\in\set{0,1}^n$, where $D_x$ is a distribution over $\set{0,1}^{m(n)}$ for some fixed polynomial $m$.
\end{definition}

\begin{definition} [$\SampBQP$]
    $\SampBQP$ is the class of sampling problems $\left(D_x\right)_{x\in\set{0, 1}^*}$ that can be (approximately) sampled by polynomial-size uniform quantum circuits. Namely, there exists a Turing machine $M$ such that for every $n \in \bbN$ and $\eps \in (0,1)$, $M(1^n, 1^{1/\eps})$ outputs a quantum circuit $C$ in $\poly(n, 1/\eps)$ time such that for every $x \in \zo^n$, the output of $C(x)$ (measured in standard basis) is $\eps$-close to $D_x$ in the total variation distance. 
\end{definition}

Note that in the above definition, there is an accuracy parameter $\eps$ and the quantum sampling algorithm only requires to output a sample that is $\eps$-close to the correct distribution in time $\poly(n,1/\eps)$.
%, instead of requiring a perfect or negligibly close sample.
\cite{aaronson_2013, Boson} discussed multiple reasons for allowing the inverse polynomial error, such as to take into account the inherent noise in conceivable physical realizations of quantum computer.
%may consist of inherent noise.
%There are multiple reasons for allowing an inverse polynomial error. For example, conceivable physical realizations of quantum computer may consist of inherent noise. We refer the readers to~\cite{aaronson_2013, Boson} for the discussion therein on why this better captures the sampling problem solvable by realistic quantum computers. 
On the other hand, it is also meaningful to require negligible error. As discussed, it is an intriguing open question to delegate quantum sampling problem with negligible error.  

%However, we note that it is also meaningful to require an exponentially small error instead of an inverse polynomial one. %We follow the definition of~\cite{aaronson_2013}, but note that  

We next define what it means for a $\QPIP_\tau$ protocol\footnote{See our full version \cite{full-version} for a formal definition of $\QPIP_\tau$.} to solve a $\SampBQP$ problem $\left(D_x\right)_{x\in\set{0, 1}^*}$.
Since sampling problems come with an accuracy parameter $\eps$, we let the prover $P$ and the verifier $V$ receive the input $x$ and $1^{1/\eps}$ as common inputs. 
Completeness is straightforward to define, which requires that when the prover $P$ is honest, the verifier $V$ should accept with high probability and output a sample $z$ distributed close to $D_x$ on input $x$. Defining soundness is more subtle. Intuitively, it requires that the verifier $V$ should never be ``cheated'' to accept and output an incorrect sample even when interacting with a malicious prover. We formalize this by a strong simulation-based definition, where we require that the joint distribution of the decision bit $d \in \set{\Acc, \Rej}$ and the output $z$ (which is $\bot$ when $d = \Rej$) is $\eps$-close (in either statistical or computational sense) to an ``ideal distribution'' $(d,z_{ideal})$, where $z_{ideal}$ is sampled from $D_x$ when $d = \Acc$ and set to $\bot$ when $d = \Rej$. Since the protocol receives the accuracy parameter $1^{1/\eps}$ as input to specify the allowed error, we do not need to introduce an additional soundness error parameter in the definition.

\begin{definition}
    \label{dfn:stats-secure-proto-sampbqp}
    Let $\Pi=(P, V)$ be a $\QPIP_\tau$ protocol.
    We say it is a protocol for the $\SampBQP$ instance $(D_x)_{x\in\zo^*}$ with completeness error $c(\cdot)$ and statistical (resp., computational) soundness if the following holds:
    \begin{itemize}
        \item On public inputs $1^\lambda$, $1^{1/\eps}$, and $x\in\zo^{\poly(\lambda)}$, $V$ outputs $(d, z)$ where $d\in\set{\Acc, \Rej}$.
            If $d=\Acc$ then $z\in\zo^{m(\abs{x})}$ where $m$ is given in \Cref{dfn:sampling-problem}, otherwise $z=\bot$.
        \item (Completeness):
            For all accuracy parameters $\eps(\lambda)=\frac{1}{\poly(\lambda)}$,
            security parameters $\lambda\in\bbN$,
            and $x\in\zo^{\poly(\lambda)}$,
            let $(d, z)\leftarrow(P, V)(1^\lambda, 1^{1/\eps}, x)$, then $d=\Rej$ with probability at most $c(\lambda)$.
        \item (Statistical soundness): For all cheating provers $P^*$,
            accuracy parameters $\eps(\lambda)=\frac{1}{\poly(\lambda)}$,
            sufficiently large $\lambda\in\bbN$, and $x\in\zo^{\poly(\lambda)}$,
            consider the following experiment:
            \begin{itemize}
                \item Let $(d, z)\leftarrow(P^*, V)(1^\lambda, 1^{1/\eps}, x)$.
                \item Define $z_{ideal}$ by
                $$\begin{cases}
                    z_{ideal}=\bot & \text{if } d=\Rej\\
                    z_{ideal}\leftarrow D_x & \text{if } d=\Acc
                \end{cases}$$.
            \end{itemize}
            It holds that $\norm{(d,z)-(d,z_{ideal})}_{\mathrm{TV}}\leq\eps$.
		\item (Computational soundness):
        For all cheating $\BQP$ provers $P^*$, $\BQP$ distinguishers $\mathsf{D}$, accuracy parameters $\eps(\lambda)=\frac{1}{\poly(\lambda)}$,
            sufficiently large $\lambda\in\bbN$, and all $x\in\zo^{\poly(\lambda)}$,
            let us define $d, z, z_{ideal}$ by the same experiment as above.
            It holds that $(d, z)$ is $\eps$-computationally indistinguishable to $(d, z_{ideal})$ over $\lambda$.
    \end{itemize}
\end{definition}

As in the case of $\BQP$, we are particularly interested in the case that $\tau = 0$, i.e., when the verifier $V$ is classical. In this case, we say that $\Pi$ is a CVQC protocol for the $\SampBQP$ problem $(D_x)_{x\in\zo^*}$.

%To the best of our knowledge, we are the first to formally define delegation of quantum sampling problems. We also note that while several constructions in the relaxed models (e.g., verifiable blind computation~\cite{FK17}) may be generalized naturally to delegate quantum sampling problem and allow the verifier to learn multi-bit outputs, it seems non-trivial to show that these constructions achieve the soundness property we just defined. Proving the soundness of these constructions for delegating $\SampBQP$ is an interesting open question. %We also note that 

