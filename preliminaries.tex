\section{Preliminaries}

\subsection{Notations}

Let $\mathcal{B}$ denote the Hilbert space corresponding to a qubit.
Let $H:\mathcal{B}^{\otimes N}\rightarrow\mathcal{B}^{\otimes N}$ be Hermitian.

\begin{definition}
	Let $H\geq0$ denote $H$ is positive semidefinite.
\end{definition}

\begin{definition}
	Let $\lambda(H)$ as the least eigenvalue of $H$.
\end{definition}

\begin{definition}
	Let the \emph{ground state} of $H$ be the eigenvector corresponding to $\lambda(H)$.
\end{definition}

\begin{definition}
	Let $H\big|_S=\prod_SH\prod_S$, where $\prod_S$ is the projection onto the subspace $S$.
\end{definition}

\begin{definition}
	Let $\ket{\widehat{t}}=\ket{11\ldots1} \ket{00\ldots0}$; $t$ $1$s followed by all $0$s. $t$ represented in unary.
\end{definition}

\begin{definition}
	Let $P(i)=\begin{pmatrix}1&0\\0&i\end{pmatrix}$, $X=\begin{pmatrix}0&1\\1&0\end{pmatrix}$, $Z=\begin{pmatrix}1&0\\0&-1\end{pmatrix}$
\end{definition}

\begin{definition}
	Let $\Lambda_c(U)$ denote the gate $U$ controlled on qubit $c$, and $\Lambda^2_{f, s}(U)$ denote the gate $U$ controlled by both $f$ and $s$. I.e. $\Lambda^2_{1, 2}(X_3)$ would be a Toffoli (CCNOT) gate.
\end{definition}

\begin{definition}
	Let $H_1, \ldots, H_n:\mathcal{B}^{\otimes N}\rightarrow\mathcal{B}^{\otimes N}$ be Hermitians each acting on at most $k$ qubits, then $$H=\sum_{j=1}^nH_j$$ is called a \emph{$k$-local Hamiltonian}.
\end{definition}

We define $\SampBQP$ based on \cite{aaronson_2013}.

\begin{definition}
	A sampling problem $S$ is a collection of probability distributions $(D_x)$, one for each input string $x\in\set{0,1}^n$, where $D_x$ is a distribution over $\set{0,1}^{p(n)}$ for some fixed polynomial $p$.
\end{definition}

\begin{definition}
	$\SampBQP$ is the class of sampling problems $S=(D_x)$ for which for all $\varepsilon$ there exists a quantum circuit constructible in time $\poly(n, \frac{1}{\varepsilon})$ that, given $x$ as input, samples from from a probability distribution $\tilde{D}_x$ such that $\norm{\tilde{D}_x-D_x}<\varepsilon$.
\end{definition}

We then define models of interation between a classical client and a quantum server. This is taken from \cite{mahadev_delegation}.

\begin{definition}
	A sampling problem $S=(D_x)$ is said to be in $\QPIP_0$ with completeness $c$ and soundness $s$ if there exists a pair of algorithms $(\bbP, \bbV)$ with the following properties with input $x$:
	\begin{itemize}
		\item $\bbP$ is a $\BQP$ machine.
		\item $\bbV$ is a $\BPP$ machine.
		\item There is a classical communication channel between $\bbP$ and $\bbV$.
		\item (Completeness) After interacting with $\bbP$, $\bbV$ accepts with probability $\geq c$.
		\item (Soundness) Given $\bbP'$, if $(\bbP', \bbV)$ accepts with probability greater than $\delta$, then conditioned on this acceptance, $\bbV$'s outputs has distribution $\tilde{D}$ with $\norm{\tilde{D}-D_x}\leq s$.
	\end{itemize}
\end{definition}

\begin{definition}
	A sampling problem $S=(D_x)$ is said to be in $\QPIP_0$ with completeness $c$ and soundness $s$ if there exists a pair of algorithms $(\bbP, \bbV)$ defined in the same way as in $\QPIP_{XZ}$ except that $\bbV$ is fully classical and cannot receive or measure qubits.
\end{definition}

\subsection{Projection Lemma}

Here is a lemma that we will use, taken from \cite{kempe_kitaev_regev_2006}.
\begin{theorem}
	Let $H_1, H_2$ be local Hamiltonians where $H_2\geq0$. Let $K=\ker H_2$.
	$$\exists J=\frac{\poly(\norm{H_1})}{\lambda(H_2|_{K^\bot})}$$
	$$\lambda(H_1+JH_2)\geq\lambda(H_1\big|_K)-\frac{1}{8}$$
\end{theorem}

\subsection{Quantum de Finetti Theorem under Local Measurements}

Here is a theorem taken from \cite{Brandão2017}.
\begin{theorem}
	\label{deFinetti}
	Let $\rho^{A_1\ldots A_k}$ be a permutation-invariant state. Then for every $0\leq l\leq k$ there is a measure $\nu$ on $D(A)$ such that
	$$\max_{\Lambda_2,\ldots,\Lambda_l\in\mathcal{M}}\norm{(id\otimes\Lambda_1\otimes\ldots\otimes\Lambda_l)\left(\rho^{A_1\ldots A_l}-\int \nu(d\sigma)\sigma^{\otimes l}\right)}_1\leq\sqrt{\frac{2l^2\ln\abs{A}}{k-l}}$$
\end{theorem}

\Ethan{Explain the terminologies. Also check do something about the integral vs weighted sum}
