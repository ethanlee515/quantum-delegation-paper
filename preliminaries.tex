\section{Preliminaries}

\subsection{Notations}

Let $\mathcal{B}$ denote the Hilbert space corresponding to a qubit.
Let $H:\mathcal{B}^{\otimes N}\rightarrow\mathcal{B}^{\otimes N}$ be Hermitian.

\begin{definition}
	Let $H\geq0$ denote $H$ is positive semidefinite.
\end{definition}

\begin{definition}
	Let $\lambda(H)$ as the least eigenvalue of $H$.
\end{definition}

\begin{definition}
	Let the \emph{ground state} of $H$ be the eigenvector corresponding to $\lambda(H)$.
\end{definition}

\begin{definition}
	Let $H\big|_S=\prod_SH\prod_S$, where $\prod_S$ is the projection onto the subspace $S$.
\end{definition}

\begin{definition}
	Let $\ket{\widehat{t}}=\ket{11\ldots1} \ket{00\ldots0}$; $t$ $1$s followed by all $0$s. $t$ represented in unary.
\end{definition}

\begin{definition}
	Let $\Lambda_c(U)$ denote the gate $U$ controlled on qubit $c$. $\Lambda_{f, s}(U)$ would be the gate $U$ controlled by both $f$ and $s$. I.e. $\Lambda{1, 2}(X_3)$ would be a Toffoli ((CCNOT) gate.
\end{definition}

\begin{definition}
	Let $P(i)=\begin{pmatrix}1&0\\0&i\end{pmatrix}$
\end{definition}

\subsection{Local Hamiltonian}

The local Hamiltonian problem is defined as follows:

\begin{definition}
	Let $H_1, \ldots, H_n:\mathcal{B}^{\otimes N}\rightarrow\mathcal{B}^{\otimes N}$ be Hermitian. Then $$H=\sum_{j=1}^nH_j$$ is called a \emph{Hamiltonian}.
	If each $H_j$ acts on at most $k$ qubits, then $H$ is called a \emph{$k$-local Hamiltonian}.
\end{definition}

The local Hamiltonian problem is given by let $a<b$ with some inverse polynomial gap, let $H=\sum H_j$, determine whether $\lambda(H)<a$ or $\lambda(H)>b$..

\begin{theorem}
	The Local Hamiltonian problem is QMA-complete.
\end{theorem}

\subsection{Projection Lemma}

Here is a lemma that we will use, taken from \cite{kempe_kitaev_regev_2006}.

Let $H_1, H_2$ be Hamiltonians where $H_2\geq0$, and $S=\ker H_2$.
Then $\exists J\in\mathbb{R}$ s.t.
$\lambda(H_1\big|_S)-\frac{1}{8}\leq
	\lambda(H_1+JH_2)\leq\lambda(H_1\big|_S)$.


