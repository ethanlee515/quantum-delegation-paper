\section{Preliminaries}

\Ethan{Check that the quantifiers are all there}

\subsection{Notations}

Let $\mathcal{B}$ be the Hilbert space corresponding to one qubit. Let $H:\mathcal{B}^{\otimes n}\rightarrow\mathcal{B}^{\otimes n}$ be Hermitian matrices. We use $H\geq0$ to denote $H$ being positive semidefinite. Let $\lambda(H)$ be the smallest eigenvalue of $H$. The ground states of $H$ are the eigenvectors corresponding to $\lambda(H)$. For matrix $H$ and subspace $S$, let $H\big|_S=\Pi_S H \Pi_S$, where $\Pi_S$ is the projector onto the subspace $S$. For a $T$-qubit Hilbert space, let the state $\ket{\widehat{t}}=\ket{1}^{\otimes t}\otimes \ket{0}^{{\otimes (T-t)}}$.
For $n$-qubit states $\rho_1, \rho_2\in\cB^{\otimes n}$,
we write $F(\rho_1, \rho_2)=\left(\tr\sqrt{\sqrt{\rho_1}\rho_2\sqrt{\rho_1}}\right)^2$ for the fidelity between $\rho_1$ and $\rho_2$.
We write $\norm{\rho_1-\rho_2}_{tr}=\frac{1}{2}\norm{\rho_1-\rho_2}_1$ for the trace distance between $\rho_1$ and $\rho_2$.
We have $\norm{\rho_1-\rho_2}_{tr}\leq\sqrt{1-F(\rho_1, \rho_2)}$.
For two densities $f_1$ and $f_2$ over the same finite domain $X$, we write $\norm{f_1-f_2}_{TV}=\frac{1}{2}\sum_{x\in X}\abs{f_1(x)-f_2(x)}$ for their total variation distance.

\begin{definition} [quantum-classical channels]
	\label{def:QCChannel}
	A quantum measurement is given by a set of matrices $\set{M_k}$ such that $M_k\geq0$ and $\sum_k M_k=\id$.
	We associate to any measurement a map $\Lambda(\rho)=\sum_k \tr(M_k\rho)\ket{k}\bra{k}$
	with $\set{\ket{k}}$ an orthonormal basis.
	This map is also called a \emph{quantum-classical channel}.
\end{definition}

The phase gate and Pauli matrices are the following matrices.

\begin{definition}
	$P(i)=\begin{pmatrix}1&0\\0&i\end{pmatrix}$,
	$X=\begin{pmatrix}0&1\\1&0\end{pmatrix}$,
	$Y=\begin{pmatrix}0&-i\\i&0\end{pmatrix}$,
	$Z=\begin{pmatrix}1&0\\0&-1\end{pmatrix}$
\end{definition}

\Ethan{I removed the definition of $\BQP$ since nobody else defines it in their papers anyways}

We define search and sampling versions of $\BQP$ based on \cite{aaronson_2013}.

\begin{definition} [sampling problem]
	A \emph{sampling problem} is a collection of probability distributions $(D_x)_{x\in\set{0, 1}^*}$, one for each input string $x\in\set{0,1}^n$, where $D_x$ is a distribution over $\set{0,1}^{p(n)}$ for some fixed polynomial $p$.
\end{definition}

\begin{definition} [$\SampBQP$]
	$\SampBQP$ is the class of sampling problems $\left(D_x\right)_{x\in\set{0, 1}^*}$ for which there exists an efficient $\BPP$ machine that, on input $(n, 1^{1/\varepsilon})$, outputs a quantum circuit (that is, a sequence of unitary gates) $C$ such that for all $x\in\zo^n$, the standard basis measurement outcome on some designated output register \Ethan{Feels incredibly wordy. Not sure if the word ``register" here feels right either.} of $C(x)$ is $\varepsilon$-close to $D_x$.
\end{definition}

\subsection{Quantum Prover Interactive Protocol (QPIP)}
We model the interaction between a (almost classical) client and a quantum server following \cite{FOCS:Mahadev18a}.

\begin{definition}
	A protocol $\Pi=(P, V)$ is said to be in $\QPIP_\tau$ if it satisfies the following properties:
	\begin{itemize}
		\item $P$ is run by the prover, a $\BQP$ machine, which also has access to a quantum channel that can transmit $\tau$ qubits to the verifier per use.
		\item $V$ is run by the verifier, which is a hybrid machine of a classical part and a limited quantum part. The classical part is a $\BPP$ machine. The quantum part is a register of $\tau$ qubits, on which the verifier can perform arbitrary quantum operations and which has access to a quantum channel which can transmit $\tau$ qubits. At any given time, the verifier is not allowed to possess more than $\tau$ qubits. The interaction between the quantum and classical parts of the verifier is the usual one: the classical part controls which operations are to be performed on the quantum register, and outcomes of measurements of the quantum register can be used as input to the classical part.
		\item There is also a classical communication channel between the prover and the verifier, which can transmit polynomially \Ethan{Is this good enough, or should I try to say poly in input length? Input isn't defined yet.} many bits to either direction. 
	\end{itemize}
\end{definition}

\begin{definition}
	We say $\Pi=(P, V)$ is a protocol for the langauge $L$ with completeness error $c$ and soundness error $s$ if for all $x\in\zo^*$ and security parameter $1^\lambda$:
	\begin{itemize}
		\item If $x\in L$, then $\Pr\left[(P, V)(x, 1^\lambda)=\Rej\right]<c(\lambda)$.
		\item If $x\notin L$, then for all cheating prover $P^*$, $\Pr\left[(P^*, V)(x, 1^\lambda)=\Acc\right]<s(\lambda)$.
	\end{itemize}
\end{definition}

For $\SampBQP$ protocols, the accuracy error is encoded in the $\SampBQP$ instance itself, so there are no soundness parameters.

\begin{definition}
	Let $\Pi=(P, V)$ be a protocol with security parameter $\lambda$ and accuracy parameter $\varepsilon=\frac{1}{\poly(\lambda)}$.
	We say $\Pi$ is a statistically sound protocol for the sampling problem $(D_x)_{x\in\zo^*}$ with completeness error $c$ if for all $x\in\zo^*$:
	\begin{itemize}
		\item At the end of the protocol, $V$ outputs $(d, z)$ where $d\in\set{\Acc, \Rej}$.
		\item Let $(d, z)\leftarrow(P, V)(x, 1^{1/\varepsilon}, 1^\lambda)$, then $d=\Rej$ with probability at most $c(\lambda)$.
		\item For all cheating prover $P^*$, let $(d, z)\leftarrow(P^*, V)(x, 1^{1/\varepsilon}, 1^\lambda)$.
			Define $z_{ideal}$ by
			$$\begin{cases}
				z_{ideal}=\bot & \text{if } d=\Rej\\
				z_{ideal}\leftarrow D_x & \text{if } d=\Acc
			\end{cases}$$
			then $\norm{(d,z)-(d,z_{ideal})}_{TV}\leq\varepsilon$.
	\end{itemize}
\end{definition}

As sometimes one only achieves computational guarantees instead of statistical ones;
we also model the corresponding situation for $\SampBQP$ protocols.

\begin{definition}
	Let $\Pi=(P, V)$ be a protocol with security parameter $\lambda$ and accuracy parameter $\varepsilon=\frac{1}{\poly(\lambda)}$.
	We say $\Pi$ is a computationally sound protocol for the sampling problem $(D_x)_{x\in\zo^*}$ with completeness error $c$ if for all $x\in\zo^*$:
	\begin{itemize}
		\item At the end of the protocol, $V$ outputs $(d, z)$ where $d\in\set{\Acc, \Rej}$.
		\item Let $(d, z)\leftarrow(P, V)(x, 1^{1/\varepsilon}, 1^\lambda)$, then $d=\Rej$ with probability at most $c(\lambda)$.
		\item For all cheating prover $P^*$, let $(d, z)\leftarrow(P^*, V)(x, 1^{1/\varepsilon}, 1^\lambda)$.
			Define $z_{ideal}$ by
			$$\begin{cases}
				z_{ideal}=\bot & \text{if } d=\Rej\\
				z_{ideal}\leftarrow D_x & \text{if } d=\Acc
			\end{cases}$$
			then for all adversary, its probability to successfully distinguish a sample of $(d, z)$ from that of $(d, z_{ideal})$ is less than $\frac{1}{2}+\frac{\varepsilon}{2}$
			\Ethan{Something like this? Or need protocol or specify prior probability is 1/2 each?}
	\end{itemize}
\end{definition}

We define blindness of a protocol as follows.

\begin{definition}
	Let $\lambda$ be a security parameter.
	Let $(\bbP, \bbV)$ be an interactive protocol with security parameter $\lambda$.
	Then it is \emph{blind} if the following ensembles are computationally indistinguishable over $\lambda$:
	\begin{itemize}
		\item $\set{\View_{P^*}(P^*, V(x))(1^\lambda)}_{\lambda\in\bbN, x\in\zo^n}$
		\item $\set{\View_{P^*}(P^*, V(0))(1^\lambda)}_{\lambda\in\bbN, x\in\zo^n}$
	\end{itemize}
\end{definition}

\subsection{Chernoff bound}

\begin{thm}
\label{thm:Chernoff}
Let $X=\sum_{i=1}^n X_i$ where $X_i$ are independent Bernoulli trials, and $\mu=\E[X]$.
Then for all $0<\delta<1$,
$$P[\abs{X-\mu}\geq\delta\mu]\leq2e^{-\frac{\mu\delta^2}{3}}.$$
\end{thm}

\subsection{Projection Lemma}

We use the projection lemma from \cite{kempe_kitaev_regev_2006}, which describes the conditions under which we can estimate the ground state energy of $H_1 + H_2$ with that of $H_1\big|_{\ker H_2}$.

\begin{thm}
	\label{thm:proj1}
	Let $H=H_1+H_2$ be the sum of two Hamiltonians operating on some Hilbert space $\cH=\cS+\cS^\bot$.
	The Hamiltonian $H_2$ is such that $\cS$ is a zero eigenspace and the eigenvectors in $\cS^\bot$ have eigenvalues at least $J>2\norm{H_1}$. Then,
	$$\lambda\left(H_1\big|_\cS\right)-\frac{\norm{H_1}^2}{J-2\norm{H_1}^2}\leq\lambda(H)\leq\lambda\left(H_1\big|_\cS\right)$$
\end{thm}

We will instead use the following formulation, which can be obtained by relabeling variables from above.

\begin{thm}
	\label{thm:projection}
	Let $H_1, H_2$ be local Hamiltonians where $H_2\geq0$. Let $K=\ker H_2$ and
	$$J=\frac{11\norm{H_1}^2}{\lambda\left(H_2\big|_{K^\bot}\right)}$$
	then we have
	$$\lambda(H_1+JH_2)\geq\lambda\left(H_1\big|_K\right)-\frac{1}{8}$$
\end{thm}
\begin{proof}
	Apply \Cref{thm:proj1} to $H=H_1+JH_2$. Note that the least nonzero eigenvalue of $JH_2$ is greater than $10\norm{H_1}$.
\end{proof}

\subsection{Quantum de Finetti Theorem under Local Measurements}

De Finetti theorem provides a way to obtain close to independent samples by taking random subsystems of a quantum system.
There are many formulations; we use the one from \cite{Brandão2017} because we need to avoid exponential dependence on number of qubits in each subsystem.
\begin{thm}
	\label{deFinetti}
	Let $\rho^{A_1\ldots A_k}$ be a permutation-invariant state on registers $A_1,\ldots,A_k$ where each register is $s$ qubits,
	then for every $0\leq l\leq k$ there exists states $\set{\rho_i}$ and $\set{p_i}\subset\bbR$ such that
	$$\max_{\Lambda_1,\ldots,\Lambda_l}
	\norm{(\Lambda_1\otimes\ldots\otimes\Lambda_l)\left(\rho^{A_1\ldots A_l}-\sum_ip_i\rho_i^{A_1}\otimes\ldots\otimes\rho_i^{A_l}\right)}_1
	\leq\sqrt{\frac{2l^2s}{k-l}}$$
	where $\Lambda_i$ are quantum-classical channels.
\end{thm}

\subsection{Quantum Homomorphic Encryption Schemes}

\def\QHE{\mathsf{QHE}}
\def\QGen{\mathsf{QHE.Keygen}}
\def\QEnc{\mathsf{QHE.Enc}}
\def\QEval{\mathsf{QHE.Eval}}
\def\QDec{\mathsf{QHE.Dec}}

We use the quantum fully homomorphic encryption scheme given in \cite{mahadev_qfhe} which is compatible with our use of a classical client. We start by presenting the interface of a homomorphic encryption scheme:
\begin{definition}
	A leveled homomophic encryption scheme is tuple of algorithms \linebreak $\mathsf{HE}=(\mathsf{HE.Keygen}, \mathsf{HE.Enc}, \mathsf{HE.Dec}, \mathsf{HE.Eval})$ with the following descriptions:
	\begin{itemize}
		\item $\mathsf{HE.Keygen}(1^\lambda, 1^L)\rightarrow(pk, sk)$
		\item $\mathsf{HE.Enc}_{pk}(\mu)\rightarrow c$
		\item $\mathsf{HE.Dec}_{sk}(c)\rightarrow \mu^*$
		\item $\mathsf{HE.Eval}_{pk}(f, c_1, \ldots, c_l)\rightarrow c_f$
	\end{itemize}
\end{definition}

$\mathsf{HE}$ also satisfies, with overwhelming probability in $\lambda$, that
$$\mathsf{HE.Dec}_{sk}(\mathsf{HE.Eval}_{pk}(f, c_1, \ldots, c_l))=f(\mathsf{HE.Dec}_{sk}(c_1),\ldots,\mathsf{HE.Dec}_{sk}(c_l))$$
where $f$ is specified by a circuit of depth at most $L$.

\Ethan{To be pedantic, the above doesn't imply Dec undoes Enc even if we sub in $f=\id$.}

We also recall the security definition for a FHE scheme.

\begin{definition}
	A FHE scheme $\mathsf{HE}$ is IND-CPA secure if, for any polynomial time adversary $\cA$, there exists a negligible function $\mu(\cdot)$ such that
	$$\abs{Pr[\cA(pk, \mathsf{HE.Enc}_{pk}(0))=1]-Pr[\cA(pk, \mathsf{HE.Enc}_{pk}(1))=1]}=\mu(\lambda)$$
	where $(pk, sk)\leftarrow\mathsf{QHE.Keygen}(1^\lambda)$
\end{definition}

The quantum homomorphic encryption scheme $\mathsf{QHE}$ from \cite{mahadev_qfhe} has additional properties that facilitate the use of classical clients.

\begin{definition}
	we call a quantum homomorphic encryption scheme \emph{classical-friendly} if it has the following properties:
	\begin{itemize}
		\item $\QGen$ can be done classically.
		\item In the case where the plaintext is classical, $\QEnc$ can be done classically.
		\item Its ciphertext can be homomorphically measured, after which $\QDec$ can also be done classically.
	\end{itemize}
\end{definition}

\begin{theorem} [from \cite{mahadev_qfhe}]
	Under the assumption that the learning with errors problem with superpolynomial noise ratio is computationally intractable for an efficient quantum machine,
	there exists a classical-friendly quantum homomorphic encryption scheme.
\end{theorem}
