\section{Preliminaries}

\subsection{Notations}

Let $\mathcal{B}$ be the Hilbert space corresponding to one qubit. Let $H:\mathcal{B}^{\otimes n}\rightarrow\mathcal{B}^{\otimes n}$ be Hermitian matrices. We use $H\geq0$ to denote $H$ being positive semidefinite. Let $\lambda(H)$ be the smallest eigenvalue of $H$. The ground states of $H$ are the eigenvectors corresponding to $\lambda(H)$. For matrix $H$ and subspace $S$, let $H\big|_S=\Pi_S H \Pi_S$, where $\Pi_S$ is the projector onto the subspace $S$. 
%For a $T$-qubit Hilbert space, let the state $\ket{\widehat{t}}=\ket{1}^{\otimes t}\otimes \ket{0}^{{\otimes (T-t)}}$.

For $n$-qubit states $\rho_1, \rho_2\in\cB^{\otimes n}$,
let $\norm{\rho_1-\rho_2}_{tr}=\frac{1}{2}\norm{\rho_1-\rho_2}_1$ denote the trace distance between $\rho_1$ and $\rho_2$. We say $\rho_1$ and $\rho_2$ are $\eps$-close if $\norm{\rho_1- \rho_2}_{tr}\leq\eps$.
For two distributions $f_1$ and $f_2$ over the same finite domain $X$, let $\norm{f_1-f_2}_{TV}=\frac{1}{2}\sum_{x\in X}\abs{f_1(x)-f_2(x)}$ denote their total variation distance,
and we say $f_1$ and $f_2$ are $\eps$-close if $\norm{f_1-f_2}_{TV}\leq\eps$.
We write $x\xleftarrow{\$}X$ when $x$ is sampled uniformly at random from the set $X$.

\begin{definition} [quantum-classical channels]
	\label{def:QCChannel}
	A quantum measurement is given by a set of matrices $\set{M_k}$ such that $M_k\geq0$ and $\sum_k M_k=\id$.
	We associate to any measurement a map $\Lambda(\rho)=\sum_k \tr(M_k\rho)\proj{k}$
	with $\{\ket{k}\}$ an orthonormal basis.
	This map is also called a \emph{quantum-classical channel}.
\end{definition}

The phase gate and Pauli matrices are the following matrices.

\begin{definition}
	$P(i)=\begin{pmatrix}1&0\\0&i\end{pmatrix}$,
	$X=\begin{pmatrix}0&1\\1&0\end{pmatrix}$,
	$Y=\begin{pmatrix}0&-i\\i&0\end{pmatrix}$,
	$Z=\begin{pmatrix}1&0\\0&-1\end{pmatrix}$.
\end{definition}

\subsection{Sampling Problems}

We define the sampling version of $\BQP$ based on \cite{aaronson_2013}.

\begin{definition} [sampling problem]
	\label{dfn:sampling-problem}
	A \emph{sampling problem} is a collection of probability distributions $(D_x)_{x\in\set{0, 1}^*}$, one for each input string $x\in\set{0,1}^n$, where $D_x$ is a distribution over $\set{0,1}^{m(n)}$ for some fixed polynomial $m$.
\end{definition}

\begin{definition} [$\SampBQP$]
	$\SampBQP$ is the class of sampling problems $\left(D_x\right)_{x\in\set{0, 1}^*}$ that can be (approximately) sampled by polynomial-size uniform quantum circuits. Namely, there exists a Turing machine $M$ such that for every $n \in \bbN$ and $\eps \in (0,1)$, $M(1^n, 1^{1/\eps})$ outputs a quantum circuit $C$ in $\poly(n, 1/\eps)$ time such that for every $x \in \zo^n$, the output of $C(x)$ (measured in standard basis) is $\eps$-close to $D_x$.
\end{definition}

\subsection{Quantum Prover Interactive Protocols (QPIP)}

We model the interaction between a (almost classical) client and a quantum server following \cite{FOCS:Mahadev18a}.

\begin{definition}
	A protocol $\Pi=(P, V)$ is said to be in $\QPIP_\tau$ if it satisfies the following properties:
	\begin{itemize}
		\item $P$ is a $\BQP$ machine, which also has access to a quantum channel that can transmit $\tau$ qubits to the verifier per use.
		\item $V$ is a hybrid machine of a classical part and a limited quantum part. The classical part is a $\BPP$ machine. The quantum part is a register of $\tau$ qubits, on which the verifier can perform arbitrary quantum operations and which has access to a quantum channel which can transmit $\tau$ qubits. At any given time, the verifier is not allowed to possess more than $\tau$ qubits. The interaction between the quantum and classical parts of the verifier is the usual one: the classical part controls which operations are to be performed on the quantum register, and outcomes of measurements of the quantum register can be used as input to the classical part.
		\item There is also a classical communication channel between the prover and the verifier, which can transmit polynomially many bits to either direction.
	\end{itemize}
\end{definition}

\begin{definition}
	Let $\Pi=(P, V)$ be a $\QPIP_\tau$ protocol.
	We say it is a prototol for a $\BQP$ language $L$ with completeness error $c(\cdot)$ and soundness error $s(\cdot)$ if the following holds:
\begin{itemize}
		\item On public inputs $1^\lambda$ and $x\in\zo^*$, the verifier outputs either $\Acc$ or $\Rej$.
		\item (Completeness): For all security parameter $\lambda\in\bbN$ and $x\in\zo^{\poly(\lambda)}$, if $x\in L$ then
		$\Pr\left[(P, V)(x, 1^\lambda)=\Rej\right]<c(\lambda)$.
		\item (Soundness): For all cheating $\BQP$ prover $P^*$, sufficiently large security parameter $\lambda\in\bbN$, and $x\in\zo^{\poly(\lambda)}$, if $x \notin L$ then $\Pr\left[(P, V)(x, 1^\lambda)=\Rej\right]<s(\lambda)$.
	\end{itemize}
\end{definition}

\begin{definition}
	\label{dfn:stats-secure-proto-sampbqp}
	Let $\Pi=(P, V)$ be a $\QPIP_\tau$ protocol.
	We say it is a protocol for the $\SampBQP$ instance $(D_x)_{x\in\zo^*}$ with completeness error $c(\cdot)$ and statistical (resp., computational) soundness if the following holds:
	\begin{itemize}
		\item On public inputs $1^\lambda$, $1^{1/\eps}$, and $x\in\zo^{\poly(\lambda)}$, $V$ outputs $(d, z)$ where $d\in\set{\Acc, \Rej}$.
			If $d=\Acc$ then $z\in\zo^{m(\abs{x})}$ where $m$ is given in \Cref{dfn:sampling-problem}, otherwise $z=\bot$.
		\item (Completeness):
			For all accuracy parameter $\eps(\lambda)=\frac{1}{\poly(\lambda)}$,
			security parameter $\lambda\in\bbN$,
			and $x\in\zo^{\poly(\lambda)}$, 
			Let $(d, z)\leftarrow(P, V)(1^\lambda, 1^{1/\eps}, x)$, then $d=\Rej$ with probability at most $c(\lambda)$.
		\item (Statistical soundness): For all cheating prover $P^*$,
		    accuracy parameter $\eps(\lambda)=\frac{1}{\poly(\lambda)}$,
			sufficiently large $\lambda\in\bbN$, and $x\in\zo^{\poly(\lambda)}$, 
			consider the following experiment:
			\begin{itemize}
				\item Let $(d, z)\leftarrow(P^*, V)(1^\lambda, 1^{1/\eps}, x)$.
				\item Define $z_{ideal}$ by
				$$\begin{cases}
					z_{ideal}=\bot & \text{if } d=\Rej\\
					z_{ideal}\leftarrow D_x & \text{if } d=\Acc
				\end{cases}$$.
			\end{itemize}
			It holds that $\norm{(d,z)-(d,z_{ideal})}_{TV}\leq\eps$. 
		\item (Computational soundness):
		For all cheating $\BQP$ prover $P^*$, $\BQP$ distinguisher $\mathsf{D}$, accuracy parameter $\eps(\lambda)=\frac{1}{\poly(\lambda)}$,
			sufficiently large $\lambda\in\bbN$, and all $x\in\zo^{\poly(\lambda)}$, 
			define $d, z, z_{ideal}$ by the same experiment as above.
			It holds that
			$$\abs{\Pr[\mathsf{D}(1^{\lambda}, d,z) = 1] - \Pr[\mathsf{D}(1^{\lambda}, d,z_{ideal}) = 1]} \leq \eps $$
	\end{itemize}
\end{definition}

\begin{remark} Note that we do not parameterize the soundness error in Definition \ref{dfn:stats-secure-proto-sampbqp}. This is because the definition of $\SampBQP$ already includes an error parameter $\eps$, where the algorithm is allowed to generate a sample with statistical error up to $\eps$ in time polynomial in $1/\eps$. Hence, in our definition of soundness, we allow the verifier's output to have error $\eps$ for any given inverse polynomial $\eps$. 
\end{remark}

We now define blindness of a protocol.
In the context of blind protocols, we consider $x$ as the verifier's private input instead of a common input.  
%the above definitions should be modified so that $x$ is the verifier's private input.
%We define blindness with an accuracy parameter $\eps$, which exists in protocols for $\SampBQP$ instances;
%one can leave it out where appropriate, such as in protocols for $\BQP$ languages.

\begin{definition}[Blindness]
	Let $\Pi=(P, V)$ be an interactive protocol with common inputs $1^\lambda$ (and $1^{1/\eps}$ for the case of $\SampBQP$ problem), and verifier's private input $x\in\zo^*$.
	We say $\Pi$ is \emph{blind} if for all cheating $\BQP$ prover $P^*$ the following ensembles are computationally indistinguishable over $\lambda$:
	\begin{itemize}
		\item $\set{\View_{P^*}(P^*, V(x))(1^\lambda, 1^{1/\eps})}_{\lambda\in\bbN, \eps\in(0,1), x\in\zo^*}$
		\item $\set{\View_{P^*}(P^*, V(0))(1^\lambda, 1^{1/\eps})}_{\lambda\in\bbN, \eps\in(0,1), x\in\zo^*}$
	\end{itemize}
\end{definition}

\begin{definition}
\label{def:comp-indis}
    \Ethan{TODO PROOFREAD ME}
    Two families of states $\set{\rho_{1,\lambda}}_{\lambda\in\bbN}$ and $\set{\rho_{2,\lambda}}_{\lambda\in\bbN}$ are \emph{$\eps$-computationally indistinguishable} if for all $\BQP$ distinguishers $\mathsf{D}$ and $\lambda\in\bbN$,
	$$\abs{\Pr\left[\mathsf{D}(\rho_{1,\lambda})=1\right]-\Pr\left[\mathsf{D}(\rho_{2,\lambda})=1\right]}\leq\eps(\lambda).$$
	We also denote $\eps$-computational indistinguishability by $\rho_{1,\lambda}\approx_{c, \eps}\rho_{2,\lambda}$.
    We drop the $\eps$ parameter when it is negligible.
    We extend computational indistinguishability to subnormalized states by taking the distinguisher $\mathsf{D}$ to output subnormalized states as appropriate.
    We also extend it to distributions by treating them as mixed states.
\end{definition}

\iffalse

Probably not necessary

\begin{lem}[triangle inequality for computational distance] \label{lem:computational-triangle0}
    Let $\rho_1, \rho_2, \rho_3\in\cB^{\otimes n}$ such that 
    \begin{align}
		\rho_1\approx_{c, \eps}\rho_2\\
		\rho_2\approx_{c, \eps'}\rho_3.
    \end{align}
    We have 
    \begin{align}
		\rho_1\approx_{c, \eps+\eps'}\rho_3.
    \end{align}
\end{lem}

\fi

\subsection{Chernoff bound}

\begin{thm}
\label{thm:Chernoff}
Let $X=\sum_{i=1}^n X_i$ where $X_i$ are independent Bernoulli trials, and $\mu=\E[X]$.
Then for all $0<\delta<1$,
$$P[\abs{X-\mu}\geq\delta\mu]\leq2e^{-\frac{\mu\delta^2}{3}}.$$
\end{thm}

\subsection{Quantum Homomorphic Encryption Schemes}

We use the quantum fully homomorphic encryption scheme given in \cite{mahadev_qfhe} which is compatible with our use of a classical client.
\begin{definition}
	A quantum leveled homomophic (public-key) encryption scheme $\QHE=(\QGen,$ \linebreak
	$\QEnc, \QEval, \QDec)$ is quadruple of quantum polynomial-time algorithms which operate as follows:
	\begin{itemize}
		\item \emph{Key generation.}
			The algorithm $(pk, sk)\leftarrow\QGen(1^\lambda, 1^L)$ takes a unary representation of the security parameter and a unary representation of the level parameter as inputs and outputs a public key $pk$ and a secret key $sk$.
		\item \emph{Encryption.}
			The algorithm $c\leftarrow\QEnc(pk, \mu)$ takes the public key $pk$ and a single bit message $\mu\in\zo$ as inputs and outputs a ciphertext $c$.
		\item \emph{Decryption.}
			The algorithm $\mu^*\leftarrow\QDec(sk, c)$ takes the secret key $sk$ and a ciphertext $c$ as inputs and outputs a message $\mu^*\in\zo$.
			It must be the case that
				$$\QDec(sk, \QEnc(pk, \mu))=\mu$$
			with overwhelming probability in $\lambda$.
		\item \emph{Homomorphic Evaluation.}
			The algorithm $c_f\leftarrow\QEval(pk, f, c_1, \ldots, c_l)$ takes the public key $pk$, a quantum circuit $f$ of depth at most $L$, and a set of $l$ ciphertexts $c_1,\cdots,c_l$ as inputs and outputs a ciphertext $c_f$.
			It must be the case that
				$$\QDec(sk, \QEval(pk, f, c_1, \ldots, c_l))=f(\QDec(sk, c_1),\ldots,\QDec(sk, c_l))$$
			with overwhelming probability in $\lambda$.
	\end{itemize}
\end{definition}

\begin{definition}[Compactness and Full Homomorphism]
	A quantum homomorphic encryption scheme $\QHE$ is compact if
	there exists a polynomial $s$ in $\lambda$ such that the output length of $\QEval$ is at most $s$ bits long (regardless of $f$ or
	the number of inputs). A compact scheme is (pure) fully homomorphic if it can evaluate any efficiently computable
	boolean function.
\end{definition}

We also recall the security definition for a FHE scheme.

\begin{definition}
	A scheme $\QHE$ is IND-CPA secure if, for any polynomial time adversary $\cA$, there exists a negligible function $\mu(\cdot)$ such that
	$$\abs{Pr[\cA(pk, \mathsf{HE.Enc}_{pk}(0))=1]-Pr[\cA(pk, \mathsf{HE.Enc}_{pk}(1))=1]}\leq\mu(\lambda)$$
	where $(pk, sk)\leftarrow\QGen(1^\lambda)$
\end{definition}

The quantum homomorphic encryption scheme $\mathsf{QHE}$ from \cite{mahadev_qfhe} has additional properties that facilitate the use of classical clients.

\begin{definition} \label{def:classical-friendly}
	we call a quantum homomorphic encryption scheme \emph{classical-friendly} if it has the following properties:
	\begin{itemize}
		\item $\QGen$ is a classical algorithm
		\item In the case where the plaintext is classical, $\QEnc$ can be done classically.
		\item When the underlying message is classical, the corresponding ciphertext is also classical and can be decrypted classically.
	\end{itemize}
\end{definition}

\begin{theorem} [Theorem 1.1 in \cite{mahadev_qfhe}]
	Under the assumption that the learning with errors problem with superpolynomial noise ratio is computationally intractable for an efficient quantum machine,
	there exists a classical-friendly compact quantum leveled homomorphic encryption scheme.
\end{theorem}
