\section{Preliminaries}

\subsection{Notations}

Let $\mathcal{B}$ be the Hilbert space corresponding to one qubit. Let $H:\mathcal{B}^{\otimes N}\rightarrow\mathcal{B}^{\otimes N}$ be Hermitian matrices. We use $H\geq0$ to denote $H$ being positive semidefinite. Let $\lambda(H)$ be the smallest eigenvalue of $H$. The ground states of $H$ are the eigenvectors corresponding to $\lambda(H)$. For matrix $H$ and subspace $S$, let $H\big|_S=\Pi_S H \Pi_S$, where $\Pi_S$ is the projector onto the subspace $S$. For a $T$-qubit Hilbert space, let the state $\ket{\widehat{t}}=\ket{1}^{\otimes t}\otimes \ket{0}^{{\otimes (T-t)}}$.
We write $F(\rho_1, \rho_2)=\left(\tr\sqrt{\sqrt{\rho_1}\rho_2\sqrt{\rho_1}}\right)^2$ for the fidelity between $\rho_1$ and $\rho_2$.
We write $\frac{1}{2}\norm{\rho_1-\rho_2}_1$ for the trace distance between $\rho_1$ and $\rho_2$. $\forall \rho_1, \rho_2\in\cB^{\otimes N}$ we have $\frac{1}{2}\norm{\rho_1-\rho_2}_1\leq\sqrt{1-F(\rho_1, \rho_2)}$.

\begin{definition} [quantum-classical channels]
	\label{def:QCChannel}
	A quantum measurement is given by a set of matrices $\set{M_k}$ such that $M_k\geq0$ and $\sum_k M_k=\id$.
	We associate to any measurement a map $\Lambda(\rho)=\sum_k \tr(M_k\rho)\ket{k}\bra{k}$
	with $\set{\ket{k}}$ an orthonormal basis.
	This map is also called a \emph{quantum-classical channel}.
\end{definition}

The phase gate and Pauli matrices are denoted as follows.

\begin{definition}
	$P(i)=\begin{pmatrix}1&0\\0&i\end{pmatrix}$, $X=\begin{pmatrix}0&1\\1&0\end{pmatrix}$,
	$Y=\begin{pmatrix}0&-i\\i&0\end{pmatrix}$,
	$Z=\begin{pmatrix}1&0\\0&-1\end{pmatrix}$
\end{definition}

\subsection{Relevant complexity classes}

We define a few relevant complexity classes.

\begin{definition} [$\BQP$]
	$\BQP$ is the class of languages $L$ for which for all $n\in\bbN$ there exists a quantum circuit constructible in time $\poly(n)$ that, given any $x\in\set{0, 1}^n$ as input, correctly decides whether $x\in L$ at least $\frac{2}{3}$ of the time.
\end{definition}

\begin{definition} [$\FBQP$]
	A function $f:\set{0,1}^*\rightarrow\set{0,1}^*$ is in $\FBQP$ if there is a $\BQP$ machine that, $\forall x$, outputs $f(x)$ with overwhelming probability.
\end{definition}

We define search and sampling versions of $\BQP$ based on \cite{aaronson_2013}.

\begin{definition} [search problem]
	A search problem $R$ is a collection of nonempty sets $(A_x)_{x\in\set{0, 1}^*}$, one for each input string $x\in\set{0, 1}^*$, where $A_x$... \Ethan{Great, interface doesn't line up correctly}
\end{definition}

\begin{definition} [sampling problem]
	A sampling problem $S$ is a collection of probability distributions $(D_x)_{x\in\set{0, 1}^*}$, one for each input string $x\in\set{0,1}^n$, where $D_x$ is a distribution over $\set{0,1}^{p(n)}$ for some fixed polynomial $p$.
\end{definition}

\begin{definition} [$\SampBQP$]
	$\SampBQP$ is the class of sampling problems $S=\left(D_x\right)_{x\in\set{0, 1}^*}$ for which there exists a polynomial-time quantum algorithm $B$ that, given $(x, 0^{1/\varepsilon})$ as input, samples from a probability distribution $C_x$ such that $\norm{C_x-D_x}\leq\varepsilon$.
\end{definition}

\subsection{Quantum Prover Interactive Protocol (QPIP)}
We classify the interaction between a (almost classical) client and a quantum server for sampling problems, extending the classification by \cite{FOCS:Mahadev18a}.

\begin{definition}
	A sampling problem $S=(D_x)_{x\in\set{0, 1}^*}$ is said to be in $\QPIP_\tau$ with completeness $c$ and soundness $(\delta, s)$ if there exists a protocol $(\bbP, \bbV)$ with the following properties with input $x$:
	\begin{itemize}
		\item $\bbP$ is run by the prover, a $\BQP$ machine, which also has access to a quantum channel that can transmit $\tau$ qubits to the verifier per use.
		\item $\bbV$ is run by the verifier, which is a hybrid machine of a classical part and a limited quantum part. The classical part is a $\BPP$ machine. The quantum part is a register of $\tau$ qubits, on which the verifier can perform arbitrary quantum operations and which has access to a quantum channel which can transmit $\tau$ qubits. At any given time, the verifier is not allowed to possess more than $\tau$ qubits. The interaction between the quantum and classical parts of the verifier is the usual one: the classical part controls which operations are to be performed on the quantum register, and outcomes of measurements of the quantum register can be used as input to the classical part.
		\item There is also a classical communication channel between the prover and the verifier, which can transmit $\poly(\abs{x})$ many bits to either direction. 
		\item (Completeness) After interacting with $\bbP$, $\bbV$ accepts with probability $\geq c$.
		\item (Soundness) Define the ideal output distribution $D_{x, \alpha}$ as first flipping a weighted coin with head probability $\alpha$. On head, output $\bot$. On tail, sample and output from $D_x$.
			Then the protocol is sound if for all cheating prover $\bbP^*$, there exists $\alpha$ so that the output distribution of $(\bbP^*, \bbV)$ is negligibly close to $D_{x, \alpha}$.
	\end{itemize}
\end{definition}

\subsection{Semantic security for interactive protocols}
\Ethan{Just call this blindness and put this under interactive protocols}

We present the security definition for interactive protocols:

\begin{dfn}
	Let $\lambda$ be a security parameter.
	Let $(\bbP, \bbV)$ be an interactive protocol with security parameter $\lambda$.
	Then it is IND-CPA secure if $\forall x\in\set{0,1}^n$ no polynomial time adversary $\cA$ can win \protoref{indcpa} with probability better than $\frac{1}{2}+\negl(\lambda)$
\end{dfn}

\begin{protocol}{Attack against semantic security}
	\label{proto:indcpa}
	\begin{enumerate}
		\item The challenge picks $b\in\set{0,1}$ at random
		\item If $b=0$, the challenger runs the protocol with the adversary, acting as the verifier with input $0^n$
		\item Otherwise, the challenger runs the protocol with the adversary, acting as the verifier with input $x$
		\item $\cA$ attempts to guess $b$
	\end{enumerate}
\end{protocol}

\subsection{Chernoff bound}

Taken from \href{http://math.mit.edu/~goemans/18310S15/chernoff-notes.pdf}{here}.

\begin{thm}
\label{thm:Chernoff}
Let $X=\sum_{i=1}^n X_i$ where $X_i$ are i.i.d. Bernoulli trials, and $\mu=\E[X]$.
Then for all $0<\delta<1$,
$$P[\abs{X-\mu}\geq\delta\mu]\leq2e^{-\frac{\mu\delta^2}{3}}$$
\end{thm}

\subsection{Projection Lemma}

We use the projection lemma from \cite{kempe_kitaev_regev_2006}, which describes the conditions under which we can estimate the ground state energy of $H_1 + H_2$ with that of $H_1\big|_{\ker H_2}$.

\begin{thm}
	Let $H=H_1+H_2$ be the sum of two Hamiltonians operating on some Hilbert space $\cH=\cS+\cS^\bot$.
	The Hamiltonian $H_2$ is such that $\cS$ is a zero eigenspace and the eigenvectors in $\cS^\bot$ have eigenvalues at least $J>2\norm{H_1}$. Then,
	$$\lambda\left(H_1\big|_\cS\right)-\frac{\norm{H_1}^2}{J-2\norm{H_1}^2}\leq\lambda(H)\leq\lambda\left(H_1\big|_\cS\right)$$
\end{thm}

We will instead use the following formulation, which can be obtained by relabeling variables from above.

\begin{thm}
	\label{thm:projection}
	Let $H_1, H_2$ be local Hamiltonians where $H_2\geq0$. Let $K=\ker H_2$ and
	$$J=\frac{10\norm{H_1}^2}{\lambda\left(H_2\big|_{K^\bot}\right)}$$
	then we have
	$$\lambda(H_1+JH_2)\geq\lambda\left(H_1\big|_K\right)-\frac{1}{8}$$
\end{thm}

\subsection{Quantum de Finetti Theorem under Local Measurements}

De Finetti theorem provides a way to obtain close to independent samples by taking random subsystems of a quantum system.
There are many formulations; we use the one from \cite{Brandão2017} because we need to avoid exponential dependence on number of qubits in each subsystem.
\begin{thm}
	\label{deFinetti}
	Let $\rho^{A_1\ldots A_k}$ be a permutation-invariant state on registers $A_1,\ldots,A_k$ where each register is $s$ qubits,
	then for every $0\leq l\leq k$ there exists states $\set{\rho_i}$ and $\set{p_i}\subset\bbR$ such that
	$$\max_{\Lambda_1,\ldots,\Lambda_l}
	\norm{(\Lambda_1\otimes\ldots\otimes\Lambda_l)\left(\rho^{A_1\ldots A_l}-\sum_ip_i\rho_i^{A_1}\otimes\ldots\otimes\rho_i^{A_l}\right)}_1
	\leq\sqrt{\frac{2l^2s}{k-l}}$$
	where $\Lambda_i$ are quantum-classical channels.
\end{thm}

\subsection{Quantum Homomorphic Encryption Schemes}

\def\QHE{\mathsf{QHE}}
\def\QGen{\mathsf{QHE.Keygen}}
\def\QEnc{\mathsf{QHE.Enc}}
\def\QEval{\mathsf{QHE.Eval}}
\def\QDec{\mathsf{QHE.Dec}}

We use the quantum fully homomorphic encryption scheme given in \cite{mahadev_qfhe} which is compatible with our use of a classical client. We start by presenting the interface of a homomorphic encryption scheme:
\begin{dfn}
	A leveled homomophic encryption scheme is tuple of algorithms $\mathsf{HE}=(\mathsf{HE.Keygen}, \mathsf{HE.Enc}, \mathsf{HE.Dec}, \mathsf{HE.Eval})$ with the following descriptions:
	\begin{itemize}
		\item $\mathsf{HE.Keygen}(1^\lambda, 1^L)\rightarrow(pk, sk)$
		\item $\mathsf{HE.Enc}_{pk}(\mu)\rightarrow c$
		\item $\mathsf{HE.Dec}_{sk}(c)\rightarrow \mu^*$
		\item $\mathsf{HE.Eval}_{pk}(f, c_1, \ldots, c_l)\rightarrow c_f$
	\end{itemize}
\end{dfn}

$\mathsf{HE}$ also satisfies, with overwhelming probability in $\lambda$, that
$$\mathsf{HE.Dec}_{sk}(\mathsf{HE.Eval}_{pk}(f, c_1, \ldots, c_l)=f(\mathsf{HE.Dec}_{sk}(c_0),\ldots,\mathsf{HE.Dec}_{sk}(c_l))$$
where $f$ is specified by a circuit of depth at most $L$.

\Ethan{To be pedantic, the above doesn't imply Dec undoes Enc even if we sub in $f=\id$.}

We also recall the security definition for a FHE scheme.

\begin{dfn}
	A FHE scheme $\mathsf{HE}$ is IND-CPA secure if, for any polynomial time adversary $\cA$, there exists a negligible function $\mu(\cdot)$ such that
	$$\abs{Pr[\cA(pk, \mathsf{HE.Enc}_{pk}(0))=1]-Pr[\cA(pk, \mathsf{HE.Enc}_{pk}(1))=1]}=\mu(\lambda)$$
	where $(pk, sk)\leftarrow\mathsf{QHE.Keygen}(1^\lambda)$
\end{dfn}

The quantum homomorphic encryption scheme $\mathsf{QHE}$ from \cite{mahadev_qfhe} has additional properties that facilitates the use of classical clients:
\begin{itemize}
	\item $\QGen$ can be done classically.
	\item In the case where the plaintext is classical, $\QEnc$ can be done classically.
	\item Its ciphertext takes the form $(X^xZ^z\rho Z^zX^x, c_{x, z})$, where $\rho$ is the plaintext and $c_{x, z}$ is a ciphertext that decodes to $(x, z)$ under a certain classical homomorphic encryption scheme.
\end{itemize}
