\section{Delegation Protocol for Fully Classical Client}

\hannote{In this section, we modify the construction of \cite{mahadev_delegation} to show that a $\QPIP_1$ protocol for $\SampBQP$ implies a $\QPIP_0$ for $\SampBQP$.}

We then extend our scheme for $\QPIP_0$ using results from \cite{mahadev_delegation}.

Here we combine \autoref{QPIP1thm} with results from \cite{mahadev_delegation} to create a delegation protocol for $\SampBQP$ for fully classical clients. Combining them will cost polynomially rounds, so we use results from \cite{parallelrep} that allows parallel repetition to construct a constant-round protocol.


%Subprotocol for Quantum Measurements
\subsection{$\QPIP_0$ for BQP?}
\hannote{Ethan old description before parallel rep.}
As a warm-up, we  restate the construction in \cite{mahadev_delegation}, which shows that $\QPIP_1$ protocol for $\BQP$ implies a $\QPIP_0$ for $\BQP$.

Let $\rho$ be an $n$-qubit state. Let $h$ be an n-bits string called the \emph{basis choice}. That is, $h_i=0$ indicates that the $i$-th qubit of $\rho$ is to be measured in the standard basis, while $i=1$ indicates Hadamard basis measurement instead. Let $D_{\rho, h}$ be the distribution of the corresponding measurement results.




\begin{theorem}
    \label{Mahadev_QPIP0_Protocol_Interface}
	Under the assumption that the learning with errors problem with superpolynomial noise ratio is computationally intractable for an efficient quantum machine, there exists a $\QPIP_0$ protocol $(\bbV, \bbP)$ with the following properties:
	\begin{itemize}
	    \item The protocol runs either a ``Hadamard round" or a ``test round".
	    \item With a test round, the verifier outputs nothing other than accept/reject
	    \item With a Hadamard round, for prover $\bbP'$, the verifier also obtains a sample from $D_{\bbP', h}$. We define $D^C_{\bbP', h}$ as that distribution conditioned on acceptance.
	    \item (completeness) $\forall\rho\in\mathcal{B}^{\otimes n}$ and $h\in\set{0,1}^n$, there is a prover with negligible reject probability. Furthermore, $D^C_{\bbP, h}$ has negligible total variation distance to $D_{\rho, h}$
	    \item (soundness) Fix $\bbP'$ and $h$. Let $p_{h, H}$ be its rejection probability under a Hadamard round, and $p_{h, T}$ be its rejection probability under a test round. Then $\exists\tilde\bbP$ s.t. $\norm{D^C_{\bbP', h} - D_{\tilde\bbP, h}}\leq p_{h,H} + \sqrt{p_{h, T}}+\mu$ for some negligible $\mu$, and $\exists\rho$ s.t. $D_{\tilde\bbP, h}$ is computationally indistinguishable from $D_{\rho, h}$.
		\item This is a 4-round protocol. The prover does not know whether it is a test or Hadamard round until the end of 3rd round.
		\begin{enumerate}
			\item The verifier generates $(k, td)$, ``key" and ``trapdoor". It sends the key to the prover.
			\item The prover computes a classical ``commitment" $y$ and sends it to the verifier.
			\item The verifier sends $c\in\set{0, 1}$, where $c=0$ corresponds to test round and $c=1$ corresponds to Hadamard round.
			\item The prover computes and sends a classical string $a$
		\end{enumerate}
	\end{itemize}
\end{theorem}

\hannote{import urmila protocol description.. up}
Here, we recall the Mahadev's protocol \cite{FOCS:Mahadev18a}. We only give a high-level description of the protocol and properties of it and omit the details since they are not needed to show our result. 

The protocol is run between a quantum prover $\pro$ and a classical verifier $\ver$ on a common input $x$. The aim of the protocol is to enable a verifier to classically verify $x\in \lang$ for a BQP language $\lang$ with the help of interactions with a quantum prover.
The protocol is a 4-round protocol where the first message is sent from $\ver$ to $\pro$. 
We denote the $i$-th message generation algorithm by $\ver_i$ for $i\in\{1,3\}$ or $\pro_i$ for $i\in \{2,4\}$ and denote the verifier's final decision algorithm by $\ver_\out$.
Then a high-level description of the protocol is given below.
\begin{description}
\item[$\ver_1$:] On input the security parameter $1^\secpar$ and $x$, it generates a pair $(\key,\td)$ of a``key" and ``trapdoor", sends $\key$ to $\pro$, and keeps $\td$ as its internal state.
\item[$\pro_2$:] On input $x$ and $\key$, it generates a classical ``commitment" $\comy$ along with a quantum state $\ket{\st_\pro}$, sends $\comy$ to $\pro$, and keeps $\ket{\st_\pro}$ as its internal state.
\item[$\ver_3$:] It randomly picks $c\sample \bit$ and sends $c$ to $\pro$.\footnote{The third message is just a public-coin, and does not depend on the transcript so far or $x$.}
Following the terminology in \cite{FOCS:Mahadev18a}, we call the case of $c=0$ the ``test round" and the case of $c=1$ the ``Hadamard round".
\item[$\pro_4$:] On input $\ket{\st_\pro}$ and $c$, it generates a classical string $\ans$ and sends $\ans$ to $\pro$.
\item[$\ver_\out$:] On input $\key$, $\td$, $y$, $c$, and $\ans$, it returns $\top$ indicating acceptance or $\bot$ indicating rejection.
In case $c=0$, the verification can be done publicly, that is, $\ver_\out$ need not take $\td$ as input.
\end{description}

For the protocol, we have the following properties:\\
\noindent\textbf{Completeness:}
For all $x\in \lang$, we have $\Pr[\langle \pro,\ver \rangle(x)]=\bot]= \negl(\secpar)$.\\
\noindent\textbf{Soundness:}
If the LWE problem is hard for quantum polynomial-time algorithms, then for any $x\notin \lang$ and a quantum polynomial-time cheating prover $\pro^*$, we have  $\Pr[\langle \pro^*,\ver \rangle(x)]=\bot]\leq 3/4$.

We need a slightly different form of soundness implicitly shown in \cite{FOCS:Mahadev18a}, which roughly says that if a cheating prover can pass the ``test round" (i.e., the case of $c=0$) with overwhelming probability, then it can pass the ``Hadamard round" (i.e., the case of $c=1$) only with a negligible probability. 
\begin{lemma}[implicit in \cite{FOCS:Mahadev18a}]\label{lem:Mah_soundness}
If the LWE problem is hard for quantum polynomial-time algorithms, then for any $x\notin \lang$ and a quantum polynomial-time cheating prover $\pro^*$ such that  $\Pr[\langle \pro^*,\ver \rangle(x)]=\bot\mid c=0]=\negl(\secpar)$, we have $\Pr[\langle \pro^*,\ver \rangle(x)]=\top\mid c=1]=\negl(\secpar)$.
\end{lemma}

We will also use the following simple lemma:
\begin{lemma}\label{fact:perfectly_pass_test}
There exists an efficient prover that passes the test round with probability $1$ (but passes the Hadamard round with probability $0$) even if $x\notin \lang$. 
\end{lemma}


\subsection{Mahadev's classical "commitment" of $XZ$ measurement}

In her groundbreaking work, Mahadev~\cite{mahadev_delegation} gave the following 4-round interactive protocol to let a $\BQP$ machine "commit a $XZ$ measurement" to a classical machine. 

\begin{protocol}{Mahadev's 4 round commitment}
\label{proto:urmila4}
\begin{enumerate}
    \item Verifier $\bbV$ is a classical machine. Prover $\bbP$ is a $\BQP$ machine. 
%    \item Both the verifier and the prover know a $\BQP$ circuit $C$ and an input string $x$.
    \item Both party know the number of qubits $n$ and a security parameter $\lambda=o(n)$.
    \item Verifier has a string $h \in \{0,1\}^n$ that specifies the measurement he wants to make: if $h_i=0$, he wants a $X$ measurement on $i$-th qubit. If $h_i=1$, he wants a $Z$ measurement on $i$-th qubit.  
    \item In the first round, the verifier generate a secret key and  public key pair $(sk,pk)$ from $h$ and $\lambda$ and sent $pk$ to the prover.
    \item In the second round, the prover pick some $n$-qubit quantum state $\rho$ and generate a "commitment" $y$ from $pk$ and $\rho$. He send $y$ to the verifier and keep the leftover quantum state $\sigma$.
    \item In the third round, the verifier send a uniformly random challenge bit $c\in \{0,1\}$. The prover needs to do a "Testing round" if $c=0$. The prover needs to do a "Hadamard round" if $c=1$.
    \item In the fourth round. The prover generates a classical string $a$ from $c$ and $\sigma$ and send to back to the prover.
    \item If $c=0$, the verifier decides whether he is going to reject from $(pk,a,y)$. If $c=1$ the verifier computes an $n$-bit string $z$ from $(sk,h,y,a)$.
\end{enumerate}
\end{protocol}


\begin{lemma}[binding property of  Protocol~\ref{proto:urmila4}\cite{mahadev_delegation}]
If a prover's strategy\hannote{need clarification} result in passing the testing round with probability $1-\negl(n)$, then there must exist a $\rho$ such that $z$ is computationally indistinguishable with $\negl(n)$ error \hannote{add notation for comp indist.} from $M_{XZ}(\rho,h)$\hannote{add notation}, unless LWE can be efficiently solved by a quantum computer. 
\end{lemma}

\begin{rmk}
Note that even a honest prover has complete freedom in choosing $\rho$, so protocol~\ref{proto:urmila4} didn't make the prover commit to a particular state. What it actually does is let the prover commit to a particular $XZ$ measurement on $\rho$. This is good enough for our application, since we will feed the $XZ$ measurement results into the verification of $\QPIP_1$ protocol to ensure that the prover pick the history state as $\rho$.
\end{rmk}

\subsection{$\QPIP_0$ protocol for $\SampBQP$}
%Parallel Repetition of the Measurement Subprotocol

The following protocol is a $\QPIP_0$ protocol for $\SampBQP$

$\forall c\in\bbN$ Soundness = $O(T^{-c})$

Given inverse poly p(T), we can parameterize the protocol to have soundness p(T)


%	\label{ProtoQPIP1}


\begin{protocol}{$\QPIP_0$ for $\SampBQP$}
\label{proto:QPIP0samp}
\begin{enumerate}
%    \item Verifier start with description of quantum circuit $C$, input string $x$ 
    \item Both verifier and prover start with the description of number of qubits $n$, quantum circuit $C$, input string $x$, and  a soundness parameter $\varepsilon$.
    \item Run $m=O(1/\varepsilon)$ copies of protocol~\ref{proto:urmila4} in parallel, with following modifications. On the second round, for each copy, the prover choose the state specified by Protocol~\ref{ProtoQPIP1}  (with error $\eps$) as $\rho$. On the third round, the verifier uniformly randomly pick an $i\in [m]$, do a Hadamard round on $i$-th copy, and has do test rounds on all other copies.
    \item \label{step:multi-testing}Verifier rejects if the results on any of the test rounds is rejected. 
    \item If all testing rounds are accepted, verifier run the verification procedure of Protocol~\ref{ProtoQPIP1} on the string $z$ decoded from the Hadamard round.
\end{enumerate}
\end{protocol}

\begin{theorem}\label{thm:urmila-binding}
    Protocol~\ref{proto:QPIP0samp} is a $\QPIP_0$ protocol for $\SampBQP$ with $\negl(n)$ completeness error and $\eps$ computational soundness error\hannote{definition}. 
\end{theorem}

\begin{definition}[computationally indistinguishable]
\hannote{move later}
Let  $D_1$ and $D_2$ be distributions over $n$-bit string.  We say $D_1$ and $D_2$  are $\eps$-Q-computationally indistinguishable if for all quantum circuit $A$ of size $\poly(n)$, $$\L\|\Pr[A({D_1})=1]-\Pr[A({D_2})=1]  \R\| \leq \eps,$$  where $A(D_1)$ means $A$ takes a sample from $D_1$ as part of its input. 
\end{definition}
%Let $O^{D_1}$ and $O^{D_2}$ be oracles giving samples from $D_1$ and $D_2$.

\begin{rmk}
If $\norm{D_1-D_2}_1\leq \eps$, $D_1$ and $D_2$ are $\eps$-Q-computationally indistinguishable.
\end{rmk}



\begin{proof}




 [[Partition lemma -> Parallel repetition]], except for probability $\eps$, either the verifier reject at step~\ref{step:multi-testing} or $z$ is $\negl(n)$-computationally indistinguishable to $M_{XZ}(\rho,h)$ by Theorem~\ref{thm:urmila-binding}. 
 
 By theorem~\ref{QPIP1thm}, $M_{XZ}(\rho,h)$  $\eps$-close to $C(x)$. Therefore\hannote{?}, $M_{XZ}(\rho,h)$  and  $C(x)$ $z$ are $\eps$-Q-computationally indistinguishable, and thus $z$ and $C(x)$ are  $2\eps$-Q-computationally indistinguishable.
 
 
\end{proof}
===========


\hannote{import 3.5. didn't import the proof, too long.}

First, We present a slight adaption of Lemma 3.5 of \cite{parallelrep}  that partitions the prover's internal state. We omit the proof here.


%$\frac{\gamma_0}{T}=1/\poly(\secpar)$
\begin{lemma}[partition lemma]\label{lem:partition}
Let $(U_0,U)$ be a prover's strategy in Protocol~\ref{proto:QPIP0samp}.  Let $i\in[m]$, 
$\gamma_0 \in[0,1]$, and $T\in \mathbb{N}$ such that $\gamma_0=1/\poly(n)$ and $T=\poly(n)$. Let $\gamma$ be a random number sampled uniformly from $\L[\frac{\gamma_0}{T},\frac{2\gamma_0}{T},\dots,\frac{T\gamma_0}{T}\R]$. 

There exists an $\poly(n)$-sized quantum circuit $G_{i,\gamma}$ such that for all (possibly sub-normalized) \hannote{$n$-qubit?} quantum state $\ket{\psi}_{\regX,\regZ}$,  
\begin{align*}
    G_{i,\gamma} \ket{0^m}_{\regC}\ket{\psi}_{\regX,\regZ}\ket{0^t}_{ph}\ket{0}_{th}\ket{0}_{in} = \\ z_0\ket{0^m}_{\regC}\ket{\psi_{0}}_{\regX,\regZ}\ket{0^t01}_{ph,th,in}+  z_1\ket{0^m}_{\regC}\ket{\psi_{1}}_{\regX,\regZ}\ket{0^t11}_{ph,th,in} + \ket{\psi'_{err}}
\end{align*}
\hannote{what's wrong with all these registers} where $t$ is the number of qubits in the register $ph$, $z_0,z_1\in \mathbb{C}$ such that $|z_0|=|z_1|=1$, and 
$z_0$, $z_1$, \hannote{$Z$ just phase factor?}$\ket{\psi_0}_{\regX,\regZ}$, $\ket{\psi_1}_{\regX,\regZ}$, and $\ket{\psi_{err}}_{\regX,\regZ}$ may depend on $\gamma$.

Furthermore, the following properties are satisfied.
%
\begin{enumerate}
    \item \label{partition-property-1}  Define $\ket{\psi_{err}}_{\regX,\regZ}\defeq \ket{\psi}_{\regX,\regZ} - \ket{\psi_{0}}_{\regX,\regZ}- \ket{\psi_{1}}_{\regX,\regZ}$.\hannote{?} We have  $$E_{\gamma}[\|\ket{\psi_{err}}_{\regX,\regZ}\|^2]\leq \frac{6}{T}+\negl(n)$$. 
    $$E_{\gamma}[\|\ket{\psi}_{\regX,\regZ}\|^2 -\|\ket{\psi_0}_{\regX,\regZ}\|^2 -\|\ket{\psi_1}_{\regX,\regZ}\|^2]\leq \frac{6}{T}+\negl(n)$$
\item For any fixed $\gamma$, $\Pr[M_{ph,th,in}\circ \ket{\psi'_{err}} \in \{0^t01,0^t11\}] =0$. %where $M_{ph,th,in}$ is the computational-basis measurement in the register $(ph,th,in)$. %and $t$ is the number of qubits in $ph$.   
This implies that if we apply the measurement $M_{ph,th,in}$ on $\frac{G_{i,\gamma} \ket{0^m}_{\regC}\ket{\psi}_{\regX,\regZ}\ket{0^t}_{ph}\ket{0}_{th}\ket{0}_{in}}{\|\ket{\psi}_{\regX,\regZ}\|}$, then the outcome is $0^tb1$ with probability $\|\ket{\psi_b}_{\regX,\regZ}\|^2$ and the resulting state in the register $(\regX,\regZ)$  is $\frac{\ket{\psi_b}_{\regX,\regZ}}{\|\ket{\psi_b}_{\regX,\regZ}\|}$ ignoring a global phase factor.

    % \item For any fixed $\gamma$, $E_{b\in \{0,1\}} [\|\ket{\psi_b}_{\regX,\regZ}\|^2]\leq \frac{1}{2}\|\ket{\psi}_{\regX,\regZ}\|^2$. \hannote{redundant?}
    
        \item 
For any fixed $\gamma$ and $c\in [m]$ such that $c\neq i$\hannote{check}\hannote{for all k, y}, we have 
\begin{align*}
\Pr\left[M_{\regX_i}\circ U\frac{\ket{c}_{\regC}\ket{\psi_0}_{\regX,\regZ}}{\|\ket{\psi_0}_{\regX,\regZ}\|}\in \Acc_{k_i,y_i}\right]\leq (m-1)\gamma+\negl(\secpar).
\end{align*}
    \item 
For any fixed $\gamma$, there exists an efficient quantum algorithm $\ext_i$ such that 
\begin{align*}  
  \Pr\left[\ext_i\left(\frac{\ket{0^m}_{\regC}\ket{\psi_1}_{\regX,\regZ}}{\|\ket{\psi_1}_{\regX,\regZ}\|}\right)\in \Acc_{k_i,y_i}\right]=1-\negl(\secpar).
  \end{align*}   \hannote{$M_x?$}
\end{enumerate}
\end{lemma}

\iffalse
\hannote{duplicate version by ethan..}
\begin{lem}
	Let $(U_0, U)$ be some prover's strategy.
	Let $m=O(\log n)$ \Ethan{$n$ is problem size?}, $i\in[m]$, $\gamma_0\in[0, 1]$, and $T\in\bbN$ such that $\frac{\gamma_0}{T}=\frac{1}{\poly(n)}$.
	Let $\gamma$ be sampled uniformly from $\set{\frac{j\gamma_0}{T}}^T_{j=1}$ \Ethan{This is a set, right? Doesn't look like one in the original document}.
	Then, there exists an efficient quantum procedure $G_{i,\gamma}$ such that for any (possibly sub-normalized) quantum state $\ket{\psi}_{X, Z}$,

	$$TODO-LONG-EQ$$

	where $t$ is the number of qubits in the register $ph$ \Ethan{Where is this register defined?},
	$z_0, z_1\in\bbC$ such that $\abs{z_0}=\abs{z_1}=1$, and $z_0$, $z_1$, $\ket{\psi_0}_{X, Z}$, $\ket{\psi_1}_{X, Z}$, $\ket{\psi_{err}}_{X, Z}$ may depend on $\gamma$.
	Furthermore, the following properties are satisfied.
	\begin{enumerate}
		\item Let $\ket{\psi_{err}}=\ket{\psi}-\ket{\psi_0}-\ket{\psi_1}$, then
			$$\E_\gamma[\norm{\ket{\psi_{err}}}^2]\leq\frac{1}{T}+\negl(n)$$
		\item TODO Some long property
		\item For any fixed $\gamma$,
			$$\E_{b\in\set{0, 1}}\left[\norm{\ket{\psi_{b}}_{X, Z}}^2\right]\leq\frac{1}{2}\norm{\ket{\psi}_{X, Z}}^2$$
		\item For any fixed $\gamma$ and $c\in\set{0, 1}^m$ such that $c_i=0$, \Ethan{Where's this $M_{X_i}$ defined?}
			$$\Pr\left[M_{X_i}\circ U\frac{\ket{c}_C\ket{\psi_0}_{X,Z}}{\norm{\psi_0}_{X,Z}}\in A_{k_i, y_i}\right]\leq2^{m-1}\gamma+\negl(n)$$
		\item For any fixed $\gamma$, there exists an efficient quantum algorithm $\cE_i$ such that
			$$\Pr\left[\cE_i\left(\frac{\ket{0^m}_C\ket{\psi_1}_{X,Z}}{\norm{\psi_1}_{X,Z}}\in A_{k_i, y_i}\right)\right]\leq1-\negl(n)$$
	\end{enumerate}
\end{lem}



\begin{theorem}
    
 If there exists a $\QPIP_1$ protocol $(\bbV, \bbP)$ for $\SampBQP$ with soundness blah and completeness blah there exists a $\QPIP_0$ protocol $(\bbV', \bbP')$ for $\SampBQP$ with soundness blah' and completeness blah'
	
\end{theorem}


\fi
==========

Choose the $T$ in Lemma~\ref{lem:partition} to be $\Theta(m^3)$ \hannote{$\gamma_0$?}. By Property~\ref{partition-property-1} of Lemma~\ref{lem:partition}, 
$$\forall i, \L(1- 1/m^2\R)\text{fraction of }\gamma$$ 
satisfies that 
$$\norm{\ket{\psi_{err}}}^2 \leq 1/m+\negl(n). $$
Therefore, there exist an $\gamma$ such that 
$$\forall i,\, \norm{\ket{\psi_{err}}}^2 \leq 1/m+\negl(n). $$

Denote $G_{i,\gamma}$ with  $\gamma$ found above as $G_i$. \hannote{finding $\gamma$ is not computationally efficient.}

====


\Ethan{Below is 3.6. Looks kinda useful but I need to think through this $c$ thing to make sure it'll be fine even if it's not randomly generated.}
\hannote{How about include 3.6 in the theorem}



Here we follow \cite{parallelrep}.

We characterize a prover by two unitaries $(U_0, U)$ acting on the space $\cH_C\otimes\cH_X\otimes\cH_Z\otimes\cH_Y\otimes\cH_K$.

\Ethan{I'll elaborate on above later, after figuring out exactly what I need or don't need}

Each coordinate can be accepted and rejected separately, and the entire protocol accepts only if all coordinates accept.

Let $A_{k_i, y_i}$ be the set of $a_i$s accepted by the verifier in the test round when the first and second messages are $k_i$ and $y_i$, respectively.



\Ethan{Do we even need this lemma 3.5, or can we just use 3.6...}


\begin{lem}
	Let $m=O(\log n)$ \Ethan{$n$ is problem size?}, $\gamma_0\in[0, 1]$, and $T\in\bbN$ such that $\frac{\gamma_0}{T}=\frac{1}{\poly(n)}$.
	Let $\gamma_i\xleftarrow{\$}\set{\frac{j\gamma_0}{T}}^T_{j=1}$ \Ethan{This is a set, right? Doesn't look like one in the original document} for each $i\in[m]$.
	Then a state $\ket{psi}_{X, Z}$ can be partitioned as follows: \Ethan{Just gonna hard-code $c=0^m$ here for now}
	$$\ket{\psi}=\ket{\psi_0}+\ket{\psi_{10}}+\ket{\psi_{110}}+\ldots+\ket{\psi_{11\ldots10}}+\ket{\psi_{11\ldots11}}+\ket{\psi_{err}}$$
	where the way of partitioning may depend on the choices of $\hat\gamma=(\gamma_i)_{i=1}^m$.
	Furthermore, the following properties are satisfied. \Ethan{Takes a bit of parsing to see how is our case different here since $c$ might be defined differently for us}
	\begin{enumerate}
		\item stuff
		\item For any fixed $\hat\gamma$... 
		\item For any fixed $\hat\gamma$...
		\item For any fixed $c$,
			$$\E_{\hat\gamma}\left[\norm{\ket{\psi_{err}}_{X,Z}}^2\right]\leq\frac{6m^2}{T}+\negl(n)$$
		\item stuff
	\end{enumerate}
\end{lem}
    asdf




\begin{cor}
    
	Under the assumption that the learning with errors problem with superpolynomial noise ratio is computationally intractable for an efficient quantum machine, there exists a $\QPIP_0$ protocol $(\bbV, \bbP)$ for $\SampBQP$ with soundness blah and completeness blah
	
\end{cor}
