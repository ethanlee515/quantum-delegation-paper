\section{Constant-Round and Blind $\QPIP_1$ protocol for $\SampBQP$}

We now present a constant-round, blind, and verifiable delegation scheme for $\SampBQP$ as an application of our $\QPIP_0$ construction for $\SampBQP$,
which we achieve by using the homomorphic encryption scheme from \cite{mahadev_qfhe} together with our construction.

\subsection{Our Delegation Protocol}

Now we compile our $\QPIP_1$ protocol \myprotoref{protoQPIP1} using the quantum homomorphic encryption $\mathsf{QHE}$ \Ethan{TODO maybe wrap this as a def} in a similar way as what we did for our blind $\BQP$ delegation protocol in \cref{sec:BlindQBP}

The construction is obvious, \Ethan{Maybe still write it out...? Or maybe not?} so we directly go into analysis. \Ethan{Need to remember security param if we write this out}

\begin{thm}
	Compiling \cref{proto:QPIP0samp} using $\mathsf{QHE}$ preserves its completeness and soundness.
\end{thm}
\begin{proof}
	A honest prover still gets accepted with overwhelming probability by the correctness of $\mathsf{QHE}$. \Ethan{Not sure if too hand-wavy}

	As for soundness, a cheating prover cannot do better under this compiliation, since each message still decodes to some plaintext.

	In other words, anything a cheating prover can do after the compiliation, there exists a corresponding strategy in the version before.
	\Ethan {Ahhhh this is not rigorous at all}
\end{proof}

\begin{thm}
	\myprotoref{ProtoPriv} is IND-CPA secure.
\end{thm}
\begin{proof}
	The verifier's first message is encrypted into $\tilde{x}$ in an IND-CPA way as a ciphertext.
	The verifier's second message is the basis choice of Hadamard or test rounds, which can be done using public coins.

	The verifier only sends the prover these two messages, so it follows \Ethan{hopefully?} easily that the protocol itself is also IND-CPA. \Ethan{Maybe not rigorous enough?}
\end{proof}
