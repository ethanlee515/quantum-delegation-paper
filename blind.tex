\section{Constant-Round, Blind, and Verifiable Delegation}

We now present a constant-round, blind, and verifiable delegation scheme for $\SampBQP$ as an application of our $\QPIP_0$ construction for $\SampBQP$,
which we achieve by using the homomorphic encryption scheme from \cite{mahadev_qfhe} together with our construction.

\subsection{Homomorphic Encryption with Classical Client}

We first present the quantum full homomorphic encryption (QFHE) scheme, $\mathsf{QHE}=(\mathsf{QHE.Keygen}, \mathsf{QHE.Enc}, \mathsf{QHE.Dec}, \mathsf{QHE.Eval})$ given in \cite{mahadev_qfhe} which allows the use of a classical client. Specifically, it has the following extra properties:
\begin{itemize}
	\item $\mathsf{QHE.Keygen}$ can be done classically.
	\item In the case where the plaintext is classical, $\mathsf{QHE.Enc}$ can be done classically.
\end{itemize}

Furthermore, one could either decode or measure first, in the following sense:
\begin{lemma}
	\label{decodeMeasureOrder}
	Let $\Lambda$ be a product of $X$ and $Z$ measurements. Then there exists $\Lambda'$ product of $X$ and $Z$ measurements and $\widehat{\mathsf{QHE.Dec}}$ a BPP algorithm so that
		$$\Lambda\circ\mathsf{QHE.Dec}=\widehat{\mathsf{QHE.Dec}}\circ\Lambda'$$
\end{lemma}

\subsection{Our Delegation Protocol}

Now we're ready to construct our delegation protocol.

\begin{algorithm}
	\caption{Verifiable, secure, and constant round delegation}
	\label{ProtoPriv}
	\begin{algorithmic}[1]
		\Procedure{Delegation}{C, x}
			\State Compute $\mathsf{QHE.Keygen}\rightarrow(pk, evk, sk)$
			\State Compute $\tilde{x}=\mathsf{Enc}_{pk}(x)$
			\State Let $\tilde{C}$ be the quantum circuit that takes $\tilde{x}$ as input to evaluate $\mathsf{QHE.Eval}_{evk}(C, \tilde{x})$
			\State Delegate $\tilde{C}(\tilde{x})$ to the server using our $\QPIP_0$ protocol.
			\State Request $\Lambda'$ based on \autoref{decodeMeasureOrder}, where $\Lambda$ is the basis choice if $C(x)$ were delegated.
			\State Let $y$ be the measurement results from the server
			\State \Return $\widehat{\mathsf{QHE.Dec}_{sk}}(y)$
		\EndProcedure
	\end{algorithmic}
\end{algorithm}

\begin{thm}
    \label{QPIP1thm}
	\myprotoref{ProtoPriv} can evaluate any $L\in\SampBQP$ with negligible completeness and soundness $(O(T^{-c}), O(T^{-c}))$ for any constant $c$.
\end{thm}
\begin{proof}
	Suppose $\bbV$ accepts $\bbP'$ at least $\delta$ of the time.

	Then the $\QPIP_0$ delegation must accept at least that often too,
	so the client will receive a sample from $\mathsf{Eval}_{evk}(C, \tilde{x})$ with at most inverse poly error. \Ethan{security param?}

	The verifier ends up with a state within inverse poly distance to
		$$\widehat{\mathsf{QHE.Dec}_{sk}}\circ\Lambda'(\tilde{C}(\tilde{x}))$$
		$$=\Lambda\circ\mathsf{QHE.Dec}_{sk}(\tilde{C}(\tilde{x}))$$
		$$=\Lambda\circ\mathsf{QHE.Dec}_{sk}(\mathsf{QHE.Eval}_{evk}(C, \tilde{x}))$$

	By the properties of homomoprhic encryption, with overwhelming probability this is indistinguishable to \Ethan{Check if it's true} $\Lambda(C(x))$
\end{proof}

\begin{thm}
	\myprotoref{ProtoPriv} is IND-CPA secure.
\end{thm}
\begin{proof}
	The verifier's first message is encrypted into $\tilde{x}$ in an IND-CPA way as a ciphertext.
	The verifier's second message is the basis choice of Hadamard or test rounds, which can be done using public coins.

	The verifier only sends the prover these two messages, so it follows \Ethan{hopefully?} easily that the protocol itself is also IND-CPA. \Ethan{Maybe not rigorous enough?}
\end{proof}
