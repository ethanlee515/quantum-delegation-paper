\section{Constant-Round and Blind $\QPIP_1$ protocol for $\SampBQP$}

\def\Samp{\mathsf{Samp}}
\nc{\PiSamp}{\ensuremath{\Pi_\Samp}}
\nc{\VSamp}{\ensuremath{V_\Samp}}
\nc{\PSamp}{\ensuremath{P_\Samp}}
\nc{\PSampstar}{\ensuremath{P_\Samp^*}}
\nc{\cVSamp}[1]{\ensuremath{\cV_{\Samp,#1}}}
\nc{\cPSamp}[1]{\ensuremath{\cP_{\Samp,#1}}}

\def\SampB{\mathsf{Samp, blind}}
\nc{\PiSampB}{\ensuremath{\Pi_\SampB}}
\nc{\VSampB}{\ensuremath{V_\SampB}}
\nc{\PSampB}{\ensuremath{P_\SampB}}

Now we compile our $\QPIP_1$ protocol (\cref{ProtoQPIP1}) using the quantum homomorphic encryption $\mathsf{QHE}$ from \cref{decodeMeasureOrder} in the same way as what we did for our blind $\BQP$ delegation protocol (\cref{proto:BlindBQP}) in \cref{sec:BlindBQP}.

First we restate \cref{ProtoQPIP1} in more convenient terms.

\begin{protocol}{Our 1-message $\QPIP_1$ protocol $\PiSamp$ for $\SampBQP$ language $L$}
	Common input: $x\in\set{0,1}^n$. 
	\begin{enumerate}
		\item The prover generates a certificate state $\rho\leftarrow\cPSamp{1}(x)$ and sends it to the verifier qubit-by-qubit.
		\item The verifier generates $\Lambda \leftarrow \cVSamp{2}(x)$, where $\Lambda=\Lambda_1\otimes\ldots\otimes\Lambda_m$;
			each $\Lambda_i$ is a single-qubit $X$ or $Z$ measurement and $m=\poly(n)$ is the length of the prover's certificate state.
			It applies $\Lambda$ to $\rho$ qubit-by-qubit.
			Let $w \in \set{0,1}^m$ denote the measurement outcome.
			It verifies the measurement outcome and produces a decision $o\leftarrow\cVSamp{3}(x, \Lambda, w)$ where $o\in\set{acc, rej}$ and $\cVSamp{3}$ is a classical polynomial-time algorithm.
	\end{enumerate}
\end{protocol}

Note that it is of the same form as its $\BQP$ counterpart $\PiSamp$ found in \cref{subsec:BlindBQP}.

So we can compile it into $\PiSampB$ to achieve blindness in the same way:

\begin{protocol}{Blind $\QPIP_1$ 1-message protocol $\PiSampB = (\PSampB, \VSampB)$ for $\SampBQP$ language $L$}
	\label{proto:BlindSampBQP}
	Verifier's input: $x\in\set{0,1}^n$.
	
	Let $d=\poly(n)$ be the depth of the circuit corresponding to $\cPSamp{1}$.
	\begin{enumerate}
		\item The verifier generates $(pk, sk, evk)\leftarrow\mathsf{QHE.Keygen}(1^n, 1^d)$, where $d$ is the depth of the circuit corresponding to $\cPSamp{1}(x)$.
			It then generates and sends $ct_x\leftarrow\mathsf{QHE.Enc}_{pk}(x)$ to the prover.
			It also sends $evk$ to the prover.
		\item The prover homomorphically evaluates $\cPSamp{1}$ on input $ct_x$, generating $\sigma\leftarrow\mathsf{QHE.Eval}_{evk}(\cPSamp{1}, ct_x)$.
			It sends $\sigma$ to the verifier qubit-by-qubit.
		\item The verifier generates $\Lambda\leftarrow\cVSamp{2}(x)$,
			$\Lambda'\leftarrow\cM_\QHE$,
			$ct_w\leftarrow\Lambda'(\sigma)$,
			$w\leftarrow\widehat{\mathsf{QHE.Dec}_{sk}}(ct_w)$,
			and $o\leftarrow\cVSamp{3}(x, \Lambda, w)$.
			It outputs $o$.
	\end{enumerate}
\end{protocol}

\begin{thm}
	$\PiSampB$ has the same completeness and soundness guarantees as $\PiSamp$ and is blind.
\end{thm}
\begin{prf}
	The proof of \cref{thm:BlindBQP} works, up to changing some labels.
	Note that we don't have negligible soundness errors here, since $\PiSamp$ doesn't either.
	The proof only shows that the compiliation preserves soundness.
\end{prf}
