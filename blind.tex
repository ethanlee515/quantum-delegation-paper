\section{Constant-Round, Blind, and Verifiable Delegation}

We now present a constant-round, blind, and verifiable delegation scheme for $\SampBQP$ as an application of our $\QPIP_0$ construction for $\SampBQP$,
which we achieve by using the homomorphic encryption scheme from \cite{mahadev_qfhe} together with our construction.

\subsection{Our Delegation Protocol}

Now we compile our $\QPIP_1$ protocol \myprotoref{protoQPIP1} using the quantum homomorphic encryption $\mathsf{QHE}$ \Ethan{TODO maybe wrap this as a def} in a similar way as what we did for our blind $\BQP$ delegation protocol in \cref{sec:BlindQBP}

The construction is obvious, \Ethan{Maybe still write it out...? Or maybe not?} so we directly go into analysis.

\begin{thm}
	Compiling \cref{proto:QPIP0samp} using $\mathsf{QHE}$ preserves its completeness and soundness.
\end{thm}
\begin{proof}
	A honest prover still gets accepted with overwhelming probability by the correctness of $\mathsf{QHE}$. \Ethan{Not sure if too hand-wavy}

	\Ethan{Check how much of the following still applies before erasing}
	Suppose $\bbV$ accepts $\bbP'$ at least $\delta$ of the time.

	Then the $\QPIP_0$ delegation must accept at least that often too,
	so the client will receive a sample from $\mathsf{Eval}_{evk}(C, \tilde{x})$ with at most inverse poly error. \Ethan{security param?}

	The verifier ends up with a state within inverse poly distance to
		$$\widehat{\mathsf{QHE.Dec}_{sk}}\circ\Lambda'(\tilde{C}(\tilde{x}))$$
		$$=\Lambda\circ\mathsf{QHE.Dec}_{sk}(\tilde{C}(\tilde{x}))$$
		$$=\Lambda\circ\mathsf{QHE.Dec}_{sk}(\mathsf{QHE.Eval}_{evk}(C, \tilde{x}))$$

	By the properties of homomoprhic encryption, with overwhelming probability this is indistinguishable to \Ethan{Check if it's true} $\Lambda(C(x))$
\end{proof}

\begin{thm}
	\myprotoref{ProtoPriv} is IND-CPA secure.
\end{thm}
\begin{proof}
	The verifier's first message is encrypted into $\tilde{x}$ in an IND-CPA way as a ciphertext.
	The verifier's second message is the basis choice of Hadamard or test rounds, which can be done using public coins.

	The verifier only sends the prover these two messages, so it follows \Ethan{hopefully?} easily that the protocol itself is also IND-CPA. \Ethan{Maybe not rigorous enough?}
\end{proof}
