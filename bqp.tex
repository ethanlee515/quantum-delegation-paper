\section{Constant-Round Blind Delegation for BQP}

\Ethan{Need to define QPIP for both decision and sampling problems and point out the differences then}

We'll define the following protocol for $\BQP$ which we later generalize.

Let $L\in\BQP$, and $x\in\set{0,1}^*$.

There exists some certificate for both $x\in L$ and $x\notin L\Rightarrow x\in\overline{L}$, since $L, \overline{L}\in\QMA$.
In fact, they have the same certificate, and the certificate can be check using only $X$ and $Z$ measurements, which we'll call $\ket{\psi_x}$. \Ethan{TODO cite textbook for this}

\begin{thm}
	Let $n$ be large enough to achieve negligible completeness and soundness for the corresponding $\QMA$ certificate checks.
	Then \protoref{BQP} is a $\QPIP_1$ and constant round protocol for $\BQP$ with negligible completeness and soundness.
\end{thm}

\floatname{algorithm}{Protocol}
\begin{algorithm}
	\caption{Protocol for $\BQP$}
	\label{proto:BQP}
	\begin{algorithmic}[1]
		\State The verifier sends $x$ to the prover
		\State The prover sends $2n$ copies of $\ket{\psi_x}$ to the verifier
		\State The verifier uses $n$ copies to check for $x\in L$, and the other $n$ copies to check for $x\in\overline{L}$
	\end{algorithmic}
\end{algorithm}

Now we generalize the previous construction to achieve blindness.
We denote the prover's action as $\cF$. That is, $\cF(x)=\ket{\psi_x}^{\otimes 2n}$.

\begin{algorithm}
	\caption{Blind Protocol for $\BQP$}
	\label{proto:BlindBQP}
	\begin{algorithmic}[1]
		\State Let $\lambda$ be the security parameter.
		\State Let $n$ be large enough to achieve negligible completeness and soundness for the corresponding $\QMA$ certificate checks.
		\State The verifier runs $\mathsf{QHE.Keygen}(\lambda)$ to get $pk, sk$.
		\State The verifier encrypts $x$ to get $\mathsf{QHE.Enc}_{pk}(x)$
		\State The verifier sends $c=\mathsf{QHE.Enc}_{pk}(x)$ and $pk$ to the prover
		\State The prover sends $\mathsf{QHE.Eval}(\cF, c)$ to the verifier
		\State The verifier measures the incoming ciphertext according to \autoref{decodeMeasureOrder} and how it would've checked the $\QMA$ certificates
		\State The verifier decodes the measurement results
		\State The verifier uses $n$ copies to check for $x\in L$, and the other $n$ copies to check for $x\in\overline{L}$
	\end{algorithmic}
\end{algorithm}

As we will see in the theorems below, \protoref{BlindBQP} is a $\QPIP_1$, constant round, and IND-CPA protocol for $\BQP$ with negligible completeness and soundness.

\begin{thm}
	\protoref{BlindBQP} has negligible completeness
\end{thm}
\begin{proof}
	Based on the correctness of $\mathsf{QHE}$, the verifier would successfully decode $\cF(c)$ with overwhelming probability.
	The subsequent underlying QMA check also has overwhelming probability of acceptance.
\end{proof}

\begin{thm}
	\protoref{BlindBQP} has negligible soundness
\end{thm}
\begin{proof}
	The verifier's actions is equivalent to first decoding the prover's message then measuring it.

	Suppose the prover's message decodes to some $\rho$. Since it is a quantum state, the soundness of the underlying $\QMA$ instance can be used.
	
	That is, there doesn't exists a valid certificate $\rho$ that has non-negligible probability of acceptance.
\end{proof}

\begin{thm}
	\protoref{BlindBQP} is IND-CPA
\end{thm}
\begin{proof}
	The only thing the prover learns is $c$ and $pk$, so this follows directly from $\mathsf{QHE}$'s security.
\end{proof}

Now we can simply apply the following results from \Ethan{TODO cite}...

\begin{thm}
	There is a way to compile a $\QPIP_1$ protocol with the following properties:
	\begin{itemize}
		\item classical inputs
		\item only X/Z measurements
		\item negl. completeness and soundness
	\end{itemize}
	into a $\QPIP_0$ protocol with negl. completeness and soundness
\end{thm}

With this result, we now achieve...
\begin{thm}
	There exists...
\end{thm}
\begin{proof}
	Just compile A through B...
\end{proof}

\floatname{algorithm}{Algorithm}
