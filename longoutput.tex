\section{Long Output scheme}

Here we describe the scheme between the classical client and an honest quantum server. We mostly follow the logic in \cite{mahadev_delegation}. We first follow \cite{kempe_kitaev_regev_2006} to reduce the problem to an instance of local Hamiltonian.

We pick our universal gate set to be Hadamard gate and controlled phase gate following \cite{quant-ph/0301040}:
$$H=\frac{1}{\sqrt{2}}\begin{pmatrix}1&1\\1&-1\end{pmatrix}$$
$$\Lambda(P(i))=\begin{pmatrix}1&0&0&0\\0&1&0&0\\0&0&1&0\\0&0&0&i\end{pmatrix}$$
This set is proven to be universal in \cite{kitaev_1997}.

\subsection{preprocess the circuit}

Here we first reduce the circuit to real gates and states only. The proof is taken from \cite{quant-ph/0301040}.

We consider the transform $T\ket{\psi}=\ket{0}\otimes\Re\ket{\psi}+\ket{1}\otimes\Im\ket{\psi}$, where $\Re$ and $\Im$ denote real and imaginary parts respectively. Note that after this transformation, the state is real. This transformation preserves measurement results in the standard basis. Under the transform, Hadamard gates are unchanged, but controlled phase gates become a combination of Hadamard and Toffoli gates. Mathematically,
$$T\circ H_f=H_f\circ T$$
$$T\circ\Lambda_f(P(i)_s)=\Lambda^2_{f,s}(X_0)\Lambda^2_{f,s}(Z_0)\circ T=\Lambda^2_{f,s}(X_0)H_0\Lambda^2_{f,s}(X_0)H_0\circ T$$
Where $\ket{\psi}$ is 1-indexed, and the added qubit is in the 0th position. 

We then double (TODO: maybe multiply by $\gamma>1$ instead of doubling?) the size of the circuit by padding identity matrices at the end. This is so that the final output has a higher chance to be measured in the final result in the following reduction.

After these transformations, let $x$ be the new circuit input, and $U_T...U_1$ be the circuit to evaluate.

\subsection{Constructing a local Hamiltonian instance}

We then attempt to construct a local Hamiltonian with ground state that encodes the history of the computation: $$\phi=\sum_{t=0}^TU_t...U_1\ket{x}\otimes\ket{\hat{t}}$$
We do so by ensuring states perpendicular to it have high eigenvalues. The base construction comes from \cite{kitaev2002classical}. The simplification is taken from \cite{biamonte_love_2008}.

First, we ensure that the invalid clock states have high eigenvalues by applying the following Hamiltonian to the time register.
$$H_{clock}=\sum_{t=1}^{T-1}\ket{01}\bra{01}_{t,t+1}$$
We also set $S_{legal}=\ker H_{clock}$. Note that $S_{legal}$ is precisely the valid clock states; that is, valid unary representations in the time register. Note that $H_{clock}$ can also be written in terms of linear combinations of tensors of $X$, $Z$, and $I$ gates. TODO check if this is obvious. If not, explain why.
$$H_{clock}=\frac{1}{4}((T-1)I + Z_1 - Z_T - \sum_t^{T-1}Z_tZ_{t+1})$$

Then, we ensure that the initial condition is set up correctly. Let $n$ be the number of the qubits in the circuit.
$$H_{in}=\sum_i^n(I-\ket{x_i}\bra{x_i})\otimes\ket{0}\bra{0}_1$$
Where $x_i$ is the $i$th bit of $x$. The kernel of this is precisely where everything is set up consistently with $\phi$ in time step $t=0$. Furthermore, this can be written in terms of linear combinations of tensors of $X$, $Z$, and $I$ gates as
$$H_{in}=\frac{1}{4}\sum_{i=1}^n(I-(-1)^{x_i}Z_i)\otimes(I+Z_1)$$

TODO define $S_{in}$ when projection lemma is used

Then, we ensure that the gates are applied correctly.
$$H_{prop1}=\sum_{t\in T_1}H_{prop,t}$$
$$H_{prop,t}=I\otimes\ket{\widehat{t}}\bra{\widehat{t}}
	+I\otimes\ket{\widehat{t-1}}\bra{\widehat{t-1}}
	-U_t\otimes\ket{\widehat{t}}\bra{\widehat{t-1}}
	-U_t^\dagger\otimes\ket{\widehat{t-1}}\bra{\widehat{t}}$$

I probably need to explain why this works but TODO I'll leave it for another day.

Note that $U^\dagger=U$, since the gates are either Hadamard or Toffoli. Additionally, $\frac{1}{2}(I-Z_{t-1})$ annihilates time steps before $t-1$. $\frac{1}{2}(I+Z_{t+1})$ similarly annihilates steps $t+1$ and after.
So we can write
$$H_{prop,t}=\frac{I}{4}\otimes(I-Z_{t-1})(I+Z_{t+1})-\frac{U}{4}\otimes(I-Z_{t-1})X_t(I+Z_{t+1})$$

\subsection{2-local ZX Hamiltonians}

We then follow \cite{biamonte_love_2008} in writing each terms of $H$ into only $Z$ and $X$ measurements...

This probably involves some kind of basis arguments on the Hamiltonians and maybe some stuff that's a bit more ugly. This should not change the state. I'll have to write this out later...

\subsection{Checking computation}

We then follow \cite{mahadev_delegation}. That is, the server would commit copies of the ground state of the Hamiltonian for the classical client to measure. The client would check whether the computation is done correctly using a modified Hamiltonian from \cite{kempe_kitaev_regev_2006}. 

And then, this in fact involves some probabilistic analysis shenanigans that we haven't fully figured out yet.

\subsection{Getting the Output}

Thanks to having padded the circuit with identity matrices at the end, if we measured $T>\frac{T}{2}$ on the time register, we would know that the other registers include the output of the required quantum computation. Now I would analyze the chances of getting this right, maybe in terms of fidelity or in terms of distance between this measured distribution and the ideal distribution...


